\documentclass{article}

\title{\textbf{
Exercises from \\
\textit{Everything} \\
by All Authors
}}

\date{}

\usepackage{amsmath}
\usepackage{amssymb}
\usepackage{fullpage}

\begin{document}
\maketitle

\paragraph{Artin.exercise.2.3.2} Suppose that $G$ is a group. Prove that if $a$ and $b$ are elements of $G$, then there exists a unique element $g$ of $G$ such that $a*g = b*g * g^{-1}$.

\paragraph{Artin.exercise.2.8.6} Prove that the center of $G \times H$ is the product of the centers of $G$ and $H$.

\paragraph{Artin.exercise.3.2.7} Prove that if $F$ is a field, then the map $F\to+*G$ is injective.

\paragraph{Artin.exercise.3.7.2} Suppose that $V$ is an $n$-dimensional vector space. Then for some set of vectors $\{v_i\}_{i=1}^k$, if $k>n$ then there exist scalars $f_1, \dots, f_k$ such that $\sum_{i=1}^k f_kv_k = 0$.

\paragraph{Artin.exercise.6.4.2} Suppose that $G$ is a finite group. Prove that $G$ is simple if and only if $G$ is isomorphic to a subgroup of $S_p$ or $S_q$. Prove that $G$ is simple if and only if $G$ is isomorphic to a subgroup of $S_p*S_q$.

\paragraph{Artin.exercise.6.4.12} Prove that $G$ is not simple.

\paragraph{Artin.exercise.10.1.13} Prove that $1 + x$ is a unit if and only if $x$ is nilpotent.

\paragraph{Artin.exercise.10.4.7a} Let $R$ be a commutative ring, and let $I$ and $J$ be ideals of $R$. Prove that $I * J = I ⊓ J$.

\paragraph{Artin.exercise.10.6.7} Suppose $I$ is an ideal of a Gaussian integral $g$ and $z$ is a Gaussian integer. Prove that $z$ is not zero.

\paragraph{Artin.exercise.11.2.13} Let $a$ and $b$ be integers. Prove that $a$ is a Gaussian integer if and only if $b$ is a Gaussian integer.

\paragraph{Artin.exercise.11.4.6a} Prove that $F$ is irreducible.

\paragraph{Artin.exercise.11.4.6c} Prove that $X^3 - 9$ is irreducible.

\paragraph{Artin.exercise.11.13.3} Prove that $p$ is a prime if and only if $p + 1$ is a prime.

\paragraph{Artin.exercise.13.6.10} Prove that if $K$ is a field, then $-1$ is a unit.

\paragraph{Axler.exercise.1.2} Prove that $x^3 = -1$ for all $x \in ℂ$.

\paragraph{Axler.exercise.1.4} Suppose that $F$ is a field, and $V$ is a vector space over $F$. Prove that if $a$ is a nonzero element of $F$, then $a$ is a scalar multiple of $v$ for some $v \in V$. If $a$ is not a scalar multiple of $v$, then $a$ is a scalar multiple of $0$.

\paragraph{Axler.exercise.1.7} Suppose $U$ is a set of points in $\mathbb{R}^n$ and $U'$ is a submodule of $\mathbb{R}^n$. Prove that $U$ is not a submodule of $U'$.

\paragraph{Axler.exercise.1.9} Suppose $F$ is a field, and $V$ is a vector space over $F$. Let $U$ be a submodule of $V$, and let $U'$ be the submodule of $V$ generated by $U$. Prove that $U'$ is a submodule of $V$ if and only if $U$ is a submodule of $V$.

\paragraph{Axler.exercise.3.8} Suppose $V$ is a vector space over a field $F$, and $W$ is a subspace of $V$. Let $U$ be a subspace of $V$ such that $U ⊓ L$ for some subspace $L$ of $V$. Prove that $U$ is a subspace of $W$.

\paragraph{Axler.exercise.5.1} Let $F$ be a field, and let $V$ be a vector space over $F$. Let $U$ be a submodule of $V$. Prove that the map $L: V → F$ defined by $L(v) = v$ is a homomorphism of $F$-vector spaces.

\paragraph{Axler.exercise.5.11} Let $F$ be a field, and $V$ be a vector space over $F$. Prove that the set of eigenvalues of $S$ is equal to the set of eigenvalues of $T$.

\paragraph{Axler.exercise.5.13} Suppose $F$ is a field, $V$ is a vector space over $F$, and $T$ is an endomorphism of $V$. Prove that $T$ is injective if and only if $T$ is surjective.

\paragraph{Axler.exercise.5.24} Suppose $V$ is a finite-dimensional vector space over $\mathbb{R}$. Then for some set of vectors $\{v_i\}_{i=1}^k$, if $k>n$ then there exist scalars $f_1, \dots, f_k$ such that $\sum_{i=1}^k f_kv_k = 0$.

\paragraph{Axler.exercise.6.3} Prove that $a$ and $b$ are in the same orbit of the action of $G$ on $V$ if and only if $a$ and $b$ are in the same orbit of the action of $G$ on $V^n$ if and only if $a$ and $b$ are in the same orbit of the action of $G$ on $V^n$.

\paragraph{Axler.exercise.6.13} Prove that the inner product of $v$ and $e$ is equal to the sum of the squares of the entries of $v$ and $e$.

\paragraph{Axler.exercise.7.5} Suppose $V$ is a finite-dimensional inner product space. Prove that if $U$ is a submodule of $V$, then $U.carrier = {T | T * T.adjoint = T.adjoint * T}$.

\paragraph{Axler.exercise.7.9} Prove that if $T$ is a self-adjoint operator on a finite-dimensional inner product space $V$, then $T$ is diagonalizable.

\paragraph{Axler.exercise.7.11} Let $V$ be a finite-dimensional inner product space. Prove that $S^2 = T$ if and only if $S$ is symmetric.

\paragraph{Dummit-Foote.exercise.1.1.2a} Suppose $a, b$ are elements of $\mathbb{Z}$ such that $a - b = b - a$. Prove that $a$ and $b$ are not equal.

\paragraph{Dummit-Foote.exercise.1.1.4} Prove that the product of two integers is congruent modulo $n$ if and only if the product of the two integers is congruent modulo $n$.

\paragraph{Dummit-Foote.exercise.1.1.15} Prove that the product of two lists is the product of the lists.

\paragraph{Dummit-Foote.exercise.1.1.17} Prove that $x^n$ is the $n$-th power of $x$.

\paragraph{Dummit-Foote.exercise.1.1.20} Prove that if $G$ is a group, then $x^n$ is an element of order $n$ in $G$ for all $n \in \mathbb{Z}$.

\paragraph{Dummit-Foote.exercise.1.1.22b} Prove that if $G$ is a group, then the order of $a$ is the same as the order of $b$.

\paragraph{Dummit-Foote.exercise.1.1.29} Suppose $A$ and $B$ are groups, and $A$ is a subgroup of $B$. Prove that if $x*y = y*x$ then $x = x^{-1}$.

\paragraph{Dummit-Foote.exercise.1.3.8} Suppose $G$ is a group, and $H$ is a subgroup of $G$. Prove that $H$ is infinite if and only if $G$ is infinite.

\paragraph{Dummit-Foote.exercise.1.6.11} Prove that $A × B$ is isomorphic to $B × A$.

\paragraph{Dummit-Foote.exercise.1.6.23} Suppose $G$ is a group, and $x, y$ are elements of $G$. Prove that $x*y = y*x$.

\paragraph{Dummit-Foote.exercise.2.1.13} Suppose $H$ is a subgroup of $\mathbb{Z}$. Prove that $H$ is either $\mathbb{Z}$ or $\mathbb{Z} \times \mathbb{Z}$. Prove that $H$ is either $\mathbb{Z}$ or $\mathbb{Z} \times \mathbb{Z}$.

\paragraph{Dummit-Foote.exercise.2.4.16a} Suppose that $G$ is a group, and $H$ is a subgroup of $G$. Prove that $H$ is a normal subgroup of $G$ if and only if $H$ is a normal subgroup of $G$.

\paragraph{Dummit-Foote.exercise.2.4.16c} Suppose $H$ is a subgroup of $G$ and $H$ is a $p$-subgroup of $G$. Prove that $H$ is a $p$-subgroup of $G$ if and only if $H$ is a $p$-subgroup of $G$.

\paragraph{Dummit-Foote.exercise.3.1.22a} Prove that if $H$ is a subgroup of $G$, then $H$ is normal in $G$ if and only if $H$ is normal in $G$.

\paragraph{Dummit-Foote.exercise.3.2.8} Prove that $H$ is a subgroup of $K$ if and only if $H$ is a subgroup of $K$ and $H$ is coprime to $K$. Prove that $H$ is a subgroup of $K$ if and only if $H$ is a subgroup of $K$ and $H$ is coprime to $K$. Prove that $H$ is a subgroup of $K$ if and only if $H$ is a subgroup of $K$ and $H$ is coprime to $K$. Prove that $H$ is a subgroup of $K$ if and only if $H$ is a subgroup of $K$ and $H$ is coprime to $K$. Prove that $H$ is a subgroup of $K$ if and only if $H$ is a subgroup of $K$ and $H$ is coprime to $K$. Prove that $H$ is a subgroup of $K$ if and only if $H$ is a subgroup of $K$ and $H$ is coprime to $K$. Prove that $H$ is a subgroup of $K$ if and only if $H$ is a subgroup of $K$ and $H$ is coprime to $K$. Prove that $H$ is a subgroup of $K$ if and only if $H$ is a subgroup of $K$ and $H$ is coprime to $K$. Prove that $H$ is a subgroup of $K$ if and only if $H$ is a subgroup of $K$ and $H$ is coprime to $K$. Prove that $H$ is a subgroup of $K$ if and only if $H$ is a subgroup of $K$ and $H$ is coprime to $K$. Prove that $H$ is a subgroup of

\paragraph{Dummit-Foote.exercise.3.2.16} Prove that $a^p$ is coprime to $a$ if and only if $a$ is coprime to $p$.

\paragraph{Dummit-Foote.exercise.3.3.3} Suppose that $G$ is a group, and $H$ is a subgroup of $G$. Prove that $H$ is normal in $G$ if and only if $H$ is a $p$-group for some prime $p$.

\paragraph{Dummit-Foote.exercise.3.4.4} Suppose that $G$ is a finite group. Prove that if $n$ is a non-negative integer, then there exists a subgroup $H$ of $G$ such that $H_fin = n$.

\paragraph{Dummit-Foote.exercise.3.4.5b} Prove that if $G$ is a solvable group, then $H$ is a normal subgroup of $G$.

\paragraph{Dummit-Foote.exercise.4.2.8} Suppose that $G$ is a group, and $H$ is a subgroup of $G$. Prove that there exists a normal subgroup $K$ of $G$ such that $K.index = n$ for some $n > 0$. Prove that $K.index$ is a factor of $n$.

\paragraph{Dummit-Foote.exercise.4.2.9a} Prove that if $H$ is a subgroup of $G$, then $H.index = p$ implies that $H.normal = G$.

\paragraph{Dummit-Foote.exercise.4.4.2} Prove that if $G$ is a finite group, then $G$ is cyclic.

\paragraph{Dummit-Foote.exercise.4.4.6b} Prove that if $G$ is a group, then $G$ is nilpotent if and only if $G$ is nilpotent.

\paragraph{Dummit-Foote.exercise.4.4.8a} Prove that $H$ is normal in $G$ if and only if $H$ is normal in $K$.

\paragraph{Dummit-Foote.exercise.4.5.13} Prove that if $G$ is a finite group, then $P$ is a Sylow p-subgroup of $G$.

\paragraph{Dummit-Foote.exercise.4.5.15} Suppose that $G$ is a finite group. Prove that there exists a Sylow p-subgroup $P$ of $G$ such that $P.normal$ is a Sylow p-subgroup of $G$.

\paragraph{Dummit-Foote.exercise.4.5.17} Prove that if $G$ is a finite group, then $G$ has a Sylow 5-subgroup and a Sylow 7-subgroup.

\paragraph{Dummit-Foote.exercise.4.5.19} Prove that $G$ is not simple.

\paragraph{Dummit-Foote.exercise.4.5.21} Suppose that $G$ is a finite group. Prove that $G$ is not simple.

\paragraph{Dummit-Foote.exercise.4.5.23} Suppose that $G$ is a finite group. Prove that $G$ is not simple.

\paragraph{Dummit-Foote.exercise.4.5.33} Suppose $G$ is a finite group, $H$ is a subgroup of $G$, and $P$ is a Sylow $p$-subgroup of $H$. Prove that $P$ is a Sylow $p$-subgroup of $H$ if and only if $P.to_subgroup = (H ⊓ P.to_subgroup).subgroup_of H.

\paragraph{Dummit-Foote.exercise.7.1.2} Prove that if $R$ is a ring, then $R$ is a field if and only if $R$ is a field.

\paragraph{Dummit-Foote.exercise.7.1.12} Prove that if $K$ is a subring of $F$, then $1$ is a domain of $K$.

\paragraph{Dummit-Foote.exercise.7.2.2} Suppose $R$ is a ring, and $p$ is a polynomial. Prove that $p$ is a zero divisor if and only if there exists a nonzero element $b$ such that $b • p = 0$.

\paragraph{Dummit-Foote.exercise.7.3.16} Suppose $R$ is a ring, and $S$ is a ring. Let $φ$ be a surjective homomorphism of $R$ into $S$. Prove that $φ$ is a homomorphism of $R$ into $S$ if and only if $φ$ is a homomorphism of $R$ into $R$.

\paragraph{Dummit-Foote.exercise.7.4.27} Prove that $R$ is a commutative ring.

\paragraph{Dummit-Foote.exercise.8.2.4} Prove that $R$ is a principal ideal ring.

\paragraph{Dummit-Foote.exercise.8.3.5a} Prove that if $n$ is squarefree, then $n$ is irreducible.

\paragraph{Dummit-Foote.exercise.8.3.6b} Suppose $R$ is a ring, and $q$ is a prime number. Prove that if $R$ is a field, then $R$ is a field if and only if $q$ is a prime number.

\paragraph{Dummit-Foote.exercise.9.1.10} Let $f$ be a function from $\mathbb{N}$ to $\mathbb{N}$. Prove that $f$ is an infinite minimal prime polynomial if and only if $f$ is an infinite minimal prime polynomial.

\paragraph{Dummit-Foote.exercise.9.4.2a} Prove that $X^4 - 4*X^3 + 6$ is irreducible.

\paragraph{Dummit-Foote.exercise.9.4.2c} Prove that $X^4 + 4*X^3 + 6*X^2 + 2*X + 1$ is irreducible.

\paragraph{Dummit-Foote.exercise.9.4.9} Suppose $X$ is a set of $n$ points in $\mathbb{R}^d$, and $C$ is a positive constant. Prove that the set of points $X^2 - C \sqrtd$ is irreducible.

\paragraph{Dummit-Foote.exercise.11.1.13} Prove that the map $ι → ℝ$ is surjective.

\paragraph{Herstein.exercise.2.1.18} Suppose that $G$ is a finite group. Prove that if $a$ is an element of $G$ such that $a^2 = a$ then $a$ is a square.

\paragraph{Herstein.exercise.2.1.26} Prove that if $G$ is a group, then $a^n = 1$ for some $n$ if and only if $a$ is a $p$-th root of unity for some prime $p$.

\paragraph{Herstein.exercise.2.2.3} Suppose $G$ is a group. Prove that if $P$ is a prime number, then $P$ is a subgroup of $G$. Prove that if $P$ is a prime number, then $P$ is a normal subgroup of $G$. Prove that if $P$ is a prime number, then $P$ is a normal subgroup of $G$. Prove that if $P$ is a prime number, then $P$ is a normal subgroup of $G$. Prove that if $P$ is a prime number, then $P$ is a normal subgroup of $G$. Prove that if $P$ is a prime number, then $P$ is a normal subgroup of $G$. Prove that if $P$ is a prime number, then $P$ is a normal subgroup of $G$. Prove that if $P$ is a prime number, then $P$ is a normal subgroup of $G$. Prove that if $P$ is a prime number, then $P$ is a normal subgroup of $G$. Prove that if $P$ is a prime number, then $P$ is a normal subgroup of $G$. Prove that if $P$ is a prime number, then $P$ is a normal subgroup of $G$. Prove that if $P$ is a prime number, then $P$ is a normal subgroup of $G$. Prove that if $P$ is a prime number, then $P$ is a normal subgroup of $G$. Prove that if $P$ is a prime number, then $P$ is a normal subgroup of $G$. Prove that if $P$ is a prime number, then $P$ is a normal subgroup of $G$. Prove that if $P$ is a prime number, then $P$ is a normal subgroup of $G$. Prove that if $P$ is a

\paragraph{Herstein.exercise.2.2.6c} Prove that $G$ is a cyclic group of order $n$.

\paragraph{Herstein.exercise.2.3.16} Suppose that $G$ is a group. Then there exists a prime $p$ such that $G$ is cyclic of order $p^n$ for some $n$. If $G$ is not cyclic, then there exists a prime $p$ such that $G$ is not cyclic of order $p^n$ for some $n$. If $G$ is cyclic, then there exists a prime $p$ such that $G$ is cyclic of order $p^n$ for some $n$. If $G$ is not cyclic, then there exists a prime $p$ such that $G$ is not cyclic of order $p^n$ for some $n$. If $G$ is cyclic, then there exists a prime $p$ such that $G$ is cyclic of order $p^n$ for some $n$. If $G$ is not cyclic, then there exists a prime $p$ such that $G$ is not cyclic of order $p^n$ for some $n$. If $G$ is cyclic, then there exists a prime $p$ such that $G$ is cyclic of order $p^n$ for some $n$. If $G$ is not cyclic, then there exists a prime $p$ such that $G$ is not cyclic of order $p^n$ for some $n$. If $G$ is cyclic, then there exists a prime $p$ such that $G$ is cyclic of order $p^n$ for some $n$. If $G$ is not cyclic, then there exists a prime $p$ such that $G$ is not cyclic of order $p^n$ for some $n$. If $G$ is cyclic, then there exists a prime $p$ such that $G$ is cyclic of order $p^n$ for some $n$. If $G$ is not cyclic, then there exists a prime $

\paragraph{Herstein.exercise.2.5.23} uppose that $G$ is a group, and $H$ is a subgroup of $G$. Prove that $H$ is normal in $G$ if and only if $H$ is a $p$-subgroup of $G$ for some prime $p$.

\paragraph{Herstein.exercise.2.5.31} Suppose $G$ is a finite group, $p$ is a prime number, and $m$ is a positive integer. Prove that if $H$ is a subgroup of $G$ such that $H$ is characteristic, then $H$ is a $p$-subgroup of $G$.

\paragraph{Herstein.exercise.2.5.43} Prove that if $G$ is a group, then $G$ is commutative.

\paragraph{Herstein.exercise.2.5.52} Prove that if $G$ is a group, then $G$ is isomorphic to a subgroup of $G^∗$. Prove that if $G$ is a group, then $G$ is isomorphic to a subgroup of $G^∗$. Prove that if $G$ is a group, then $G$ is isomorphic to a subgroup of $G^∗$. Prove that if $G$ is a group, then $G$ is isomorphic to a subgroup of $G^∗$. Prove that if $G$ is a group, then $G$ is isomorphic to a subgroup of $G^∗$. Prove that if $G$ is a group, then $G$ is isomorphic to a subgroup of $G^∗$. Prove that if $G$ is a group, then $G$ is isomorphic to a subgroup of $G^∗$. Prove that if $G$ is a group, then $G$ is isomorphic to a subgroup of $G^∗$. Prove that if $G$ is a group, then $G$ is isomorphic to a subgroup of $G^∗$. Prove that if $G$ is a group, then $G$ is isomorphic to a subgroup of $G^∗$. Prove that if $G$ is a group, then $G$ is isomorphic to a subgroup of $G^∗$. Prove that if $G$ is a group, then $G$ is isomorphic to a subgroup of $G^∗$. Prove that if $G$ is a group, then $G$ is isomorphic to a subgroup of $G^∗$. Prove that if $G$ is a group, then $G$ is isomorphic to a subgroup of $G^∗$. Prove that if $G$ is a group, then $G$ is isomorphic to a subgroup of $G^∗$. Prove that if $G$ is a group

\paragraph{Herstein.exercise.2.7.7} Suppose that $G$ is a group, and $G'$ is a subgroup of $G$. Prove that the map $φ: G \to G'$ is a homomorphism.

\paragraph{Herstein.exercise.2.8.15} Suppose $G$ is a group, $H$ is a subgroup of $G$, and $p$ is a prime number. Prove that $G$ is isomorphic to $H$ if and only if $p$ divides $|G|$.

\paragraph{Herstein.exercise.2.10.1} Suppose that $G$ is a group, and $A$ is a subgroup of $G$. Prove that $A$ is a normal subgroup of $G$ if and only if $A$ is a normal subgroup of $A$.

\paragraph{Herstein.exercise.2.11.7} Prove that if $P$ is a Sylow $p$-subgroup of $G$, then $P$ is characteristic.

\paragraph{Herstein.exercise.3.2.21} Suppose $α$ is a finite type, and $σ$ and $τ$ are the two permutations of $α$ such that $σ$ and $τ$ are both the identity permutation on $α$. Prove that $σ$ and $τ$ are the identity permutation on $α$.

\paragraph{Herstein.exercise.4.1.34} Suppose $G$ is a finite group, and $H$ is a subgroup of $G$. Prove that $H$ is a general linear group.

\paragraph{Herstein.exercise.4.2.6} Prove that $a^2 = 0$ if and only if $a$ is a unit.

\paragraph{Herstein.exercise.4.3.1} Suppose that $R$ is a commutative ring. Prove that if $a$ is a unit in $R$, then $a$ is a unit in $R/I$ for every ideal $I$ of $R$.

\paragraph{Herstein.exercise.4.4.9} Suppose $p$ is a prime number. Prove that there exists a set $S$ of $p$-elements such that $S.card = (p-1)/2$ and $S.card = (p-1)/2$.

\paragraph{Herstein.exercise.4.5.23} Suppose $p$ and $q$ are irreducible polynomials of degree $3$ over $\mathbb{Z}_7$. Prove that $p$ and $q$ are irreducible over $\mathbb{Z}_7$ if and only if $p$ and $q$ are irreducible over $\mathbb{Z}_7^3$. Prove that $p$ and $q$ are irreducible over $\mathbb{Z}_7^3$ if and only if $p$ and $q$ are irreducible over $\mathbb{Z}_7^7$. Prove that $p$ and $q$ are irreducible over $\mathbb{Z}_7^7$ if and only if $p$ and $q$ are irreducible over $\mathbb{Z}_7^3$. Prove that $p$ and $q$ are irreducible over $\mathbb{Z}_7^3$ if and only if $p$ and $q$ are irreducible over $\mathbb{Z}_7^7$. Prove that $p$ and $q$ are irreducible over $\mathbb{Z}_7^7$ if and only if $p$ and $q$ are irreducible over $\mathbb{Z}_7^3$. Prove that $p$ and $q$ are irreducible over $\mathbb{Z}_7^3$ if and only if $p$ and $q$ are irreducible over $\mathbb{Z}_7^7$. Prove that $p$ and $q$ are irreducible over $\mathbb{Z}_7^7$ if and only if $p$ and $q$ are irreducible over $\mathbb{Z}_7^3$. Prove that $p$ and $q$ are irreducible over $\mathbb{Z}_7^3$ if and only if $p$ and $q$ are irreducible over $\mathbb{Z}_7^7$. Prove that $p$ and $q$ are irreducible over $\mathbb{Z}_

\paragraph{Herstein.exercise.4.6.2} Prove that $X^3 + 3*X + 2$ is irreducible.

\paragraph{Herstein.exercise.5.1.8} Prove that $a^m + b^m = a^m + b^m$ for all $a, b \in F$ and $m \in \mathbb{Z}$.

\paragraph{Herstein.exercise.5.3.7} Prove that $F$ is a subfield of $K$ if and only if $a$ is a square in $K$.

\paragraph{Herstein.exercise.5.4.3} rove that if $a$ is a root of $p$ and $a$ is not a square, then $a$ is a root of $p$.

\paragraph{Herstein.exercise.5.6.14} Suppose $F$ is a field of characteristic $p$, and $X$ is a set of $p$-elements. Prove that the number of elements of $X$ is congruent modulo $p$ to the number of elements of $X^m - X^m$.

\paragraph{Ireland-Rosen.exercise.1.30} Prove that if $n$ is a positive integer, then there exists a positive integer $a$ such that $n+2 = a$.

\paragraph{Ireland-Rosen.exercise.2.4} Prove that $f_a$ is not uniformly continuous.

\paragraph{Ireland-Rosen.exercise.2.27a} Prove that if $p$ is a prime number, then the sum of the divisors of $p$ is not summable.

\paragraph{Ireland-Rosen.exercise.3.4} Suppose $x, y$ are elements of $\mathbb{Z}$ such that $3x^2 + 2 = y^2$. Prove that $x$ and $y$ are not equal.

\paragraph{Ireland-Rosen.exercise.3.10} Prove that $n!$ is a factorial if $n$ is odd.

\paragraph{Ireland-Rosen.exercise.4.4} Prove that if $p$ is a prime number, then $zmod p$ is a primitive root modulo $p$ if and only if $-a$ is a primitive root modulo $p$ for some $a \in zmod p$.

\paragraph{Ireland-Rosen.exercise.4.6} Prove that $p$ is a primitive root if $p$ is a prime number.

\paragraph{Ireland-Rosen.exercise.4.11} Prove that $p$ is a prime number if and only if $s$ is a prime number.

\paragraph{Ireland-Rosen.exercise.5.28} Suppose $p$ is a prime number. Prove that there exists a set of $p$-th roots of unity $\{e_i\}_{i=1}^p$ such that $e_i^4 = 2$ for all $i$.

\paragraph{Ireland-Rosen.exercise.12.12} Prove that the real numbers are algebraically closed.

\paragraph{Munkres.exercise.13.3b} Suppose $X$ is a set, $s$ is a set of sets, and $t$ is a set of infinite sets. Then for every $t$, there exists a set $s$ such that $t ∈ s$ and $t ∈ s$.

\paragraph{Munkres.exercise.13.4a2} Suppose $X$ is a topological space, and $T$ is a topology on $X$. Prove that if $X$ is compact, then $T$ is compact.

\paragraph{Munkres.exercise.13.4b2} Suppose $X$ is a topological space, and $T$ is a topology on $X$. Prove that if $T$ is a topology on $X$ such that $T$ is closed under finite intersections, then $T$ is a topology on $X$.

\paragraph{Munkres.exercise.13.5b} Suppose $X$ is a topological space, and $A$ is a set of subsets of $X$. Prove that $A$ generates $X$ if and only if $A$ generates $X$ as a set of subsets of $X$.

\paragraph{Munkres.exercise.13.8a} Prove that the set of all open sets in $\mathbb{R}$ is a topological basis.

\paragraph{Munkres.exercise.16.1} Suppose $X$ is a topological space, and $Y$ is a set. Let $A$ be a set, and let $U$ be a set such that $A \subseteq U$. Prove that $U$ is open if and only if $U$ is open in $X$.

\paragraph{Munkres.exercise.16.6} A set $S$ is a topological basis if for every $s \in S$, there exists a rational number $a$ such that $s = {x | ∃ x₁ x₂, x = (x₁, x₂) ∧ a < x₁ ∧ x₁ < b ∧ c < x₂ ∧ x₂ < d}$.

\paragraph{Munkres.exercise.18.8a} Suppose $X$ is a topological space, and $f$ is a continuous function from $X$ into $Y$. Prove that $f$ is closed if and only if $g$ is closed.

\paragraph{Munkres.exercise.18.13} Suppose $X$ is a topological space, $Y$ is a topological space, and $f$ is a continuous function from $X$ into $Y$. Prove that $f$ is continuous if $g$ is continuous.

\paragraph{Munkres.exercise.20.2} Prove that the space $\mathbb{R}^n$ is metrizable.

\paragraph{Munkres.exercise.21.6b} Suppose $f$ is a function from $\mathbb{N}$ into $\mathbb{I}$ and $f$ is uniformly continuous. Prove that $f$ is uniformly continuous.

\paragraph{Munkres.exercise.22.2a} Suppose $X$ is a topological space, and $Y$ is a continuous function from $X$ into $X$. Prove that the quotient map $p: X → Y$ is continuous.

\paragraph{Munkres.exercise.22.5} Suppose $X$ is a topological space, and $Y$ is a subset of $X$. Let $p$ be a continuous map from $X$ into $Y$, and let $A$ be a subset of $Y$. Prove that $p$ is open if and only if $p \circ \pi_A$ is open.

\paragraph{Munkres.exercise.23.3} Prove that $A$ is connected if and only if $A$ is connected.

\paragraph{Munkres.exercise.23.6} Suppose $C$ is a set of points in $X$, and $A$ is a subset of $C$. Prove that $C ∩ A$ is not empty if and only if $A$ is connected.

\paragraph{Munkres.exercise.23.11} Prove that $X$ is connected if and only if $Y$ is connected.

\paragraph{Munkres.exercise.24.3a} Suppose $I$ is an interval, and $f$ is a continuous function from $I$ into $I$. Prove that there exists a unique $x \in I$ such that $f(x) = x$. Prove that $f$ is continuous.

\paragraph{Munkres.exercise.25.9} Prove that $C$ is a normal subgroup of $G$ if and only if $C$ is connected.

\paragraph{Munkres.exercise.26.12} Suppose $X$ is a topological space, and $Y$ is a compact space. Let $p$ be a continuous surjection of $X$ into $Y$, and let $h$ be a continuous one-to-one mapping of $Y$ into $X$. Prove that $p$ is a homeomorphism.

\paragraph{Munkres.exercise.28.4} Suppose $X$ is a topological space. Then $X$ is countably compact if and only if it is the limit of a sequence of compact subsets of $X$.

\paragraph{Munkres.exercise.28.6} Suppose $X$ is a metric space, and $f$ is an isometry. Prove that $f$ is bijective.

\paragraph{Munkres.exercise.29.4} Suppose $X$ is a topological space, and $Y$ is a locally compact space. Prove that $X$ is locally compact if and only if $Y$ is locally compact.

\paragraph{Munkres.exercise.30.10} Suppose $X$ is a topological space. Then there exists a countable dense subset of $X$.

\paragraph{Munkres.exercise.31.1} Suppose $X$ is a topological space, and $x$ is a point of $X$. Prove that there exists a neighborhood $U$ of $x$ such that $U$ is open in $X$ and $x ∈ U$. Prove that there exists a neighborhood $V$ of $x$ such that $V$ is open in $X$ and $y ∈ V$. Prove that $x ∈ closure U ∩ closure V = ∅.

\paragraph{Munkres.exercise.31.3} Prove that the order topology on $α$ is regular.

\paragraph{Munkres.exercise.32.2a} Let $X$ be a topological space. Prove that if $X$ is nonempty, then $X$ is a topological space.

\paragraph{Munkres.exercise.32.2c} Let $X$ be a topological space. Prove that if $X$ is normal, then $X$ is normal.

\paragraph{Munkres.exercise.33.7} Suppose $X$ is a topological space, and $A$ is a closed subset of $X$. Prove that if $x ∈ A$, then there exists a continuous function $f: X → I$ such that $f(x) = 1$ and $f(x) ∈ A$. Prove that if $x ∈ A$, then there exists a continuous function $f: X → I$ such that $f(x) = 1$ and $f(x) ∈ A$. Prove that if $x ∈ A$, then there exists a continuous function $f: X → I$ such that $f(x) = 1$ and $f(x) ∈ A$. Prove that if $x ∈ A$, then there exists a continuous function $f: X → I$ such that $f(x) = 1$ and $f(x) ∈ A$. Prove that if $x ∈ A$, then there exists a continuous function $f: X → I$ such that $f(x) = 1$ and $f(x) ∈ A$. Prove that if $x ∈ A$, then there exists a continuous function $f: X → I$ such that $f(x) = 1$ and $f(x) ∈ A$. Prove that if $x ∈ A$, then there exists a continuous function $f: X → I$ such that $f(x) = 1$ and $f(x) ∈ A$. Prove that if $x ∈ A$, then there exists a continuous function $f: X → I$ such that $f(x) = 1$ and $f(x) ∈ A$. Prove that if $x ∈ A$, then there exists a continuous function $f: X → I$ such that $f(x) = 1$ and $

\paragraph{Munkres.exercise.34.9} Suppose $X$ is a topological space, and $X_1, X_2$ are compact spaces. Let $X_1 \cup X_2$ be the union of $X_1$ and $X_2$, and let $X_1 \cap X_2$ be the intersection of $X_1$ and $X_2$. Prove that $X$ is metrizable.

\paragraph{Munkres.exercise.43.2} Suppose $X$ is a metric space, $Y$ is a complete space, and $A$ is a set of points in $X$. Let $f$ be a continuous function from $X$ into $Y$, and let $g$ be a continuous function from $Y$ into $Z$. Prove that $f$ is uniformly continuous if $g$ is uniformly continuous.

\paragraph{Pugh.exercise.2.26} Suppose $M$ is a topological space, and $U$ is a set of points in $M$. Prove that if $x ∈ U$ then $𝓟 Uᶜ$ is a closed set.

\paragraph{Pugh.exercise.2.32a} Prove that $A$ is a clopen set.

\paragraph{Pugh.exercise.2.46} Suppose $M$ is a metric space, and $A$ and $B$ are subsets of $M$. Let $A$ be compact and $B$ be non-empty. Prove that there exists a point $a \in A$ and a point $b \in B$ such that $dist a b = 0$.

\paragraph{Pugh.exercise.2.92} Prove that if $s$ is a nonempty subset of $\mathbb{N}$ then $s$ is a compact subset of $\mathbb{N}$.

\paragraph{Pugh.exercise.3.1} Let $f$ be a function from $\mathbb{R}$ to $\mathbb{R}$. Prove that there exists a constant $c$ such that $|f(x) - f(y)| ≤ |x - y| ^ 2$ for all $x, y \in \mathbb{R}$.

\paragraph{Pugh.exercise.3.63a} Suppose $p$ is a prime number, and $f$ is a function from $\mathbb{Z}$ to $\mathbb{R}$. Prove that there exists a sequence $l_n$ of natural numbers such that $f(n) = l_n$ for all $n$ and $f(n) \to 0$ as $n \to \infty$.

\paragraph{Pugh.exercise.4.15a} Suppose that $F$ is a set of continuous functions from $\mathbb{R}$ to $\mathbb{R}$. Then for every $x \in \mathbb{R}$, there exists a sequence $(s_n)_{n \in \mathbb{N}}$ in $F$ such that $s_n \to x$ and $|s_n - x| \to 0$. Prove that $F$ is uniformly continuous.

\paragraph{Putnam.exercise.1998.b6} Suppose that $a, b, c$ are real numbers, and $a, b, c$ are not all zero. Prove that there exists a positive integer $n$ such that $n^3 + a*n^2 + b*n + c$ is not a perfect square.

\paragraph{Putnam.exercise.1999.b4} Suppose $f$ is a continuous function from $\mathbb{R}$ to $\mathbb{R}$. Prove that $f$ is differentiable at $x$ if and only if $f$ is differentiable at $f(x)$.

\paragraph{Putnam.exercise.2001.a5} Prove that $a$ is a root of the polynomial $a^n - (a+1)^n$.

\paragraph{Putnam.exercise.2014.a5} Prove that if $P$ is a polynomial of degree $n$ with $n$ distinct roots, then $P$ is coprime to $X^n$ if and only if $P$ is coprime to $X^n$.

\paragraph{Putnam.exercise.2018.a5} Suppose $f$ is a continuous function from $\mathbb{R}$ to $\mathbb{R}$. Prove that $f$ is differentiable at $0$ if and only if $f$ is differentiable at $1$ and $f(0) = f(1)$. Prove that $f$ is differentiable at $0$ if and only if $f$ is differentiable at $1$ and $f(0) = f(1)$. Prove that $f$ is differentiable at $0$ if and only if $f$ is differentiable at $1$ and $f(0) = f(1)$. Prove that $f$ is differentiable at $0$ if and only if $f$ is differentiable at $1$ and $f(0) = f(1)$. Prove that $f$ is differentiable at $0$ if and only if $f$ is differentiable at $1$ and $f(0) = f(1)$. Prove that $f$ is differentiable at $0$ if and only if $f$ is differentiable at $1$ and $f(0) = f(1)$. Prove that $f$ is differentiable at $0$ if and only if $f$ is differentiable at $1$ and $f(0) = f(1)$. Prove that $f$ is differentiable at $0$ if and only if $f$ is differentiable at $1$ and $f(0) = f(1)$. Prove that $f$ is differentiable at $0$ if and only if $f$ is differentiable at $1$ and $f(0) = f(1)$. Prove that $f$ is differentiable at $0$ if and only if $f$ is differentiable at $1$ and $f(0) = f(1)$. Prove that $f$ is differentiable at $0$ if and only if $f$ is

\paragraph{Putnam.exercise.2018.b4} Let $a$ be a real number. Prove that there exists a function $x$ such that $x 0 = a$ and $x 1 = 0$ and $x (n+1) = 2*(x n)*(x (n-1)) - x (n-2)$ for all $n \geq 2$.

\paragraph{Rudin.exercise.1.1b} Prove that $x$ and $y$ are not proportional.

\paragraph{Rudin.exercise.1.4} Let $α$ be a partial order on a set $S$. Prove that if $x$ and $y$ are elements of $S$, then $x ≤ y$ if and only if $x$ is a lower bound of $y$ in $α$.

\paragraph{Rudin.exercise.1.8} Suppose $V$ is a vector space over $\mathbb{R}$ and $r$ is a linear order on $V$. Prove that $r$ is not linear order.

\paragraph{Rudin.exercise.1.12} Let $f$ be a function from $\mathbb{N}$ into $\mathbb{R}$. Prove that $\sum_{i=1}^n f(i) < \sum_{i=1}^n f(i)$ for all $n \in \mathbb{N}$.

\paragraph{Rudin.exercise.1.14} Prove that $z^2 + z$ is an integer.

\paragraph{Rudin.exercise.1.17} Prove that the norm of the vector $x$ is equal to the sum of the norms of the vectors $x_1, \dots, x_n$.

\paragraph{Rudin.exercise.1.18b} Suppose $x$ is a real number, and $y$ is a real number. Prove that there exists a real number $z$ such that $x * y = z$.

\paragraph{Rudin.exercise.2.19a} Suppose $X$ is a metric space, and $A, B$ are subsets of $X$. Then $A$ and $B$ are separated if and only if $A$ and $B$ are closed.

\paragraph{Rudin.exercise.2.25} Suppose $K$ is a compact metric space. Prove that there exists a set $B$ of cardinality $\aleph_1$ such that $K$ is a topological basis of $B$.

\paragraph{Rudin.exercise.2.27b} Prove that $E$ is a nonempty set.

\paragraph{Rudin.exercise.2.29} Let $U$ be a set of real numbers. Prove that there exists a function $f: ℕ → U$ such that $f$ is injective and $f$ is continuous.

\paragraph{Rudin.exercise.3.2a} Let $f$ be a function from $\mathbb{R}$ to $\mathbb{R}$ such that $f(x) = x^2$ for $x \in \mathbb{R}$. Prove that $f$ is tendsto at_top at_top.

\paragraph{Rudin.exercise.3.5} Suppose $a, b$ are real numbers, and $a$ is bounded above by $b$. Prove that $a + b$ is bounded above by $a + b + \epsilon$ for some $\epsilon > 0$.

\paragraph{Rudin.exercise.3.7} Suppose $a$ is a real number. Prove that there exists a sequence of real numbers $\{a_n\}_{n\in\mathbb{N}}$ such that $a_n$ tends to $a$ as $n$ tends to infinity.

\paragraph{Rudin.exercise.3.13} Suppose $f$ is a function from $\mathbb{N}$ into $\mathbb{R}$. Prove that there exists a sequence $\{a_n\}_{n\in\mathbb{N}}$ such that $a_n \in \mathbb{N}$ and $a_n \to 0$ as $n\to\infty$ and such that $f(n) = a_n$ for all $n\in\mathbb{N}$.

\paragraph{Rudin.exercise.3.21} Let $X$ be a complete metric space. Prove that there exists a sequence of points $a_n$ in $X$ such that $E_n = \{a_n\}$ for all $n$.

\paragraph{Rudin.exercise.4.1a} Suppose $f$ is a continuous function from $\mathbb{R}$ to $\mathbb{R}$. Prove that $f$ is uniformly continuous.

\paragraph{Rudin.exercise.4.3} Suppose $X$ is a metric space, and $f$ is a continuous function from $X$ into $\mathbb{R}$. Prove that $f$ is continuous if $h$ is continuous.

\paragraph{Rudin.exercise.4.4b} Suppose $X$ is a metric space, and $f$ is a continuous function from $X$ into $X$. Prove that $f$ is uniformly continuous.

\paragraph{Rudin.exercise.4.5b} Suppose $E$ is a set of real numbers, and $f$ is a continuous function from $E$ to $\mathbb{R}$. Prove that $f$ is uniformly continuous.

\paragraph{Rudin.exercise.4.8a} Let $E$ be a set of real numbers, and let $f$ be a function from $E$ to $\mathbb{R}$. Prove that $f$ is uniformly continuous if $h$ is uniformly continuous.

\paragraph{Rudin.exercise.4.11a} Suppose $X$ is a metric space, and $f$ is a uniformly continuous function from $X$ into $Y$. Prove that $f$ is Cauchy.

\paragraph{Rudin.exercise.4.15} Prove that $f$ is monotone.

\paragraph{Rudin.exercise.4.21a} Suppose $X$ is a metric space, and $F$ is a closed subset of $X$. Prove that there exists a positive real number $\delta$ such that for all $p, q \in X$, $p ∈ K$ and $q ∈ F$, we have $dist p q ≥ δ$.

\paragraph{Rudin.exercise.5.1} Let $f$ be a function from $\mathbb{R}$ to $\mathbb{R}$. Prove that there exists a constant $c$ such that $f(x) = c$ for all $x \in \mathbb{R}$.

\paragraph{Rudin.exercise.5.3} Let $g$ be a continuous function from $\mathbb{R}$ to $\mathbb{R}$. Prove that there exists a continuous function $h$ such that $h$ is uniformly continuous and $h(x) = g(x)$ for all $x \in \mathbb{R}$.

\paragraph{Rudin.exercise.5.5} Prove that $f$ is differentiable at $x$ if $f'(x) = 1$. Prove that $f$ is differentiable at $x$ if $f'(x) = 0$. Prove that $f$ is differentiable at $x$ if $f'(x) = -1$. Prove that $f$ is differentiable at $x$ if $f'(x) = -0$. Prove that $f$ is differentiable at $x$ if $f'(x) = 0$. Prove that $f$ is differentiable at $x$ if $f'(x) = 1$. Prove that $f$ is differentiable at $x$ if $f'(x) = -1$. Prove that $f$ is differentiable at $x$ if $f'(x) = -0$. Prove that $f$ is differentiable at $x$ if $f'(x) = 0$. Prove that $f$ is differentiable at $x$ if $f'(x) = 1$. Prove that $f$ is differentiable at $x$ if $f'(x) = -1$. Prove that $f$ is differentiable at $x$ if $f'(x) = -0$. Prove that $f$ is differentiable at $x$ if $f'(x) = 0$. Prove that $f$ is differentiable at $x$ if $f'(x) = 1$. Prove that $f$ is differentiable at $x$ if $f'(x) = -1$. Prove that $f$ is differentiable at $x$ if $f'(x) = -0$. Prove that $f$ is differentiable at $x$ if $f'(x) = 0$. Prove that $f$ is differentiable at $x$ if $f'(x) = 1$. Prove that $f$ is differentiable at $x$ if $

\paragraph{Rudin.exercise.5.7} Suppose $f$ and $g$ are differentiable functions on $\mathbb{R}$ and $f$ is continuous at $0$. Prove that $f$ is differentiable at $0$ if and only if $g$ is differentiable at $0$. Prove that $f$ is differentiable at $0$ if and only if $g$ is differentiable at $0$. Prove that $f$ is differentiable at $0$ if and only if $g$ is differentiable at $0$. Prove that $f$ is differentiable at $0$ if and only if $g$ is differentiable at $0$. Prove that $f$ is differentiable at $0$ if and only if $g$ is differentiable at $0$. Prove that $f$ is differentiable at $0$ if and only if $g$ is differentiable at $0$. Prove that $f$ is differentiable at $0$ if and only if $g$ is differentiable at $0$. Prove that $f$ is differentiable at $0$ if and only if $g$ is differentiable at $0$. Prove that $f$ is differentiable at $0$ if and only if $g$ is differentiable at $0$. Prove that $f$ is differentiable at $0$ if and only if $g$ is differentiable at $0$. Prove that $f$ is differentiable at $0$ if and only if $g$ is differentiable at $0$. Prove that $f$ is differentiable at $0$ if and only if $g$ is differentiable at $0$. Prove that $f$ is differentiable at $0$ if and only if $g$ is differentiable at $0$. Prove that $f$ is differentiable at $0$ if and only if $g$ is differentiable at $0$. Prove that $f$ is differentiable at $0$ if and only if $g$ is differentiable at $

\paragraph{Rudin.exercise.5.17} Suppose $f$ is differentiable on $\mathbb{R}$ and $f(-1)=0$. Prove that there exists $x \in \mathbb{R}$ such that $f(x)=0$ and $f'(x)=1$. Prove that $f$ is differentiable on $\mathbb{R}$ if and only if $f'(x)=1$ for some $x \in \mathbb{R}$. Prove that $f$ is differentiable on $\mathbb{R}$ if and only if $f'(x)=1$ for all $x \in \mathbb{R}$. Prove that $f$ is differentiable on $\mathbb{R}$ if and only if $f'(x)=1$ for all $x \in \mathbb{R}$. Prove that $f$ is differentiable on $\mathbb{R}$ if and only if $f'(x)=1$ for all $x \in \mathbb{R}$. Prove that $f$ is differentiable on $\mathbb{R}$ if and only if $f'(x)=1$ for all $x \in \mathbb{R}$. Prove that $f$ is differentiable on $\mathbb{R}$ if and only if $f'(x)=1$ for all $x \in \mathbb{R}$. Prove that $f$ is differentiable on $\mathbb{R}$ if and only if $f'(x)=1$ for all $x \in \mathbb{R}$. Prove that $f$ is differentiable on $\mathbb{R}$ if and only if $f'(x)=1$ for all $x \in \mathbb{R}$. Prove that $f$ is differentiable on $\mathbb{R}$ if and only if $f'(x)=1$ for all $x \in \mathbb{R}$. Prove that $f$ is differentiable on $\mathbb{R}$ if and only if $f'(x)=1$

\paragraph{Shakarchi.exercise.1.13b} Suppose $f$ is a differentiable function on $\mathbb{R}$ and $a, b$ are points in $\mathbb{R}$. Prove that $f$ is differentiable at $a$ if and only if $f$ is differentiable at $b$.

\paragraph{Shakarchi.exercise.1.19a} Suppose $z$ is a complex number, and $s$ is a real number. Prove that if $z$ is a root of $s$, then $z$ is a root of $s$.

\paragraph{Shakarchi.exercise.1.19c} Suppose $z$ is a complex number, and $s$ is a real number. Prove that there exists a sequence of complex numbers $\{z_n\}_{n=1}^\infty$ such that $z_n$ converges to $s$ and $z_n$ is a complex number for each $n$.

\paragraph{Shakarchi.exercise.2.2} Prove that the limit of the sequence of functions $x \mapsto \sin x / x$ is $0$.

\paragraph{Shakarchi.exercise.2.13} Let $f$ be a function from $\mathbb{R}$ to $\mathbb{R}$. Prove that $f$ is continuous if and only if $f$ is uniformly continuous.

\paragraph{Shakarchi.exercise.3.4} Prove that the function $tendsto$ is tendsto.

\paragraph{Shakarchi.exercise.3.14} Prove that $f$ is injective if $f$ is differentiable.

\paragraph{Shakarchi.exercise.5.1} Let $f$ be a differentiable function on the open unit ball of $\mathbb{R}^n$. Prove that $f$ is uniformly continuous if $f$ is uniformly continuous on the open unit ball of $\mathbb{R}^n$.
\end{document}