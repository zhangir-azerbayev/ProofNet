
\documentclass{article}

\title{\textbf{
Exercises from \\
\textit{Topology} \\
by James Munkres
}}

\date{}

\usepackage{amsmath}
\usepackage{amssymb}
\usepackage{amsthm}

\begin{document}
\maketitle


\paragraph{Exercise 13.1} Let $X$ be a topological space; let $A$ be a subset of $X$. Suppose that for each $x \in A$ there is an open set $U$ containing $x$ such that $U \subset A$. Show that $A$ is open in $X$.
\begin{proof}
    Since, from the given hypothesis given any $x \in A$ there exists an open set containing $x$ say, $U_x$ such that $U_x \subset A$. Thus, we claim that
$$
A=\bigcup_{x \in A} U_x
$$
Observe that if we prove the above claim, then $A$ will be open, being a union of arbitrary open sets. Since, for each $x \in A, U_x \subset A \Longrightarrow \cup U_x \subset A$. For the converse, observe that given any $x \in A, x \in U_x$ and hence in the union. Thus we proved our claim, and hence $A$ is an open set.
\end{proof}



\paragraph{Exercise 13.3b} Show that the collection $$\mathcal{T}_\infty = \{U | X - U \text{ is infinite or empty or all of X}\}$$ does not need to be a topology on the set $X$.
\begin{proof}
    Let $X=\mathbb{R}, U_1=(-\infty, 0)$ and $U_2=(0, \infty)$. Then $U_1$ and $U_2$ are in $\mathcal{T}_{\infty}$ but $U_1 \cup U_2=\mathbb{R} \backslash\{0\}$ is not.
\end{proof}



\paragraph{Exercise 13.4a1} If $\mathcal{T}_\alpha$ is a family of topologies on $X$, show that $\bigcap \mathcal{T}_\alpha$ is a topology on $X$.
\begin{proof}
    Since $\emptyset$ and $X$ belong to $\mathcal{T}_\alpha$ for each $\alpha$, they belong to $\bigcap_\alpha \mathcal{T}_\alpha$. Let $\left\{V_\beta\right\}_\beta$ be a collection of open sets in $\bigcap_\alpha \mathcal{T}_\alpha$. For any fixed $\alpha$ we have $\cup_\beta V_\beta \in \mathcal{T}_\alpha$ since $\mathcal{T}_\alpha$ is a topology on $X$, so $\bigcup_\beta V_\beta \in \bigcap_\alpha \mathcal{T}_\alpha$. Similarly, if $U_1, \ldots, U_n$ are elements of $\bigcap_\alpha \mathcal{T}_\alpha$, then for each $\alpha$ we have $\bigcup_{i=1}^n U_i \in \mathcal{T}_\alpha$ and therefore $\bigcup_{i=1}^n U_i \in \bigcap_\alpha \mathcal{T}_\alpha$. It follows that $\bigcap_\alpha \mathcal{T}_\alpha$ is a topology on $X$.
\end{proof}



\paragraph{Exercise 13.4a2} If $\mathcal{T}_\alpha$ is a family of topologies on $X$, show that $\bigcup \mathcal{T}_\alpha$ does not need to be a topology on $X$.
\begin{proof}
    On the other hand, the union $\bigcup_\alpha \mathcal{T}_\alpha$ is in general not a topology on $X$. For instance, let $X=\{a, b, c\}$. Then $\mathcal{T}_1=\{\emptyset, X,\{a\}\}$ and $\mathcal{T}_2=\{\emptyset, X,\{b\}\}$ are topologies on $X$ but $\mathcal{T}_1 \cup \mathcal{T}_2=$ $\{\emptyset, X,\{a\},\{b\}\}$ is not.
\end{proof}



\paragraph{Exercise 13.4b1} Let $\mathcal{T}_\alpha$ be a family of topologies on $X$. Show that there is a unique smallest topology on $X$ containing all the collections $\mathcal{T}_\alpha$.
\begin{proof}
    (b) First we prove that there is a unique smallest topology on $X$ containing all the collections $\mathcal{T}_\alpha$. Uniqueness of such topology is clear. For each $\alpha$ let $\mathcal{B}_\alpha$ be a basis for $\mathcal{T}_\alpha$. Let $\mathcal{T}$ be the topology generated by the subbasis $\mathcal{S}=\bigcup_\alpha \mathcal{B}_\alpha$. Then the collection $\mathcal{B}$ of all finite intersections of elements of $\mathcal{S}$ is a basis for $\mathcal{T}$. Clearly $\mathcal{T}_\alpha \subset \mathcal{T}$ for all $\alpha$. We now prove that if $\mathcal{O}$ is a topology on $X$ such that $\mathcal{T}_\alpha \subset \mathcal{O}$ for all $\alpha$, then $\mathcal{T} \subset \mathcal{O}$. Given such $\mathcal{O}$, we have $\mathcal{B}_\alpha \subset \mathcal{O}$ for all $\alpha$, so $\mathcal{S} \subset \mathcal{O}$. Since $\mathcal{O}$ is a topology, it must contain all finite intersections of elements of $\mathcal{S}$, so $\mathcal{B} \subset \mathcal{O}$ and hence $\mathcal{T} \subset \mathcal{O}$. We conclude that the topology $\mathcal{T}$ generated by the subbasis $\mathcal{S}=\cup_\alpha \mathcal{B}_\alpha$ is the unique smallest topology on $X$ containing all the collections $\mathcal{T}_\alpha$.
\end{proof}



\paragraph{Exercise 13.4b2} Let $\mathcal{T}_\alpha$ be a family of topologies on $X$. Show that there is a unique largest topology on $X$ contained in all the collections $\mathcal{T}_\alpha$.
\begin{proof}
    Now we prove that there exists a unique largest topology contained in all $\mathcal{T}_\alpha$. Uniqueness of such topology is clear. Consider $\mathcal{T}=\bigcap_\alpha \mathcal{T}_\alpha$. We already know that $\mathcal{T}$ is a topology by, and clearly $\mathcal{T} \subset \mathcal{T}_\alpha$ for all $\alpha$. If $\mathcal{O}$ is another topology contained in all $\mathcal{T}_\alpha$, it must be contained in their intersection, so $\mathcal{O} \subset \mathcal{T}$. I follows that $\mathcal{T}$ is the unique largest topology contained in all $\mathcal{T}_\alpha$.
\end{proof}



\paragraph{Exercise 13.5a} Show that if $\mathcal{A}$ is a basis for a topology on $X$, then the topology generated by $\mathcal{A}$ equals the intersection of all topologies on $X$ that contain $\mathcal{A}$.
\begin{proof}
    Let $\mathcal{T}$ be the topology generated by $\mathcal{A}$ and let $\mathcal{O}$ be the intersection of all topologies on $X$ that contains $\mathcal{A}$. Clearly $\mathcal{O} \subset \mathcal{T}$ since $\mathcal{T}$ is a topology on $X$ that contain $\mathcal{A}$. Conversely, let $U \in \mathcal{T}$, so that $U$ is a union of elements of $\mathcal{A}$. Since each of this elements is also an element of $\mathcal{O}$, their union $U$ belongs to $\mathcal{O}$. Thus $\mathcal{T} \subset \mathcal{O}$ and the equality holds.
\end{proof}



\paragraph{Exercise 13.5b} Show that if $\mathcal{A}$ is a subbasis for a topology on $X$, then the topology generated by $\mathcal{A}$ equals the intersection of all topologies on $X$ that contain $\mathcal{A}$.
\begin{proof}
    If we now considered $\mathcal{A}$ as a subbasis, then the elements of $\mathcal{T}$ are union of finite intersections of elements of $\mathcal{A}$. The inclusion $\mathcal{O} \subset \mathcal{T}$ is again clear and $\mathcal{T} \subset \mathcal{O}$ holds since every union of finite intersections of elements of $\mathcal{A}$ belongs to $\mathcal{O}$.
\end{proof}



\paragraph{Exercise 13.6} Show that the lower limit topology $\mathbb{R}_l$ and $K$-topology $\mathbb{R}_K$ are not comparable.
\begin{proof}
    Let $\mathcal{T}_{\ell}$ and $\mathcal{T}_K$ denote the topologies of $\mathbb{R}_{\ell}$ and $\mathbb{R}_K$ respectively. Given the basis element $[0,1)$ for $\mathcal{T}_{\ell}$, there is no basis element for $\mathcal{T}_K$ containing 0 and contained in $[0,1)$, so $\mathcal{T}_{\ell} \not \subset \mathcal{T}_K$. Similarly, given the basis element $(-1,1) \backslash K$ for $\mathcal{T}_K$, there is no basis element for $\mathcal{T}_{\ell}$ containing 0 contained in $(-1,1) \backslash K$, so $\mathcal{T}_K \not \subset \mathcal{T}_{\ell}$.
\end{proof}



\paragraph{Exercise 13.8a} Show that the collection $\{(a,b) \mid a < b, a \text{ and } b \text{ rational}\}$ is a basis that generates the standard topology on $\mathbb{R}$.
\begin{proof}
    Exercise 13.8. (a) First note that $\mathcal{B}$ is a basis for a topology on $\mathbb{R}$. This follows from the fact that the union of its elements is all of $\mathbb{R}$ and the intersection of two elements of $\mathcal{B}$ is either empty or another element of $\mathcal{B}$. Let $\mathcal{T}$ be the standard topology on $\mathbb{R}$. Clearly the topology generated by $\mathcal{B}$ is coarser than $\mathcal{T}$. Let $U \in \mathcal{T}$ and $x \in U$. Then $U$ contains an open interval with centre $x$. Since the rationals are dense in $\mathbb{R}$ with the standard topology, there exists $q \in \mathbb{Q}$ such that $x \in(x-q, x+q) \subset U$. This proves that $\mathcal{T}$ is coarser than the topology generated by $\mathcal{B}$. We conclude that $\mathcal{B}$ generates the standard topology on $\mathbb{R}$.
\end{proof}



\paragraph{Exercise 13.8b} Show that the collection $\{(a,b) \mid a < b, a \text{ and } b \text{ rational}\}$ is a basis that generates a topology different from the lower limit topology on $\mathbb{R}$.
\begin{proof}
    (b) $\mathcal{C}$ is a basis for a topology on $\mathbb{R}$ since the union of its elements is $\mathbb{R}$ and the intersection of two elements of $\mathcal{C}$ is either empty or another element of $\mathcal{C}$. Now consider $[r, s)$ where $r$ is any irrational number and $s$ is any real number greater than $r$. Then $[r, s)$ is a basis element for the topology of $\mathbb{R}_{\ell}$, but $[r, s)$ is not a union of elements of $\mathcal{C}$. Indeed, suppose that $[r, s)=\cup_\alpha\left[a_\alpha, b_\alpha\right)$ for rationals $a_\alpha, b_\alpha$. Then $r \in\left[a_\alpha, b_\alpha\right)$ for some $\alpha$. Since $r$ is irrational we must have $a_\alpha<r$, but then $a_\alpha \notin[r, s)$, a contradiction. It follows that the topology generated by $\mathcal{C}$ is strictly coarser than the lower limit topology on $\mathbb{R}$.
\end{proof}



\paragraph{Exercise 16.1} Show that if $Y$ is a subspace of $X$, and $A$ is a subset of $Y$, then the topology $A$ inherits as a subspace of $Y$ is the same as the topology it inherits as a subspace of $X$.
\begin{proof}
    Exercise 16.1. Let $\mathcal{T}$ be the topology $A$ inherits as a subspace of $Y$, and $\mathcal{O}$ be the topology it inherits as a subspace of $X$. A (standard) basis element for $\mathcal{T}$ has the form $U \cap A$ where $U$ is open in $Y$, so is of the form $(Y \cap V) \cap A=V \cap A$ where $V$ is open in $X$. Therefore every basis element for $\mathcal{T}$ is also a basis element for $\mathcal{O}$. Conversely, a (standard) basis element for $\mathcal{O}$ have the form $W \cap A=W \cap Y \cap A$ where $W$ is open in $X$. Since $W \cap Y$ is open in $Y$, this is a basis element for $\mathcal{T}$, so every basis element for $\mathcal{O}$ is a basis element for $\mathcal{T}$. It follows that $\mathcal{T}=\mathcal{O}$.
\end{proof}



\paragraph{Exercise 16.4} A map $f: X \rightarrow Y$ is said to be an open map if for every open set $U$ of $X$, the set $f(U)$ is open in $Y$. Show that $\pi_{1}: X \times Y \rightarrow X$ and $\pi_{2}: X \times Y \rightarrow Y$ are open maps.
\begin{proof}
    Exercise 16.4. Let $U \times V$ be a (standard) basis element for $X \times Y$, so that $U$ is open in $X$ and $V$ is open in $Y$. Then $\pi_1(U \times V)=U$ is open in $X$ and $\pi_2(U \times V)=V$ is open in $Y$. Since arbitrary maps and unions satisfy $f\left(\bigcup_\alpha W_\alpha\right)=\bigcup_\alpha f\left(W_\alpha\right)$, it follows that $\pi_1$ and $\pi_2$ are open maps.
\end{proof}



\paragraph{Exercise 16.6} Show that the countable collection \[\{(a, b) \times (c, d) \mid a < b \text{ and } c < d, \text{ and } a, b, c, d \text{ are rational}\}\] is a basis for $\mathbb{R}^2$.
\begin{proof}
    We know that $\mathcal{B}=\{(a,b)|a<b, a \text{ and } b \text rational\}$ is a basis for $\mathcal{R}$, therefore the set we are concerned with in the above question is a basis for $\mathcal{R}^2$. 
\end{proof}



\paragraph{Exercise 17.4} Show that if $U$ is open in $X$ and $A$ is closed in $X$, then $U-A$ is open in $X$, and $A-U$ is closed in $X$.
\begin{proof}
Since
$$
X \backslash(U \backslash A)=(X \backslash U) \cup A \text { and } \quad X \backslash(A \backslash U)=(X \backslash A) \cup U,
$$
it follows that $X \backslash(U \backslash A)$ is closed in $X$ and $X \backslash(A \backslash U)$ is open in $X$.
\end{proof}



\paragraph{Exercise 18.8a} Let $Y$ be an ordered set in the order topology. Let $f, g: X \rightarrow Y$ be continuous. Show that the set $\{x \mid f(x) \leq g(x)\}$ is closed in $X$.
\begin{proof}
    We prove that $U=\{x \mid g(x)<f(x)\}$ is open in $X$. Let $a \in U$, so that $g(a)<f(a)$. If there is an element $c$ between $g(a)$ and $f(a)$, then $a \in g^{-1}((-\infty, c)) \cap f^{-1}((c,+\infty))$. If there are no elements between $g(a)$ and $f(a)$, then $a \in g^{-1}\left((-\infty, f(a)) \cap f^{-1}((g(a),+\infty))\right.$. Note that all these preimages are open since $f$ and $g$ are continuous. Thus $U=V \cup W$, where
$$
V=\bigcup_{c \in X} g^{-1}((-\infty, c)) \cap f^{-1}((c,+\infty)) \quad \text { and } \quad W=\bigcup_{\substack{g(a)<f(a) \\(g(a), f(a))=\emptyset}} g^{-1}\left((-\infty, f(a)) \cap f^{-1}((g(a),+\infty))\right.
$$
are open in $X$. So $U$ is open in $X$ and therefore $X \backslash U=\{x \mid f(x) \leq g(x)\}$ is closed in $X$.
\end{proof}



\paragraph{Exercise 18.8b} Let $Y$ be an ordered set in the order topology. Let $f, g: X \rightarrow Y$ be continuous. Let $h: X \rightarrow Y$ be the function $h(x)=\min \{f(x), g(x)\}.$ Show that $h$ is continuous.
\begin{proof}
    Let $A=\{x \mid f(x) \leq g(x)\}$ and $B=\{x \mid g(x) \leq f(x)\}$. Then $A$ and $B$ are closed in $X$ by (a), $A \cap B=\{x \mid f(x)=g(x)\}$, and $X=A \cup B$. Since $f$ and $g$ are continuous, their restrictions $f^{\prime}: A \rightarrow Y$ and $g^{\prime}: B \rightarrow Y$ are continuous. It follows from the pasting lemma that
$$
h: X \rightarrow Y, \quad h(x)=\min \{f(x), g(x)\}= \begin{cases}f^{\prime}(x) & \text { if } x \in A \\ g^{\prime}(x) & \text { if } x \in B\end{cases}
$$
is continuous
\end{proof}



\paragraph{Exercise 18.13} Let $A \subset X$; let $f: A \rightarrow Y$ be continuous; let $Y$ be Hausdorff. Show that if $f$ may be extended to a continuous function $g: \bar{A} \rightarrow Y$, then $g$ is uniquely determined by $f$.
\begin{proof}
    Let $h, g: \bar{A} \rightarrow Y$ be continuous extensions of $f$. Suppose that there is a point $x \in \bar{A}$ such that $h(x) \neq g(x)$. Since $h=g$ on $A$, we must have $x \in A^{\prime}$. Since $Y$ is Hausdorff, there is a neighbourhood $U$ of $h(x)$ and a neighbourhood $V$ of $g(x)$ such that $U \cap V=\emptyset$. Since $h$ and $g$ are continuous, $h^{-1}(U) \cap g^{-1}(V)$ is a neighbourhood of $x$. Since $x \in A^{\prime}$, there is a point $y \in h^{-1}(U) \cap g^{-1}(V) \cap A$ different from $x$. But $h=g$ on $A$, so $g^{-1}(V) \cap A=h^{-1}(V) \cap A$ and hence $y \in h^{-1}(U) \cap h^{-1}(V)=h^{-1}(U \cap V)=\emptyset$, a contradiction. It follows that $h=g$ on $\bar{A}$.
\end{proof}



\paragraph{Exercise 19.6a} Let $\mathbf{x}_1, \mathbf{x}_2, \ldots$ be a sequence of the points of the product space $\prod X_\alpha$.  Show that this sequence converges to the point $\mathbf{x}$ if and only if the sequence $\pi_\alpha(\mathbf{x}_i)$ converges to $\pi_\alpha(\mathbf{x})$ for each $\alpha$.
\begin{proof}
    For each $n \in \mathbb{Z}_{+}$, we write $\mathbf{x}_n=\left(x_n^\alpha\right)_\alpha$, so that $\pi_\alpha\left(\mathbf{x}_n\right)=x_n^\alpha$ for each $\alpha$.
First assume that the sequence $\mathbf{x}_1, \mathbf{x}_2, \ldots$ converges to $\mathbf{x}=\left(x_\alpha\right)_\alpha$ in the product space $\prod_\alpha X_\alpha$. Fix an index $\beta$ and let $U$ be a neighbourhood of $\pi_\beta(\mathbf{x})=x_\beta$. Let $V=\prod_\alpha U_\alpha$, where $U_\alpha=X_\alpha$ for each $\alpha \neq \beta$ and $U_\beta=U$. Then $V$ is a neighbourhood of $\mathbf{x}$, so there exists $N \in \mathbb{Z}_{+}$such that $\mathbf{x}_n \in V$ for all $n \geq N$. Therefore $\pi_\beta\left(\mathbf{x}_n\right)=x_n^\beta \in U$ for all $n \geq N$. Since $U$ was arbitrary, it follows that $\pi_\beta\left(\mathbf{x}_1\right), \pi_\beta\left(\mathbf{x}_2\right), \ldots$ converges to $\pi_\beta(\mathbf{x})$. Since $\beta$ was arbitrary, this holds for all indices $\alpha$.
\end{proof}



\paragraph{Exercise 20.2} Show that $\mathbb{R} \times \mathbb{R}$ in the dictionary order topology is metrizable.
\begin{proof}
    The dictionary order topology on $\mathbb{R} \times \mathbb{R}$ is the same as the product topology $\mathbb{R}_d \times \mathbb{R}$, where $\mathbb{R}_d$ denotes $\mathbb{R}$ with the discrete topology. We know that $\mathbb{R}_d$ and $\mathbb{R}$ are metrisable. Thus, it suffices to show that the product of two metrisable spaces is metrisable.
So let $X$ and $Y$ be metrisable spaces, with metrics $d$ and $d^{\prime}$ respectively. On $X \times Y$, define
$$
\rho(x \times y, w \times z)=\max \left\{d(x, w), d^{\prime}(y, z)\right\} .
$$
Then $\rho$ is a metric on $X \times Y$; it remains to prove that it induces the product topology on $X \times Y$. If $B_d\left(x, r_1\right) \times B_d\left(y, r_2\right)$ is a basis element for the product space $X \times Y$, and $r=\min \left\{r_1, r_2\right\}$, then $x \times y \in B_\rho(x \times y, r) \subset B_d\left(x, r_1\right) \times B_d\left(y, r_2\right)$, so the product topology is coarser than the $\rho$-topology. Conversely, if $B_\rho(x \times y, \delta)$ is a basis element for the $\rho$-topology, then $x \times y \in B_d(x, \delta) \times B_{d^{\prime}}(y, \delta) \subset$ $B_\rho(x \times y, \delta)$, so the product topology is finer than the $\rho$-topology. It follows that both topologies are equal, so the product space $X \times Y$ is metrisable.
\end{proof}



\paragraph{Exercise 21.6a} Define $f_{n}:[0,1] \rightarrow \mathbb{R}$ by the equation $f_{n}(x)=x^{n}$. Show that the sequence $\left(f_{n}(x)\right)$ converges for each $x \in[0,1]$.
\begin{proof}
If $0 \leq x<1$ is fixed, then $f_n(x) \rightarrow 0$ as $n \rightarrow \infty$. As $f_n(1)=1$ for all $n, f_n(1) \rightarrow 1$. Thus $\left(f_n\right)_n$ converges to $f:[0,1] \rightarrow \mathbb{R}$ given by $f(x)=0$ if $x=0$ and $f(1)=1$. The sequence
\end{proof}

\paragraph{Exercise 21.6b} Define $f_{n}:[0,1] \rightarrow \mathbb{R}$ by the equation $f_{n}(x)=x^{n}$. Show that the sequence $\left(f_{n}\right)$ does not converge uniformly.
\begin{proof}
    The sequence $\left(f_n\right)_n$ does not converge uniformly, since given $0<\varepsilon<1$ and $N \in \mathbb{Z}_{+}$, for $x=\varepsilon^{1 / N}$ we have $d\left(f_N(x), f(x)\right)=\varepsilon$. We can also apply Theorem 21.6: the convergence is not uniform since $f$ is not continuous.
\end{proof}



\paragraph{Exercise 21.8} Let $X$ be a topological space and let $Y$ be a metric space. Let $f_{n}: X \rightarrow Y$ be a sequence of continuous functions. Let $x_{n}$ be a sequence of points of $X$ converging to $x$. Show that if the sequence $\left(f_{n}\right)$ converges uniformly to $f$, then $\left(f_{n}\left(x_{n}\right)\right)$ converges to $f(x)$.
\begin{proof}
    Let $d$ be the metric on $Y$. Let $V$ be a neighbourhood of $f(x)$, and let $\varepsilon>0$ be such that $f(x) \in B_d(f(x), \varepsilon) \subset V$. Since $\left(f_n\right)_n$ converges uniformly to $f$, there exists $N_1 \in \mathbb{Z}_{+}$such that $d\left(f_n(x), f(x)\right)<\varepsilon / 2$ for all $x \in X$ and all $n \geq N_1$, so that $d\left(f_n\left(x_n\right), f\left(x_n\right)\right)<\varepsilon / 2$ for all $n \geq N_1$. Moreover, $f$ is continuous, so there exists $N_2 \in \mathbb{Z}_{+}$such that $d\left(f\left(x_n\right), f(x)\right)<\varepsilon / 2$ for all $n \geq N_2$. Thus, if $N>\max \left\{N_1, N_2\right\}$, then
$$
d\left(f_n\left(x_n\right), f(x)\right) \leq d\left(f_n\left(x_n\right), f\left(x_n\right)\right)+d\left(f\left(x_n\right), f(x)\right)<\frac{\varepsilon}{2}+\frac{\varepsilon}{2}=\varepsilon
$$
for all $n \geq N$, so $f_n\left(x_n\right) \in V$ for all $n \geq N$. It follows that $\left(f_n\left(x_n\right)\right)_n$ converges to $f(x)$.
\end{proof}



\paragraph{Exercise 22.2a} Let $p: X \rightarrow Y$ be a continuous map. Show that if there is a continuous map $f: Y \rightarrow X$ such that $p \circ f$ equals the identity map of $Y$, then $p$ is a quotient map.
\begin{proof}
Let $1_Y: Y \rightarrow Y$ be the identity map in $Y$. If $U$ is a subset of $Y$ and $p^{-1}(U)$ is open in $X$, then $f^{-1}\left(p^{-1}(U)\right)=1_Y^{-1}(U)=U$ is open in $Y$ by continuity of $f$. Thus $p$ is a quotient map.
\end{proof}



\paragraph{Exercise 22.2b} If $A \subset X$, a retraction of $X$ onto $A$ is a continuous map $r: X \rightarrow A$ such that $r(a)=a$ for each $a \in A$. Show that a retraction is a quotient map.
\begin{proof}
The inclusion map $i: A \rightarrow X$ is continuous and $r \circ i=1_A$ is the identity. Thus $r$ is a quotient map by (a).
\end{proof}



\paragraph{Exercise 22.5} Let $p \colon X \rightarrow Y$ be an open map. Show that if $A$ is open in $X$, then the map $q \colon A \rightarrow p(A)$ obtained by restricting $p$ is an open map.
\begin{proof}
Let $U$ be open in $A$. Since $A$ is open in $X, U$ is open in $X$ as well, so $p(U)$ is open in $Y$. Since $q(U)=p(U)=p(U) \cap p(A)$, the set $q(U)$ is open in $p(A)$. Thus $q$ is an open map.
\end{proof}



\paragraph{Exercise 23.2} Let $\left\{A_{n}\right\}$ be a sequence of connected subspaces of $X$, such that $A_{n} \cap A_{n+1} \neq \varnothing$ for all $n$. Show that $\bigcup A_{n}$ is connected.
\begin{proof}
    Suppose that $\bigcup_n A_n=B \cup C$, where $B$ and $C$ are disjoint open subsets of $\bigcup_n A_n$. Since $A_1$ is connected and a subset of $B \cup C$, by Lemma $23.2$ it lies entirely within either $B$ or $C$. Without any loss of generality, we may assume $A_1 \subset B$. Note that given $n$, if $A_n \subset B$ then $A_{n+1} \subset B$, for if $A_{n+1} \subset C$ then $A_n \cap A_{n+1} \subset B \cap C=\emptyset$, in contradiction with the assumption. By induction, $A_n \subset B$ for all $n \in \mathbb{Z}_{+}$, so that $\bigcup_n A_n \subset B$. It follows that $\bigcup_n A_n$ is connected.
\end{proof}



\paragraph{Exercise 23.3} Let $\left\{A_{\alpha}\right\}$ be a collection of connected subspaces of $X$; let $A$ be a connected subset of $X$. Show that if $A \cap A_{\alpha} \neq \varnothing$ for all $\alpha$, then $A \cup\left(\bigcup A_{\alpha}\right)$ is connected.
\begin{proof}
    For each $\alpha$ we have $A \cap A_\alpha \neq \emptyset$, so each $A \cup A_\alpha$ is connected by Theorem 23.3. In turn $\left\{A \cup A_\alpha\right\}_\alpha$ is a collection of connected spaces that have a point in common (namely any point in $A)$, so $\bigcup_\alpha\left(A \cup A_\alpha\right)=A \cup\left(\bigcup_\alpha A_\alpha\right)$ is connected. 
\end{proof}



\paragraph{Exercise 23.4} Show that if $X$ is an infinite set, it is connected in the finite complement topology.
\begin{proof}
    Suppose that $A$ is a non-empty subset of $X$ that is both open and closed, i.e., $A$ and $X \backslash A$ are finite or all of $X$. Since $A$ is non-empty, $X \backslash A$ is finite. Thus $A$ cannot be finite as $X \backslash A$ is infinite, so $A$ is all of $X$. Therefore $X$ is connected.
\end{proof}



\paragraph{Exercise 23.6} Let $A \subset X$. Show that if $C$ is a connected subspace of $X$ that intersects both $A$ and $X-A$, then $C$ intersects $\operatorname{Bd} A$.
\begin{proof}
    Suppose that $C \cap B d A=C \cap \bar{A} \cap \overline{X-A}=\emptyset$. Then $C \cap A$ and $C \cap(X \backslash A)$ are a pair of disjoint non-empty sets whose union is all of $C$, neither of which contains a limit point of the other. Indeed, if $C \cap(X-A)$ contains a limit point $x$ of $C \cap A$, then $x \in C \cap(X-A) \cap A^{\prime} \subset C \cap \bar{A} \cap \overline{X-A}=\emptyset$, a contradiction, and similarly $C \cap A$ does not contain a limit point of $C \cap(X-A)$. Then $C \cap A$ and $C \cap(X-A)$ constitute a separation of $C$, contradicting the fact that $C$ is connected (Lemma 23.1).
\end{proof}



\paragraph{Exercise 23.9} Let $A$ be a proper subset of $X$, and let $B$ be a proper subset of $Y$. If $X$ and $Y$ are connected, show that $(X \times Y)-(A \times B)$ is connected.
\begin{proof}
This is similar to the proof of Theorem 23.6. Take $c \times d \in(X \backslash A) \times(Y \backslash B)$. For each $x \in X \backslash A$, the set
$$
U_x=(X \times\{d\}) \cup(\{x\} \times Y)
$$
is connected since $X \times\{d\}$ and $\{x\} \times Y$ are connected and have the common point $x \times d$. Then $U=\bigcup_{x \in X \backslash A} U_x$ is connected because it is the union of the connected spaces $U_x$ which have the point $c \times d$ in common. Similarly, for each $y \in Y \backslash B$ the set
$$
V_y=(X \times\{y\}) \cup(\{c\} \times Y)
$$
is connected, so $V=\bigcup_{y \in Y \backslash B} V_y$ is connected. Thus $(X \times Y) \backslash(A \times B)=U \cup V$ is connected since $c \times d$ is a common point of $U$ and $V$.
\end{proof}



\paragraph{Exercise 23.11} Let $p: X \rightarrow Y$ be a quotient map. Show that if each set $p^{-1}(\{y\})$ is connected, and if $Y$ is connected, then $X$ is connected.
\begin{proof}
    Suppose that $U$ and $V$ constitute a separation of $X$. If $y \in p(U)$, then $y=p(x)$ for some $x \in U$, so that $x \in p^{-1}(\{y\})$. Since $p^{-1}(\{y\})$ is connected and $x \in U \cap p^{-1}(\{y\})$, we have $p^{-1}(\{y\}) \subset U$. Thus $p^{-1}(\{y\}) \subset U$ for all $y \in p(U)$, so that $p^{-1}(p(U)) \subset U$. The inclusion $U \subset p^{-1}(p(U))$ if true for any subset and function, so we have the equality $U=p^{-1}(p(U))$ and therefore $U$ is saturated. Similarly, $V$ is saturated. Since $p$ is a quotient map, $p(U)$ and $p(V)$ are disjoint non-empty open sets in $Y$. But $p(U) \cup p(V)=Y$ as $p$ is surjective, so $p(U)$ and $p(V)$ constitute a separation of $Y$, contradicting the fact that $Y$ is connected. We conclude that $X$ is connected.
\end{proof}



\paragraph{Exercise 24.2} Let $f: S^{1} \rightarrow \mathbb{R}$ be a continuous map. Show there exists a point $x$ of $S^{1}$ such that $f(x)=f(-x)$.
\begin{proof}
    Let $f: S^1 \rightarrow \mathbb{R}$ be continuous. Let $x \in S^1$. If $f(x)=f(-x)$ we are done, so assume $f(x) \neq f(-x)$. Define $g: S^1 \rightarrow \mathbb{R}$ by setting $g(x)=f(x)-f(-x)$. Then $g$ is continuous. Suppose $f(x)>f(-x)$, so that $g(x)>0$. Then $-x \in S^1$ and $g(-x)<0$. By the intermediate value theorem, since $S^1$ is connected and $g(-x)<0<g(x)$, there exists $y \in S^1$ such that $g(y)=0$. i.e, $f(y)=f(-y)$. Similarly, if $f(x)<f(-x)$, then $g(x)<0<g(-x)$ and again the intermediate value theorem gives the result.
\end{proof}



\paragraph{Exercise 24.3a} Let $f \colon X \rightarrow X$ be continuous. Show that if $X = [0, 1]$, there is a point $x$ such that $f(x) = x$. (The point $x$ is called a fixed point of $f$.)
\begin{proof}
    If $f(0)=0$ or $f(1)=1$ we are done, so suppose $f(0)>0$ and $f(1)<1$. Let $g:[0,1] \rightarrow[0,1]$ be given by $g(x)=f(x)-x$. Then $g$ is continuous, $g(0)>0$ and $g(1)<0$. Since $[0,1]$ is connected and $g(1)<0<g(0)$, by the intermediate value theorem there exists $x \in(0,1)$ such that $g(x)=0$, that is, such that $f(x)=x$.
\end{proof}



\paragraph{Exercise 25.4} Let $X$ be locally path connected. Show that every connected open set in $X$ is path connected.
\begin{proof}
    Let $U$ be a open connected set in $X$. By Theorem 25.4, each path component of $U$ is open in $X$, hence open in $U$. Thus, each path component in $U$ is both open and closed in $U$, so must be empty or all of $U$. It follows that $U$ is path-connected.
\end{proof}



\paragraph{Exercise 25.9} Let $G$ be a topological group; let $C$ be the component of $G$ containing the identity element $e$. Show that $C$ is a normal subgroup of $G$.
\begin{proof}
    Given $x \in G$, the maps $y \mapsto x y$ and $y \mapsto y x$ are homeomorphisms of $G$ onto itself. Since $C$ is a component, $x C$ and $C x$ are both components that contain $x$, so they are equal. Hence $x C=C x$ for all $x \in G$, so $C$ is a normal subgroup of $G$.
\end{proof}



\paragraph{Exercise 26.11} Let $X$ be a compact Hausdorff space. Let $\mathcal{A}$ be a collection of closed connected subsets of $X$ that is simply ordered by proper inclusion. Then $Y=\bigcap_{A \in \mathcal{A}} A$ is connected.
\begin{proof}
 Since each $A \in \mathcal{A}$ is closed, $Y$ is closed. Suppose that $C$ and $D$ form a separation of $Y$. Then $C$ and $D$ are closed in $Y$, hence closed in $X$. Since $X$ is compact, $C$ and $D$ are compact by Theorem 26.2. Since $X$ is Hausdorff, by Exercise 26.5, there exist $U$ and $V$ open in $X$ and disjoint containing $C$ and $D$, respectively. We show that
$$
\bigcap_{A \in \mathcal{A}}(A \backslash(U \cup V))
$$
is not empty. Let $\left\{A_1, \ldots, A_n\right\}$ be a finite subcollection of elements of $\mathcal{A}$. We may assume that $A_i \subsetneq A_{i+1}$ for all $i=1, \ldots, n-1$. Then
$$
\bigcap_{i=1}^n\left(A_i \backslash(U \cup V)\right)=A_1 \backslash(U \cup V) \text {. }
$$
Suppose that $A_1 \backslash(U \cup V)=\emptyset$. Then $A_1 \subset U \cup V$. Since $A_1$ is connected and $U \cap V=\emptyset, A_1$ lies within either $U$ or $V$, say $A_1 \subset U$. Then $Y \subset A_1 \subset U$, so that $C=Y \cap C \subset Y \cap V=\emptyset$, contradicting the fact that $C$ and $D$ form a separation of $Y$. Hence, $\bigcap_{i=1}^n\left(A_i \backslash(U \cup V)\right)$ is non-empty. Therefore, the collection $\{A \backslash(U \cup V) \mid A \in \mathcal{A}\}$ has the finite intersection property, so
$$
\bigcap_{A \in \mathcal{A}}(A \backslash(U \cup V))=\left(\bigcap_{A \in \mathcal{A}} A\right) \backslash(U \cup V)=Y \backslash(U \cup V)
$$
is non-empty. So there exists $y \in Y$ such that $y \notin U \cup V \supset C \cup D$, contradicting the fact that $C$ and $D$ form a separation of $Y$. We conclude that there is no such separation, so that $Y$ is connected.
\end{proof}



\paragraph{Exercise 26.12} Let $p: X \rightarrow Y$ be a closed continuous surjective map such that $p^{-1}(\{y\})$ is compact, for each $y \in Y$. (Such a map is called a perfect map.) Show that if $Y$ is compact, then $X$ is compact.
\begin{proof}
    We first show that if $U$ is an open set containing $p^{-1}(\{y\})$, then there is a neighbourhood $W$ of $y$ such that $p^{-1}(W)$ is contained in $U$. Since $X-U$ is closed in $X$, $p(X-U)$ is closed in $Y$ and does not contain $y$, so $W=Y \backslash p(X \backslash U)$ is a neighbourhood of $y$. Moreover, since $X \backslash U \subset p^{-1}(p(X \backslash U)$ ) (by elementary set theory), we have
$$
p^{-1}(W)=p^{-1}(Y \backslash p(X \backslash U))=p^{-1}(Y) \backslash p^{-1}(p(X \backslash U)) \subset X \backslash(X \backslash U)=U .
$$
Now let $\mathcal{A}$ be an open covering of $X$. For each $y \in Y$, let $\mathcal{A}_y$ be a subcollection of $\mathcal{A}$ such that
$$
p^{-1}(\{y\}) \subset \bigcup_{A \in \mathcal{A}_y} A .
$$
Since $p^{-1}(\{y\})$ is compact, there exists a finite subcollection of $\mathcal{A}_y$ that also covers $p^{-1}(\{y\})$, say $\left\{A_y^1, \ldots, A_y^{n_y}\right\}$. Thus $\bigcup_{i=1}^{n_y} A_y^i$ is open and contains $p^{-1}(\{y\})$, so there exists a neighbourhood $W_y$ of $y$ such that $p^{-1}\left(W_y\right)$ is contained in $\bigcup_{i=1}^{n_y} A_y^i$. Then $\left\{W_y\right\}_{y \in Y}$ is an open covering of $Y$, so there exist $y_1, \ldots, y_k \in Y$ such that $\left\{W_{y_j}\right\}_{j=1}^k$ also covers $Y$. Then
$$
X=p^{-1}(Y) \subset p^{-1}\left(\bigcup_{j=1}^k W_{y_j}\right)=\bigcup_{j=1}^k p^{-1}\left(W_{y_j}\right) \subset \bigcup_{j=1}^k\left(\bigcup_{i=1}^{n_{y_j}} A_{y_j}^i\right)
$$
so
$$
\left\{A_{y_j}^i\right\}_{\substack{j=1, \ldots, k . \\ i=1, \ldots, n_{y_j}}}
$$
is a finite subcollection of $\mathcal{A}$ that also covers $X$. Therefore, $X$ is compact.
\end{proof}



\paragraph{Exercise 27.4} Show that a connected metric space having more than one point is uncountable.
\begin{proof}
    The distance function $d: X \times X \rightarrow \mathbb{R}$ is continuous by Exercise 20.3(a), so given $x \in X$, the function $d_x: X \rightarrow \mathbb{R}$ given by $d_x(y)=d(x, y)$ is continuous by Exercise 19.11. Since $X$ is connected, the image $d_x(X)$ is a connected subspace of $\mathbb{R}$, and contains 0 since $d_x(x)=0$. Thus, if $y \in X$ and $y \neq x$, then $d_x(X)$ contains the set $[0, \delta]$, where $\delta=d_x(y)>0$. Therefore $X$ must be uncountable.
\end{proof}



\paragraph{Exercise 28.4} A space $X$ is said to be countably compact if every countable open covering of $X$ contains a finite subcollection that covers $X$. Show that for a $T_1$ space $X$, countable compactness is equivalent to limit point compactness.
\begin{proof}
    First let $X$ be a countable compact space. Note that if $Y$ is a closed subset of $X$, then $Y$ is countable compact as well, for if $\left\{U_n\right\}_{n \in \mathbb{Z}_{+}}$is a countable open covering of $Y$, then $\left\{U_n\right\}_{n \in \mathbb{Z}_{+}} \cup(X \backslash Y)$ is a countable open covering of $X$; there is a finite subcovering of $X$, hence a finite subcovering of $Y$. Now let $A$ be an infinite subset. We show that $A$ has a limit point. Let $B$ be a countable infinite subset of $A$. Suppose that $B$ has no limit point, so that $B$ is closed in $X$. Then $B$ is countable compact. Since $B$ has no limit point, for each $b \in B$ there is a neighbourhood $U_b$ of $b$ that intersects $B$ in the point $b$ alone. Then $\left\{U_b\right\}_{b \in B}$ is an open covering of $B$ with no finite subcovering, contradicting the fact that $B$ is countable compact. Hence $B$ has a limit point, so that $A$ has a limit point as well. Since $A$ was arbitrary, we deduce that $X$ is limit point compact. (Note that the $T_1$ property is not necessary in this direction.)

Now assume that $X$ is a limit point compact $T_1$ space. We show that $X$ is countable compact. Suppose, on the contrary, that $\left\{U_n\right\}_{n \in \mathbb{Z}_{+}}$is a countable open covering of $X$ with no finite subcovering. For each $n$, take a point $x_n$ in $X$ not in $U_1 \cup \cdots \cup U_n$. By assumption, the infinite set $A=\left\{x_n \mid n \in \mathbb{Z}_{+}\right\}$has a limit point $y \in X$. Since $\left\{U_n\right\}_{n \in \mathbb{Z}_{+}}$covers $X$, there exists $N \in \mathbb{Z}_{+}$such that $y \in U_1 \cup \cdots \cup U_N$. Now $X$ is $T_1$, so for each $i=1, \ldots, N$ there exists a neighbourhood $V_i$ of $y$ that does not contain $x_i$. Then
$$
V=\left(V_1 \cap \cdots \cap V_N\right) \cap\left(U_1 \cup \cdots \cup U_N\right)
$$
is a neighbourhood of $y$ that does not contain any of the points $x_i$, contradicting the fact that $y$ is a limit point of $A$. It follows that every countable open covering of $X$ must have a finite subcovering, so $X$ is countable compact.
\end{proof}



\paragraph{Exercise 28.5} Show that X is countably compact if and only if every nested sequence $C_1 \supset C_2 \supset \cdots$ of closed nonempty sets of X has a nonempty intersection.
\begin{proof}
We could imitate the proof of Theorem 26.9, but we prove directly each direction. First let $X$ be countable compact and let $C_1 \supset C_2 \supset \cdots$ be a nested sequence of closed nonempty sets of $X$. For each $n \in \mathbb{Z}_{+}, U_n=X \backslash C_n$ is open in $X$. Then $\left\{U_n\right\}_{n \in \mathbb{Z}_{+}}$is a countable collection of open sets with no finite subcollection covering $X$, for if $U_{i_1} \cup \cdots \cup U_{1_n}$ covers $X$, then $C_{i_1} \cap \cdots \cap C_{i_n}$ is empty, contrary to the assumption. Hence $\left\{U_n\right\}_{n \in \mathbb{Z}_{+}}$does not cover $X$, so there exist $x \in X \backslash \bigcup_{n \in \mathbb{Z}_{+}} U_n=\bigcap_{n \in Z_{+}}\left(X \backslash U_n\right)=\bigcap_{n \in Z_{+}} C_n$.

Conversely, assume that every nested sequence $C_1 \supset C_2 \supset \cdots$ of closed non-empty sets of $X$ has a non-empty intersection and let $\left\{U_n\right\}_{n \in \mathbb{Z}_{+}}$be a countable open covering of $X$. For each $n$, let $V_n=U_1 \cup \cdots \cup U_n$ and $C_n=X \backslash V_n$. Suppose that no finite subcollection of $\left\{U_n\right\}_{n \in \mathbb{Z}_{+}}$covers $X$. Then each $C_n$ is non-empty, so $C_1 \supset C_2 \supset \cdots$ is a nested sequence of non-empty closed sets and $\bigcap_{n \in \mathbb{Z}_{+}} C_n$ is non-empty by assumption. Then there exists $x \in \bigcap_{n \in \mathbb{Z}_{+}} C_n$, so that $x \notin V_n$ for all $n$, contradicting the fact that $\left\{U_n\right\}_{n \in \mathbb{Z}_{+}}$covers $X$. It follows that there exists $N \in \mathbb{Z}_{+}$such that $C_N=\emptyset$, so that $X=V_N$ and hence some finite subcollection of $\left\{U_n\right\}_{n \in \mathbb{Z}_{+}}$covers $X$. We deduce that $X$ is countable compact.
\end{proof}



\paragraph{Exercise 28.6} Let $(X, d)$ be a metric space. If $f: X \rightarrow X$ satisfies the condition $d(f(x), f(y))=d(x, y)$ for all $x, y \in X$, then $f$ is called an isometry of $X$. Show that if $f$ is an isometry and $X$ is compact, then $f$ is bijective and hence a homeomorphism.
\begin{proof}
    Note that $f$ is an imbedding. It remains to prove that $f$ is surjective. Suppose it is not, and let $a \in f(X)$. Since $X$ is compact, $f(X)$ is compact and hence closed (every metric space is Hausdorff). Thus, there exists $\varepsilon>0$ such that the $\varepsilon^{-}$ neighbourhood of $a$ is contained in $X \backslash f(X)$. Set $x_1=a$, and inductively $x_{n+1}=f\left(x_n\right)$ for $n \in \mathbb{Z}_{+}$. We show that $d\left(x_n, x_m\right) \geq \varepsilon$ for $n \neq m$. Indeed, we may assume $n<m$. If $n \geq 1$, then $d\left(x_n, x_m\right)=$ $d\left(f^{-1}\left(x_n\right), f^{-1}\left(x_m\right)\right)=d\left(x_{n-1}, x_{m-1}\right)$. By induction it follows that $d\left(x_n, x_m\right)=d\left(x_{n-i}, x_{m-i}\right)$ for all $i \geq 1$, and hence $d\left(x_n, x_m\right)=d\left(a, x_{m-n}\right)=d\left(a, f\left(x_{m-n-1}\right)\right)$. Since $f\left(x_{m-n-1}\right) \in f(X)$ and $B(a, \varepsilon) \cap f(X)=\emptyset$, we have $d\left(x_n, x_m\right) \geq \varepsilon$, as claimed. Thus $\left\{x_n\right\}_{n \in \mathbb{Z}_{+}}$is a sequence with no convergent subsequence, so $X$ is not sequentially compact. This contradicts the fact that $X$ is compact. Therefore $f$ is surjective and hence a homeomorphism.
\end{proof}



\paragraph{Exercise 29.1} Show that the rationals $\mathbb{Q}$ are not locally compact.
\begin{proof}
    First, we prove that each set $\mathbb{Q} \cap[a, b]$, where $a, b$ are irrational numbers, is not compact. Indeed, since $\mathbb{Q} \cap[a, b]$ is countable, we can write $\mathbb{Q} \cap[a, b]=\left\{q_1, q_2, \ldots\right\}$. Then $\left\{U_i\right\}_{i \in \mathbb{Z}_{+}}$, where $U_i=\mathbb{Q} \cap\left[a, q_i\right)$ for each $i$, is an open covering of $\mathbb{Q} \cap[a, b]$ with no finite subcovering. Now let $x \in \mathbb{Q}$ and suppose that $\mathbb{Q}$ is locally compact at $x$. Then there exists a compact set $C$ containing a neighbourhood $U$ of $x$. Then $U$ contains a set $\mathbb{Q} \cap[a, b]$ where $a, b$ are irrational numbers. Since this set is closed and contained in the compact $C$, it follows $\mathbb{Q} \cap[a, b]$ is compact, a contradiction. Therefore, $\mathbb{Q}$ is not locally compact.
\end{proof}



\paragraph{Exercise 29.4} Show that $[0, 1]^\omega$ is not locally compact in the uniform topology.
\begin{proof}
    Consider $\mathbf{0} \in[0,1]^\omega$ and suppose that $[0,1]^\omega$ is locally compact at $\mathbf{0}$. Then there exists a compact $C$ containing an open ball $B=B_\rho(\mathbf{0}, \varepsilon) \subset[0,1]^\omega$. Note that $\bar{B}=[0, \varepsilon]^\omega$. Then $[0, \varepsilon]^\omega$ is closed and contained in the compact $C$, so it is compact. But $[0, \varepsilon]^\omega$ is homeomorphic to $[0,1]^\omega$, which is not compact by Exercise 28.1. This contradiction proves that $[0,1]^\omega$ is not locally compact in the uniform topology.
\end{proof}



\paragraph{Exercise 29.10} Show that if $X$ is a Hausdorff space that is locally compact at the point $x$, then for each neighborhood $U$ of $x$, there is a neighborhood $V$ of $x$ such that $\bar{V}$ is compact and $\bar{V} \subset U$.
\begin{proof}
    Let $U$ be a neighbourhood of $x$. Since $X$ is locally compact at $x$, there exists a compact subspace $C$ of $X$ containing a neighbourhood $W$ of $x$. Then $U \cap W$ is open in $X$, hence in $C$. Thus, $C \backslash(U \cap W)$ is closed in $C$, hence compact. Since $X$ is Hausdorff, there exist disjoint open sets $V_1$ and $V_2$ of $X$ containing $x$ and $C \backslash(U \cap W)$ respectively. Let $V=V_1 \cap U \cap W$. Since $\bar{V}$ is closed in $C$, it is compact. Furthermore, $\bar{V}$ is disjoint from $C \backslash(U \cap W) \supset C \backslash U$, so $\bar{V} \subset U$.
\end{proof}



\paragraph{Exercise 30.10} Show that if $X$ is a countable product of spaces having countable dense subsets, then $X$ has a countable dense subset.
\begin{proof}
    Let $\left(X_n\right)$ be spaces having countable dense subsets $\left(A_n\right)$. For each $n$, fix an arbitrary $x_n \in X_n$. Consider the subset $A$ of $X$ defined by
$$
A=\bigcup\left\{\prod U_n: U_n=A_n \text { for finitely many } n \text { and is }\left\{x_n\right\} \text { otherwise }\right\} .
$$
This set is countable because the set of finite subsets of $\mathbb{N}$ is countable and each of the inner sets is countable. Now, let $x \in X$ and $V=\prod V_n$ be a basis element containing $x$ such that each $V_n$ is open in $X_n$ and $V_n=X_n$ for all but finitely many $n$. For each $n$, if $V_n \neq X_n$, choose a $y_n \in\left(A_n \cap V_n\right)$ (such a $y_n$ exists since $A_n$ is dense in $\left.X_n\right)$. Otherwise, let $y_n=x_n$. Then $\left(y_n\right) \in(A \cap V)$, proving that $A$ is dense in $X$.
\end{proof}



\paragraph{Exercise 30.13} Show that if $X$ has a countable dense subset, every collection of disjoint open sets in $X$ is countable.
\begin{proof}
    Let $\mathcal{U}$ be a collection of disjoint open sets in $X$ and let $A$ be a countable dense subset of $X$.
Since $A$ is dense in $X$, every $U \in \mathcal{U}$ intesects $S$. Therefore, there exists a point $x_U \in U \cap S$.
Let $U_1, U_2 \in \mathcal{U}, U_1 \neq U_2$. Then $x_{U_1} \neq x_{U_2}$ since $U_1 \cap U_2=\emptyset$.
Thus, the function $\mathcal{U} \rightarrow S$ given by $U \mapsto x_U$ is injective and therefore, since $S$ is countable, it follows that $\mathcal{U}$ is countable.
\end{proof}



\paragraph{Exercise 31.1} Show that if $X$ is regular, every pair of points of $X$ have neighborhoods whose closures are disjoint.
\begin{proof}
    Let $x, y \in X$ be two points such that $x \neq y$. Since $X$ is regular (and thus Hausdorff), there exist disjoint open sets $U, V \subseteq X$ such that $x \in U$ and $y \in V$.
Note that $y \notin \bar{U}$. Otherwise $V$ must intersect $U$ in a point different from $y$ since $V$ is an open neighborhood of $y$, which is a contradiction since $U$ and $V$ are disjoint.
Since $X$ is regular and $\bar{U}$ is closed, there exist disjoint open sets $U^{\prime}, V^{\prime} \subseteq X$ such that $\bar{U} \subseteq U^{\prime}$ and $y \in V^{\prime}$.

And now $U$ and $V^{\prime}$ are neighborhoods of $x$ and $y$ whose closures are disjoint. If $\bar{U} \cap \overline{V^{\prime}} \neq \emptyset$, then it follows that $U^{\prime} \supseteq U$ intersects $\overline{V^{\prime}}$. Since $U^{\prime}$ is open, it follows that $U^{\prime}$ intersects $V^{\prime}$, which is a contradiction.
\end{proof}



\paragraph{Exercise 31.2} Show that if $X$ is normal, every pair of disjoint closed sets have neighborhoods whose closures are disjoint.
\begin{proof}
    Let $A$ and $B$ be disjoint closed sets. Then there exist disjoint open sets $U$ and $V$ containing $A$ and $B$ respectively.

Since $X \backslash V$ is closed and contains $U$, the closure of $U$ is contained in $X \backslash V$ hence $B$ and closure of $U$ are disjoint.

Repeat steps 1 and 2 for $B$ and $\bar{U}$ instead of $A$ and $B$ respectively and you will have open set $V^{\prime}$ which contains $B$ and its closure doesn't intersect with $\bar{U}$.
\end{proof}



\paragraph{Exercise 31.3} Show that every order topology is regular.
\begin{proof}
    Let $X$ be an ordered set.
First we show that $X$ is a $T_1$-space. For $x \in X$ we have that
$$
X \backslash\{x\}=\langle-\infty, x\rangle \cup\langle x,+\infty\rangle
$$
which is an open set as an union of two open intervals. Therefore, the set $\{x\}$ is closed.
Step 2
2 of 3
Now to prove that $X$ is regular we use Lemma $\mathbf{3 1 . 1 .}$
Let $x \in X$ be any point and $U \subseteq X$ any open neighborhood of $x$. Then there exist $a, b \in X$ such that $x \in\langle a, b\rangle \subseteq U$. Now we have four possibilities.
1. If there exist $x_1, x_2 \in U$ such that $a<x_1<x<x_2<b$, then
$$
x \in\left\langle x_1, x_2\right\rangle \subseteq \overline{\left\langle x_1, x_2\right\rangle} \subseteq\left[x_1, x_2\right] \subseteq\langle a, b\rangle \subseteq U
$$
2. If there exists $x_1 \in U$ such that $a<x_1<x$, but there's no $x_2 \in U$ such that $x<x_2<b$, then
$$
x \in\left\langle x_1, b\right\rangle=\left(x_1, x\right] \subseteq \overline{\left(x_1, x\right]} \subseteq\left[x_1, x\right] \subseteq\langle a, b\rangle \subseteq U
$$
3. If there exists $x_2 \in U$ such that $x<x_2<b$, but there's no $x_1 \in U$ such that $a<x_1<x$, then
$$
x \in\left\langle a, x_2\right\rangle=\left[x, x_2\right) \subseteq \overline{\left[x, x_2\right)} \subseteq\left[x, x_2\right] \subseteq\langle a, b\rangle \subseteq U
$$
4. If there's no $x_1 \in U$ such that $a<x_1<x$ and no $x_2 \in U$ such that $x<x_2<b$, then
$$
x \in\langle a, b\rangle=\{x\}=\overline{\{x\}}=\{x\} \subseteq U
$$
We have that $\overline{\{x\}}=\{x\}$ because $X$ is a $T_1$-space.
In all four cases we proved that there exists an open interval $V$ such that $x \in V \subseteq \bar{V} \subseteq U$, so $X$ is regular.
\end{proof}



\paragraph{Exercise 32.1} Show that a closed subspace of a normal space is normal.
\begin{proof}
    Let $X$ be a normal space and $Y$ a closed subspace of $X$.
First we shows that $Y$ is a $T_1$-space.
Let $y \in Y$ be any point. Since $X$ is normal, $X$ is also a $T_1$ space and therefore $\{y\}$ is closed in $X$.
Then it follows that $\{y\}=\{y\} \cap Y$ is closed in $Y$ (in relative topology).
Now let's prove that $X$ is a $T_4$-space.
Let $F, G \subseteq Y$ be disjoint closed sets. Since $F$ and $G$ are closed in $Y$ and $Y$ is closed in $X$, it follows that $F$ and $G$ are closed in $X$.

Since $X$ is normal, $X$ is also a $T_4$-space and therefore there exist disjoint open sets $U, V \subseteq$ $X$ such that $F \subseteq U$ and $G \subseteq V$.
However, then $U \cap Y$ and $V \cap Y$ are open disjoint sets in $Y$ (in relative topology) which separate $F$ and $G$.
\end{proof}



\paragraph{Exercise 32.2a} Show that if $\prod X_\alpha$ is Hausdorff, then so is $X_\alpha$. Assume that each $X_\alpha$ is nonempty.
\begin{proof}
    Suppose that $X=\prod_\beta X_\beta$ is Hausdorff and let $\alpha$ be any index.
Let $x, y \in X_\alpha$ be any points such that $x \neq y$. Since all $X_\beta$ are nonempty, there exist points $\mathbf{x}, \mathbf{y} \in X$ such that $x_\beta=y_\beta$ for every $\beta \neq \alpha$ and $x_\alpha=x, y_\alpha=y$.
Since $x \neq y$, it follows that $\mathbf{x} \neq \mathbf{y}$. Since $X$ is Hausdorff, there exist open disjoint sets $U, V \subseteq X$ such that $\mathbf{x} \in U$ and $\mathbf{y} \in V$.
For $\beta \neq \alpha$ we have that $x_\beta=y_\beta \in \pi_\beta(U) \cap \pi_\beta(V)$, hence $\pi_\beta(U)$ and $\pi_\beta(V)$ are not disjoint.
Since $U$ and $V$ are disjoint, it follows that $\pi_\alpha(U) \cap \pi_\alpha(V)=\emptyset$.
We also have that $x \in \pi_\alpha(U)$ and $y \in \pi_\alpha(V)$ and since the projections are open maps, it follows that the sets $\pi_\alpha(U)$ and $\pi_\alpha(V)$ are open.
This proves that $x$ and $y$ can be separated by open sets, so $X_\alpha$ is Hausdorff.
\end{proof}



\paragraph{Exercise 32.2b} Show that if $\prod X_\alpha$ is regular, then so is $X_\alpha$. Assume that each $X_\alpha$ is nonempty.
\begin{proof}
    Suppose that $X=\prod_\beta X_\beta$ is regular and let $\alpha$ be any index.
We have to prove that $X_\alpha$ satisfies the $T_1$ and the $T_3$ axiom.
Since $X$ is regular, it follows that $X$ is Hausdorff, which then implies that $X_\alpha$ is Hausdorff. However, this implies that $X_\alpha$ satisfies the $T_1$ axiom.

Let now $F \subseteq X_\alpha$ be a closed set and $x \in X_\alpha \backslash F$ a point.
Then $\prod_\beta F_\beta$, where $F_\alpha=F$ and $F_\beta=X_\beta$ for $\beta \neq \alpha$, is a closed set in $X$ since $\left(\prod_\beta F_\beta\right)^c=\prod_\beta U_\beta$, where $U_\alpha=F^c$ and $U_\beta=X_\beta$ for $\beta \neq \alpha$, which is an open set because it is a base element for the product topology.
Since all $X_\beta$ are nonempty, there exists a point $\mathbf{x} \in X$ such that $x_\alpha=x$. Then $\mathbf{x} \notin \prod_\beta F_\beta$.
Now since $X$ is regular (and therefore satisfies the $T_3$ axiom), there exist disjoint open sets $U, V \subseteq X$ such that $\mathbf{x} \in U$ and $\prod_\beta F_\beta \subseteq V$.

Now for every $\beta \neq \alpha$ we have that $x_\beta \in X_\beta=\pi_\beta(V)$. However, since $x_\beta \in \pi_\beta(U)$, it follows that $\pi_\beta(U) \cap \pi_\beta(V) \neq \emptyset$.
Then $U \cap V=\emptyset$ implies that $\pi_\alpha(U) \cap \pi_\alpha(V)=\emptyset$.. Also, $x \in \pi_\alpha(U)$ and $F \subseteq \pi_\alpha(V)$ and $\pi_\alpha(U), \pi_\alpha(V)$ are open sets since $\pi_\alpha$ is an open map.
Therefore, $X_\alpha$ satisfies the $T_3$ axiom.
\end{proof}



\paragraph{Exercise 32.2c} Show that if $\prod X_\alpha$ is normal, then so is $X_\alpha$. Assume that each $X_\alpha$ is nonempty.
\begin{proof}
    Suppose that $X=\prod_\beta X_\beta$ is normal and let $\alpha$ be any index.
Since $X$ is normal, it follows that $X$ is Hausdorff (or regular), which then implies that $X_\alpha$ is Hausdorff (or regular). This imples that $X_\alpha$ satisfies the $T_1$ axiom.
Now the proof that $X_\alpha$ satisfies the $T_4$ axiom is the same as for regular spaces.
If $F, G \subseteq X_\alpha$ are disjoint closed sets, then $\prod_\beta F_\beta$ and $\prod_\beta G_\beta$, where $F_\alpha=F, G_\alpha=G$ and $F_\beta=G_\beta=X_\beta$ for $\beta \neq \alpha$, are disjoint closed sets in $X$.
Since $X$ is normal (and therefore satisfies the $T_4$ axiom), there exist disjoint open sets $U, V \subseteq X$ such that $\prod_\beta F_\beta \subseteq U$ and $\prod_\beta G_\beta \subseteq V$
Then $\pi_\alpha(U)$ and $\pi_\alpha(V)$ are disjoint open sets in $X_\alpha$ such that $F \subseteq \pi_\alpha(U)$ and $G \subseteq \pi_\alpha(V)$.
\end{proof}



\paragraph{Exercise 32.3} Show that every locally compact Hausdorff space is regular.
\begin{proof}
    Let $X$ be a LCH space.
Then it follows that for every $x \in X$ and for every open neighborhood $U \subseteq X$ of $x$ there exists an open neighborhood $V \subseteq X$ of $x$ such that $\bar{V} \subseteq U$ (and $\bar{V}$ is compact, but this is not important here).
Since $X$ is a Hausdorff space, it satisfies the $T_1$ axiom.
Then it follows that $X$ is regular.
\end{proof}



\paragraph{Exercise 33.7} Show that every locally compact Hausdorff space is completely regular.
\begin{proof}
    $X$ is a subspace of a compact Hausdorff space $Y$, its one-point compactification. $Y$ is normal, and so by the Urysohn lemma $Y$ is completely regular. Therefore by corollary $X$ is completely regular.
\end{proof}



\paragraph{Exercise 33.8} Let $X$ be completely regular, let $A$ and $B$ be disjoint closed subsets of $X$. Show that if $A$ is compact, there is a continuous function $f \colon X \rightarrow [0, 1]$ such that $f(A) = \{0\}$ and $f(B) = \{1\}$.
\begin{proof}
    Since $X$ is completely regular $\forall a \in A, \exists f_a: X \rightarrow[0,1]: f_a(a)=0$ and $f_a(B)=\{1\}$. For some $\epsilon_a \in(0,1)$ we have that $U_a:=f_a^{-1}([0, \epsilon))$ is an open neighborhood of $a$ that does not intersect $B$. We therefore have an open covering $\left\{U_a \mid a \in A\right\}$ of $A$, so since $A$ is compact we have a finite subcover $\left\{U_{a_i} \mid 1 \leq i \leq m\right\}$. For each $1 \leq i \leq m$ define
$$
\begin{aligned}
\tilde{f}_{a_i}: X & \rightarrow[0,1] \\
x & \mapsto \frac{\max \left(f_{a_i}(x), \epsilon_{a_i}\right)-\epsilon_{a_i}}{1-\epsilon_{a_i}}
\end{aligned}
$$
so that $\forall x \in U_{a_i}: \tilde{f}_{a_i}(x)=0$ and $\forall x \in B, \forall 1 \leq i \leq m: \tilde{f}_{a_i}(x)=1$, and define $f:=$ $\prod_{i=1}^m \tilde{f}_{a_i}$. Then since $A \subset \cup_{i=1}^m U_{a_i}$ we have that $f(A)=\{0\}$ and also we have $f(B)=\{1\}$.
\end{proof}



\paragraph{Exercise 34.9} Let $X$ be a compact Hausdorff space that is the union of the closed subspaces $X_1$ and $X_2$. If $X_1$ and $X_2$ are metrizable, show that $X$ is metrizable.
\begin{proof}
    Both $X_1$ and $X_2$ are compact, Hausdorff and metrizable, so by exercise 3 they are second countable, i.e. there are countable bases $\left\{U_{i, n} \subset X_i \mid n \in \mathbb{N}\right\}$ for $i \in\{1,2\}$. By the same exercise it is enough to show that $X$ is second countable. If $X_1 \cap X_2=\emptyset$ both $X_1$ and $X_2$ are open and the union $\left\{U_{i, n} \mid i \in\{1,2\} ; n \in \mathbb{N}\right\}$ of their countable bases form a countable base for $X$.

Suppose now $X_1 \cap X_2 \neq \emptyset$. Let $x \in X$ and $U \subset X$ be an open neighborhood of $x$. If $x \in X_i-X_j=X-$ $X_j$ then $U \cap X_i$ is open in $X_i$ and there is a basis neighborhood $U_{i, n}$ of $x$ such that $x \in U_{i, n} \cap X-X_j$ is an open neighborhood of $x$ in the open subset $X-X_j$, so $U_{i, n} \cap X-X_j$ is also open in $X$.
Suppose now that $x \in X_1 \cap X_2$. We have that $U \cap X_i$ is open in $X_i$ so there is a basis neighborhood $U_{i, n_i}$ contained in $U \cap X_i$. By definition of sub-space topology there is some open subset $V_{i, n_i} \subset X$ such that $U_{i, n_i}=$ $X_i \cap V_{i, n_i}$. Then
$$
x \in V_{1, n_1} \cap V_{2, n_2}=\left(V_{1, n_1} \cap V_{2, n_2} \cap X_1\right) \cup\left(V_{1, n_1} \cap V_{2, n_2} \cap X_2\right)=\left(U_{1, n_1} \cap V_{2, n_2}\right) \cup\left(V_{1, n_1} \cap U_{2, n_2}\right) \subset U
$$
Therefore the open subsets $U_{i, n} \cap X-X_j$ and $V_{1, n_1} \cap V_{2, n_2}$ form a countable base for $X$.
\end{proof}



\paragraph{Exercise 38.6} Let $X$ be completely regular. Show that $X$ is connected if and only if the Stone-Čech compactification of $X$ is connected.
\begin{proof}
    The closure of a connected set is connected, so if $X$ is connected so is $\beta(X)$
Suppose $X$ is the union of disjoint open subsets $U, V \subset X$. Define the continuous map
$$
\begin{aligned}
& f: X \rightarrow\{0,1\} \\
& x \mapsto \begin{cases}0, & x \in U \\
1, & x \in V\end{cases}
\end{aligned}
$$
By the fact that $\{0,1\}$ is compact and Hausdorff we can extend $f$ to a surjective map $\bar{f}: \beta(X) \rightarrow\{0,1\}$ such that $\bar{f}^{-1}(\{0\})$ and $\bar{f}^{-1}(\{1\})$ are disjoint open sets that cover $\beta(X)$, which makes this space not-connected.
\end{proof}



\paragraph{Exercise 43.2} Let $(X, d_X)$ and $(Y, d_Y)$ be metric spaces; let $Y$ be complete. Let $A \subset X$. Show that if $f \colon A \rightarrow Y$ is uniformly continuous, then $f$ can be uniquely extended to a continuous function $g \colon \bar{A} \rightarrow Y$, and $g$ is uniformly continuous.
\begin{proof}
    Let $\left(X, d_X\right)$ and $\left(Y, d_Y\right)$ be metric spaces; let $Y$ be complete. Let $A \subset X$. It is also given that $f: A \longrightarrow X$ is a uniformly continuous function. Then we have to show that $f$ can be uniquely extended to a continuous function $g: \bar{A} \longrightarrow Y$, and $g$ is uniformly coninuous.
We define the function $g$ as follows:
$$
g(x)= \begin{cases}f(x) & \text { if } x \in A \\ \lim _{n \rightarrow \infty} f\left(x_n\right) & \text { if } x \in \bar{A},\end{cases}
$$
where $\left\{x_n\right\} \subset A$ is some sequence in $A$ such that $\lim _{n \rightarrow \infty} x_n=x$. Now we have to check that the above definition is well defined. We first note that since $\left\{x_n\right\} \subset A$ is convergent in $X,\left\{x_n\right\}$ is Cauchy in $A$ and since $f$ is uniformly continuous on $A,\left\{f\left(x_n\right)\right\}$ is Cauchy in $Y$. Further, since $Y$ is complete $\left\{f\left(x_n\right)\right\}$ is convergent. So let us now consider two sequences $\left\{x_n\right\}$ and $\left\{y_n\right\}$ such that $\lim _{n \rightarrow \infty} x_n=\lim _{n \rightarrow \infty} y_n=x$. Then we need to prove that $\lim _{n \rightarrow \infty} f\left(x_n\right)=\lim _{n \rightarrow \infty} f\left(y_n\right)=f(x)$. Let
$$
\lim _{n \rightarrow \infty} f\left(x_n\right)=a \text { and } b=\lim _{n \rightarrow \infty} f\left(y_n\right)
$$
Now since $f$ is uniformly continuous for any given $\epsilon>0$ there exists $\delta>0$ such that
$$
d_Y(f(x), f(y))<\epsilon \text { whenever } d_X(x, y)<\delta \text { and } x, y \in A
$$
So for this $\delta>0$ there exists $N \in \mathbb{N}$ such that
$$
d_X\left(x_n, x\right)<\frac{\delta}{2} \text { and } d_X\left(y_n, x\right)<\frac{\delta}{2}, \text { foe all } n \geq N .
$$
Therefore, we have that for all $n \geq N$,
$$
d_X\left(x_n, y_n\right)<\delta
$$
Thus the equation (1) yields us that
$$
d_Y\left(f\left(x_n\right), f\left(y_n\right)\right)<\epsilon \text { for all } n \geq N .
$$
Now since $\lim _{n \rightarrow \infty} f\left(x_n\right)=a$ and $b=\lim _{n \rightarrow \infty} f\left(y_n\right)$, so for the above $\epsilon>0$ we have a natural number $K \geq N$ such that
$$
\begin{gathered}
d_Y\left(f\left(x_n\right), a\right)<\epsilon \text { for all } n \geq K \text { and } \\
d_Y\left(f\left(y_n\right), b\right)<\epsilon \text { for all } n \geq K .
\end{gathered}
$$
Moreover, since $K \geq N$, from $(2)$ we get
$$
d_Y\left(f\left(x_n\right), f\left(y_n\right)\right)<\epsilon \text { for all } n \geq K .
$$
Now we calculate the following, for $n \geq K$,
$$
\begin{array}{rlr}
d_Y(a, b) & \leq & d_Y\left(a, f\left(x_n\right)\right)+d_Y\left(f\left(x_n\right), f\left(y_n\right)\right)+d\left(f\left(y_n\right), b\right) \\
& < & \epsilon+\epsilon+\epsilon \text { by }(3),(4) \text { and }(5) \\
& = & 3 \epsilon
\end{array}
$$
where the first inequality holds because of triangular inequality. Since $\epsilon>0$ is arbitrary the above calculation shows that $d_Y(a, b)=0$. Thus, the above definition is independent of the choice of the sequence $\left\{x_n\right\}$ and hence the map $g$ is well defined. Moreover, from the construction it follows that $g$ is continuous on $\bar{A}$.
Moreover, we observe that $g$ is unique extension of $f$ by the construction.
So it remains to show that $g$ is uniformly continuous. In order to that we take a Cauchy sequence $\left\{a_n\right\} \subset \bar{A}$. Then since $\bar{A}$ is a closed set so the sequence $\left\{a_n\right\}$ is convergent and hence $\left\{g\left(a_n\right)\right\}$ is also a convergent sequence as $g$ is continuous on $\bar{A}$. So $\left\{g\left(a_n\right)\right\}$ is a Cauchy sequence in $Y$. Since a function is uniformly continuous if and only if it sends Cauchy sequences to Cauchy sequences, we conclude that $g$ is uniformly continuous.
\end{proof}
\end{document}
