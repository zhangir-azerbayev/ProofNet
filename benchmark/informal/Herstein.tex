\documentclass{article}

\title{\textbf{
Exercises from \\
\textit{Abstract Algebra} \\
by I. N. Herstein
}}

\date{}

\usepackage{amsmath}
\usepackage{amssymb}
\usepackage{amsthm}

\begin{document}
\maketitle

\paragraph{Exercise 2.1.18} If $G$ is a finite group of even order, show that there must be an element $a \neq e$ such that $a=a^{-1}$.
\begin{proof}
    First note that $a=a^{-1}$ is the same as saying $a^2=e$, where $e$ is the identity. I.e. the statement is that there exists at least one element of order 2 in $G$.
Every element $a$ of $G$ of order at least 3 has an inverse $a^{-1}$ that is not itself -- that is, $a \neq a^{-1}$. So the subset of all such elements has an even cardinality (/size). There's exactly one element with order 1 : the identity $e^1=e$. So $G$ contains an even number of elements -call it $2 k$-- of which an even number are elements of order 3 or above -- call that $2 n$ where $n<k$-- and exactly one element of order 1 . Hence the number of elements of order 2 is
$$
2 k-2 n-1=2(k-n)-1
$$
This cannot equal 0 as $2(k-n)$ is even and 1 is odd. Hence there's at least one element of order 2 in $G$, which concludes the proof.
\end{proof}


\paragraph{Exercise 2.1.21} Show that a group of order 5 must be abelian.
\begin{proof}
    Suppose $G$ is a group of order 5 which is not abelian. Then there exist two non-identity elements $a, b \in G$ such that $a * b \neq$ $b * a$. Further we see that $G$ must equal $\{e, a, b, a * b, b * a\}$. To see why $a * b$ must be distinct from all the others, not that if $a *$ $b=e$, then $a$ and $b$ are inverses and hence $a * b=b * a$.
Contradiction. If $a * b=a$ (or $=b$ ), then $b=e$ (or $a=e$ ) and $e$ commutes with everything. Contradiction. We know by supposition that $a * b \neq b * a$. Hence all the elements $\{e, a, b, a * b, b * a\}$ are distinct.

Now consider $a^2$. It can't equal $a$ as then $a=e$ and it can't equal $a * b$ or $b * a$ as then $b=a$. Hence either $a^2=e$ or $a^2=b$.
Now consider $a * b * a$. It can't equal $a$ as then $b * a=e$ and hence $a * b=b * a$. Similarly it can't equal $b$. It also can't equal $a * b$ or $b * a$ as then $a=e$. Hence $a * b * a=e$.

So then we additionally see that $a^2 \neq e$ because then $a^2=e=$ $a * b * a$ and consequently $a=b * a$ (and hence $b=e$ ). So $a^2=b$. But then $a * b=a * a^2=a^2 * a=b * a$. Contradiction.
Hence starting with the assumption that there exists an order 5 abelian group $G$ leads to a contradiction. Thus there is no such group.
\end{proof}


\paragraph{Exercise 2.1.26} If $G$ is a finite group, prove that, given $a \in G$, there is a positive integer $n$, depending on $a$, such that $a^n = e$.
\begin{proof}
    Because there are only a finite number of elements of $G$, it's clear that the set $\left\{a, a^2, a^3, \ldots\right\}$ must be a finite set and in particular, there should exist some $i$ and $j$ such that $i \neq j$ and $a^i=a^j$. WLOG suppose further that $i>j$ (just reverse the roles of $i$ and $j$ otherwise). Then multiply both sides by $\left(a^j\right)^{-1}=a^{-j}$ to get
$$
a^i * a^{-j}=a^{i-j}=e
$$
Thus the $n=i-j$ is a positive integer such that $a^n=e$.
\end{proof}


\paragraph{Exercise 2.1.27} If $G$ is a finite group, prove that there is an integer $m > 0$ such that $a^m = e$ for all $a \in G$.
\begin{proof}
    Let $n_1, n_2, \ldots, n_k$ be the orders of all $k$ elements of $G=$ $\left\{a_1, a_2, \ldots, a_k\right\}$. Let $m=\operatorname{lcm}\left(n_1, n_2, \ldots, n_k\right)$. Then, for any $i=$ $1, \ldots, k$, there exists an integer $c$ such that $m=n_i c$. Thus
$$
a_i^m=a_i^{n_i c}=\left(a_i^{n_i}\right)^c=e^c=e
$$
Hence $m$ is a positive integer such that $a^m=e$ for all $a \in G$.
\end{proof}


\paragraph{Exercise 2.2.3} If $G$ is a group in which $(a b)^{i}=a^{i} b^{i}$ for three consecutive integers $i$, prove that $G$ is abelian.
\begin{proof}
    Let $G$ be a group, $a, b \in G$ and $i$ be any integer. Then from given condition,
$$
\begin{aligned}
(a b)^i & =a^i b^i \\
(a b)^{i+1} & =a^{i+1} b^{i+1} \\
(a b)^{i+2} & =a^{i+2} b^{i+2}
\end{aligned}
$$
From first and second, we get
$$
a^{i+1} b^{i+1}=(a b)^i(a b)=a^i b^i a b \Longrightarrow b^i a=a b^i
$$
From first and third, we get
$$
a^{i+2} b^{i+2}=(a b)^i(a b)^2=a^i b^i a b a b \Longrightarrow a^2 b^{i+1}=b^i a b a
$$
This gives
$$
a^2 b^{i+1}=a\left(a b^i\right) b=a b^i a b=b^i a^2 b
$$
Finally, we get
$$
b^i a b a=b^i a^2 b \Longrightarrow b a=a b
$$
This shows that $G$ is Abelian.
\end{proof}


\paragraph{Exercise 2.2.5} Let $G$ be a group in which $(a b)^{3}=a^{3} b^{3}$ and $(a b)^{5}=a^{5} b^{5}$ for all $a, b \in G$. Show that $G$ is abelian.
\begin{proof}
 We have
$$
\begin{aligned}
& (a b)^3=a^3 b^3, \text { for all } a, b \in G \\
\Longrightarrow & (a b)(a b)(a b)=a\left(a^2 b^2\right) b \\
\Longrightarrow & a(b a)(b a) b=a\left(a^2 b^2\right) b \\
\Longrightarrow & (b a)^2=a^2 b^2, \text { by cancellation law. }
\end{aligned}
$$
Again,
$$
\begin{aligned}
& (a b)^5=a^5 b^5, \text { for all } a, b \in G \\
\Longrightarrow & (a b)(a b)(a b)(a b)(a b)=a\left(a^4 b^4\right) b \\
\Longrightarrow & a(b a)(b a)(b a)(b a) b=a\left(a^4 b^4\right) b \\
\Longrightarrow & (b a)^4=a^4 b^4, \text { by cancellation law. }
\end{aligned}
$$
Now by combining two cases we have
$$
\begin{aligned}
& (b a)^4=a^4 b^4 \\
\Longrightarrow & \left((b a)^2\right)^2=a^2\left(a^2 b^2\right) b^2 \\
\Longrightarrow & \left(a^2 b^2\right)^2=a^2\left(a^2 b^2\right) b^2 \\
\Longrightarrow & \left(a^2 b^2\right)\left(a^2 b^2\right)=a^2\left(a^2 b^2\right) b^2 \\
\Longrightarrow & a^2\left(b^2 a^2\right) b^2=a^2\left(a^2 b^2\right) b^2 \\
\Longrightarrow & b^2 a^2=a^2 b^2, \text { by cancellation law. } \\
\Longrightarrow & b^2 a^2=(b a)^2, \text { since }(b a)^2=a^2 b^2 \\
\Longrightarrow & b(b a) a=(b a)(b a) \\
\Longrightarrow & b(b a) a=b(a b) a \\
\Longrightarrow & b a=a b, \text { by cancellation law. }
\end{aligned}
$$
It follows that, $a b=b a$ for all $a, b \in G$. Hence $G$ is abelian
\end{proof}


\paragraph{Exercise 2.2.6c} Let $G$ be a group in which $(a b)^{n}=a^{n} b^{n}$ for some fixed integer $n>1$ for all $a, b \in G$. For all $a, b \in G$, prove that $\left(a b a^{-1} b^{-1}\right)^{n(n-1)}=e$. 
\begin{proof}
    We start with the following two intermediate results.
(1) $(a b)^{n-1}=b^{n-1} a^{n-1}$.
(2) $a^n b^{n-1}=b^{n-1} a^n$.
To prove (1), notice by the given condition for all $a, b \in G$
$(b a)^n=b^n a^n$, for some fixed integers $n>1$.
Then,
$(b a)^n=b^n a^n \Longrightarrow b .(a b)(a b) \ldots .(a b) . a=b\left(b^{n-1} a^{n-1}\right) a$, where $(a b)$ occurs $n-1$ times $\Longrightarrow(a b)^{n-1}=b^{n-1} a^{n-1}$, by cancellation law.
Hence, for all $a, b \in G$
$$
(a b)^{n-1}=b^{n-1} a^{n-1} .
$$
To prove (2), notice by the given condition for all $a, b \in G$
$(b a)^n=b^n a^n$, for some fixed integers $n>1$.
Then we have
$$
\begin{aligned}
& (b a)^n=b^n a^n \\
\Longrightarrow & b \cdot(a b)(a b) \ldots(a b) \cdot a=b\left(b^{n-1} a^{n-1}\right) a, \text { where }(a b) \text { occurs } n-1 \text { times } \\
\Longrightarrow & (a b)^{n-1}=b^{n-1} a^{n-1}, \text { by cancellation law } \\
\Longrightarrow & (a b)^{n-1}(a b)=\left(b^{n-1} a^{n-1}\right)(a b) \\
\Longrightarrow & (a b)^n=b^{n-1} a^n b \\
\Longrightarrow & a^n b^n=b^{n-1} a^n b, \text { given condition } \\
\Longrightarrow & a^n b^{n-1}=b^{n-1} a^n, \text { by cancellation law. }
\end{aligned}
$$
Therefore for all $a, b \in G$ we have
$$
a^n b^{n-1}=b^{n-1} a^n
$$
In order to show that
$$
\left(a b a^{-1} b^{-1}\right)^{n(n-1)}=e, \text { for all } a, b \in G
$$
it is enough to show that
$$
(a b)^{n(n-1)}=(b a)^{n(n-1)}, \quad \forall x, y \in G .
$$
Step 3
This is because of
$$
\begin{aligned}
(a b)^{n(n-1)}=(b a)^{n(n-1)} & \left.\Longrightarrow(b a)^{-1}\right)^{n(n-1)}(a b)^{n(n-1)}=e \\
& \Longrightarrow\left(a^{-1} b^{-1}\right)^{n(n-1)}(a b)^{n(n-1)}=e \\
& \Longrightarrow\left(\left(a^{-1} b^{-1}\right)^n\right)^{n-1}\left((a b)^n\right)(n-1)=e \\
& \Longrightarrow\left((a b)^n\left(a^{-1} b^{-1}\right)^n\right)^{n-1}=e, \text { by }(1) \\
& \Longrightarrow\left(a b a^{-1} b^{-1}\right)^{n(n-1)}=e, \text { ( given condition) }
\end{aligned}
$$
Now, it suffices to show that
$$
(a b)^{n(n-1)}=(b a)^{n(n-1)}, \quad \forall x, y \in G .
$$
Now, we have
$$
\begin{aligned}
(a b)^{n(n-1)} & =\left(a^n b^n\right)^{n-1}, \text { by the given condition } \\
& =\left(a^n b^{n-1} b\right)^{n-1} \\
& =\left(b^{n-1} a^n b\right)^{n-1}, \text { by }(2) \\
& =\left(a^n b\right)^{n-1}\left(b^{n-1}\right)^{n-1}, \text { by }(1) \\
& =b^{n-1}\left(a^n\right)^{n-1}\left(b^{n-1}\right)^{n-1}, \text { by }(1) \\
& =\left(b^{n-1}\left(a^{n-1}\right)^n\right)\left(b^{n-1}\right)^{n-1} \\
& =\left(a^{n-1}\right)^n b^{n-1}\left(b^{n-1}\right)^{n-1}, \text { by }(2) \\
& =\left(a^{n-1}\right)^n\left(b^{n-1}\right)^n \\
& =\left(a^{n-1} b^{n-1}\right)^n, \text { by }(1) \\
& =(b a)^{n(n-1)}, \text { by }(1) .
\end{aligned}
$$
This completes our proof.
\end{proof}


\paragraph{Exercise 2.3.17} If $G$ is a group and $a, x \in G$, prove that $C\left(x^{-1} a x\right)=x^{-1} C(a) x$
\begin{proof}
    Note that
$$
C(a):=\{x \in G \mid x a=a x\} .
$$
Let us assume $p \in C\left(x^{-1} a x\right)$. Then,
$$
\begin{aligned}
& p\left(x^{-1} a x\right)=\left(x^{-1} a x\right) p \\
\Longrightarrow & \left(p x^{-1} a\right) x=x^{-1}(a x p) \\
\Longrightarrow & x\left(p x^{-1} a\right)=(a x p) x^{-1} \\
\Longrightarrow & \left(x p x^{-1}\right) a=a\left(x p x^{-1}\right) \\
\Longrightarrow & x p x^{-1} \in C(a) .
\end{aligned}
$$
Therefore,
$$
p \in C\left(x^{-1} a x\right) \Longrightarrow x p x^{-1} \in C(a) .
$$
Thus,
$$
C\left(x^{-1} a x\right) \subset x^{-1} C(a) x .
$$
Let us assume
$$
q \in x^{-1} C(a) x .
$$
Then there exists an element $y$ in $C(a)$ such that
$$
q=x^{-1} y x
$$
Now,
$$
y \in C(a) \Longrightarrow y a=a y .
$$
Also,
$$
q\left(x^{-1} a x\right)=\left(x^{-1} y x\right)\left(x^{-1} a x\right)=x^{-1}(y a) x=x^{-1}(y a) x=\left(x^{-1} y x\right)\left(x^{-1} a x\right)=\left(x^{-1} y x\right) q .
$$
Therefore,
$$
q\left(x^{-1} a x\right)=\left(x^{-1} y x\right) q
$$
So,
$$
q \in C\left(x^{-1} a x\right) .
$$
Consequently we have
$$
x^{-1} C(a) x \subset C\left(x^{-1} a x\right) .
$$
It follows from the aforesaid argument
$$
C\left(x^{-1} a x\right)=x^{-1} C(a) x .
$$
This completes the proof.
\end{proof}


\paragraph{Exercise 2.3.19} If $M$ is a subgroup of $G$ such that $x^{-1} M x \subset M$ for all $x \in G$, prove that actually $x^{-1} M x=M$. 

\paragraph{Exercise 2.3.16} If a group $G$ has no proper subgroups, prove that $G$ is cyclic of order $p$, where $p$ is a prime number.
\begin{proof}
    Case-1: $G=(e), e$ being the identity element in $G$. Then trivially $G$ is cyclic.
    Case-2: $G \neq(e)$. Then there exists an non-identity element in $G.$ Let us consider an non-identity element in $G$, say $a\neq (e)$. Now look at the cyclic subgroup generated by $a$, that is, $\langle a\rangle$. Since
    $a\neq (e) \in G,\langle a\rangle$ is a subgroup of $G$.
If $G \neq\langle a\rangle$ then $\langle a\rangle$ is a proper non-trivial subgroup of $G$, which is an impossibility. Therfore we must have
$$
G=\langle a\rangle .
$$
This implies, $G$ is a cyclic group generated by $a$. Then it follows that every non-identity element of $G$ is a generator of $G$. Now we claim that $G$ is finite.
\end{proof}


\paragraph{Exercise 2.3.21} If $A, B$ are subgroups of $G$ such that $b^{-1} Ab \subset A$ for all $b \in B$, show that $AB$ is a subgroup of $G$.
\begin{proof}
    Proof: Let us consider any two elements $p$ and $q$ in $A B$. Then there exist elements $a_1, a_2 \in A$ and $b_1, b_2 \in B$ such that
$$
p=a_1 b_1 \text { and } q=a_2 b_2 .
$$
Now,
$$
\begin{aligned}
p q^{-1} & =\left(a_1 b_1\right)\left(a_2 b_2\right)^{-1} \\
& =\left(a_1 b_1\right)\left(b_2^{-1} a_2^{-1}\right) \\
& =a_1\left(b_1 b_2^{-1} a_2^{-1} b_2 b_1^{-1}\right) b_1 b_2^{-1} .
\end{aligned}
$$
Since $b^{-1} A b \subset A$, for all $b \in B$, we have
$$
b_1 b_2{ }^{-1} a_2{ }^{-1} b_2 b_1^{-1} \in A .
$$
\end{proof}


\paragraph{Exercise 2.3.22} If $A$ and $B$ are finite subgroups, of orders $m$ and $n$, respectively, of the abelian group $G$, prove that $AB$ is a subgroup of order $mn$ if $m$ and $n$ are relatively prime.
\begin{proof}
    Proof: Firstly we show that $A B$ forms a subgroup of the abelian group $G$. Let us consider $p \in A B, q \in A B$ and $p=a_1 b_1, q=a_2 b_2$, for some $a_1, a_2 \in A$ and $b_1, b_2 \in B$. Then,
$$
\begin{aligned}
p q & =\left(a_1 b_1\right)\left(a_2 b_2\right) \\
& =a_1\left(b_1 a_2\right) b_2 \\
& =a_1\left(a_2 b_1\right) b_2, \text { since } G \text { is abelian } \\
& =\left(a_1 a_2\right)\left(b_1 b_2\right) \in A B .
\end{aligned}
$$
Therefore,
$$
p, q \in A B \Longrightarrow p q \in A B .
$$
Also,
$$
p^{-1}=\left(a_1 b_1\right)^{-1}=\left(b_1\right)^{-1}\left(a_1\right)^{-1}=\left(a_1\right)^{-1}\left(b_1\right)^{-1} \in A B .
$$
So $A B$ is a subgroup of $G$.
\end{proof}


\paragraph{Exercise 2.3.28} Let $M, N$ be subgroups of $G$ such that $x^{-1}Mx \subset M$ and $x^{-1} N x \subset N$ for all $x \in G$. Prove that $MN$ is a subgroup of $G$ and that $x^{-1} (MN)x \subset MN$ for all $x \in G$.
\begin{proof}
    Proof: First we assert that $M N$ is a subgroup of $G$. Let us consider two elements
$$
x, y \in M N \text {. }
$$
Then, there exists $m_1, m_2 \in M$ and $n_1, n_2 \in N$ such that
$$
x=m_1 n_1 \text { and } y=m_2 n_2 .
$$
Now we need to show that $x y^{-1} \in M N$. Now,
$$
\begin{gathered}
x y^{-1}=m_1 n_1\left(m_2 n_2\right)^{-1} \\
=m_1 n_1 n_2^{-1} m_2^{-1} \\
=m_1 m_2^{-1}\left(m_2 n_1 n_2^{-1} m_2^{-1}\right) .
\end{gathered}
$$
Since,
$n_1, n_2 \in N$, then $n_1 n_2^{-1} \in N$ and this implies $m_2 n_1 n_2^{-1} m_2^{-1} \in N$.
Consequently,
$$
x y^{-1}=m_1 m_2^{-1}\left(m_2 n_1 n_2^{-1} m_2^{-1}\right) \in M N .
$$
Thus,
$$
x, y \in M N \Longrightarrow x y \in M N .
$$
Hence, $M N$ is a subgroup of $G$.
\end{proof}


\paragraph{Exercise 2.3.29} If $M$ is a subgroup of $G$ such that $x^{-1}Mx \subset M$ for all $x \in G$, prove that actually $x^{-1}Mx = M$.
\begin{proof}
    Proof: To prove $x^{-1} M x=M$, it suffices to show that
$$
M \subset x^{-1} M x
$$
Let us consider an element $m$ in $M$. Then,
$$
m=x^{-1}\left(x m x^{-1}\right) x, \text { for any } x \in G .
$$
Since $G$ is a group,
$$
x \in G \Longrightarrow x^{-1} \in G .
$$
So,
$$
x m x^{-1}=\left(x^{-1}\right)^{-1} m x^{-1} \in x^{-1} M x(\subset M) \Longrightarrow x m x^{-1} \in M
$$
It follows that
$$
m=x^{-1}\left(x m x^{-1}\right) x \in x^{-1} M x .
$$
Thus,
$$
m \in M \Longrightarrow m \in x^{-1} M x .
$$
Consequently,
$$
M \subset x^{-1} M x
$$
Thence,
$$
M=x^{-1} M x
$$
This completes the proof.
\end{proof}


\paragraph{Exercise 2.4.8} If every right coset of $H$ in $G$ is a left coset of $H$ in $G$, prove that $aHa^{-1} = H$ for all $a \in G$.
\begin{proof}
    Proof: We have
$$
H a=b H, \text { for } a, b \in G .
$$
Then there exist $h_1, h_2 \in H$ such that
$$
h_1 a=b h_2 .
$$
Hence,
$$
\begin{aligned}
h_1 a=b h_2 & \Longrightarrow b=h_1 a h_2^{-1} \\
& \Longrightarrow b H=h_1 a h_2^{-1} H \\
& \Longrightarrow H a=h_1 a h_2^{-1} H \\
& \Longrightarrow H a=h_1 a H \\
& \Longrightarrow h_1^{-1} H a=a H \\
& \Longrightarrow H a=a H
\end{aligned}
$$
Therefore,
$$
\begin{aligned}
& H a=a H, \text { for all } a \in G \\
\Longrightarrow & H=a H a^{-1} .
\end{aligned}
$$
This completes the proof.
\end{proof}


\paragraph{Exercise 2.4.26} Let $G$ be a group, $H$ a subgroup of $G$, and let $S$ be the set of all distinct right cosets of $H$ in $G$, $T$ the set of all left cosets of $H$ in $G$. Prove that there is a 1-1 mapping of $S$ onto $T$.
\begin{proof}
    It suffices to show that there is a bijection betwwen the set of all distinct left cosets of $H$ in $G$ and the set of all distinct right cosets of $H$ in $G$
Let us consider an element $a$ in $G$.
Let us define a mapping
$$
f: S \rightarrow T
$$
by the assignment
$$
f(H a)=a^{-1} H, H a \in S .
$$
First we show that the mapping $f$ is well defined in the sense that if
$$
H x=H a \text { then } x^{-1} H=a^{-1} H .
$$
Now
$$
\begin{aligned}
H x=H a & \Longrightarrow x \in H a \\
& \Longrightarrow x a^{-1} \in H \\
& \Longrightarrow\left(x^{-1}\right)^{-1} a^{-1} \in H \\
& \Longrightarrow a^{-1} \in x^{-1} H \\
& \Longrightarrow a^{-1} H=x^{-1} H .
\end{aligned}
$$
Therefore $f$ assigns a unique coset in $T$ to a unique coset in $S$.
We now prove that $f$ is one-one.
Let $H a, H b \in S$ and $H a \neq H b$.
Then,
$$
\begin{aligned}
f(H a)=f(H b) & \Longrightarrow a^{-1} H=b^{-1} H \\
& \Longrightarrow a^{-1} \in b^{-1} H \\
& \Longrightarrow\left(b^{-1}\right)^{-1} a^{-1} \in H \\
& \Longrightarrow b a^{-1} \in H \\
& \Longrightarrow b \in H a \Longrightarrow H b=H a .
\end{aligned}
$$
So,
$$
H a \neq H b \Longrightarrow f(H a) \neq f(H b) .
$$
This proves that $f$ is one-one.
In order to prove the $f$ is onto, let us take an element $a H$ in $T$. The pre-image of $a H$ is $H a^{-1}$ in $S$, since
$$
f\left(H a^{-1}\right)=\left(a^{-1}\right)^{-1} H=a H .
$$
Therefore, $f$ is onto.
Consequently, $f$ is a bijection from $S$ to $T$.
Hence we get an one-one mapping $S$ onto $T$.
This completes the proof.
\end{proof}


\paragraph{Exercise 2.4.32} Let $G$ be a finite group, $H$ a subgroup of $G$. Let $f(a)$ be the least positive $m$ such that $a^m \in H$. Prove that $f(a) \mid o(a)$, where $o(a)$ is an order of $a$.
\begin{proof}
    Let us assume that
$$
o(a)=n .
$$
Then by Division Algorithm, there exist $q$ and $r$ such that
$$
n=q f(a)+r, \text { where } 0 \leq r<f(a) .
$$
Since $o(a)=n$, we have
$$
\begin{aligned}
a^n= & \Longrightarrow(a)^{q f(a)} \cdot a^r=e \\
& \Longrightarrow\left(a^{f(a)}\right)^q \cdot a^r=e
\end{aligned}
$$
Now,
$$
a^{f(a)} \in H \Longrightarrow\left(a^{f(a)}\right)^q \in H \Longrightarrow a^r \in H \text {, as } e \in H .
$$
The minimality of $f(a)$ that $a^{f(a)} \in H$, forced $r=0$. It follows that
$$
n=q f(a) .
$$
Therefore,
$$
f(a) \mid o(a)
$$
This completes the proof.
\end{proof}


\paragraph{Exercise 2.4.36} If $a > 1$ is an integer, show that $n \mid \varphi(a^n - 1)$, where $\phi$ is the Euler $\varphi$-function.
\begin{proof}
    Proof: We have $a>1$. First we propose to prove that
$$
\operatorname{Gcd}\left(a, a^n-1\right)=1 .
$$
If possible, let us assume that
$\operatorname{Gcd}\left(a, a^n-1\right)=d$, where $d>1$.
Then
$d$ divides $a$ as well as $a^n-1$.
Now,
$d$ divides $a \Longrightarrow d$ divides $a^n$.
This is an impossibility, since $d$ divides $a^n-1$ by our assumption. Consequently, $d$ divides 1 , which implies $d=1$. Hence we are contradict to the fact that $d>1$. Therefore
$$
\operatorname{Gcd}\left(a, a^n-1\right)=1 .
$$
Then $a \in U_{a^n-1}$, where $U_n$ is a group defined by
$$
U_n:=\left\{\bar{a} \in \mathbb{Z}_n \mid \operatorname{Gcd}(a, n)=1\right\} .
$$
We know that order of an element divides the order of the group. Here order of the group $U_{a^n-1}$ is $\phi\left(a^n-1\right)$ and $a \in U_{a^n-1}$. This follows that $\mathrm{o}(a)$ divides $\phi\left(a^n-1\right)$.
\end{proof}

\paragraph{Exercise 2.5.23} Let $G$ be a group such that all subgroups of $G$ are normal in $G$. If $a, b \in G$, prove that $ba = a^jb$ for some $j$.
\begin{proof}
    Let $G$ be a group where each subgroup is normal in $G$. let $a, b \in G$.
$$
\begin{aligned}
    \langle a\rangle\triangleright  G  &\Rightarrow b \cdot\langle a\rangle=\langle a\rangle \cdot b . \\
& \Rightarrow \quad b \cdot a=a^j \cdot b \text { for some } j \in \mathbb{Z}.
\end{aligned}
$$
(hence for $a_1 b \in G \quad a^j b=b \cdot a$ ).
\end{proof}


\paragraph{Exercise 2.5.30} Suppose that $|G| = pm$, where $p \nmid m$ and $p$ is a prime. If $H$ is a normal subgroup of order $p$ in $G$, prove that $H$ is characteristic.
\begin{proof}
    Let $G$ be a group of order $p m$, such that $p \nmid m$. Now, Given that $H$ is a normal subgroup of order $p$. Now we want to prove that $H$ is a characterestic subgroup, that is $\phi(H)=H$ for any automorphism $\phi$ of $G$. Now consider $\phi(H)$. Clearly $|\phi(H)|=p$. Suppose $\phi(H) \neq H$, then $H \cap \phi (H)=\{ e\}$. Consider $H \phi(H)$, this is a subgroup of $G$ as $H$ is normal. Also $|H \phi(H)|=p^2$. By lagrange's theorem then $p^2 \mid$ $p m \Longrightarrow p \mid m$ - contradiction. So $\phi(H)=H$, and $H$ is characterestic subgroup of $G$
\end{proof}


\paragraph{Exercise 2.5.31} Suppose that $G$ is an abelian group of order $p^nm$ where $p \nmid m$ is a prime.  If $H$ is a subgroup of $G$ of order $p^n$, prove that $H$ is a characteristic subgroup of $G$.
\begin{proof}
    Let $G$ be an abelian group of order $p^n m$, such that $p \nmid m$. Now, Given that $H$ is a subgroup of order $p^n$. Since $G$ is abelian $H$ is normal. Now we want to prove that $H$ is a characterestic subgroup, that is $\phi(H)=H$ for any automorphism $\phi$ of $G$. Now consider $\phi(H)$. Clearly $|\phi(H)|=p^n$. Suppose $\phi(H) \neq H$, then $|H \cap \phi(H)|=p^s$, where $s<n$. Consider $H \phi(H)$, this is a subgroup of $G$ as $H$ is normal. Also $|H \phi(H)|=\frac{|H||\phi(H)|}{|H \cap \phi(H)|}=\frac{p^{2 n}}{p^s}=p^{2 n-s}$, where $2 n-s>n$. By lagrange's theorem then $p^{2 n-s}\left|p^n m \Longrightarrow p^{n-s}\right| m \Longrightarrow p \mid m$-contradiction. So $\phi(H)=H$, and $H$ is characterestic subgroup of $G$.
\end{proof}


\paragraph{Exercise 2.5.37} If $G$ is a nonabelian group of order 6, prove that $G \simeq S_3$.
\begin{proof}
    Suppose $G$ is a non-abelian group of order 6 . We need to prove that $G \cong S_3$. Since $G$ is non-abelian, we conclude that there is no element of order 6. Now all the nonidentity element has order either 2 or 3 . All elements cannot be order 3 .This is because except the identity elements there are 5 elements, but order 3 elements occur in pair, that is $a, a^2$, both have order 3 , and $a \neq a^2$. So, this is a contradiction, as there are only 5 elements. So, there must be an element of order 2 . All elements of order 2 will imply that $G$ is abelian, hence there is also element of order 3 . Let $a$ be an element of order 2 , and $b$ be an element of order 3 . So we have $e, a, b, b^2$, already 4 elements. Now $a b \neq e, b, b^2$. So $a b$ is another element distinct from the ones already constructed. $a b^2 \neq e, b, a b, b^2, a$. So, we have got another element distinct from the other. So, now $ G=\left\{e, a, b, b^2, a b, a b^2\right\}$. Also, ba must be equal to one of these elements. But $b a \neq e, a, b, b^2$. Also if $b a=a b$, the group will become abelian. so $b a=a b^2$. So what we finally get is $G=\left\langle a, b \mid a^2=e=b^3, b a=a b^2\right\rangle$. Hence $G \cong S_3$.
\end{proof}


\paragraph{Exercise 2.5.43} Prove that a group of order 9 must be abelian.
\begin{proof}
    We use the result from problem 40 which is as follows: Suppose $G$ is a group, $H$ is a subgroup and $|G|=n$ and $n \nmid\left(i_G(H)\right) !$. Then there exists a normal subgroup $\$ K \backslash$ neq $\{$ e $\} \$$ and $K \subseteq H$.
So, we have now a group $G$ of order 9. Suppose that $G$ is cyclic, then $G$ is abelian and there is nothing more to prove. Suppose that $G$ s not cyclic,then there exists an element $a$ of order 3 , and $A=\langle a\rangle$. Now $i_G(A)=3$, now $9 \nmid 3$ !, hence by the above result there is a normal subgroup $K$, non-trivial and $K \subseteq A$. But $|A|=3$, a prime order subgroup, hence has no non-trivial subgroup, so $K=A$. So $A$ is normal subgroup. Now since $G$ is not cyclic any non-identity element is of order 3.So Let $a(\neq$ $e) \in G$.Consider $A=\langle a\rangle$. As shown before $A$ is normal. $a$ commutes with any if its powers. Now Let $b \in G$ such that $b \notin A$. Then $b a b^{-1} \in A$ and hence $b a b^{-1}=a^i$.This implies $a=b^3 a b^{-3}=a^{i^3} \Longrightarrow a^{i^3-1}=e$. So, 3 divides $i^3-1$. Also by fermat's little theorem 3 divides $i^2-1$.So 3 divides $i-1$. But $0 \leq i \leq 2$. So $i=1$, is the only possibility and hence $a b=b a$. So $a \in Z(G)$ as $b$ was arbitrary. Since $a$ was arbitrary $G=Z(G)$. Hence $G$ is abelian.
\end{proof}


\paragraph{Exercise 2.5.44} Prove that a group of order $p^2$, $p$ a prime, has a normal subgroup of order $p$.

\paragraph{Exercise 2.5.52} Let $G$ be a finite group and $\varphi$ an automorphism of $G$ such that $\varphi(x) = x^{-1}$ for more than three-fourths of the elements of $G$. Prove that $\varphi(y) = y^{-1}$ for all $y \in G$, and so $G$ is abelian.

\paragraph{Exercise 2.6.15} If $G$ is an abelian group and if $G$ has an element of order $m$ and one of order $n$, where $m$ and $n$ are relatively prime, prove that $G$ has an element of order $mn$.

\paragraph{Exercise 2.7.3} Let $G$ be the group of nonzero real numbers under multiplication and let $N = \{1, -1\}$. Prove that $G / N \simeq$ positive real numbers under multiplication.

\paragraph{Exercise 2.7.7} If $\varphi$ is a homomorphism of $G$ onto $G'$ and $N \triangleleft G$, show that $\varphi(N) \triangleleft G'$.

\paragraph{Exercise 2.8.7} If $G$ is a group with subgroups $A, B$ of orders $m, n$, respectively, where $m$ and $n$ are relatively prime, prove that the subset of $G$, $AB = \{ab \mid a \in A, b \in B\}$, has $mn$ distinct elements.

\paragraph{Exercise 2.8.12} Prove that any two nonabelian groups of order 21 are isomorphic.

\paragraph{Exercise 2.8.15} Prove that if $p > q$ are two primes such that $q \mid p - 1$, then any two nonabelian groups of order $pq$ are isomorphic.

\paragraph{Exercise 2.9.2} If $G_1$ and $G_2$ are cyclic groups of orders $m$ and $n$, respectively, prove that $G_1 \times G_2$ is cyclic if and only if $m$ and $n$ are relatively prime.

\paragraph{Exercise 2.10.1} Let $A$ be a normal subgroup of a group $G$, and suppose that $b \in G$ is an element of prime order $p$, and that $b \not\in A$. Show that $A \cap (b) = (e)$.

\paragraph{Exercise 2.11.6} If $P$ is a $p$-Sylow subgroup of $G$ and $P \triangleleft G$, prove that $P$ is the only $p$-Sylow subgroup of $G$.

\paragraph{Exercise 2.11.7} If $P \triangleleft G$, $P$ a $p$-Sylow subgroup of $G$, prove that $\varphi(P) = P$ for every automorphism $\varphi$ of $G$.

\paragraph{Exercise 2.11.22} Show that any subgroup of order $p^{n-1}$ in a group $G$ of order $p^n$ is normal in $G$.

\paragraph{Exercise 3.2.21} If $\sigma, \tau$ are two permutations that disturb no common element and $\sigma \tau = e$, prove that $\sigma = \tau = e$.

\paragraph{Exercise 3.2.23} Let $\sigma, \tau$ be two permutations such that they both have decompositions into disjoint cycles of cycles of lengths $m_1, m_2, \ldots, m_k$. Prove that for some permutation $\beta, \tau = \beta \sigma \beta^{-1}$.

\paragraph{Exercise 3.3.2} If $\sigma$ is a $k$-cycle, show that $\sigma$ is an odd permutation if $k$ is even, and is an even permutation if $k$ is odd.

\paragraph{Exercise 3.3.9} If $n \geq 5$ and $(e) \neq N \subset A_n$ is a normal subgroup of $A_n$, show that $N$ must contain a 3-cycle.

\paragraph{Exercise 4.1.19} Show that there is an infinite number of solutions to $x^2 = -1$ in the quaternions.

\paragraph{Exercise 4.1.34} Let $T$ be the group of $2\times 2$ matrices $A$ with entries in the field $\mathbb{Z}_2$ such that $\det A$ is not equal to 0. Prove that $T$ is isomorphic to $S_3$, the symmetric group of degree 3.

\paragraph{Exercise 4.2.5} Let $R$ be a ring in which $x^3 = x$ for every $x \in R$. Prove that $R$ is commutative.

\paragraph{Exercise 4.2.6} If $a^2 = 0$ in $R$, show that $ax + xa$ commutes with $a$.

\paragraph{Exercise 4.2.9} Let $p$ be an odd prime and let $1 + \frac{1}{2} + ... + \frac{1}{p - 1} = \frac{a}{b}$, where $a, b$ are integers. Show that $p \mid a$.

\paragraph{Exercise 4.3.1} If $R$ is a commutative ring and $a \in R$, let $L(a) = \{x \in R \mid xa = 0\}$. Prove that $L(a)$ is an ideal of $R$.

\paragraph{Exercise 4.3.25} Let $R$ be the ring of $2 \times 2$ matrices over the real numbers; suppose that $I$ is an ideal of $R$. Show that $I = (0)$ or $I = R$.

\paragraph{Exercise 4.4.9} Show that $(p - 1)/2$ of the numbers $1, 2, \ldots, p - 1$ are quadratic residues and $(p - 1)/2$ are quadratic nonresidues $\mod p$.

\paragraph{Exercise 4.5.12} If $F \subset K$ are two fields and $f(x), g(x) \in F[x]$ are relatively prime in $F[x]$, show that they are relatively prime in $K[x]$.

\paragraph{Exercise 4.5.16} Let $F = \mathbb{Z}_p$ be the field of integers $\mod p$, where $p$ is a prime, and let $q(x) \in F[x]$ be irreducible of degree $n$. Show that $F[x]/(q(x))$ is a field having at exactly $p^n$ elements.

\paragraph{Exercise 4.5.23} Let $F = \mathbb{Z}_7$ and let $p(x) = x^3 - 2$ and $q(x) = x^3 + 2$ be in $F[x]$. Show that $p(x)$ and $q(x)$ are irreducible in $F[x]$ and that the fields $F[x]/(p(x))$ and $F[x]/(q(x))$ are isomorphic.

\paragraph{Exercise 4.5.25} If $p$ is a prime, show that $q(x) = 1 + x + x^2 + \cdots x^{p - 1}$ is irreducible in $Q[x]$.

\paragraph{Exercise 4.6.2} Prove that $f(x) = x^3 + 3x + 2$ is irreducible in $Q[x]$.

\paragraph{Exercise 4.6.3} Show that there is an infinite number of integers a such that $f(x) = x^7 + 15x^2 - 30x + a$ is irreducible in $Q[x]$.

\paragraph{Exercise 5.1.8} If $F$ is a field of characteristic $p \neq 0$, show that $(a + b)^m = a^m + b^m$, where $m = p^n$, for all $a, b \in F$ and any positive integer $n$.

\paragraph{Exercise 5.2.20} Let $V$ be a vector space over an infinite field $F$. Show that $V$ cannot be the set-theoretic union of a finite number of proper subspaces of $V$.

\paragraph{Exercise 5.3.7} If $a \in K$ is such that $a^2$ is algebraic over the subfield $F$ of $K$, show that a is algebraic over $F$.

\paragraph{Exercise 5.3.10} Prove that $\cos 1^{\circ}$  is algebraic over $\mathbb{Q}$.

\paragraph{Exercise 5.4.3} If $a \in C$ is such that $p(a) = 0$, where $p(x) = x^5 + \sqrt{2}x^3 + \sqrt{5}x^2 + \sqrt{7}x + \sqrt{11}$, show that $a$ is algebraic over $\mathbb{Q}$ of degree at most 80.

\paragraph{Exercise 5.5.2} Prove that $x^3 - 3x - 1$ is irreducible over $\mathbb{Q}$.

\paragraph{Exercise 5.6.3} Let $\mathbb{Q}$ be the rational field and let $p(x) = x^4 + x^3 + x^2 + x + 1$.  Show that there is an extension $K$ of $Q$ with $[K:Q] = 4$ over which $p(x)$ splits into linear factors.

\paragraph{Exercise 5.6.14} If $F$ is of characteristic $p \neq 0$, show that all the roots of $x^m - x$, where $m = p^n$, are distinct.

\end{document}
