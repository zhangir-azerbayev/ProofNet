\documentclass{article}

\title{ProofNet NL Statements}
\author{Zhangir Azerbayev}
\date{Summer 2022}

\usepackage{amsmath}
\usepackage{amssymb}
\usepackage{mathrsfs}
\usepackage{parskip}
\usepackage{fullpage}

\begin{document}
\maketitle
\section*{df}
\subsection*{chapter 1}
\subsubsection*{section 1}
1a: Prove that the the operation $\star$ on $\mathbb{Z}$ defined by $a \star b=a-b$ is not associative. 

2a: Prove the the operation $\star$ on $\mathbb{Z}$ defined by $a\star b=a-b$ is not commutative. 

3: Prove that the addition of residue classes $\mathbb{Z}/n\mathbb{Z}$ is associative. 

4: Prove that the multiplication of residue class $\mathbb{Z}/n\mathbb{Z}$ is associative. 

5: Prove that for all $n>1$ that $\mathbb{Z}/n\mathbb{Z}$ is not a group under multiplication of residue classes. 

15: Prove that $(a_1a_2\dots a_n)^{-1} = a_n^{-1}a_{n-1}^{-1}\dots a_1^{-1}$ for all $a_1, a_2, \dots, a_n\in G$. 

16: Let $x$ be an element of $G$. Prove that $x^2=1$ if and only if $|x|$ is either $1$ or $2$. 

17: Let $x$ be an element of $G$. Prove that if $|x|=n$ for some positive integer $n$ then $x^{-1}=x^{n-1}$. 

18: Let $x$ and $y$ be elements of $G$. Prove that $xy=yx$ if and only if $y^{-1}xy=x$ if and only if $x^{-1}y^{-1}xy=1$.

20: For $x$ an element in $G$ show that $x$ and $x^{-1}$ have the same order.

22a: If $x$ and $g$ are elements of the group $G$, prove that $|x|=\left|g^{-1} x g\right|$. 

22b: Deduce that $|a b|=|b a|$ for all $a, b \in G$.

25: Prove that if $x^{2}=1$ for all $x \in G$ then $G$ is abelian.

29: Prove that $A \times B$ is an abelian group if and only if both $A$ and $B$ are abelian.

34: If $x$ is an element of infinite order in $G$, prove that the elements $x^{n}, n \in \mathbb{Z}$ are all distinct.

\subsubsection*{section 3}
8: Prove that if $\Omega=\{1,2,3, \ldots\}$ then $S_{\Omega}$ is an infinite group

\subsubsection*{section 6}
4: Prove that the multiplicative groups $\mathbb{R}-\{0\}$ and $\mathbb{C}-\{0\}$ are not isomorphic.

11: Let $A$ and $B$ be groups. Prove that $A \times B \cong B \times A$.

17: Let $G$ be any group. Prove that the map from $G$ to itself defined by $g \mapsto g^{-1}$ is a homomorphism if and only if $G$ is abelian.

23: Let $G$ be a finite group which possesses an automorphism $\sigma$ such that $\sigma(g)=g$ if and only if $g=1$. If $\sigma^{2}$ is the identity map from $G$ to $G$, prove that $G$ is abelian. 

\subsubsection*{section 7}
5: Prove that the kernel of an action of the group $G$ on a set $A$ is the same as the kernel of the corresponding permutation representation $G\to S_A$. 

6: Prove that a group $G$ acts faithfully on a set $A$ if and only if the kernel of the action is the set consisting only of the identity. 

\subsection*{chapter 2}
\subsubsection*{section 1}
5: Prove that $G$ cannot have a subgroup $H$ with $|H|=n-1$, where $n=|G|>2$.

13: Let $H$ be a subgroup of the additive group of rational numbers with the property that $1 / x \in H$ for every nonzero element $x$ of $H$. Prove that $H=0$ or $\mathbb{Q}$.

\subsubsection*{section 2}
4: Prove that if $H$ is a subgroup of $G$ then $H$ is generated by the set $H-\{1\}$.

13: Prove that the multiplicative group of positive rational numbers is generated by the set $\left\{\frac{1}{p} \mid \text{$p$ is a prime} \right\}$. 

16a: A subgroup $M$ of a group $G$ is called a maximal subgroup if $M \neq G$ and the only subgroups of $G$ which contain $M$ are $M$ and $G$. Prove that if $H$ is a proper subgroup of the finite group $G$ then there is a maximal subgroup of $G$ containing $H$.

16c: Show that if $G=\langle x\rangle$ is a cyclic group of order $n \geq 1$ then a subgroup $H$ is maximal if and only $H=\left\langle x^{p}\right\rangle$ for some prime $p$ dividing $n$.

\subsection*{Chapter 3}
\subsubsection*{section 1}
3a: Let $A$ be an abelian group and let $B$ be a subgroup of $A$. Prove that $A / B$ is abelian.

22a: Prove that if $H$ and $K$ are normal subgroups of a group $G$ then their intersection $H \cap K$ is also a normal subgroup of $G$.

22b: Prove that the intersection of an arbitrary nonempty collection of normal subgroups of a group is a normal subgroup (do not assume the collection is countable).

\subsubsection*{section 2}
8: Prove that if $H$ and $K$ are finite subgroups of $G$ whose orders are relatively prime then $H \cap K=1$.

11: Let $H \leq K \leq G$. Prove that $|G: H|=|G: K| \cdot|K: H|$ (do not assume $G$ is finite).

16: Use Lagrange's Theorem in the multiplicative group $(\mathbb{Z} / p \mathbb{Z})^{\times}$to prove Fermat's Little Theorem: if $p$ is a prime then $a^{p} \equiv a(\bmod p)$ for all $a \in \mathbb{Z}$.

21a: Prove that $\mathbb{Q}$ has no proper subgroups of finite index.

\subsubsection*{section 3}
3: Prove that if $H$ is a normal subgroup of $G$ of prime index $p$ then for all $K \leq G$ either $K \leq H$ or $G=H K$ and $|K: K \cap H|=p$.

\subsubsection*{section 4}
1: Prove that if $G$ is an abelian simple group then $G \cong Z_{p}$ for some prime $p$ (do not assume $G$ is a finite group).

4: Use Cauchy's Theorem and induction to show that a finite abelian group has a subgroup of order $n$ for each positive divisor $n$ of its order. 

5a: Prove that subgroups of a solvable group are solvable.

5b: Prove that quotient groups of a solvable group are solvable.

11: Prove that if $H$ is a nontrivial normal subgroup of the solvable group $G$ then there is a nontrivial subgroup $A$ of $H$ with $A \unlhd G$ and $A$ abelian.

\subsection*{Chapter 4}
\subsubsection*{section 2}
8: Prove that if $H$ has finite index $n$ then there is a normal subgroup $K$ of $G$ with $K \leq H$ and $|G: K| \leq n$ !.

9a: Prove that if $p$ is a prime and $G$ is a group of order $p^{\alpha}$ for some $\alpha \in \mathbb{Z}^{+}$, then every subgroup of index $p$ is normal in $G$. 

14: Let $G$ be a finite group of composite order $n$ with the property that $G$ has a subgroup of order $k$ for each positive integer $k$ dividing $n$. Prove that $G$ is not simple.

\subsubsection*{section 3}
5: If the center of $G$ is of index $n$, prove that every conjugacy class has at most $n$ elements.

26: Let $G$ be a transitive permutation group on the finite set $A$ with $|A|>1$. Show that there is some $\sigma \in G$ such that $\sigma(a) \neq a$ for all $a \in A$ (such an element $\sigma$ is called fixed point free ).

27: Let $g_{1}, g_{2}, \ldots, g_{r}$ be representatives of the conjugacy classes of the finite group $G$ and assume these elements pairwise commute. Prove that $G$ is abelian.


\subsubsection*{section 4}
2: Prove that if $G$ is a n abelian group of order $p q$, where $p$ and $q$ are distinct primes, then $G$ is cyclic.

6a: Prove that characteristic subgroups are normal.

7: If $H$ is the unique subgroup of a given order in a group $G$ prove $H$ is characteristic in $G$.

8a: Let $G$ be a group with subgroups $H$ and $K$ with $H \leq K$. Prove that if $H$ is characteristic in $K$ and $K$ is normal in $G$ then $H$ is normal in $G$.

\subsubsection*{section 5}
1: Prove that if $P \in \mathrm{Syl}_{p}(G)$ and $H$ is a subgroup of $G$ containing $P$ then $P \in\mathrm{Syl}_{p}(H)$.

13: Prove that a group of order 56 has a normal Sylow $p$-subgroup for some prime $p$ dividing its order.

14: Prove that a group of order 312 has a normal Sylow $p$-subgroup for some prime $p$ dividing its order.

15: Prove that a group of order 351 has a normal Sylow $p$-subgroup for some prime $p$ dividing its order.

16: Let $|G|=p q r$, where $p, q$ and $r$ are primes with $p<q<r$. Prove that $G$ has a normal Sylow subgroup for either $p, q$ or $r$.

17: Prove that if $|G|=105$ then $G$ has a normal Sylow 5 -subgroup and a normal Sylow 7-subgroup.

18: Prove that a group of order 200 has a normal Sylow 5-subgroup.

19: Prove that if $|G|=6545$ then $G$ is not simple.

20: Prove that if $|G|=1365$ then $G$ is not simple.

21: Prove that if $|G|=2907$ then $G$ is not simple.

22: Prove that if $|G|=132$ then $G$ is not simple.

23: Prove that if $|G|=462$ then $G$ is not simple.

28: Let $G$ be a group of order 105. Prove that if a Sylow 3-subgroup of $G$ is normal then $G$ is abelian.

33: Let $P$ be a normal Sylow $p$-subgroup of $G$ and let $H$ be any subgroup of $G$. Prove that $P \cap H$ is the unique Sylow $p$-subgroup of $H$.

\subsection*{Chapter 5}
\subsubsection*{section 4}
2: Prove that a subgroup $H$ of $G$ is normal if and only if $[G, H] \leq H$.

\subsection*{Chapter 7}
\subsubsection{section 1}
2: Prove that if $u$ is a unit in $R$ then so is $-u$.

11: Prove that if $R$ is an integral domain and $x^{2}=1$ for some $x \in R$ then $x=\pm 1$.

12: Prove that any subring of a field which contains the identity is an integral domain.

15: A ring $R$ is called a Boolean ring if $a^{2}=a$ for all $a \in R$. Prove that every Boolean ring is commutative.

\subsubsection*{section 2}
2: Let $p(x)=a_{n} x^{n}+a_{n-1} x^{n-1}+\cdots+a_{1} x+a_{0}$ be an element of the polynomial ring $R[x]$. Prove that $p(x)$ is a zero divisor in $R[x]$ if and only if there is a nonzero $b \in R$ such that $b p(x)=0$. 

4: Prove that if $R$ is an integral domain then the ring of formal power series $R[[x]]$ is also an integral domain.

12: Let $G=\left\{g_{1}, \ldots, g_{n}\right\}$ be a finite group. Prove that the element $N=g_{1}+g_{2}+\ldots+g_{n}$ is in the center of the group ring $R G$.

\subsubsection*{section 3}
16: Let $\varphi: R \rightarrow S$ be a surjective homomorphism of rings. Prove that the image of the center of $R$ is contained in the center of $S$. 

28: Prove that an integral domain has characteristic $p$, where $p$ is either a prime or 0

37: An ideal $N$ is called nilpotent if $N^{n}$ is the zero ideal for some $n \geq 1$. Prove that the ideal $p \mathbb{Z} / p^{m} \mathbb{Z}$ is a nilpotent ideal in the ring $\mathbb{Z} / p^{m} \mathbb{Z}$.

\subsubsection*{section 4}
27: Let $R$ be a commutative ring with $1 \neq 0$. Prove that if $a$ is a nilpotent element of $R$ then $1-a b$ is a unit for all $b \in R$.

\section*{Chapter 8}
\subsubsection*{section 1}
12: Let $N$ be a positive integer. Let $M$ be an integer relatively prime to $N$ and let $d$ be an integer relatively prime to $\varphi(N)$, where $\varphi$ denotes Euler's $\varphi$-function. Prove that if $M_{1} \equiv M^{d}(\bmod N)$ then $M \equiv M_{1}^{d^{\prime}}(\bmod N)$ where $d^{\prime}$ is the inverse of $d$ $\bmod \varphi(N): d d^{\prime} \equiv 1(\bmod \varphi(N))$

\subsubsection*{section 2}
4: Let $R$ be an integral domain. Prove that if the following two conditions hold then $R$ is a Principal Ideal Domain: (i) any two nonzero elements $a$ and $b$ in $R$ have a greatest common divisor which can be written in the form $r a+s b$ for some $r, s \in R$, and (ii) if $a_{1}, a_{2}, a_{3}, \ldots$ are nonzero elements of $R$ such that $a_{i+1} \mid a_{i}$ for all $i$, then there is a positive integer $N$ such that $a_{n}$ is a unit times $a_{N}$ for all $n \geq N$.

\subsubsection*{section 3}
4: Prove that if an integer is the sum of two rational squares, then it is the sum of two integer squares (for example, $13=(1 / 5)^{2}+(18 / 5)^{2}=2^{2}+3^{2}$ ).

5a: Let $R=\mathbb{Z}[\sqrt{-n}]$ where $n$ is a squarefree integer greater than 3. Prove that $2, \sqrt{-n}$ and $1+\sqrt{-n}$ are irreducibles in $R$.

6a: Prove that the quotient ring $\mathbb{Z}[i] /(1+i)$ is a field of order 2 .

6b: Let $q \in \mathbb{Z}$ be a prime with $q \equiv 3 \bmod 4$. Prove that the quotient ring $\mathbb{Z}[i] /(q)$ is a field with $q^{2}$ elements.

\subsection*{chapter 9}
\subsubsection*{section 1}
6: Prove that $(x, y)$ is not a principal ideal in $\mathbb{Q}[x, y]$.

10: Prove that the ring $\mathbb{Z}\left[x_{1}, x_{2}, x_{3}, \ldots\right] /\left(x_{1} x_{2}, x_{3} x_{4}, x_{5} x_{6}, \ldots\right)$ contains infinitely many minimal prime ideals (cf. Exercise 36 of Section 7.4).

\subsubsection*{section 3}
2: Prove that if $f(x)$ and $g(x)$ are polynomials with rational coefficients whose product $f(x) g(x)$ has integer coefficients, then the product of any coefficient of $g(x)$ with any coefficient of $f(x)$ is an integer.

\subsubsection*{section 4}
2a: Prove that $x^4-4x^3+6$ is irreducible in $\mathbb{Z}[x]$. 

2b: Prove that $x^6+30x^5-15x^+6x-120$ is irreducible in $\mathbb{Z}[x]$. 

2c: Prove that $x^4+4x^3+6x^2+2x+1$ is irreducible in $\mathbb{Z}[x]$. 

2d: Prove that $\frac{(x+2)^p-2^p}{x}$, where $p$ is an odd prime, is irreducible in $\mathbb{Z}[x]$. 

9: Prove that the polynomial $x^{2}-\sqrt{2}$ is irreducible over $\mathbb{Z}[\sqrt{2}]$. You may assume that $\mathbb{Z}[\sqrt{2}]$ is a U.F.D. 

11: Prove that $x^2+y^2-1$ is irreducible in $\mathbb{Q}[x,y]$. 
\section*{Rudin}
\subsection*{chapter 1}
1: If $r$ is rational $(r \neq 0)$ and $x$ is irrational, prove that $r+x$ and $r x$ are irrational.

2: Prove that there is no rational number whose square is $12$. 

4: Let $E$ be a nonempty subset of an ordered set; suppose $\alpha$ is a lower bound of $E$ and $\beta$ is an upper bound of $E$. Prove that $\alpha \leq \beta$.

5: Let $A$ be a nonempty set of real numbers which is bounded below. Let $-A$ be the set of all numbers $-x$, where $x \in A$. Prove that $\inf A=-\sup (-A)$

8: Prove that no order can be defined in the complex field that turns it into an ordered field. Hint: $-1$ is a square.

14: If $z$ is a complex number such that $|z|=1$, that is, such that $z \bar{z}=1$, compute $|1+z|^{2}+|1-z|^{2}$. 

17: Prove that
$|\mathbf{x}+\mathbf{y}|^{2}+|\mathbf{x}-\mathbf{y}|^{2}=2|\mathbf{x}|^{2}+2|\mathbf{y}|^{2}$ if $\mathbf{x} \in R^{k}$ and $\mathbf{y} \in R^{k}$.

18a: If $k \geq 2$ and $\mathbf{x} \in R^{k}$, prove that there exists $\mathbf{y} \in R^{k}$ such that $\mathbf{y} \neq 0$ but $\mathbf{x} \cdot \mathbf{y}=0$

25: Prove that every compact metric space $K$ has a countable base. 

27a: Suppose $E\subset\mathbb{R}^k$ is uncountable, and let $P$ be the set of condensation points of $E$. Prove that $P$ is perfect. 

27b: Suppose $E\subset\mathbb{R}^k$ is uncountable, and let $P$ be the set of condensation points of $E$. Prove that at most countably many point of $E$ are not in $P$. 

28: Prove that every closed set in a separable metric space is the union of a (possibly empty) perfect set and a set which is at most countable. 

29: Prove that every open set in $\mathbb{R}$ is the union of an at most countable collection of disjoint segments. 

\subsection*{chapter 2}
19a: If $A$ and $B$ are disjoint closed sets in some metric space $X$, prove that they are separated.

24: Let $X$ be a metric space in which every infinite subset has a limit point. Prove that $X$ is separable. Hint: Fix $\delta>0$, and pick $x_{1} \in X$. Having chosen $x_{1}, \ldots, x_{J} \in X$,

\subsection*{chapter 3}
1a: Prove that convergence of $\left\{s_{n}\right\}$ implies convergence of $\left\{\left|s_{n}\right|\right\}$. 

3: If $s_{1}=\sqrt{2}$, and $s_{n+1}=\sqrt{2+\sqrt{s_{n}}} \quad(n=1,2,3, \ldots),$ prove that $\left\{s_{n}\right\}$ converges, and that $s_{n}<2$ for $n=1,2,3, \ldots$.

5: For any two real sequences $\left\{a_{n}\right\},\left\{b_{n}\right\}$, prove that $\limsup _{n \rightarrow \infty}\left(a_{n}+b_{n}\right) \leq \limsup _{n \rightarrow \infty} a_{n}+\limsup _{n \rightarrow \infty} b_{n},$ provided the sum on the right is not of the form $\infty-\infty$.

7: Prove that the convergence of $\Sigma a_{n}$ implies the convergence of $\sum \frac{\sqrt{a_{n}}}{n}$ if $a_n\geq 0$.

8: If $\Sigma a_{n}$ converges, and if $\left\{b_{n}\right\}$ is monotonic and bounded, prove that $\Sigma a_{n} b_{n}$ converges.

13: Prove that the Cauchy product of two absolutely convergent series converges absolutely.

20: 20. Suppose $\left\{p_{n}\right\}$ is a Cauchy sequence in a metric space $X$, and some subsequence $\left\{p_{n l}\right\}$ converges to a point $p \in X$. Prove that the full sequence $\left\{p_{n}\right\}$ converges to $p$.

21: If $\left\{E_{n}\right\}$ is a sequence of closed nonempty and bounded sets in a complete metric space $X$, if $E_{n} \supset E_{n+1}$, and if $\lim _{n \rightarrow \infty} \operatorname{diam} E_{n}=0,$ then $\bigcap_{1}^{\infty} E_{n}$ consists of exactly one point.

22: Suppose $X$ is a nonempty complete metric space, and $\left\{G_{n}\right\}$ is a sequence of dense open subsets of $X$. Prove Baire's theorem, namely, that $\bigcap_{1}^{\infty} G_{n}$ is not empty. Hint: Find a shrinking sequence of neighborhoods $E_{n}$ such that $E_{n} \subset G_{n}$. 

\subsection*{chapter 4}
2a: If $f$ is a continuous mapping of a metric space $X$ into a metric space $Y$, prove that $f(\overline{E}) \subset \overline{f(E)}$ for every set $E \subset X$. ($\overline{E}$ denotes the closure of $E$). 

3: Let $f$ be a continuous real function on a metric space $X$. Let $Z(f)$ (the zero set of $f$ ) be the set of all $p \in X$ at which $f(p)=0$. Prove that $Z(f)$ is closed.

4a: Let $f$ and $g$ be continuous mappings of a metric space $X$ into a metric space $Y$, and let $E$ be a dense subset of $X$. Prove that $f(E)$ is dense in $f(X)$. 

5a: If $f$ is a real continuous function defined on a closed set $E \subset R^{1}$, prove that there exist continuous real functions $g$ on $R^{1}$ such that $g(x)=f(x)$ for all $x \in E$.

6: If $f$ is defined on $E$, the graph of $f$ is the set of points $(x, f(x))$, for $x \in E$. In particular, if $E$ is a set of real numbers, and $f$ is real-valued, the graph of $f$ is a subset of the plane. Suppose $E$ is compact, and prove that $f$ is continuous on $E$ if and only if its graph is compact.

8a: Let $f$ be a real uniformly continuous function on the bounded set $E$ in $R^{1}$. Prove that $f$ is bounded on $E$.

11a: Suppose $f$ is a uniformly continuous mapping of a metric space $X$ into a metric space $Y$ and prove that $\left\{f\left(x_{n}\right)\right\}$ is a Cauchy sequence in 
12: A uniformly continuous function of a uniformly continuous function is uniformly continuous.

14: Let $I=[0,1]$ be the closed unit interval. Suppose $f$ is a continuous mapping of $I$ into $I$. Prove that $f(x)=x$ for at least one $x \in I$.

15: Prove that every continuous open mapping of $R^{1}$ into $R^{1}$ is monotonic.

19: Suppose $f$ is a real function with domain $R^{1}$ which has the intermediate value property: If $f(a)<c<f(b)$, then $f(x)=c$ for some $x$ between $a$ and $b$. Suppose also, for every rational $r$, that the set of all $x$ with $f(x)=r$ is closed. Prove that $f$ is continuous.

21a: Suppose $K$ and $F$ are disjoint sets in a metric space $X, K$ is compact, $F$ is closed. Prove that there exists $\delta>0$ such that $d(p, q)>\delta$ if $p \in K, q \in F$. 

24: Assume that $f$ is a continuous real function defined in $(a, b)$ such that $f\left(\frac{x+y}{2}\right) \leq \frac{f(x)+f(y)}{2}$ for all $x, y \in(a, b)$. Prove that $f$ is convex.

26a: Suppose $X, Y, Z$ are metric spaces, and $Y$ is compact. Let $f$ map $X$ into $Y$, let $g$ be a continuous one-to-one mapping of $Y$ into $Z$, and put $h(x)=g(f(x))$ for $x \in X$. Prove that $f$ is uniformly continuous if $h$ is uniformly continuous.

\subsection*{chapter 5}
1: Let $f$ be defined for all real $x$, and suppose that $|f(x)-f(y)| \leq(x-y)^{2}$for all real $x$ and $y$. Prove that $f$ is constant.

2: Suppose $f^{\prime}(x)>0$ in $(a, b)$. Prove that $f$ is strictly increasing in $(a, b)$, and let $g$ be its inverse function. Prove that $g$ is differentiable, and that$g^{\prime}(f(x))=\frac{1}{f^{\prime}(x)} \quad(a<x<b)$

3: Suppose $g$ is a real function on $R^{1}$, with bounded derivative (say $\left|g^{\prime}\right| \leq M$ ). Fix $\varepsilon>0$, and define $f(x)=x+\varepsilon g(x)$. Prove that $f$ is one-to-one if $\varepsilon$ is small enough.

4: If $C_{0}+\frac{C_{1}}{2}+\cdots+\frac{C_{n-1}}{n}+\frac{C_{n}}{n+1}=0,$ where $C_{0}, \ldots, C_{n}$ are real constants, prove that the equation $C_{0}+C_{1} x+\cdots+C_{n-1} x^{n-1}+C_{n} x^{n}=0$ has at least one real root between 0 and 1 .

5: Suppose $f$ is defined and differentiable for every $x>0$, and $f^{\prime}(x) \rightarrow 0$ as $x \rightarrow+\infty$. Put $g(x)=f(x+1)-f(x)$. Prove that $g(x) \rightarrow 0$ as $x \rightarrow+\infty$.

6: Suppose (a) $f$ is continuous for $x \geq 0$, (b) $f^{\prime}(x)$ exists for $x>0$, (c) $f(0)=0$, (d) $f^{\prime}$ is monotonically increasing. Put $g(x)=\frac{f(x)}{x} \quad(x>0)$ and prove that $g$ is monotonically increasing.

7: Suppose $f^{\prime}(x), g^{\prime}(x)$ exist, $g^{\prime}(x) \neq 0$, and $f(x)=g(x)=0$. Prove that $\lim _{t \rightarrow x} \frac{f(t)}{g(t)}=\frac{f^{\prime}(x)}{g^{\prime}(x)}.$

15: Suppose $a \in R^{1}, f$ is a twice-differentiable real function on $(a, \infty)$, and $M_{0}, M_{1}, M_{2}$ are the least upper bounds of $|f(x)|,\left|f^{\prime}(x)\right|,\left|f^{\prime \prime}(x)\right|$, respectively, on $(a, \infty)$. Prove that $M_{1}^{2} \leq 4 M_{0} M_{2} .$

16: Suppose $f$ is a real, three times differentiable function on $[-1,1]$, such that
$f(-1)=0, \quad f(0)=0, \quad f(1)=1, \quad f^{\prime}(0)=0 .$ Prove that $f^{(3)}(x) \geq 3$ for some $x \in(-1,1)$.

\subsection*{chapter 5}
1: Suppose $\alpha$ increases on $[a, b], a \leq x_{0} \leq b, \alpha$ is continuous at $x_{0}, f\left(x_{0}\right)=1$, and $f(x)=0$ if $x \neq x_{0}$. Prove that $f \in \mathscr{R}(\alpha)$ and that $\int f d \alpha=0$.

2: Suppose $f \geq 0, f$ is continuous on $[a, b]$, and $\int_{a}^{b} f(x) d x=0$. Prove that $f(x)=0$ for all $x \in[a, b]$.

4: If $f(x)=0$ for all irrational $x, f(x)=1$ for all rational $x$, prove that $f \notin \mathscr{R}$ on $[a, b]$ for any $a<b$.

6: Let $P$ be the Cantor set. Let $f$ be a bounded real function on $[0,1]$ which is continuous at every point outside $P$. Prove that $f \in \mathscr{R}$ on $[0,1]$.

\section*{Munkres}
\subsection*{chapter 2}
\subsubsection*{section 13}
5a: Show that if $\mathcal{A}$ is a basis for a topology on $X$, then the topology generated by $\mathcal{A}$ equals the intersection of all topologies on $X$ that contain $\mathcal{A}$. 

\subsubsection*{section 16}
4: A map $f: X \rightarrow Y$ is said to be an open map if for every open set $U$ of $X$, the set $f(U)$ is open in $Y$. Show that $\pi_{1}: X \times Y \rightarrow X$ and $\pi_{2}: X \times Y \rightarrow Y$ are open maps.

\subsubsection*{section 17}
2: Show that if $A$ is closed in $Y$ and $Y$ is closed in $X$, then $A$ is closed in $X$.

3: Show that if $A$ is closed in $X$ and $B$ is closed in $Y$, then $A \times B$ is closed in $X \times Y$.

4: Show that if $U$ is open in $X$ and $A$ is closed in $X$, then $U-A$ is open in $X$, and $A-U$ is closed in $X$.

\subsubsection*{18}
8a: Let $Y$ be an ordered set in the order topology. Let $f, g: X \rightarrow Y$ be continuous. Show that the set $\{x \mid f(x) \leq g(x)\}$ is closed in $X$.

8b: Let $Y$ be an ordered set in the order topology. Let $f, g: X \rightarrow Y$ be continuous. Let $h: X \rightarrow Y$ be the function $h(x)=\min \{f(x), g(x)\}.$ Show that $h$ is continuous. [Hint: Use the pasting lemma.] 

13: Let $A \subset X$; let $f: A \rightarrow Y$ be continuous; let $Y$ be Hausdorff. Show that if $f$ may be extended to a continuous function $g: \bar{A} \rightarrow Y$, then $g$ is uniquely determined by $f$.

\subsubsection*{21}

6a: Define $f_{n}:[0,1] \rightarrow \mathbb{R}$ by the equation $f_{n}(x)=x^{n}$. Show that the sequence $\left(f_{n}(x)\right)$ converges for each $x \in[0,1]$. 

6b: Define $f_{n}:[0,1] \rightarrow \mathbb{R}$ by the equation $f_{n}(x)=x^{n}$. Show that the sequence $\left(f_{n}\right)$ does not converge uniformly.

8: Let $X$ be a topological space and let $Y$ be a metric space. Let $f_{n}: X \rightarrow Y$ be a sequence of continuous functions. Let $x_{n}$ be a sequence of points of $X$ converging to $x$. Show that if the sequence $\left(f_{n}\right)$ converges uniformly to $f$, then $\left(f_{n}\left(x_{n}\right)\right)$ converges to $f(x)$.

\subsubsection*{22}
1: Let $p: X \rightarrow Y$ be a continuous map. Show that if there is a continuous map $f: Y \rightarrow X$ such that $p \circ f$ equals the identity map of $Y$, then $p$ is a quotient map.

2a: Let $p: X \rightarrow Y$ be a continuous map. Show that if there is a continuous map $f: Y \rightarrow X$ such that $p \circ f$ equals the identity map of $Y$, then $p$ is a quotient map.

2b: If $A \subset X$, a retraction of $X$ onto $A$ is a continuous map $r: X \rightarrow A$ such that $r(a)=a$ for each $a \in A$. Show that a retraction is a quotient map.

3: Let $H$ be a subspace of $G$. Show that if $H$ is also a subgroup of $G$, then both $H$ and $\bar{H}$ are topological groups.

\subsubsection*{23}
2: Let $\left\{A_{n}\right\}$ be a sequence of connected subspaces of $X$, such that $A_{n} \cap A_{n+1} \neq \varnothing$ for all $n$. Show that $\bigcup A_{n}$ is connected.

3: Let $\left\{A_{\alpha}\right\}$ be a collection of connected subspaces of $X$; let $A$ be a connectea eubsen of $X$ Show that if $A \cap A_{\alpha} \neq \varnothing$ for all $\alpha$, then $A \cup\left(\bigcup \mid A_{\alpha}\right)$ is connected.

4: Show that if $X$ is an infinite set, it is connected in the finite complement topology.

6: Let $A \subset X$. Show that if $C$ is a connected subspace of $X$ that intersects both $A$ and $X-A$, then $C$ intersects $\mathrm{Bd} A$.

9: Let $A$ be a proper subset of $X$, and let $B$ be a proper subset of $Y$. If $X$ and $Y$ are connected, show that $(X \times Y)-(A \times B)$ is connected.

11: Let $p: X \rightarrow Y$ be a quotient map. Show that if each set $p^{-1}(\{y\})$ is connected, and if $Y$ is connected, then $X$ is connected.

12: Let $Y \subset X$; let $X$ and $Y$ be connected. Show that if $A$ and $B$ form a separation of $X-Y$, then $Y \cup A$ and $Y \cup B$ are connected.

\subsubsection*{24}
2: Let $f: S^{1} \rightarrow \mathbb{R}$ be a continuous map. Show there exists a point $x$ of $S^{1}$ such that $f(x)=f(-x)$.

\subsubsection*{25}
9: Let $G$ be a topological group; let $C$ be the component of $G$ containing the identity element $e$. Show that $C$ is a normal subgroup of $G$.

\subsubsection*{26}
9: Theorem. Let $A$ and $B$ be subspaces of $X$ and $Y$, respectively; let $N$ be an open set in $X \times Y$ containing $A \times B$. If $A$ and $B$ are compact, then there exist open sets $U$ and $V$ in $X$ and $Y$, respectively, such that $A \times B \subset U \times V \subset N .$

11: Theorem. Let $X$ be a compact Hausdorff space. Let A be a collection of closed connected subsets of $X$ that is simply ordered by proper inclusion. Then $Y=\bigcap_{A \in \mathcal{A}} A$ is connected.

12: Let $p: X \rightarrow Y$ be a closed continuous surjective map such that $p^{-1}(\{y\})$ is compact, for each $y \in Y$. (Such a map is called a perfect map.) Show that if $Y$ is compact, then $X$ is compact.

\subsubsection*{27}
4: Show that a connected metric space having more than one point is uncountable.

\subsubsection*{28}
6: Let $(X, d)$ be a metric space. If $f: X \rightarrow X$ satisfies the condition $d(f(x), f(y))=d(x, y)$ for all $x, y \in X$, then $f$ is called an isometry of $X$. Show that if $f$ is an isometry and $X$ is compact, then $f$ is bijective and hence a homeomorphism. 

\subsubsection*{29}
10: Show that if $X$ is a Hausdorff space that is locally compact at the point $x$, then for each neighborhood $U$ of $x$, there is a neighborhood $V$ of $x$ such that $\bar{V}$ is compact and $\bar{V} \subset U$.

\section*{irelandrosen}
\subsection*{chapter 1}
27: For all odd $n$ show that $8 \mid n^{2}-1$.

30: Prove that $\frac{1}{2}+\frac{1}{3}+\cdots+\frac{1}{n}$ is not an integer.

31: Show that 2 is divisible by $(1+i)^{2}$ in $\mathbb{Z}[i]$.

\subsection*{chapter 2}
4: If $a$ is a nonzero integer, then for $n>m$ show that $\left(a^{2^{n}}+1, a^{2^{m}}+1\right)=1$ or 2 depending on whether $a$ is odd or even. 

21: Define $\wedge(n)=\log p$ if $n$ is a power of $p$ and zero otherwise. Prove that $\sum_{A \mid n} \mu(n / d) \log d$ $=\wedge(n)$.

27a: Show that $\sum^{\prime} 1 / n$, the sum being over square free integers, diverges.

\subsection*{chapter 3}
1: Show that there are infinitely many primes congruent to $-1$ modulo 6 .

4: Show that the equation $3 x^{2}+2=y^{2}$ has no solution in integers.

5: Show that the equation $7 x^{3}+2=y^{3}$ has no solution in integers.

10: If $n$ is not a prime, show that $(n-1) ! \equiv 0(n)$, except when $n=4$.

14: Let $p$ and $q$ be distinct odd primes such that $p-1$ divides $q-1$. If $(n, p q)=1$, show that $n^{q-1} \equiv 1(p q)$.

18: Let $N$ be the number of solutions to $f(x) \equiv 0(n)$ and $N_{i}$ be the number of solutions to $f(x) \equiv 0\left(p_{i}^{a_{i}}\right)$. Prove that $N=N_{1} N_{2} \cdots N_{i}$.

20: Show that $x^{2} \equiv 1\left(2^{b}\right)$ has one solution if $b=1$, two solutions if $b=2$, and four solutions if $b \geq 3$.

\subsection*{chapter 4}
4: Consider a prime $p$ of the form $4 t+1$. Show that $a$ is a primitive root modulo $p$ iff - $a$ is a primitive root modulo $p$.

5: Consider a prime $p$ of the form $4 t+3$. Show that $a$ is a primitive root modulo $p$ iff $-a$ has order $(p-1) / 2$.

6: If $p=2^{n}+1$ is a Fermat prime, show that 3 is a primitive root modulo $p$.

8: Let $p$ be an odd prime. Show that $a$ is a primitive root module $p$ iff $a^{(p-1) / q} \not \equiv 1(p)$ for all prime divisors $q$ of $p-1$.

9: Show that the product of all the primitive roots modulo $p$ is congruent to $(-1)^{\phi(p-1)}$ modulo $p$.

10: Show that the sum of all the primitive roots modulo $p$ is congruent to $\mu(p-1)$ modulo $p$.

11: Prove that $1^{k}+2^{k}+\cdots+(p-1)^{k} \equiv 0(p)$ if $p-1 \nmid k$ and $-1(p)$ if $p-1 \mid k$.

22: If $a$ has order 3 modulo $p$, show that $1+a$ has order 6 .

24: Show that $a x^{m}+b y^{n} \equiv c(p)$ has the same number of solutions as $a x^{m^{\prime}}+b y^{n^{\prime}} \equiv c(p)$, where $m^{\prime}=(m, p-1)$ and $n^{\prime}=(n, p-1)$.

\subsection*{chapter 5}
2: Show that the number of solutions to $x^{2} \equiv a(p)$ is given by $1+(a / p)$.

3: Suppose that $p \nmid a$. Show that the number of solutions to $a x^{2}+b x+c \equiv 0(p)$ is given by $1+\left(\left(b^{2}-4 a c\right) / p\right)$.

4: Prove that $\sum_{a=1}^{p-1}(a / p)=0$.

5: Prove that $\sum_{\substack{p-1 \\ x=0}}((a x+b) / p)=0$ provided that $p \nmid a .$

6: Show that the number of solutions to $x^{2}-y^{2} \equiv a(p)$ is given by $\sum_{y=0}^{p-1}\left(1+\left(\left(y^{2}+a\right) / p\right)\right) .$

7: By calculating directly show that the number of solutions to $x^{2}-y^{2} \equiv a(p)$ is $p-1$ if $p \nmid a$ and $2 p-1$ if $p \mid a$. 

13: Show that any prime divisor of $x^{4}-x^{2}+1$ is congruent to 1 modulo 12 .

27: Suppose that $f$ is such that $b \equiv a f(p)$. Show that $f^{2} \equiv-1(p)$ and that $2^{(p-1) / 4} \equiv$ $f^{a b / 2}(p)$

28: Show that $x^{4} \equiv 2(p)$ has a solution for $p \equiv 1(4)$ iff $p$ is of the form $A^{2}+64 B^{2}$.

37: Show that if $a$ is negative then $p \equiv q(4 a), p \times a$ implies $(a / p)=(a / q)$.

\subsection*{chapter 6}
18: Show that there exist algebraic numbers of arbitrarily high degree.

\subsection*{chapter 7}
6: Let $K \supset F$ be finite fields with $[K: F]=3$. Show that if $\alpha \in F$ is not a square in $F$, it is not a square in $K$.

24: Suppose that $f(x) \in \mathbb{Z} / p \mathbb{Z}[x]$ has the property that $f(x+y)=f(x)+$ $f(y) \in \mathbb{Z} / p \mathbb{Z}[x, y]$. Show that $f(x)$ must be of the form $a_{0} x+a_{1} x^{p}+a_{2} x^{p^{2}}+$ $\cdots+a_{m} x^{p^{m}}$.

\subsection*{chapter 12}
12: Show that $\sin (\pi / 12)$ is an algebraic number.

19: Show that a finite integral domain is a field.

22: Let $F \subset E$ be algebraic number fields. Show that any isomorphism of $F$ into $\mathbb{C}$ extends in exactly $[E: F]$ ways to an isomorphism of $E$ into $\mathbb{C}$.

30: Let $p$ be an odd prime and consider $\mathbb{Q}(\sqrt{p})$. If $q \neq p$ is prime show that $\sigma_{q}(\sqrt{p})=$ $(p / q) \sqrt{p}$ where $\sigma_{q}$ is the Frobenius automorphism at a prime ideal in $\mathbb{Q}(\sqrt{p})$ lying above $q$.

\subsection*{chapter 18}
1: Show that $165 x^{2}-21 y^{2}=19$ has no integral solution.

4: Show that 1729 is the smallest positive integer expressible as the sum of two different integral cubes in two ways.

32: Let $d$ be a square-free integer $d \equiv 1$ or 2 modulo 4 . Show that if $x$ and $y$ are integers such that $y^{2}=x^{3}-d$ then $(x, 2 d)=1$.

\section*{steinshakarchi}
\subsection*{chapter 1}
13: Suppose that $f$ is holomorphic in an open set $\Omega$. Prove that if $|f|$ is constant, then $f$ is constant. 

\subsection*{chapter 2}
2: Show that $\int_{0}^{\infty} \frac{\sin x}{x} d x=\frac{\pi}{2}$.

9: Let $\Omega$ be a bounded open subset of $\mathbb{C}$, and $\varphi: \Omega \rightarrow \Omega$ a holomorphic function. Prove that if there exists a point $z_{0} \in \Omega$ such that $\varphi\left(z_{0}\right)=z_{0} \quad \text { and } \quad \varphi^{\prime}\left(z_{0}\right)=1$ then $\varphi$ is linear.

\subsection*{chapter 5}
1: Prove that if $f$ is holomorphic in the unit disc, bounded and not identically zero, and $z_{1}, z_{2}, \ldots, z_{n}, \ldots$ are its zeros $\left(\left|z_{k}\right|<1\right)$, then $\sum_{n}\left(1-\left|z_{n}\right|\right)<\infty$. 

3: Show that $\sum \frac{z^{n}}{(n !)^{\alpha}}$ is an entire function of order $1 / \alpha$.

\section{cambridgetripos}
\subsection*{2022}
\subsubsection*{IA}
1-II-9D-a: Let $a_{n}$ be a sequence of real numbers. Show that if $a_{n}$ converges, the sequence $\frac{1}{n} \sum_{k=1}^{n} a_{k}$ also converges and $\lim _{n \rightarrow \infty} \frac{1}{n} \sum_{k=1}^{n} a_{k}=\lim _{n \rightarrow \infty} a_{n}$.

1-II-10D-c: Let a function $g:(0, \infty) \rightarrow \mathbb{R}$ be continuous and bounded. Show that for every $T>0$ there exists a sequence $x_{n}$ such that $x_{n} \rightarrow \infty$ and $\lim _{n \rightarrow \infty}\left(g\left(x_{n}+T\right)-g\left(x_{n}\right)\right)=0 .$

4-I-1E-a: By considering numbers of the form $3 p_{1} \ldots p_{k}-1$, show that there are infinitely many primes of the form $3 n+2$ with $n \in \mathbb{N}$.

4-I-2D-a: Prove that $\sqrt[3]{2}+\sqrt[3]{3}$ is irrational.

\subsubsection*{IB}
3-II-11G-b: Let $f: \mathbb{R}^{2} \rightarrow \mathbb{R}^{2}$ be the map given by
$f(x, y)=\left(\frac{\cos x+\cos y-1}{2}, \cos x-\cos y\right) .$ Prove that $f$ has a fixed point. 

\subsection*{2018}
\subsection*{IA}
1-I-3E-b: Let $f: \mathbb{R} \rightarrow(0, \infty)$ be a decreasing function. Let $x_{1}=1$ and $x_{n+1}=x_{n}+f\left(x_{n}\right)$. Prove that $x_{n} \rightarrow \infty$ as $n \rightarrow \infty$.

\section{pugh}
\subsection*{chapter 2}
5: Prove that a set $U \subset M$ is open if and only if none of its points are limits of its complement.

11: Let $\mathcal{T}$ be the collection of open subsets of a metric space $\mathrm{M}$, and $\mathcal{K}$ the collection of closed subsets. Show that there is a bijection from $\mathcal{T}$ onto $\mathcal{K}$.

13a: Show that every subset of $\mathbb{N}$ is clopen. 

32a: Let $(p_n)$ be a sequence and $f:\mathbb{N}\to\mathbb{N}$ a bijection. The sequence $(q_k)_{k\in\mathbb{N}}$ with $q_k=p_{f(k)}$ is called a rearrangement of $(p_n)$. Show that if $f$ is an injection, the limit of a sequence is unaffected by rearrangement. 

32b: Let $(p_n)$ be a sequence and $f:\mathbb{N}\to\mathbb{N}$ a bijection. The sequence $(q_k)_{k\in\mathbb{N}}$ with $q_k=p_{f(k)}$ is called a rearrangement of $(p_n)$. Show that if $f$ is a surjection, the limit of a sequence is unaffected by rearrangement. 

38: Let $\|\hspace{4pt}\|$ be any norm on $\mathbb{R}^{m}$ and let $B=\left\{x \in \mathbb{R}^{m}:\|x\| \leq 1\right\}$. Prove that $B$ is compact. 

44: Suppose that $M$ is compact and that $\mathcal{U}$ is an open covering of $M$ which is "redundant" in the sense that each $p \in M$ is contained in at least two members of $\mathcal{U}$. Show that $\mathcal{U}$ reduces to a finite subcovering with the same property.

54: Show that if $S$ is connected, it is not true in general that its interior is connected. 

79: Prove that if $M$ is nonempty compact, locally path-connected and connected then it is path-connected.

105: A metric on $M$ is an ultrametric if for all $x, y, z \in M$, $d(x, z) \leq \max \{d(x, y), d(y, z)\} .$ Show that a metric space with an ultrametric is totally disconnected. 

\subsection*{chapter 3}
1: Assume that $f: \mathbb{R} \rightarrow \mathbb{R}$ satisfies $|f(t)-f(x)| \leq|t-x|^{2}$ for all $t, x$. Prove that $f$ is constant.

4: Prove that $\sqrt{n+1}-\sqrt{n} \rightarrow 0$ as $n \rightarrow \infty$.

14c-i: Show that the bump function $\beta(x)=e^{2} e(1-x) \cdot e(x+1)$ is smooth. 

14c-ii: Show that the bump function $\beta(x)=e^{2} e(1-x) \cdot e(x+1)$ is identically 0 outside the interval $(-1, 1)$. 

\section*{putnam}
\subsection*{2021}
b4: Let $F_{0}, F_{1}, \ldots$ be the sequence of Fibonacci numbers, with $F_{0}=0, F_{1}=1$, and $F_{n}=F_{n-1}+F_{n-2}$ for $n \geq 2$. For $m>2$, let $R_{m}$ be the remainder when the product $\prod_{k=1}^{F_{m}-1} k^{k}$ is divided by $F_{m}$. Prove that $R_{m}$ is also a Fibonacci number.

\subsection*{2020}
b5: B5 For $j \in\{1,2,3,4\}$, let $z_{j}$ be a complex number with $\left|z_{j}\right|=1$ and $z_{j} \neq 1$. Prove that $3-z_{1}-z_{2}-z_{3}-z_{4}+z_{1} z_{2} z_{3} z_{4} \neq 0 .$

\subsection*{2018}
a5: Let $f: \mathbb{R} \rightarrow \mathbb{R}$ be an infinitely differentiable function satisfying $f(0)=0, f(1)=1$, and $f(x) \geq 0$ for all $x \in$ $\mathbb{R}$. Show that there exist a positive integer $n$ and a real number $x$ such that $f^{(n)}(x)<0$.

b2: Let $n$ be a positive integer, and let $f_{n}(z)=n+(n-1) z+$ $(n-2) z^{2}+\cdots+z^{n-1}$. Prove that $f_{n}$ has no roots in the closed unit disk $\{z \in \mathbb{C}:|z| \leq 1\}$.

b4: Given a real number $a$, we define a sequence by $x_{0}=1$, $x_{1}=x_{2}=a$, and $x_{n+1}=2 x_{n} x_{n-1}-x_{n-2}$ for $n \geq 2$. Prove that if $x_{n}=0$ for some $n$, then the sequence is periodic.

b6: Let $S$ be the set of sequences of length 2018 whose terms are in the set $\{1,2,3,4,5,6,10\}$ and sum to 3860 . Prove that the cardinality of $S$ is at most $2^{3860} \cdot\left(\frac{2018}{2048}\right)^{2018} .$

\subsection*{2017}
b3: Suppose that $f(x)=\sum_{i=0}^{\infty} c_{i} x^{i}$ is a power series for which each coefficient $c_{i}$ is 0 or 1 . Show that if $f(2 / 3)=3 / 2$, then $f(1 / 2)$ must be irrational.

\section*{axler}
\subsection*{chapter 2}
1: Prove that if $\left(v_{1}, \ldots, v_{n}\right)$ spans $V$, then so does the list $\left(v_{1}-v_{2}, v_{2}-v_{3}, \ldots, v_{n-1}-v_{n}, v_{n}\right)$obtained by subtracting from each vector (except the last one) the following vector.

2: Prove that if $\left(v_{1}, \ldots, v_{n}\right)$ is linearly independent in $V$, then so is the list $\left(v_{1}-v_{2}, v_{2}-v_{3}, \ldots, v_{n-1}-v_{n}, v_{n}\right)$ obtained by subtracting from each vector (except the last one) the following vector.

6: Prove that the real vector space consisting of all continuous realvalued functions on the interval $[0,1]$ is infinite dimensional.

\subsection*{chapter 3}
1: Show that every linear map from a one-dimensional vector space to itself is multiplication by some scalar. More precisely, prove that if $\operatorname{dim} V=1$ and $T \in \mathcal{L}(V, V)$, then there exists $a \in \mathbf{F}$ such that $T v=a v$ for all $v \in V$.

8: Suppose that $V$ is finite dimensional and that $T \in \mathcal{L}(V, W)$. Prove that there exists a subspace $U$ of $V$ such that $U \cap$ null $T=\{0\}$ and range $T=\{T u: u \in U\}$.

9: Prove that if $T$ is a linear map from $\mathbf{F}^{4}$ to $\mathbf{F}^{2}$ such that null $T=\left\{\left(x_{1}, x_{2}, x_{3}, x_{4}\right) \in \mathbf{F}^{4}: x_{1}=5 x_{2}\right.$ and $\left.x_{3}=7 x_{4}\right\}$, then $T$ is surjective.

10: Prove that there does not exist a linear map from $\mathbf{F}^{5}$ to $\mathbf{F}^{2}$ whose mull space equals $\left\{\left(x_{1}, x_{2}, x_{3}, x_{4}, x_{5}\right) \in \mathbf{F}^{5}: x_{1}=3 x_{2} \text { and } x_{3}=x_{4}=x_{5}\right\} .$

11: Prove that if there exis ts a linear map on $V$ whose null space and range are both finite dimensional, then $V$ is finite dimensional.

\subsection*{chapter 4}
4: Suppose $p \in \mathcal{P}(\mathbf{C})$ has degree $m$. Prove that $p$ has $m$ distinct roots if and only if $p$ and its derivative $p^{\prime}$ have no roots in common.

\subsection*{chapter 5}
1: Suppose $T \in \mathcal{L}(V)$. Prove that if $U_{1}, \ldots, U_{m}$ are subspaces of $V$ invariant under $T$, then $U_{1}+\cdots+U_{m}$ is invariant under $T$.

4: Suppose that $S, T \in \mathcal{L}(V)$ are such that $S T=T S$. Prove that null $(T-\lambda I)$ is invariant under $S$ for every $\lambda \in \mathbf{F}$.

11: Suppose $S, T \in \mathcal{L}(V)$. Prove that $S T$ and $T S$ have the same eigenvalues.

12: Suppose $T \in \mathcal{L}(V)$ is such that every vector in $V$ is an eigenvector of $T$. Prove that $T$ is a scalar multiple of the identity operator.

13: Suppose $T \in \mathcal{L}(V)$ is such that every subspace of $V$ with dimension $\operatorname{dim} V-1$ is invariant under $T$. Prove that $T$ is a scalar multiple of the identity operator.

20: Suppose that $T \in \mathcal{L}(V)$ has $\operatorname{dim} V$ distinct eigenvalues and that $S \in \mathcal{L}(V)$ has the same eigenvectors as $T$ (not necessarily with the same eigenvalues). Prove that $S T=T S$.

24: Suppose $V$ is a real vector space and $T \in \mathcal{L}(V)$ has no eigenvalues. Prove that every subspace of $V$ invariant under $T$ has even dimension.

\subsection*{chapter 6}
2: Suppose $u, v \in V$. Prove that $\langle u, v\rangle=0$ if and only if $\|u\| \leq\|u+a v\|$for all $a \in \mathbf{F}$.

3: Prove that $\left(\sum_{j=1}^{n} a_{j} b_{j}\right)^{2} \leq\left(\sum_{j=1}^{n} j a_{j}{ }^{2}\right)\left(\sum_{j=1}^{n} \frac{b_{j}{ }^{2}}{j}\right)$ for all real numbers $a_{1}, \ldots, a_{n}$ and $b_{1}, \ldots, b_{n}$.

7: Prove that if $V$ is a complex inner-product space, then $\langle u, v\rangle=\frac{\|u+v\|^{2}-\|u-v\|^{2}+\|u+i v\|^{2} i-\|u-i v\|^{2} i}{4}$ for all $u, v \in V$.

13: Suppose $\left(e_{1}, \ldots, e_{m}\right)$ is an or thonormal list of vectors in $V$. Let $v \in V$. Prove that $\|v\|^{2}=\left|\left\langle v, e_{1}\right\rangle\right|^{2}+\cdots+\left|\left\langle v, e_{m}\right\rangle\right|^{2}$ if and only if $v \in \operatorname{span}\left(e_{1}, \ldots, e_{m}\right)$.

16: Suppose $U$ is a subspace of $V$. Prove that $U^{\perp}=\{0\}$ if and only if $U=V$

17: Prove that if $P \in \mathcal{C}(V)$ is such that $P^{2}=P$ and every vector in null $P$ is orthogonal to every vector in range $P$, then $P$ is an orthogonal projection.

18: Prove that if $P \in \mathcal{L}(V)$ is such that $P^{2}=P$ and $\|P v\| \leq\|v\|$ for every $v \in V$, then $P$ is an or thogonal projection.

19: Suppose $T \in \mathcal{L}(V)$ and $U$ is a subspace of $V$. Prove that $U$ is invariant under $T$ if and only if $P_{U} T P_{U}=T P_{U}$.

20: Suppose $T \in \mathcal{L}(V)$ and $U$ is a subspace of $V$. Prove that $U$ and $U^{\perp}$ are both invariant under $T$ if and only if $P_{U} T=T P_{U}$.

29: Suppose $T \in \mathcal{L}(V)$ and $U$ is a subspace of $V$. Prove that $U$ is invariant under $T$ if and only if $U^{\perp}$ is invariant under $T^{*}$.

\subsection*{chapter 7}
4: Suppose $P \in \mathcal{L}(V)$ is such that $P^{2}=P$. Prove that $P$ is an orthogonal projection if and only if $P$ is self-adjoint.

5: Show that if $\operatorname{dim} V \geq 2$, then the set of normal operators on $V$ is not a subspace of $\mathcal{L}(V)$.

6: Prove that if $T \in \mathcal{L}(V)$ is normal, then $\mathrm{range} T=\mathrm{range} T^{*}.$

8: Prove that there does not exist a self-adjoint operator $T \in \mathcal{L}\left(\mathbf{R}^{3}\right)$ such that $T(1,2,3)=(0,0,0)$ and $T(2,5,7)=(2,5,7)$.

9: Prove that a normal operator on a complex inner-product space is self-adjoint if and only if all its eigenvalues are real.

10: Suppose $V$ is a complex inner-product space and $T \in \mathcal{L}(V)$ is a normal operator such that $T^{9}=T^{8}$. Prove that $T$ is self-adjoint and $T^{2}=T$.

11: Suppose $V$ is a complex inner-product space. Prove that every normal operator on $V$ has a square root. (An operator $S \in \mathcal{L}(V$ ) is called a square root of $T \in \mathcal{L}(V)$ if $S^{2}=T$.)

14: Suppose $T \in \mathcal{L}(V)$ is self-adjoint, $\lambda \in \mathbf{F}$, and $\epsilon>0$. Prove that if there exists $v \in V$ such that $\|v\|=1$ and $\|T v-\lambda v\|<\epsilon,$ then $T$ has an eigenvalue $\lambda^{\prime}$ such that $\left|\lambda-\lambda^{\prime}\right|<\epsilon$.

15: Suppose $U$ is a finite-dimensional real vector space and $T \in$ $\mathcal{L}(U)$. Prove that $U$ has a basis consisting of eigenvectors of $T$ if and only if there is an inner product on $U$ that makes $T$ into a self-adjoint operator.

17: Prove that the sum of any two positive operators on $V$ is positive.

18: Prove that if $T \in \mathcal{L}(V)$ is positive, then so is $T^{k}$ for every positive integer $k$.

\section{herstein}
\subsection*{chapter 2}
\subsubsection*{section 1}
18: If $G$ is a finite group of even order, show that there must be an element $a \neq e$ such that $a=a^{-1}$.
\subsubsection*{section 2}
3: If $G$ is a group in which $(a b)^{i}=a^{i} b^{i}$ for three consecutive integers $i$, prove that $G$ is abelian.

5: Let $G$ be a group in which $(a b)^{3}=a^{3} b^{3}$ and $(a b)^{5}=a^{5} b^{5}$ for all $a, b \in G$. Show that $G$ is abelian.
\end{document}
