\documentclass{article}

\title{\textbf{
Exercises from \\
\textit{Cambridge Tripos}
}}

\date{}

\usepackage{amsmath}
\usepackage{amssymb}

\begin{document}
\maketitle

\paragraph{Exercise 2022.IA.1-II-9D-a} Let $a_{n}$ be a sequence of real numbers. Show that if $a_{n}$ converges, the sequence $\frac{1}{n} \sum_{k=1}^{n} a_{k}$ also converges and $\lim _{n \rightarrow \infty} \frac{1}{n} \sum_{k=1}^{n} a_{k}=\lim _{n \rightarrow \infty} a_{n}$.

\paragraph{Exercise 2022.IA.4-I-1E-a} Show that there are infinitely many primes of the form $3 n+2$ with $n \in \mathbb{N}$.
\begin{proof}
    The general strategy is to find a (large) number $n$ that is relatively prime to each of the existing list of such primes, and is also congruent to 2 modulo 3 . The prime factorization of $n$ cannot consist only of primes congruent to 1 modulo 3 , since the product of any number of such is still 1 modulo 3 . Hence there must be some prime factor of $n$ that is congruent to 2 modulo 3 , which must be not on our list by the construction of $n$.
Now, how to construct such an $n$ ? Suppose the finite list is $\left\{p_1, p_2, \ldots, p_k\right\}$. If $k$ is even, then take $n=p_1 p_2 \cdots p_k+1$. If $k$ is odd, then take $n=\left(p_1 p_2 \cdots p_k\right) p_k+1$.
\end{proof}


\paragraph{Exercise 2022.IA.4-I-2D-a} Prove that $\sqrt[3]{2}+\sqrt[3]{3}$ is irrational.

\paragraph{Exercise 2022.IB.3-II-13G-a-i} Let $U \subset \mathbb{C}$ be a (non-empty) connected open set and let $f_n$ be a sequence of holomorphic functions defined on $U$. Suppose that $f_n$ converges uniformly to a function $f$ on every compact subset of $U$. Show that $f$ is holomorphic in $U$.

\paragraph{Exercise 2022.IB.3-II-11G-b} Let $f: \mathbb{R}^{2} \rightarrow \mathbb{R}^{2}$ be the map given by $f(x, y)=\left(\frac{\cos x+\cos y-1}{2}, \cos x-\cos y\right)$. Prove that $f$ has a fixed point.


\paragraph{Exercise 2021.IIB.3-I-1G-i} Let $G$ be a finite group, and let $H$ be a proper subgroup of $G$ of index $n$. Show that there is a normal subgroup $K$ of $G$ such that $|G / K|$ divides $n$ ! and $|G / K| \geqslant n$

\paragraph{Exercise 2018.IA.1-I-3E-b} Let $f: \mathbb{R} \rightarrow(0, \infty)$ be a decreasing function. Let $x_{1}=1$ and $x_{n+1}=x_{n}+f\left(x_{n}\right)$. Prove that $x_{n} \rightarrow \infty$ as $n \rightarrow \infty$.

\end{document}
