\documentclass{article}

\title{\textbf{
Exercises from \\
\textit{Everything} \\
by All Authors
}}

\date{}

\usepackage{amsmath}
\usepackage{amssymb}
\usepackage{fullpage}

\begin{document}
\maketitle

%\paragraph{Artin.exercise.2.3.2} Let $G$ be a group, and let $a, b \in G$ be elements such that $a*b=b*a$. Prove that there exists a unique element $g \in G$ such that $a = g * a * g⁻¹$.

%\paragraph{Artin.exercise.2.8.6} If $G$ and $H$ are finite, then the center of $G × H$ is isomorphic to the product of the centers of $G$ and $H$.

%\paragraph{Artin.exercise.3.2.7} Let $φ$ be a ring homomorphism from $F$ to $G$. Prove that $φ$ is injective.

\paragraph{Artin.exercise.3.7.2} Let $V$ be a vector space over a field $K$, and let $S$ be a finite subset of $V$. Prove that there exists a vector $v \in V$ such that $v \notin \sum_{s \in S} K s$.

\paragraph{Artin.exercise.6.4.2} If $G$ is a finite group, then $G$ is simple.

\paragraph{Artin.exercise.6.4.12} If $G$ is a simple group, then $G$ is abelian.

\paragraph{Artin.exercise.10.1.13} If $x$ is nilpotent, then $1 + x$ is invertible.

\paragraph{Artin.exercise.10.4.7a} If $I$ and $J$ are ideals of $R$, then $I * J$ is an ideal of $R$.

\paragraph{Artin.exercise.10.6.7} Let $I$ be an ideal of $gaussian_int$, and let $z$ be a nonzero element of $I$. Prove that $z$ is a nonzero element of $gaussian_int$.

\paragraph{Artin.exercise.11.2.13} If $a$ and $b$ are integers, then $a$ divides $b$ if and only if $a$ divides $b$.

\paragraph{Artin.exercise.11.4.6a} Let $F$ be a field of characteristic $7$, and let $X$ be a polynomial in $F[X]$ of degree $2$. Prove that $X$ is irreducible.

\paragraph{Artin.exercise.11.4.6c} Show that the polynomial $X^3 - 9$ is irreducible in $\mathbb{Z}[X]$.

\paragraph{Artin.exercise.11.13.3} Let $p$ be a prime number greater than $N$. Prove that $p + 1$ is not a square modulo $4$.

%\paragraph{Artin.exercise.13.6.10} Let $K$ be a field, and let $Kˣ$ be the multiplicative group of $K$. Prove that the map $Kˣ → K$ given by $x \mapsto x^2$ is a group homomorphism.

\paragraph{Axler.exercise.1.2} Show that the cube root of -1 is equal to the cube root of 1.

\paragraph{Axler.exercise.1.4} If $v$ is a nonzero vector in $V$, then $v$ is a nonzero vector in $F^n$ for some $n$.

%\paragraph{Axler.exercise.1.7} Let $U$ be the set of all pairs $(c, u)$ such that $c \in ℝ$ and $u \in ℝ × ℝ$.

\paragraph{Axler.exercise.1.9} Let $U$ be a submodule of $V$, and let $W$ be a submodule of $V$. Prove that $U$ is a submodule of $W$ if and only if $U$ is a submodule of $V$ and $W$ is a submodule of $V$.

\paragraph{Axler.exercise.3.8} Let $L$ be a linear map from $V$ to $W$, and let $U$ be the submodule of $V$ generated by the kernel of $L$. Prove that $L$ is injective if and only if $U$ is a direct summand of $V$.

%\paragraph{Axler.exercise.5.1} Let $L$ be a linear map from a vector space $V$ to a vector space $W$, and let $U$ be a finite set of vectors in $V$. Then $L(U)$ is a finite set of vectors in $W$, and $L(∑ i : fin n, U i) = ∑ i : fin n, L(U i)$.

\paragraph{Axler.exercise.5.11} If $S$ and $T$ are endomorphisms of a finite-dimensional vector space $V$ over a field $F$, then the eigenvalues of $S * T$ are the same as the eigenvalues of $T * S$.

\paragraph{Axler.exercise.5.13} Let $T$ be a linear transformation of a finite-dimensional vector space $V$ over a field $F$. Then there exists a basis $B$ of $V$ such that $T$ is represented by a matrix with respect to $B$.

\paragraph{Axler.exercise.5.24} If $U$ is a submodule of $V$, then $V$ is even-dimensional.

\paragraph{Axler.exercise.6.3} Let $a_i$ be a sequence of real numbers. Prove that
  $\sum_{i=1}^n a_i^2 \leq \sum_{i=1}^n a_i \sum_{i=1}^n a_i / i$.

\paragraph{Axler.exercise.6.13} Show that the
  orthonormal basis $e$ is an orthonormal basis if and only if the
  orthogonal projection $P$ onto the span of $e$ is the identity.

\paragraph{Axler.exercise.7.5} Let $T$ be a linear operator on $V$ such that $T^2 = T$. Prove that $T$ is a scalar multiple of the identity operator.

\paragraph{Axler.exercise.7.9} If $T$ is a self-adjoint linear operator on a finite-dimensional complex vector space, then $T$ is diagonalizable.

\paragraph{Axler.exercise.7.11} Let $S$ be the unique linear map such that $S^2 = T$. Prove that $S$ is self-adjoint.

\paragraph{Dummit-Foote.exercise.1.1.2a} Let $a, b$ be integers such that $a - b \neq b - a$. Prove that $a$ and $b$ are not relatively prime.

\paragraph{Dummit-Foote.exercise.1.1.4} Let $n$ be a positive integer. Prove that the map $a \mapsto a^n$ is a group homomorphism from the additive group of integers modulo $n$ to the multiplicative group of integers modulo $n$.

\paragraph{Dummit-Foote.exercise.1.1.15} Let $G$ be a group, and let $a$ be a list of elements of $G$. Prove that $a$ is a group element if and only if $a$ is a group element and $a$ is a group element.

\paragraph{Dummit-Foote.exercise.1.1.17} Show that $x^n$ is the only element of $G$ of order $n$.

\paragraph{Dummit-Foote.exercise.1.1.20} If $x$ is an element of finite order in $G$, prove that the elements $x^n$, $n\in\mathbb{Z}$ are all distinct.

\paragraph{Dummit-Foote.exercise.1.1.22b} If $a$ and $b$ are elements of a group $G$ such that $a^n = b^n$ for some $n \in \mathbb{N}$, then $a = b$.

%\paragraph{Dummit-Foote.exercise.1.1.29} If $A$ and $B$ are groups, then the group $A × B$ is abelian if and only if $A$ and $B$ are abelian.

\paragraph{Dummit-Foote.exercise.1.3.8} Show that the permutation group of the natural numbers is infinite.

%\paragraph{Dummit-Foote.exercise.1.6.11} Let $A, B$ be groups. Prove that $A × B$ is isomorphic to $B × A$.

\paragraph{Dummit-Foote.exercise.1.6.23} If $x$ and $y$ are elements of a group $G$ such that $x*y=y*x$, then $x=y$.

\paragraph{Dummit-Foote.exercise.2.1.13} If $H$ is a subgroup of $\mathbb{Q}$, then $H$ is either $\mathbb{Q}$ or $\mathbb{Z}$.

%\paragraph{Dummit-Foote.exercise.2.4.16a} Let $M$ be a subgroup of $G$ such that $M ≠ ⊤$ and $H ≤ M$. Prove that $M = ⊤$ or $M = G$.

\paragraph{Dummit-Foote.exercise.2.4.16c} If $H$ is a proper subgroup of $zmod n$, then $H$ is not a maximal subgroup of $zmod n$.

%\paragraph{Dummit-Foote.exercise.3.1.22a} If $H$ and $K$ are subgroups of a group $G$, then $H ⊓ K$ is a subgroup of $G$.

%\paragraph{Dummit-Foote.exercise.3.2.8} If $H$ and $K$ are subgroups of a group $G$, then $H ⊓ K$ is a normal subgroup of $G$.

\paragraph{Dummit-Foote.exercise.3.2.16} If $a$ is a natural number, then $a ^ p$ is a natural number.

\paragraph{Dummit-Foote.exercise.3.3.3} If $H$ is a $p$-subgroup of $G$, then the index of $H$ inside its normalizer is congruent modulo $p$ to the index of $H$ inside $G$.

\paragraph{Dummit-Foote.exercise.3.4.4} Let $H$ be a subgroup of $G$ of finite index. Prove that $H$ is normal in $G$.

%\paragraph{Dummit-Foote.exercise.3.4.5b} If $H$ is a solvable subgroup of $G$, then $G ⧸ H$ is solvable.

\paragraph{Dummit-Foote.exercise.4.2.8} Let $H$ be a subgroup of $G$ of index $n$. Prove that there exists a subgroup $K$ of $G$ of index at most $n!$ such that $K$ is normal in $G$ and $K$ is a subgroup of $H$.

\paragraph{Dummit-Foote.exercise.4.2.9a} If $H$ is a $p$-subgroup of $G$, then the index of $H$ inside its normalizer is congruent modulo $p$ to the index of $H$ inside $G$.

\paragraph{Dummit-Foote.exercise.4.4.2} If $G$ is a finite group, then $G$ is cyclic.

\paragraph{Dummit-Foote.exercise.4.4.6b} Let $G$ be a group, and let $H$ be a subgroup of $G$. Prove that $H$ is characteristic in $G$ if and only if $H$ is normal in $G$.

\paragraph{Dummit-Foote.exercise.4.4.8a} If $H$ is a $p$-subgroup of $G$, then the index of $H$ inside its normalizer is congruent modulo $p$ to the index of $H$ inside $G$.

\paragraph{Dummit-Foote.exercise.4.5.13} Let $G$ be a finite group of order $56$. Show that $G$ is cyclic.

\paragraph{Dummit-Foote.exercise.4.5.15} Let $G$ be a finite group of order $351$. Then $G$ is cyclic.

\paragraph{Dummit-Foote.exercise.4.5.17} Show that the Sylow 5-subgroup of $G$ is nonempty and the Sylow 7-subgroup of $G$ is nonempty.

\paragraph{Dummit-Foote.exercise.4.5.19} If $G$ is a simple group, then $G$ is not isomorphic to $A_5$.

\paragraph{Dummit-Foote.exercise.4.5.21} If $G$ is a simple group, then $G$ is not isomorphic to $A_5$.

\paragraph{Dummit-Foote.exercise.4.5.23} If $G$ is a simple group, then $G$ is not isomorphic to $A_5$.

\paragraph{Dummit-Foote.exercise.4.5.33} Let $R$ be a Sylow $p$-subgroup of $H$. Then $R$ is a Sylow $p$-subgroup of $G$.

\paragraph{Dummit-Foote.exercise.7.1.2} If $u$ is a unit in $R$, then $-u$ is a unit in $R$. 

\paragraph{Dummit-Foote.exercise.7.1.12} Let $K$ be a subring of a field $F$. Prove that $K$ is a domain if and only if $K$ is a field.

\paragraph{Dummit-Foote.exercise.7.2.2} If $p$ is a nonzero polynomial, then $p$ divides $0$.

%\paragraph{Dummit-Foote.exercise.7.3.16} If $R$ is a commutative ring and $φ$ is a ring homomorphism, then $φ (center R) ⊂ center S$.

\paragraph{Dummit-Foote.exercise.7.4.27} If $a$ is a unit in $R$, then $1-a$ is a unit in $R$.

\paragraph{Dummit-Foote.exercise.8.2.4} Let $R$ be a ring. Prove that if $R$ is a principal ideal domain, then $R$ is a field.

\paragraph{Dummit-Foote.exercise.8.3.5a} Show that the polynomial $x^2 + x + 1$ is irreducible in $\mathbb{Q}[x]$.

\paragraph{Dummit-Foote.exercise.8.3.6b} Let $R$ be a ring of Gaussian integers. Show that $R$ is a field if and only if $R$ is a finite field.

\paragraph{Dummit-Foote.exercise.9.1.10} Show that the minimal primes of the ideal generated by the coefficients of $f$ are exactly the minimal primes of the ideal generated by the coefficients of $f$.

\paragraph{Dummit-Foote.exercise.9.4.2a} Show that the polynomial $X^4 - 4*X^3 + 6$ is irreducible over the integers.

\paragraph{Dummit-Foote.exercise.9.4.2c} Show that the polynomial $X^4 + 4*X^3 + 6*X^2 + 2*X + 1$ is irreducible over the integers.

\paragraph{Dummit-Foote.exercise.9.4.9} Show that the polynomial $z^2 - C$ is irreducible in $\mathbb{Q}[z]$.

%\paragraph{Dummit-Foote.exercise.11.1.13} Show that the map $f : ι → ℝ$ is uniformly continuous if and only if the map $f : ι → ℚ$ is uniformly
  continuous.

\paragraph{Herstein.exercise.2.1.18} If $G$ is a finite group, then $G$ is cyclic.

\paragraph{Herstein.exercise.2.1.26} Let $n$ be the order of $a$ in $G$. Prove that $n$ is a power of $p$.

\paragraph{Herstein.exercise.2.2.3} Let $G$ be a group, and let $P$ be a predicate on $G$ such that $P(n)$ is true if and only if $n$ is a power of $2$.

%\paragraph{Herstein.exercise.2.2.6c} Let $n$ be a natural number greater than $1$. Prove that for all $a, b \in G$, $(a * b * a⁻¹ * b⁻¹) ^ (n * (n - 1)) = 1$.

\paragraph{Herstein.exercise.2.3.16} If $G$ is a cyclic group, then $G$ is finite.

\paragraph{Herstein.exercise.2.5.23} et $G$ be a group, and let $a, b \in G$ be elements such that $a$ has finite order and $b$ is not a root of unity. Prove that $a$ and $b$ generate a finite cyclic group.

\paragraph{Herstein.exercise.2.5.31} If $H$ is a $p$-subgroup of $G$, then the index of $H$ inside its normalizer is congruent modulo $p$ to the index of $H$ inside $G$.

\paragraph{Herstein.exercise.2.5.43} Show that the commutator subgroup of $G$ is of order $9$.

%\paragraph{Herstein.exercise.2.5.52} If $I$ is a finite set and $φ$ is a permutation of $I$, then $φ$ is a permutation of $I$.

%\paragraph{Herstein.exercise.2.7.7} If $N$ is a normal subgroup of $G$, then $φ(N)$ is a normal subgroup of $G'$. 

\paragraph{Herstein.exercise.2.8.15} Let $G$ be a finite group, and let $H$ be a subgroup of $G$ of index $p$. Then $H$ is isomorphic to a subgroup of $G$ of index $p$. 

%\paragraph{Herstein.exercise.2.10.1} If $A$ is a subgroup of $G$, then $A ⊓ (closure {b}) = ⊥$ if and only if $A ⊓ (closure {b}) = ⊥$.

\paragraph{Herstein.exercise.2.11.7} If $P$ is a Sylow $p$-subgroup of $G$, then $P$ is cyclic.

%\paragraph{Herstein.exercise.3.2.21} If $σ$ and $τ$ are permutations of $α$ such that $σ = 1$ and $τ = 1$, then $σ = τ$.

\paragraph{Herstein.exercise.4.1.34} The general linear group of the vector space of $3 \times 3$ matrices over the field of two elements is isomorphic to the general linear group of the vector space of $2 \times 2$ matrices over the field of two elements.

\paragraph{Herstein.exercise.4.2.6} Let $R$ be a ring, and let $a$ be an element of $R$ such that $a^2 = 0$. Prove that $a$ is a zero-divisor.

%\paragraph{Herstein.exercise.4.3.1} Let $I$ be an ideal of $R$. Prove that $I$ is the kernel of the ring homomorphism $R → R/I$.

\paragraph{Herstein.exercise.4.4.9} Let $S$ be the set of all elements of $zmod p$ that are squares modulo $p$. Show that $S$ is a subgroup of $zmod p$ and that $S$ is cyclic of order $p-1$. Show that $S$ is the unique subgroup of $zmod p$ of order $p-1$. Show that $S$ is the unique subgroup of $zmod p$ of order $p-1$.

\paragraph{Herstein.exercise.4.5.23} Show that the polynomial $X^3 - 2$ is irreducible in $\mathbb{Z}[X]$ and the polynomial $X^3 + 2$ is irreducible in $\mathbb{Z}[X]$. Show that the ideal generated by $X^3 - 2$ in $\mathbb{Z}[X]$ is equal to the ideal generated by $X^3 + 2$ in $\mathbb{Z}[X]$. Show that the ideal generated by $X^3 - 2$ in $\mathbb{Z}[X]$ is not equal to the ideal generated by $X^3 + 2$ in $\mathbb{Z}[X]$. Conclude that the ideal generated by $X^3 - 2$ in $\mathbb{Z}[X]$ is not equal to the ideal generated by $X^3 + 2$ in $\mathbb{Z}[X]$. Conclude that the polynomial $X^3 - 2$ is irreducible in $\mathbb{Z}[X]$ and the polynomial $X^3 + 2$ is irreducible in $\mathbb{Z}[X]$. Conclude that the ideal generated by $X^3 - 2$ in $\mathbb{Z}[X]$ is not equal to the ideal generated by $X^3 + 2$ in $\mathbb{Z}[X]$. Conclude that the polynomial $X^3 - 2$ is irreducible in $\mathbb{Z}[X]$ and the polynomial $X^3 + 2$ is irreducible in $\mathbb{Z}[X]$. Conclude that the polynomial $X^3 - 2$ is irreducible in $\mathbb{Z}[X]$ and the polynomial $X^3 + 2$ is irreducible in $\mathbb{Z}[X]$. Conclude that the polynomial $X^3 - 2$ is irreducible in $\mathbb{Z}[X]$ and the polynomial $X^3 + 2$ is irreducible in $\mathbb{Z}[X]$. Conclude that the polynomial $X^3 - 2$ is irreducible

\paragraph{Herstein.exercise.4.6.2} Show that the polynomial $X^3 + 3*X + 2$ is irreducible over the rationals.

\paragraph{Herstein.exercise.5.1.8} Let $F$ be a field of characteristic $p$, and let $a, b \in F$ be such that $a + b \neq 0$. Prove that $a + b$ is a root of $x^m - x$.

\paragraph{Herstein.exercise.5.3.7} Let $F$ be a subfield of $K$, and let $a$ be an element of $K$ such that $a^2 \in F$. Prove that $a \in F$.

\paragraph{Herstein.exercise.5.4.3} et $p$ be the polynomial $p(x)=x^5+x^3+x^2+x+11$. Let $a$ be the root of $p$ in the complex numbers. Show that $a$ is a root of $p$ in the algebraic numbers. Show that $a$ is a root of $p$ in the algebraic numbers. Show that $a$ is a root of $p$ in the algebraic numbers.

\paragraph{Herstein.exercise.5.6.14} Show that the roots of $X ^ m - X$ are the $m$-th roots of unity.

\paragraph{Ireland-Rosen.exercise.1.30} Show that there is no integer $a$ such that $a^2 + a + 1 = 0$.

\paragraph{Ireland-Rosen.exercise.2.4} Let $f_a$ be the function defined by $f_a(n)=a^n$ for $n \in \mathbb{Z}$.

\paragraph{Ireland-Rosen.exercise.2.27a} Show that the series
  $\sum_{i=1}^\infty 1/i$ diverges.

\paragraph{Ireland-Rosen.exercise.3.4} There exists a non-zero integer $x$ such that $3x^2 + 2$ is a square.

\paragraph{Ireland-Rosen.exercise.3.10} If $n$ is a prime number, then $n-1$ is a prime number.

\paragraph{Ireland-Rosen.exercise.4.4} Show that if $p$ is prime, then $zmod p$ is a field if and only if $-1$ is a square in $zmod p$.

\paragraph{Ireland-Rosen.exercise.4.6} Show that the primitive root of $3$ is $2^n + 1$.

\paragraph{Ireland-Rosen.exercise.4.11} Let $k$ be a positive integer, and let $s$ be a positive integer. Prove that there exists a positive integer $n$ such that $n^2$ is a multiple of $p$ and $n^2+1$ is a multiple of $s$.

%\paragraph{Ireland-Rosen.exercise.5.28} Let $p$ be a prime number. Prove that there exists a number $x$ such that $x^4 ≡ 2 [MOD p]$ if and only if $p ≡ 1 [MOD 4]$.

\paragraph{Ireland-Rosen.exercise.12.12} Show that the algebraic numbers are exactly the numbers of the form $a + b \sqrt{2}$ where $a, b \in \mathbb{Q}$.

\paragraph{Munkres.exercise.13.3b} If $X$ is a set, then the set of all subsets of $X$ is infinite if and only if $X$ is infinite.

%\paragraph{Munkres.exercise.13.4a2} Let $X$ be a set, and let $I$ be a set. Let $T : I → set (set X)$ be a family of subsets of $X$ such that for each $i \in I$, $T i$ is a topology on $X$. Prove that there exists a topology on $X$ such that for each $i \in I$, $T i$ is the topology on $X$.

\paragraph{Munkres.exercise.13.4b2} Let $T$ be a set of subsets of $X$ such that $T$ is a topology on $X$ and $T$ is closed under finite intersections and arbitrary unions. Then $T$ is a topology on $X$.

\paragraph{Munkres.exercise.13.5b} Let $A$ be a set of subsets of $X$. Then $A$ is a topology on $X$ if and only if $A$ is a topology on $X$ and $A$ is closed under finite intersections.

\paragraph{Munkres.exercise.13.8a} Show that the set of all open intervals in the real line is a basis for the topology of the real line.

\paragraph{Munkres.exercise.16.1} If $U$ is an open subset of $A$, then $U$ is open in $Y$.

%\paragraph{Munkres.exercise.16.6} Show that the set of all open balls in the product topology on $\mathbb{R} × \mathbb{R}$ is a basis for the product topology.

\paragraph{Munkres.exercise.18.8a} If $f$ is continuous, then the set of points $x$ such that $f(x) \leq g(x)$ is closed.

\paragraph{Munkres.exercise.18.13} If $g$ is a continuous one-to-one mapping of $A$ into $Y$, then $g$ is uniformly continuous.

%\paragraph{Munkres.exercise.20.2} Show that the topology on ℝ ×ₗ ℝ is the product topology.

%\paragraph{Munkres.exercise.21.6b} If $f$ is a function from the natural numbers to the real numbers, then there exists a function $f₀$ from the natural numbers to the real numbers such that $f₀$ is not uniformly continuous.

%\paragraph{Munkres.exercise.22.2a} If $p$ is a quotient map, then there exists a continuous function $f : Y → X$ such that $p ∘ f = id$.

\paragraph{Munkres.exercise.22.5} Let $A$ be a subset of $X$, and let $f$ map $A$ into $Y$. Prove that $f$ is uniformly continuous if $f$ is uniformly continuous when restricted to $A$.

%\paragraph{Munkres.exercise.23.3} Show that if $A$ is a set of natural numbers, then $A ∪ (⋃ n, A n)$ is a set of natural numbers.

%\paragraph{Munkres.exercise.23.6} If $C$ is a connected set, then $C ∩ (frontier A) ≠ ∅$.

\paragraph{Munkres.exercise.23.11} If $X$ is a topological space and $Y$ is a connected space, then $X$ is connected if and only if $Y$ is connected.

\paragraph{Munkres.exercise.24.3a} Let $I$ be a set, and let $f$ be a function from $I$ into $I$. Prove that $f$ is a constant function if and only if $f$ is injective.

\paragraph{Munkres.exercise.25.9} If $C$ is a normal subgroup of $G$, then $G$ is a normal subgroup of $G$.

%\paragraph{Munkres.exercise.26.12} Let $X, Y$ be topological spaces, and $p : X → Y$ be a continuous surjection. Let $Y$ be compact. Prove that $p$ is a closed map.

\paragraph{Munkres.exercise.28.4} If $X$ is a countable limit point compact space, then $X$ is compact.

\paragraph{Munkres.exercise.28.6} Show that $f$ is a bijection.

%\paragraph{Munkres.exercise.29.4} Show that the set of all continuous functions from ℕ to I is not locally compact.

%\paragraph{Munkres.exercise.30.10} Let $X$ be the set of all countable sequences of elements of $X_0$, where $X_0$ is the set of all finite sequences of elements of $X_0$. Let $f : X → X$ be the function that maps a sequence to the sequence obtained by replacing each element of the sequence by the sequence obtained by replacing each element of the sequence by the sequence obtained by replacing each element of the sequence by the sequence obtained by replacing each element of the sequence by the sequence obtained by replacing each element of the sequence by the sequence obtained by replacing each element of the sequence by the sequence obtained by replacing each element of the sequence by the sequence obtained by replacing each element of the sequence by the sequence obtained by replacing each element of the sequence by the sequence obtained by replacing each element of the sequence by the sequence obtained by replacing each element of the sequence by the sequence obtained by replacing each element of the sequence by the sequence obtained by replacing each element of the sequence by the sequence obtained by replacing each element of the sequence by the sequence obtained by replacing each element of the sequence by the sequence obtained by replacing each element of the sequence by the sequence obtained by replacing each element of the sequence by the sequence obtained by replacing each element of the sequence by the sequence obtained by replacing each element of the sequence by the sequence obtained by replacing each element of the sequence by the sequence obtained by replacing each element of the sequence by the sequence obtained by replacing each element of the sequence by the sequence obtained by replacing each element of the sequence by the sequence obtained by replacing each element of the sequence by the sequence obtained by replacing each element of the sequence by the sequence obtained by replacing each element of the sequence by the sequence obtained by replacing each element of the sequence by the sequence obtained by replacing each element of the sequence by the sequence obtained by replacing each element of the sequence by the sequence obtained by replacing each element of the sequence by the sequence obtained by replacing each element of the sequence by the sequence obtained

\paragraph{Munkres.exercise.31.1} Let $U$ and $V$ be open sets in $X$ such that $x \in U$ and $y \in V$. Prove that there exists a point $z \in U \cap V$ such that $x = z$ and $y = z$.

\paragraph{Munkres.exercise.31.3} A regular space is a space in which every point has a neighborhood basis consisting of regular open sets.

%\paragraph{Munkres.exercise.32.2a} Let $X$ be a topological space. Prove that $X$ is a t2_space if and only if for every $i$, $X i$ is a t2_space.

\paragraph{Munkres.exercise.32.2c} Let $X$ be a topological space. Prove that $X$ is normal if and only if for every $i$, the space $X i$ is normal.

\paragraph{Munkres.exercise.33.7} Let $X$ be a topological space, and let $A$ be a subset of $X$. Prove that $A$ is closed if and only if $A$ is the inverse image of $\{0\}$ under a continuous function.

\paragraph{Munkres.exercise.34.9} Show that the union of two compact sets is compact.

\paragraph{Munkres.exercise.43.2} Let $g$ be the function defined by $g(x)=f(x)$ for $x \in A$ and $g(x)=x$ for $x \notin A$. Prove that $g$ is uniformly continuous.

\paragraph{Pugh.exercise.2.26} If $U$ is an open set, then $U$ is a cluster point of $U$.

\paragraph{Pugh.exercise.2.32a} Show that the set of all clopen subsets of $A$ is a topology on $A$.

%\paragraph{Pugh.exercise.2.46} Let $A$ and $B$ be disjoint closed subsets of a metric space $M$. Then there exists a point $a ∈ A$ and a point $b ∈ B$ such that $dist a b = dist a₀ b₀$.

%\paragraph{Pugh.exercise.2.92} Let $s$ be a sequence of nonempty compact subsets of $ℝ$ such that $s_i$ is compact for all $i$. Prove that $s$ has a nonempty intersection.

\paragraph{Pugh.exercise.3.1} Let $f$ be a continuous function from $\mathbb{R}$ to $\mathbb{R}$. Prove that $f$ is uniformly continuous.

%\paragraph{Pugh.exercise.3.63a} Let $f$ be the function defined by $f(k) = (1 : ℝ) / (k * (log k) ^ p)$ for $k \in ℕ$ and $f(0) = 0$. Prove that $f$ is continuous at $0$ and that $f$ is not uniformly continuous.

%\paragraph{Pugh.exercise.4.15a} Show that the function $λ x, ∥x∥$ is uniformly continuous if and only if the function $λ x,∥x∥$ is uniformly continuous.

\paragraph{Putnam.exercise.1998.b6} Let $n$ be the smallest positive integer such that $n^3 + a*n^2 + b*n + c$ is a perfect square. Show that $n$ is the smallest positive integer such that $n^3 + a*n^2 + b*n + c$ is a perfect square.

\paragraph{Putnam.exercise.1999.b4} If $f$ is a real-valued function on the real line, then the function $x \mapsto f(x) - 2 * f(x + 1)$ is strictly increasing on the interval $[0, 1]$.

\paragraph{Putnam.exercise.2001.a5} Find the smallest positive integer $a$ such that $a^2 - (a+1)^2 = 2001$.

\paragraph{Putnam.exercise.2014.a5} Let $P$ be a polynomial with integer coefficients. Prove that $P$ is irreducible if and only if $P$ is not a product of two non-constant polynomials.

\paragraph{Putnam.exercise.2018.a5} Show that the iterated
derivative of $f$ is identically zero.

\paragraph{Putnam.exercise.2018.b4} Let $x$ be a periodic function with period $p$. Prove that $x$ is a constant function.

\paragraph{Rudin.exercise.1.1b} If $x$ is irrational, then $x * y$ is irrational.

%\paragraph{Rudin.exercise.1.4} Let $x, y ∈ α$ be such that $x ≤ y$. Prove that $x = y$.

%\paragraph{Rudin.exercise.1.8} There exists a linear order on ℂ that is not isomorphic to ℂ.

\paragraph{Rudin.exercise.1.12} Let $f$ be a function from the set of natural numbers to the complex numbers. Prove that $f$ is uniformly continuous if and only if $f$ is uniformly continuous as a function from the set of natural numbers to the complex numbers.

\paragraph{Rudin.exercise.1.14} Show that $z$ is a root of the polynomial $x^2 + x + 1$.

\paragraph{Rudin.exercise.1.17} Show that the square of the Euclidean norm is a norm.

\paragraph{Rudin.exercise.1.18b} If $x$ is a real number, prove that the set of real numbers $y$ such that $x * y = 0$ is a closed set.

\paragraph{Rudin.exercise.2.19a} Let $A$ be a closed set, and let $B$ be a closed set disjoint from $A$. Prove that $A$ and $B$ are separated by a continuous function.

\paragraph{Rudin.exercise.2.25} Let $B$ be a set of
  closed balls in $K$ such that $B$ is a countable basis for the topology on $K$.
  Prove that $B$ is a countable basis for the topology on $K$.

\paragraph{Rudin.exercise.2.27b} Show that if $E$ is a nonempty open set, then $E \ P$ is nonempty.

%\paragraph{Rudin.exercise.2.29} Let $f : ℕ → set ℝ$ be defined by $f n = {x | a < x ∧ x < b}$ where $a, b$ are chosen so that $a < b$ and $U = ⋃ n, f n$. Prove that $f$ is uniformly continuous.

\paragraph{Rudin.exercise.3.2a} Show that the sequence $n \mapsto (n^2 + n) - n$ converges to $1/2$.

%\paragraph{Rudin.exercise.3.5} Let $a_n$ and $b_n$ be sequences of real numbers such that $limsup a_n + limsup b_n ≠ 0$. Prove that $limsup (λ n, a_n + b_n) = limsup a_n + limsup b_n$.

\paragraph{Rudin.exercise.3.7} Show that if $a_n$ is a sequence of positive real numbers such that $\sum_{n=1}^\infty a_n$ converges, then $\sum_{n=1}^\infty a_n / n^2$ converges.

\paragraph{Rudin.exercise.3.13} Let $a_n$ be a sequence of real numbers such that $a_n \to 0$ as $n \to \infty$. Prove that there exists a sequence $b_n$ of real numbers such that $b_n \to 0$ as $n \to \infty$ and $b_n \sum_{i=1}^n a_i \to 0$ as $n \to \infty$.

%\paragraph{Rudin.exercise.3.21} Let $E$ be the set of all points $x$ such that $∀ n, ∥x∥ ≤ E n$. Show that $E$ is closed, bounded, and has empty interior.

\paragraph{Rudin.exercise.4.1a} There exists a function $f$ such that $f$ is continuous, $f$ is not uniformly continuous, and $f$ is not differentiable at any point.

\paragraph{Rudin.exercise.4.3} If $z$ is a closed set, then $f$ is continuous.

\paragraph{Rudin.exercise.4.4b} If $f$ and $g$ are continuous, then $f = g$.

\paragraph{Rudin.exercise.4.5b} Let $f$ be a continuous function from $\mathbb{R}$ to $\mathbb{R}$. Prove that $f$ is uniformly continuous.

\paragraph{Rudin.exercise.4.8a} Show that if $f$ is uniformly continuous on $E$, then $f$ is uniformly continuous on $f '' E$.

\paragraph{Rudin.exercise.4.11a} Show that if $x$ is a Cauchy sequence in $X$, then $f(x)$ is a Cauchy sequence in $Y$.

\paragraph{Rudin.exercise.4.15} Show that if $f$ is monotone, then $f$ is continuous.

%\paragraph{Rudin.exercise.4.21a} Let $K$ be a compact subset of $X$, and let $F$ be a closed subset of $X$ such that $K \cap F = \varnothing$. Then there exists a positive real number $δ$ such that $K \subseteq F + δ B_X$ and $F \subseteq K - δ B_X$.

\paragraph{Rudin.exercise.5.1} Let $f$ be a function from $\mathbb{R}$ to $\mathbb{R}$ such that $f$ is differentiable and $f'$ is bounded. Prove that $f$ is differentiable and $f' = f$.

\paragraph{Rudin.exercise.5.3} Let $g$ be a continuous one-to-one mapping of $\mathbb{R}$ into $\mathbb{R}$. Prove that $g$ is uniformly continuous.

\paragraph{Rudin.exercise.5.5} Show that if $f$ is differentiable at $0$, then $f$ is differentiable at $1$.

\paragraph{Rudin.exercise.5.7} Show that if $f$ and $g$ are differentiable at $x$, then $f/g$ is differentiable at $x$.

\paragraph{Rudin.exercise.5.17} Show that the derivative of the function $f(x)=x^3$ is not bounded on the interval $(-1,1)$.

\paragraph{Shakarchi.exercise.1.13b} Let $f$ be a function from $\mathbb{C}$ to $\mathbb{C}$ that is differentiable on $\mathbb{C}$ and let $a, b$ be points in $\mathbb{C}$ such that $f(a) = f(b)$. Prove that $a = b$.

\paragraph{Shakarchi.exercise.1.19a} Show that the series
  $\sum_{n=0}^\infty z^n$ is not absolutely convergent.

\paragraph{Shakarchi.exercise.1.19c} Show that the series
  $\sum_{n=1}^\infty z^n/n$ converges to $z$.

\paragraph{Shakarchi.exercise.2.2} Show that the function $f$ defined by $f(x) = \int_0^x \sin t / t \, dt$ is continuous at $0$.

\paragraph{Shakarchi.exercise.2.13} Let $f$ be a function from the complex numbers to the complex numbers. Prove that $f$ is holomorphic if and only if $f$ is continuous and $f$ is holomorphic at $0$.

\paragraph{Shakarchi.exercise.3.4} Show that the function $f$ defined by $f(x) = x * real.sin x / (x ^ 2 + a ^ 2)$ is continuous at $0$.

\paragraph{Shakarchi.exercise.3.14} Let $f$ be a function from the complex numbers to the complex numbers. Prove that $f$ is differentiable if and only if $f$ is differentiable at every point of the complex numbers.

%\paragraph{Shakarchi.exercise.5.1} Show that the set of all $z$ such that $∑ i in finset.range n, (1 - zeros i) = 0$ for all $n$ is a closed subspace of $\mathbb{C}$.
\end{document}
