\documentclass{article}

\title{\textbf{
Exercises from \\
\textit{Abstract Algebra} \\
by I. N. Herstein
}}

\date{}

\usepackage{amsmath}
\usepackage{amssymb}
\usepackage{amsthm}

\begin{document}
\maketitle


\paragraph{Exercise 2.1.18} If $G$ is a finite group of even order, show that there must be an element $a \neq e$ such that $a=a^{-1}$.
\begin{proof}
    First note that $a=a^{-1}$ is the same as saying $a^2=e$, where $e$ is the identity. I.e. the statement is that there exists at least one element of order 2 in $G$.
Every element $a$ of $G$ of order at least 3 has an inverse $a^{-1}$ that is not itself -- that is, $a \neq a^{-1}$. So the subset of all such elements has an even cardinality (/size). There's exactly one element with order 1 : the identity $e^1=e$. So $G$ contains an even number of elements -call it $2 k$-- of which an even number are elements of order 3 or above -- call that $2 n$ where $n<k$-- and exactly one element of order 1 . Hence the number of elements of order 2 is
$$
2 k-2 n-1=2(k-n)-1
$$
This cannot equal 0 as $2(k-n)$ is even and 1 is odd. Hence there's at least one element of order 2 in $G$, which concludes the proof.
\end{proof}



\paragraph{Exercise 2.1.21} Show that a group of order 5 must be abelian.
\begin{proof}
    Suppose $G$ is a group of order 5 which is not abelian. Then there exist two non-identity elements $a, b \in G$ such that $a * b \neq$ $b * a$. Further we see that $G$ must equal $\{e, a, b, a * b, b * a\}$. To see why $a * b$ must be distinct from all the others, not that if $a *$ $b=e$, then $a$ and $b$ are inverses and hence $a * b=b * a$.
Contradiction. If $a * b=a$ (or $=b$ ), then $b=e$ (or $a=e$ ) and $e$ commutes with everything. Contradiction. We know by supposition that $a * b \neq b * a$. Hence all the elements $\{e, a, b, a * b, b * a\}$ are distinct.

Now consider $a^2$. It can't equal $a$ as then $a=e$ and it can't equal $a * b$ or $b * a$ as then $b=a$. Hence either $a^2=e$ or $a^2=b$.
Now consider $a * b * a$. It can't equal $a$ as then $b * a=e$ and hence $a * b=b * a$. Similarly it can't equal $b$. It also can't equal $a * b$ or $b * a$ as then $a=e$. Hence $a * b * a=e$.

So then we additionally see that $a^2 \neq e$ because then $a^2=e=$ $a * b * a$ and consequently $a=b * a$ (and hence $b=e$ ). So $a^2=b$. But then $a * b=a * a^2=a^2 * a=b * a$. Contradiction.
Hence starting with the assumption that there exists an order 5 abelian group $G$ leads to a contradiction. Thus there is no such group.
\end{proof}



\paragraph{Exercise 2.1.26} If $G$ is a finite group, prove that, given $a \in G$, there is a positive integer $n$, depending on $a$, such that $a^n = e$.
\begin{proof}
    Because there are only a finite number of elements of $G$, it's clear that the set $\left\{a, a^2, a^3, \ldots\right\}$ must be a finite set and in particular, there should exist some $i$ and $j$ such that $i \neq j$ and $a^i=a^j$. WLOG suppose further that $i>j$ (just reverse the roles of $i$ and $j$ otherwise). Then multiply both sides by $\left(a^j\right)^{-1}=a^{-j}$ to get
$$
a^i * a^{-j}=a^{i-j}=e
$$
Thus the $n=i-j$ is a positive integer such that $a^n=e$.
\end{proof}



\paragraph{Exercise 2.1.27} If $G$ is a finite group, prove that there is an integer $m > 0$ such that $a^m = e$ for all $a \in G$.
\begin{proof}
    Let $n_1, n_2, \ldots, n_k$ be the orders of all $k$ elements of $G=$ $\left\{a_1, a_2, \ldots, a_k\right\}$. Let $m=\operatorname{lcm}\left(n_1, n_2, \ldots, n_k\right)$. Then, for any $i=$ $1, \ldots, k$, there exists an integer $c$ such that $m=n_i c$. Thus
$$
a_i^m=a_i^{n_i c}=\left(a_i^{n_i}\right)^c=e^c=e
$$
Hence $m$ is a positive integer such that $a^m=e$ for all $a \in G$.
\end{proof}



\paragraph{Exercise 2.2.3} If $G$ is a group in which $(a b)^{i}=a^{i} b^{i}$ for three consecutive integers $i$, prove that $G$ is abelian.
\begin{proof}
    Let $G$ be a group, $a, b \in G$ and $i$ be any integer. Then from given condition,
$$
\begin{aligned}
(a b)^i & =a^i b^i \\
(a b)^{i+1} & =a^{i+1} b^{i+1} \\
(a b)^{i+2} & =a^{i+2} b^{i+2}
\end{aligned}
$$
From first and second, we get
$$
a^{i+1} b^{i+1}=(a b)^i(a b)=a^i b^i a b \Longrightarrow b^i a=a b^i
$$
From first and third, we get
$$
a^{i+2} b^{i+2}=(a b)^i(a b)^2=a^i b^i a b a b \Longrightarrow a^2 b^{i+1}=b^i a b a
$$
This gives
$$
a^2 b^{i+1}=a\left(a b^i\right) b=a b^i a b=b^i a^2 b
$$
Finally, we get
$$
b^i a b a=b^i a^2 b \Longrightarrow b a=a b
$$
This shows that $G$ is Abelian.
\end{proof}



\paragraph{Exercise 2.2.5} Let $G$ be a group in which $(a b)^{3}=a^{3} b^{3}$ and $(a b)^{5}=a^{5} b^{5}$ for all $a, b \in G$. Show that $G$ is abelian.
\begin{proof}
 We have
$$
\begin{aligned}
& (a b)^3=a^3 b^3, \text { for all } a, b \in G \\
\Longrightarrow & (a b)(a b)(a b)=a\left(a^2 b^2\right) b \\
\Longrightarrow & a(b a)(b a) b=a\left(a^2 b^2\right) b \\
\Longrightarrow & (b a)^2=a^2 b^2, \text { by cancellation law. }
\end{aligned}
$$
Again,
$$
\begin{aligned}
& (a b)^5=a^5 b^5, \text { for all } a, b \in G \\
\Longrightarrow & (a b)(a b)(a b)(a b)(a b)=a\left(a^4 b^4\right) b \\
\Longrightarrow & a(b a)(b a)(b a)(b a) b=a\left(a^4 b^4\right) b \\
\Longrightarrow & (b a)^4=a^4 b^4, \text { by cancellation law. }
\end{aligned}
$$
Now by combining two cases we have
$$
\begin{aligned}
& (b a)^4=a^4 b^4 \\
\Longrightarrow & \left((b a)^2\right)^2=a^2\left(a^2 b^2\right) b^2 \\
\Longrightarrow & \left(a^2 b^2\right)^2=a^2\left(a^2 b^2\right) b^2 \\
\Longrightarrow & \left(a^2 b^2\right)\left(a^2 b^2\right)=a^2\left(a^2 b^2\right) b^2 \\
\Longrightarrow & a^2\left(b^2 a^2\right) b^2=a^2\left(a^2 b^2\right) b^2 \\
\Longrightarrow & b^2 a^2=a^2 b^2, \text { by cancellation law. } \\
\Longrightarrow & b^2 a^2=(b a)^2, \text { since }(b a)^2=a^2 b^2 \\
\Longrightarrow & b(b a) a=(b a)(b a) \\
\Longrightarrow & b(b a) a=b(a b) a \\
\Longrightarrow & b a=a b, \text { by cancellation law. }
\end{aligned}
$$
It follows that, $a b=b a$ for all $a, b \in G$. Hence $G$ is abelian
\end{proof}



\paragraph{Exercise 2.2.6c} Let $G$ be a group in which $(a b)^{n}=a^{n} b^{n}$ for some fixed integer $n>1$ for all $a, b \in G$. For all $a, b \in G$, prove that $\left(a b a^{-1} b^{-1}\right)^{n(n-1)}=e$. 
\begin{proof}
    We start with the following two intermediate results.
(1) $(a b)^{n-1}=b^{n-1} a^{n-1}$.
(2) $a^n b^{n-1}=b^{n-1} a^n$.
To prove (1), notice by the given condition for all $a, b \in G$
$(b a)^n=b^n a^n$, for some fixed integers $n>1$.
Then,
$(b a)^n=b^n a^n \Longrightarrow b .(a b)(a b) \ldots .(a b) . a=b\left(b^{n-1} a^{n-1}\right) a$, where $(a b)$ occurs $n-1$ times $\Longrightarrow(a b)^{n-1}=b^{n-1} a^{n-1}$, by cancellation law.
Hence, for all $a, b \in G$
$$
(a b)^{n-1}=b^{n-1} a^{n-1} .
$$
To prove (2), notice by the given condition for all $a, b \in G$
$(b a)^n=b^n a^n$, for some fixed integers $n>1$.
Then we have
$$
\begin{aligned}
& (b a)^n=b^n a^n \\
\Longrightarrow & b \cdot(a b)(a b) \ldots(a b) \cdot a=b\left(b^{n-1} a^{n-1}\right) a, \text { where }(a b) \text { occurs } n-1 \text { times } \\
\Longrightarrow & (a b)^{n-1}=b^{n-1} a^{n-1}, \text { by cancellation law } \\
\Longrightarrow & (a b)^{n-1}(a b)=\left(b^{n-1} a^{n-1}\right)(a b) \\
\Longrightarrow & (a b)^n=b^{n-1} a^n b \\
\Longrightarrow & a^n b^n=b^{n-1} a^n b, \text { given condition } \\
\Longrightarrow & a^n b^{n-1}=b^{n-1} a^n, \text { by cancellation law. }
\end{aligned}
$$
Therefore for all $a, b \in G$ we have
$$
a^n b^{n-1}=b^{n-1} a^n
$$
In order to show that
$$
\left(a b a^{-1} b^{-1}\right)^{n(n-1)}=e, \text { for all } a, b \in G
$$
it is enough to show that
$$
(a b)^{n(n-1)}=(b a)^{n(n-1)}, \quad \forall x, y \in G .
$$
Step 3
This is because of
$$
\begin{aligned}
(a b)^{n(n-1)}=(b a)^{n(n-1)} & \left.\Longrightarrow(b a)^{-1}\right)^{n(n-1)}(a b)^{n(n-1)}=e \\
& \Longrightarrow\left(a^{-1} b^{-1}\right)^{n(n-1)}(a b)^{n(n-1)}=e \\
& \Longrightarrow\left(\left(a^{-1} b^{-1}\right)^n\right)^{n-1}\left((a b)^n\right)(n-1)=e \\
& \Longrightarrow\left((a b)^n\left(a^{-1} b^{-1}\right)^n\right)^{n-1}=e, \text { by }(1) \\
& \Longrightarrow\left(a b a^{-1} b^{-1}\right)^{n(n-1)}=e, \text { ( given condition) }
\end{aligned}
$$
Now, it suffices to show that
$$
(a b)^{n(n-1)}=(b a)^{n(n-1)}, \quad \forall x, y \in G .
$$
Now, we have
$$
\begin{aligned}
(a b)^{n(n-1)} & =\left(a^n b^n\right)^{n-1}, \text { by the given condition } \\
& =\left(a^n b^{n-1} b\right)^{n-1} \\
& =\left(b^{n-1} a^n b\right)^{n-1}, \text { by }(2) \\
& =\left(a^n b\right)^{n-1}\left(b^{n-1}\right)^{n-1}, \text { by }(1) \\
& =b^{n-1}\left(a^n\right)^{n-1}\left(b^{n-1}\right)^{n-1}, \text { by }(1) \\
& =\left(b^{n-1}\left(a^{n-1}\right)^n\right)\left(b^{n-1}\right)^{n-1} \\
& =\left(a^{n-1}\right)^n b^{n-1}\left(b^{n-1}\right)^{n-1}, \text { by }(2) \\
& =\left(a^{n-1}\right)^n\left(b^{n-1}\right)^n \\
& =\left(a^{n-1} b^{n-1}\right)^n, \text { by }(1) \\
& =(b a)^{n(n-1)}, \text { by }(1) .
\end{aligned}
$$
This completes our proof.
\end{proof}



\paragraph{Exercise 2.3.17} If $G$ is a group and $a, x \in G$, prove that $C\left(x^{-1} a x\right)=x^{-1} C(a) x$
\begin{proof}
    Note that
$$
C(a):=\{x \in G \mid x a=a x\} .
$$
Let us assume $p \in C\left(x^{-1} a x\right)$. Then,
$$
\begin{aligned}
& p\left(x^{-1} a x\right)=\left(x^{-1} a x\right) p \\
\Longrightarrow & \left(p x^{-1} a\right) x=x^{-1}(a x p) \\
\Longrightarrow & x\left(p x^{-1} a\right)=(a x p) x^{-1} \\
\Longrightarrow & \left(x p x^{-1}\right) a=a\left(x p x^{-1}\right) \\
\Longrightarrow & x p x^{-1} \in C(a) .
\end{aligned}
$$
Therefore,
$$
p \in C\left(x^{-1} a x\right) \Longrightarrow x p x^{-1} \in C(a) .
$$
Thus,
$$
C\left(x^{-1} a x\right) \subset x^{-1} C(a) x .
$$
Let us assume
$$
q \in x^{-1} C(a) x .
$$
Then there exists an element $y$ in $C(a)$ such that
$$
q=x^{-1} y x
$$
Now,
$$
y \in C(a) \Longrightarrow y a=a y .
$$
Also,
$$
q\left(x^{-1} a x\right)=\left(x^{-1} y x\right)\left(x^{-1} a x\right)=x^{-1}(y a) x=x^{-1}(y a) x=\left(x^{-1} y x\right)\left(x^{-1} a x\right)=\left(x^{-1} y x\right) q .
$$
Therefore,
$$
q\left(x^{-1} a x\right)=\left(x^{-1} y x\right) q
$$
So,
$$
q \in C\left(x^{-1} a x\right) .
$$
Consequently we have
$$
x^{-1} C(a) x \subset C\left(x^{-1} a x\right) .
$$
It follows from the aforesaid argument
$$
C\left(x^{-1} a x\right)=x^{-1} C(a) x .
$$
This completes the proof.
\end{proof}


\paragraph{Exercise 2.3.16} If a group $G$ has no proper subgroups, prove that $G$ is cyclic of order $p$, where $p$ is a prime number.
\begin{proof}
    Case-1: $G=(e), e$ being the identity element in $G$. Then trivially $G$ is cyclic.
    Case-2: $G \neq(e)$. Then there exists an non-identity element in $G.$ Let us consider an non-identity element in $G$, say $a\neq (e)$. Now look at the cyclic subgroup generated by $a$, that is, $\langle a\rangle$. Since
    $a\neq (e) \in G,\langle a\rangle$ is a subgroup of $G$.
If $G \neq\langle a\rangle$ then $\langle a\rangle$ is a proper non-trivial subgroup of $G$, which is an impossibility. Therfore we must have
$$
G=\langle a\rangle .
$$
This implies, $G$ is a cyclic group generated by $a$. Then it follows that every non-identity element of $G$ is a generator of $G$. Now we claim that $G$ is finite.
\end{proof}



\paragraph{Exercise 2.4.36} If $a > 1$ is an integer, show that $n \mid \varphi(a^n - 1)$, where $\phi$ is the Euler $\varphi$-function.
\begin{proof}
    Proof: We have $a>1$. First we propose to prove that
$$
\operatorname{Gcd}\left(a, a^n-1\right)=1 .
$$
If possible, let us assume that
$\operatorname{Gcd}\left(a, a^n-1\right)=d$, where $d>1$.
Then
$d$ divides $a$ as well as $a^n-1$.
Now,
$d$ divides $a \Longrightarrow d$ divides $a^n$.
This is an impossibility, since $d$ divides $a^n-1$ by our assumption. Consequently, $d$ divides 1 , which implies $d=1$. Hence we are contradict to the fact that $d>1$. Therefore
$$
\operatorname{Gcd}\left(a, a^n-1\right)=1 .
$$
Then $a \in U_{a^n-1}$, where $U_n$ is a group defined by
$$
U_n:=\left\{\bar{a} \in \mathbb{Z}_n \mid \operatorname{Gcd}(a, n)=1\right\} .
$$
We know that order of an element divides the order of the group. Here order of the group $U_{a^n-1}$ is $\phi\left(a^n-1\right)$ and $a \in U_{a^n-1}$. This follows that $\mathrm{o}(a)$ divides $\phi\left(a^n-1\right)$.
\end{proof}


\paragraph{Exercise 2.5.23} Let $G$ be a group such that all subgroups of $G$ are normal in $G$. If $a, b \in G$, prove that $ba = a^jb$ for some $j$.
\begin{proof}
    Let $G$ be a group where each subgroup is normal in $G$. let $a, b \in G$.
$$
\begin{aligned}
    \langle a\rangle\triangleright  G  &\Rightarrow b \cdot\langle a\rangle=\langle a\rangle \cdot b . \\
& \Rightarrow \quad b \cdot a=a^j \cdot b \text { for some } j \in \mathbb{Z}.
\end{aligned}
$$
(hence for $a_1 b \in G \quad a^j b=b \cdot a$ ).
\end{proof}



\paragraph{Exercise 2.5.30} Suppose that $|G| = pm$, where $p \nmid m$ and $p$ is a prime. If $H$ is a normal subgroup of order $p$ in $G$, prove that $H$ is characteristic.
\begin{proof}
    Let $G$ be a group of order $p m$, such that $p \nmid m$. Now, Given that $H$ is a normal subgroup of order $p$. Now we want to prove that $H$ is a characterestic subgroup, that is $\phi(H)=H$ for any automorphism $\phi$ of $G$. Now consider $\phi(H)$. Clearly $|\phi(H)|=p$. Suppose $\phi(H) \neq H$, then $H \cap \phi (H)=\{ e\}$. Consider $H \phi(H)$, this is a subgroup of $G$ as $H$ is normal. Also $|H \phi(H)|=p^2$. By lagrange's theorem then $p^2 \mid$ $p m \Longrightarrow p \mid m$ - contradiction. So $\phi(H)=H$, and $H$ is characterestic subgroup of $G$
\end{proof}



\paragraph{Exercise 2.5.31} Suppose that $G$ is an abelian group of order $p^nm$ where $p \nmid m$ is a prime.  If $H$ is a subgroup of $G$ of order $p^n$, prove that $H$ is a characteristic subgroup of $G$.
\begin{proof}
    Let $G$ be an abelian group of order $p^n m$, such that $p \nmid m$. Now, Given that $H$ is a subgroup of order $p^n$. Since $G$ is abelian $H$ is normal. Now we want to prove that $H$ is a characterestic subgroup, that is $\phi(H)=H$ for any automorphism $\phi$ of $G$. Now consider $\phi(H)$. Clearly $|\phi(H)|=p^n$. Suppose $\phi(H) \neq H$, then $|H \cap \phi(H)|=p^s$, where $s<n$. Consider $H \phi(H)$, this is a subgroup of $G$ as $H$ is normal. Also $|H \phi(H)|=\frac{|H||\phi(H)|}{|H \cap \phi(H)|}=\frac{p^{2 n}}{p^s}=p^{2 n-s}$, where $2 n-s>n$. By lagrange's theorem then $p^{2 n-s}\left|p^n m \Longrightarrow p^{n-s}\right| m \Longrightarrow p \mid m$-contradiction. So $\phi(H)=H$, and $H$ is characterestic subgroup of $G$.
\end{proof}



\paragraph{Exercise 2.5.37} If $G$ is a nonabelian group of order 6, prove that $G \simeq S_3$.
\begin{proof}
    Suppose $G$ is a non-abelian group of order 6 . We need to prove that $G \cong S_3$. Since $G$ is non-abelian, we conclude that there is no element of order 6. Now all the nonidentity element has order either 2 or 3 . All elements cannot be order 3 .This is because except the identity elements there are 5 elements, but order 3 elements occur in pair, that is $a, a^2$, both have order 3 , and $a \neq a^2$. So, this is a contradiction, as there are only 5 elements. So, there must be an element of order 2 . All elements of order 2 will imply that $G$ is abelian, hence there is also element of order 3 . Let $a$ be an element of order 2 , and $b$ be an element of order 3 . So we have $e, a, b, b^2$, already 4 elements. Now $a b \neq e, b, b^2$. So $a b$ is another element distinct from the ones already constructed. $a b^2 \neq e, b, a b, b^2, a$. So, we have got another element distinct from the other. So, now $ G=\left\{e, a, b, b^2, a b, a b^2\right\}$. Also, ba must be equal to one of these elements. But $b a \neq e, a, b, b^2$. Also if $b a=a b$, the group will become abelian. so $b a=a b^2$. So what we finally get is $G=\left\langle a, b \mid a^2=e=b^3, b a=a b^2\right\rangle$. Hence $G \cong S_3$.
\end{proof}



\paragraph{Exercise 2.5.43} Prove that a group of order 9 must be abelian.
\begin{proof}
    We use the result from problem 40 which is as follows: Suppose $G$ is a group, $H$ is a subgroup and $|G|=n$ and $n \nmid\left(i_G(H)\right) !$. Then there exists a normal subgroup $\$ K \backslash$ neq $\{$ e $\} \$$ and $K \subseteq H$.
So, we have now a group $G$ of order 9. Suppose that $G$ is cyclic, then $G$ is abelian and there is nothing more to prove. Suppose that $G$ s not cyclic,then there exists an element $a$ of order 3 , and $A=\langle a\rangle$. Now $i_G(A)=3$, now $9 \nmid 3$ !, hence by the above result there is a normal subgroup $K$, non-trivial and $K \subseteq A$. But $|A|=3$, a prime order subgroup, hence has no non-trivial subgroup, so $K=A$. So $A$ is normal subgroup. Now since $G$ is not cyclic any non-identity element is of order 3.So Let $a(\neq$ $e) \in G$.Consider $A=\langle a\rangle$. As shown before $A$ is normal. $a$ commutes with any if its powers. Now Let $b \in G$ such that $b \notin A$. Then $b a b^{-1} \in A$ and hence $b a b^{-1}=a^i$.This implies $a=b^3 a b^{-3}=a^{i^3} \Longrightarrow a^{i^3-1}=e$. So, 3 divides $i^3-1$. Also by fermat's little theorem 3 divides $i^2-1$.So 3 divides $i-1$. But $0 \leq i \leq 2$. So $i=1$, is the only possibility and hence $a b=b a$. So $a \in Z(G)$ as $b$ was arbitrary. Since $a$ was arbitrary $G=Z(G)$. Hence $G$ is abelian.
\end{proof}



\paragraph{Exercise 2.5.44} Prove that a group of order $p^2$, $p$ a prime, has a normal subgroup of order $p$.
\begin{proof}
    We use the result from problem 40 which is as follows: Suppose $G$ is a group, $H$ is a subgroup and $|G|=n$ and $n \nmid\left(i_G(H)\right) !$. Then there exists a normal subgroup $K \neq \{ e \}$ and $K \subseteq H$.

So, we have now a group $G$ of order $p^2$. Suppose that the group is cyclic, then it is abelian and any subgroup of order $p$ is normal. Now let us suppose that $G$ is not cyclic, then there exists an element $a$ of order $p$, and $A=\langle a\rangle$. Now $i_G(A)=p$, so $p^2 \nmid p$ ! , hence by the above result there is a normal subgroup $K$, non-trivial and $K \subseteq A$. But $|A|=p$, a prime order subgroup, hence has no non-trivial subgroup, so $K=A$. so $A$ is normal subgroup.
\end{proof}



\paragraph{Exercise 2.5.52} Let $G$ be a finite group and $\varphi$ an automorphism of $G$ such that $\varphi(x) = x^{-1}$ for more than three-fourths of the elements of $G$. Prove that $\varphi(y) = y^{-1}$ for all $y \in G$, and so $G$ is abelian.
\begin{proof}
Let us start with considering $b$ to be an arbitrary element in $A$. 

1. Show that $\left|A \cap\left(b^{-1} A\right)\right|>\frac{|G|}{2}$, where
$$
b^{-1} A=\left\{b^{-1} a \mid a \in A\right\}
$$
First notice that if we consider a map $f: A \rightarrow b^{-1} A$ defined by $f(a)=b^{-1} a$, for all $a \in A$, then $f$ is a 1-1 map and so $\left|b^{-1} A\right| \geq|A|>\frac{3}{4}|G|$. Now using inclusion-exclusion principle we have
$$
\left|A \cap\left(b^{-1} A\right)\right|=|A|+\left|b^{-1} A\right|-\left|A \cup\left(b^{-1} A\right)\right|>\frac{3}{4}|G|+\frac{3}{4}|G|-|G|=\frac{1}{2}|G|
$$
2. Argue that $A \cap\left(b^{-1} A\right) \subseteq C(b)$, where $C(b)$ is the centralizer of $b$ in $G$.

Suppose $x \in A \cap\left(b^{-1} A\right)$, that means, $x \in A$ and $x \in b^{-1} A$. Thus there exist an element $a \in A$ such that $x=$ $b^{-1} a$, which gives us $x b=a \in A$. Now notice that $x, b \in A$ and $x b \in A$, therefore we get
$$
\phi(x b)=(x b)^{-1} \Longrightarrow \phi(x) \phi(b)=(x b)^{-1} \Longrightarrow x^{-1} b^{-1}=b^{-1} x^{-1} \Longrightarrow x b=b x
$$
Therefore, we get $x b=b x$, for any $x \in A \cap\left(b^{-1} A\right)$, that means, $x \in C(b)$.

3. Argue that $C(b)=G$.
We know that centralizer of an element in a group $G$ is a subgroup (See Page 53). Therefore $C(b)$ is a subgroup of $G$. From statements $\mathbf{1}$ and $\mathbf{2}$, we have
$$
|C(b)| \geq\left|A \cap\left(b^{-1} A\right)\right|>\frac{|G|}{2}
$$
We need to use the following remark to argue $C(b)=G$ from the above step.
Remark. Let $G$ be a finite group and $H$ be a subgroup with more then $|G| / 2$ elements then $H=G$.

Proof of Remark. Suppose $|H|=p$ Then by Lagrange Theorem, there exist an $n \in \mathbb{N}$, such that $|G|=n p$, as $|H|$ divide $|G|$. Now by hypothesis $p>\frac{G]}{2}$ gives us,
$$
p>\frac{|G|}{2} \Longrightarrow n p>\frac{n|G|}{2} \Longrightarrow n<2 \Longrightarrow n=1
$$
Therefore we get $H=G$.

Now notice that $C(b)$ is a subgroup of $G$ with $C(b)$ having more than $|G| / 2$ elements. Therefore, $C(b)=G$.

4. Show that $A \in Z(G)$.

We know that $x \in Z(G)$ if and only if $C(a)=G$. Now notice that, for any $b \in A$ we have $C(b)=G$. Therefore, every element of $A$ is in the center of $G$, that means, $A \subseteq Z(G)$.

5. 5how that $Z(G)=G$.

As it is given that $|A|>\frac{3|G|}{4}$ and $A \leq|Z(G)|$, therefore we get
$$
|Z(G)|>\frac{3}{4}|G|>\frac{1}{2}|G| .
$$
As $Z(G)$ is a subgroup of $G$, so by the above Remark we have $Z(G)=G$. Hence $G$ is abelian.

6. Finally show that $A=G$.

First notice that $A$ is a subgroup of $G$. To show this let $p, q \in A$. Then we have
$$
\phi(p q)=\phi(p) \phi(q)=p^{-1} q^{-1}=(q p)^{-1}=(p q)^{-1}, \quad \text { As } G \text { is abelian. }
$$
Therefore, $p q \in A$ and so we have $A$ is a subgroup of $G$. Again by applying the above remark. we get $A=G$. Therefore we have
$$
\phi(y)=y^{-1}, \quad \text { for all } y \in G
$$

\end{proof}



\paragraph{Exercise 2.6.15} If $G$ is an abelian group and if $G$ has an element of order $m$ and one of order $n$, where $m$ and $n$ are relatively prime, prove that $G$ has an element of order $mn$.
\begin{proof}
Let $G$ be an abelian group, and let $a$ and $b$ be elements in $G$ of order $m$ and $n$, respectively, where $m$ and $n$ are relatively prime. We will show that the product $ab$ has order $mn$ in $G$, which will prove that $G$ has an element of order $mn$.

To show that $ab$ has order $mn$, let $k$ be the order of $ab$ in $G$. We have $a^m = e$, $b^n = e$, and $(ab)^k = e$, where $e$ denotes the identity element of $G$. Since $G$ is abelian, we have
$$(ab)^{mn} = a^{mn}b^{mn} = e \cdot e = e.$$
Thus, $k$ is a divisor of $mn$.

Now, observe that $a^k = b^{-k}$. Since $m$ and $n$ are relatively prime, there exist integers $x$ and $y$ such that $mx + ny = 1$. Taking $kx$ on both sides of the equation, we get $a^{kx} = b^{-kx}$, or equivalently, $(a^k)^x = (b^k)^{-x}$. It follows that $a^{kx} = (a^m)^{xny} = e$, and similarly, $b^{ky} = (b^n)^{mxk} = e$. Therefore, $m$ divides $ky$ and $n$ divides $kx$. Since $m$ and $n$ are relatively prime, it follows that $mn$ divides $k$. Hence, $k = mn$, and $ab$ has order $mn$ in $G$. This completes the proof.
\end{proof}

\paragraph{Exercise 2.7.7} If $\varphi$ is a homomorphism of $G$ onto $G'$ and $N \triangleleft G$, show that $\varphi(N) \triangleleft G'$.
\begin{proof}
We first claim that $\varphi(N)$ is a subgroup of $G'$. To see this, note that since $N$ is a subgroup of $G$, the identity element $e_G$ of $G$ belongs to $N$. Therefore, the element $\varphi(e_G) \in \varphi(N)$, so $\varphi(N)$ is a non-empty subset of $G'$.

Now, let $a', b' \in \varphi(N)$. Then there exist elements $a, b \in N$ such that $\varphi(a) = a'$ and $\varphi(b) = b'$. Since $N$ is a subgroup of $G$, we have $a, b \in N$, so $ab^{-1} \in N$. Thus, we have
$$\varphi(ab^{-1}) = \varphi(a) \varphi(b^{-1}) = a'b'^{-1} \in \varphi(N),$$
which shows that $a', b' \in \varphi(N)$ implies $a'b'^{-1} \in \varphi(N)$. Therefore, $\varphi(N)$ is a subgroup of $G'$.

Next, we will show that $\varphi(N)$ is a normal subgroup of $G'$. Let $\varphi(N) = N'$, a subgroup of $G'$. Let $x' \in G'$ and $h' \in N'$. Since $\varphi$ is onto, there exist elements $x \in G$ and $h \in N$ such that $\varphi(x) = x'$ and $\varphi(h) = h'$.

Since $N$ is a normal subgroup of $G$, we have $xhx^{-1} \in N$. Thus,
$$\varphi(xhx^{-1}) = \varphi(x)\varphi(h)\varphi(x^{-1}) = x'h'x'^{-1} \in \varphi(N),$$
which shows that $x' \in G'$ and $h' \in N'$ implies $x'h'x'^{-1} \in \varphi(N)$. Therefore, $\varphi(N)$ is a normal subgroup of $G'$. This completes the proof.
\end{proof}

\paragraph{Exercise 2.8.12} Prove that any two nonabelian groups of order 21 are isomorphic.
\begin{proof}
    By Cauchy's theorem we have that if $G$ is a group of order 21 then it has an element $a$ of order 3 and an element $b$ of order 7. By exercise 2.5.41 we have that the subgroup generated by $b$ is normal, so there is some $i=0,1,2,3,4,5,6$ such that $a b a^{-1}=b^i$. We know $i \neq$ 0 since that implies $a b=a$ and so that $b=e$, a contradiction, and we know $i \neq 1$ since then $a b=b a$ and this would imply $G$ is abelian, which we are assuming is not the case.
Now, $a$ has order 3 so we must have $b=a^3 b a^{-3}=b^{i^3}$ mod 7 , and so $i$ is restricted by the modular equation $i^3 \equiv 1 \bmod 7$
\begin{center}
\begin{tabular}{|c|c|}
\hline$x$ & $x^3 \bmod 7$ \\
\hline 2 & 1 \\
\hline 3 & 6 \\
\hline 4 & 1 \\
\hline 5 & 6 \\
\hline 6 & 6 \\
\hline
\end{tabular}
\end{center}
Therefore the only options are $i=2$ and $i=4$. Now suppose $G$ is such that $a b a^{-1}=b^2$ and let $G^{\prime}$ be another group of order 21 with an element $c$ of order 3 and an element $d$ of order 7 such that $c d c^{-1}=d^4$. We now prove that $G$ and $G^{\prime}$ are isomorphic. Define
$$
\begin{aligned}
\phi: G & \rightarrow G^{\prime} \\
a & \mapsto c^{-1} \\
b & \mapsto d
\end{aligned}
$$
since $a$ and $c^{-1}$ have the same order and $b$ and $d$ have the same order this is a well defined function. Since
$$
\begin{aligned}
\phi(a) \phi(b) \phi(a)^{-1} & =c^{-1} d c \\
& =\left(c d^{-1} c^{-1}\right)^{-1} \\
& =\left(d^{-4}\right)^{-1} \\
& =d^4 \\
& =\left(d^2\right)^2 \\
& =\phi(b)^2
\end{aligned}
$$
$\phi$ is actually a homomorphism. For any $c^i d^j \in G^{\prime}$ we have $\phi\left(a^{-i} b^j\right)=c^i d^j$ so $\phi$ is onto and $\phi\left(a^i b^j\right)=c^{-i} d^j=e$ only if $i=j=0$, so $\phi$ is 1-to-l. Therefore $G$ and $G^{\prime}$ are isomorphic and so up to isomorphism there is only one nonabelian group of order 21 .
\end{proof}


\paragraph{Exercise 2.8.15} Prove that if $p > q$ are two primes such that $q \mid p - 1$, then any two nonabelian groups of order $pq$ are isomorphic.
\begin{proof}
    For a nonabelian group of order $p q$, the structure of the group $G$ is set by determining the relation $a b a^{-1}=b^{k^{\frac{p-1}{q}}}$ for some generator $k$ of the cyclic group. Here we are using the fact that $k^{\frac{p-1}{q}}$ is a generator for the unique subgroup of order $q$ in $U_p$ (a cyclic group of order $m$ has a unique subgroup of order $d$ for each divisor $d$ of $m$ ). The other possible generators of this subgroup are $k^{\frac{l(p-1)}{q}}$ for each $1 \leq l \leq q-1$, so these give potentially new group structures. Let $G^{\prime}$ be a group with an element $c$ of order $q$, an element $d$ of order $p$ with structure defined by the relation $c d c^{-1}=d^{k^{\frac{l(p-1)}{q}}}$. We may then define
$$
\begin{aligned}
\phi: G^{\prime} & \rightarrow G \\
c & \mapsto a^l \\
d & \mapsto b
\end{aligned}
$$
since $c$ and $a^l$ have the same order and $b$ and $d$ have the same order this is a well defined function.
Since
$$
\begin{aligned}
\phi(c) \phi(d) \phi(c)^{-1} & =a^l b a^{-l} \\
& =b^{\left(k^{\frac{p-1}{q}}\right)^l} \\
& =b^{k^{\frac{l(p-1)}{q}}} \\
& =\phi(d)^{k^{\frac{l(p-1)}{q}}}
\end{aligned}
$$
$\phi\left(c^i d^j\right)=a^{l i} b^j=e$ only if $i=j=0$, so $\phi$ is 1-to-l. Therefore $G$ and $G^{\prime}$ are isomorphic and so up to isomorphism there is only one nonabelian group of order $p q$.
\end{proof}



\paragraph{Exercise 2.9.2} If $G_1$ and $G_2$ are cyclic groups of orders $m$ and $n$, respectively, prove that $G_1 \times G_2$ is cyclic if and only if $m$ and $n$ are relatively prime.
\begin{proof}
    The order of $G \times H$ is $n$. $m$. Thus, $G \times H$ is cyclic iff it has an element with order n. $m$. Suppose $\operatorname{gcd}(n . m)=1$. This implies that $g^m$ has order $n$, and analogously $h^n$ has order $m$. That is, $g \times h$ has order $n$. $m$, and therefore $G \times H$ is cyclic.

Suppose now that $\operatorname{gcd}(n . m)>1$. Let $g^k$ be an element of $G$ and $h^j$ be an element of $H$. Since the lowest common multiple of $n$ and $m$ is lower than the product $n . m$, that is, $\operatorname{lcm}(n, m)<n$. $m$, and since $\left(g^k\right)^{l c m(n, m)}=e_G,\left(h^j\right)^{l c m(n, m)}=e_H$, we have $\left(g^k \times h^j\right)^{l c m(n, m)}=e_{G \times H}$. It follows that every element of $G \times H$ has order lower than $n . m$, and therefore $G \times H$ is not cyclic.
\end{proof}



\paragraph{Exercise 2.10.1} Let $A$ be a normal subgroup of a group $G$, and suppose that $b \in G$ is an element of prime order $p$, and that $b \not\in A$. Show that $A \cap (b) = (e)$.
\begin{proof}
If $b \in G$ has order $p$, then $(b)$ is a cyclic group of order $p$. Since $A$ is a subgroup of $G$, we have $A \cap (b)$ is a subgroup of $G$. Also, $A \cap (b) \subseteq (b)$. So $A \cap (b)$ is a subgroup of $(b)$. Since $(b)$ is a cyclic group of order $p$, the only subgroups of $(b)$ are $(e)$ and $(b)$ itself.

Therefore, either $A \cap (b) = (e)$ or $A \cap (b) = (b)$. If $A \cap (b) = (e)$, then we are done. Otherwise, if $A \cap (b) = (b)$, then $A \subseteq (b)$. Since $A$ is a subgroup of $G$ and $A \subseteq (b)$, it follows that $A$ is a subgroup of $(b)$.

Since the only subgroups of $(b)$ are $(e)$ and $(b)$ itself, we have either $A = (e)$ or $A = (b)$. If $A = (e)$, then $A \cap (b) = (e)$ and we are done. But if $A = (b)$, then $b \in A$ as $b \in (b)$, which contradicts our hypothesis that $b \notin A$. So $A \neq (b)$.

Hence $A \cap (b) \neq (b)$. Therefore, $A \cap (b) = (e)$. This completes our proof.
\end{proof}



\paragraph{Exercise 2.11.6} If $P$ is a $p$-Sylow subgroup of $G$ and $P \triangleleft G$, prove that $P$ is the only $p$-Sylow subgroup of $G$.
\begin{proof}
    let $G$ be a group and $P$ a sylow-p subgroup. Given $P$ is normal. By sylow second theorem the sylow-p subgroups are conjugate. Let $K$ be any other sylow-p subgroup. Then there exists $g \in G$ such that $K=g P g^{-1}$. But since $P$ is normal $K=g P g^{-1}=P$. Hence the sylow-p subgroup is unique.
\end{proof}


\paragraph{Exercise 2.11.7} If $P \triangleleft G$, $P$ a $p$-Sylow subgroup of $G$, prove that $\varphi(P) = P$ for every automorphism $\varphi$ of $G$.
\begin{proof}
    Let $\phi$ be an automorphism of $G$. Let $P$ be a normal sylow p-subgroup. $\phi(P)$ is also a sylow-p subgroup. But since $P$ is normal, it is unique. Hence $\phi(P)=P$.
\end{proof}



\paragraph{Exercise 2.11.22} Show that any subgroup of order $p^{n-1}$ in a group $G$ of order $p^n$ is normal in $G$.
\begin{proof}
Proof: First we prove the following lemma.

\textbf{Lemma:} If $G$ is a finite $p$-group with $|G|>1$, then $Z(G)$, the center of $G$, has more than one element; that is, if $|G|=p^k$ with $k\geq 1$, then $|Z(G)|>1$.

\textit{Proof of the lemma:} Consider the class equation
$$
|G|=|Z(G)|+\sum_{a \notin Z(G)}[G: C(a)],
$$
where $C(a)$ denotes the centralizer of $a$ in $G$. If $G=Z(G)$, then the lemma is immediate. Suppose $Z(G)$ is a proper subset of $G$ and consider an element $a\in G$ such that $a\notin Z(G)$. Then $C(a)$ is a proper subgroup of $G$. Since $C(a)$ is a subgroup of a $p$-group, $[G:C(a)]$ is divisible by $p$ for all $a\notin Z(G)$. This implies that $p$ divides $|G|=|Z(G)|+\sum_{a\notin Z(G)} [G:C(a)]$.

Since $p$ also divides $|G|$, it follows that $p$ divides $|Z(G)|$. Hence, $|Z(G)|>1$. $\Box$

This proves our \textbf{lemma}.

We will prove the result by induction on $n$.
If $n=1$, the $G$ is a cyclic group of prime order and hence every subgroup of $G$ is normal in $G$. Thus, the result is true for $n=1$.
Suppose the result is true for all groups of order $p^m$, where $1 \leq m<n$.
Let $H$ be a subgroup of order $p^{n-1}$.
Consider $N(H)=\{g \in H: g H=H g\}$.
If $H \neq N(H)$, then $|N(H)|>p^{n-1}$. Thus, $|N(H)|=p^n$ and $N(H)=G$.
In this case $H$ is normal in $G$.
Let $H=N(H)$. Then $Z(G)$, the center of $G$, is a subset of $H$ and $Z(G) \neq$ $\{e\}$.
By Cauchy's theorem and the above Claim, there exists $a \in Z(G)$ such that $o(a)=p$.
Let $K=\langle a\rangle$, a cyclic group generated by $a$.
Then $K$ is a normal subgroup of $G$ of order $p$. Now, $|H / K|=p^{n-2}$ and $|G / K|=p^{n-1}$.
Thus, by induction hypothesis, $H / K$ is a normal subgroup of $G / K$.
\end{proof}



\paragraph{Exercise 3.2.21} If $\sigma, \tau$ are two permutations that disturb no common element and $\sigma \tau = e$, prove that $\sigma = \tau = e$.
\begin{proof}
    Note that $\sigma \tau=e$ can equivalentnly be phrased as $\tau$ being the inverse of $\sigma$. Our statement is then equivalent to the statement that an inverse of a nonidentity permutation disturbs at least one same element as that permutation. To prove this, let $\sigma$ be a nonidentity permutation, then let $\left(i_1 \cdots i_n\right)$ be a cycle in $\sigma$. Then we have that
$$
\sigma\left(i_1\right)=i_2, \sigma\left(i_2\right)=i_2, \ldots, \sigma\left(i_{n-1}\right)=i_n, \sigma\left(i_n\right)=i_1,
$$
but then also
$$
i_1=\tau\left(i_2\right), i_2=\tau\left(i_3\right), \ldots, i_{n-1}=\tau\left(i_n\right), i_n=\tau\left(i_1\right),
$$
i.e. its inverse disturbs $i_1, \ldots, i_n$.
\end{proof}



\paragraph{Exercise 4.1.19} Show that there is an infinite number of solutions to $x^2 = -1$ in the quaternions.
\begin{proof}
Let $x=a i+b j+c k$ then
$$
x^2=(a i+b j+c k)(a i+b j+c k)=-a^2-b^2-c^2=-1
$$
This gives $a^2+b^2+c^2=1$ which has infinitely many solutions for $-1<a, b, c<1$.
\end{proof}


\paragraph{Exercise 4.1.34} Let $T$ be the group of $2\times 2$ matrices $A$ with entries in the field $\mathbb{Z}_2$ such that $\det A$ is not equal to 0. Prove that $T$ is isomorphic to $S_3$, the symmetric group of degree 3.
\begin{proof}
    The order of $T$ is $2^4-2^3-2^2+2=6$; we now find those six matrices:
$$
\begin{array}{ll}
A_1=\left(\begin{array}{ll}
1 & 0 \\
0 & 1
\end{array}\right), & A_2=\left(\begin{array}{ll}
0 & 1 \\
1 & 0
\end{array}\right) \\
A_3=\left(\begin{array}{ll}
1 & 0 \\
1 & 1
\end{array}\right), & A_4=\left(\begin{array}{ll}
1 & 1 \\
0 & 1
\end{array}\right) \\
A_5=\left(\begin{array}{ll}
0 & 1 \\
1 & 1
\end{array}\right), & A_6=\left(\begin{array}{ll}
1 & 1 \\
1 & 0
\end{array}\right)
\end{array}
$$
with orders $1,2,2,2,3,3$ respectively.
Note that $S_3$ is composed of elements
$$
\text{ id, (1 2), (1 3), (2 3), (1 2 3), (1 3 2)} 
$$
with orders 1, 2, 2, 2, 3, 3 respectively. Also note that, by Problem 17 of generate $S_3$. We also have that $\left(\begin{array}{llll}1 & 3 & 2\end{array}\right)=\left(\begin{array}{llll}1 & 2 & 3\end{array}\right)\left(\begin{array}{lll}1 & 2 & 3\end{array}\right)$, that $\left(\begin{array}{lll}1 & 3\end{array}\right)=\left(\begin{array}{lll}1 & 2 & 3\end{array}\right)\left(\begin{array}{ll}1 & 2\end{array}\right)$, $\left(\begin{array}{ll}1 & 2\end{array}\right)\left(\begin{array}{lll}1 & 2 & 3\end{array}\right)=\left(\begin{array}{ll}2 & 3\end{array}\right)$ and $\left(\begin{array}{lll}1 & 2\end{array}\right)\left(\begin{array}{ll}1 & 2\end{array}\right)=\mathrm{id}$

Now we can check that $\tau\left(A_2\right)=\left(\begin{array}{ll}1 & 2\end{array}\right), \tau\left(A_5\right)=\left(\begin{array}{lll}1 & 2 & 3\end{array}\right)$ induces an isomorphism. We compute
$$
\begin{aligned}
& \tau\left(A_1\right)=\tau\left(A_2 A_2\right)=\tau\left(A_2\right) \tau\left(A_2\right)=\mathrm{id} \\
& \tau\left(A_3\right)=\tau\left(A_5 A_2\right)=\tau\left(A_5\right) \tau\left(A_2\right)=\left(\begin{array}{llll}
1 & 2 & 3
\end{array}\right)\left(\begin{array}{lll}
1 & 2
\end{array}\right)=\left(\begin{array}{ll}
1 & 3
\end{array}\right) \\
& \tau\left(A_4\right)=\tau\left(A_2 A_5\right)=\tau\left(A_2\right) \tau\left(A_5\right)=\left(\begin{array}{lll}
1 & 2
\end{array}\right)\left(\begin{array}{lll}
1 & 2 & 3
\end{array}\right)=\left(\begin{array}{ll}
2 & 3
\end{array}\right) \\
& \tau\left(A_6\right)=\tau\left(A_5 A_5\right)=\tau\left(A_5\right) \tau\left(A_5\right)=\left(\begin{array}{lll}
1 & 3 & 2
\end{array}\right)
\end{aligned}
$$
Thus we see that $\tau$ extendeds to an isomorphism, since $A_2$ and $A_5$ generate $T$, so that $\tau\left(A_i A_j\right)=\tau\left(A_i\right) \tau\left(A_j\right)$ follows from writing $A_i$ and $A_j$ in terms of $A_2$ and $A_5$ and using the equlities and relations shown above.
\end{proof}


\paragraph{Exercise 4.2.5} Let $R$ be a ring in which $x^3 = x$ for every $x \in R$. Prove that $R$ is commutative.
\begin{proof}
To begin with
$$
2 x=(2 x)^3=8 x^3=8 x .
$$
Therefore $6 x=0 \quad \forall x$.
Also
$$
(x+y)=(x+y)^3=x^3+x^2 y+x y x+y x^2+x y^2+y x y+y^2 x+y^3
$$
and
$$
(x-y)=(x-y)^3=x^3-x^2 y-x y x-y x^2+x y^2+y x y+y^2 x-y^3
$$
Subtracting we get
$$
2\left(x^2 y+x y x+y x^2\right)=0
$$
Multiply the last relation by $x$ on the left and right to get
$$
2\left(x y+x^2 y x+x y x^2\right)=0 \quad 2\left(x^2 y x+x y x^2+y x\right)=0 .
$$
Subtracting the last two relations we have
$$
2(x y-y x)=0 .
$$
We then show that $3\left(x+x^2\right)=0 \forall x$. You get this from
$$
x+x^2=\left(x+x^2\right)^3=x^3+3 x^4+3 x^5+x^6=4\left(x+x^2\right) .
$$
In particular
$$
3\left(x+y+(x+y)^2\right)=3\left(x+x^2+y+y^2+x y+y x\right)=0
$$
we end-up with $3(x y+y x)=0$. But since $6 x y=0$, we have $3(x y-y x)=0$. Then subtract $2(x y-y x)=0$ to get $x y-y x=0$.
\end{proof}

\paragraph{Exercise 4.2.6} If $a^2 = 0$ in $R$, show that $ax + xa$ commutes with $a$.
\begin{proof}
We need to show that
$$
a(a x+x a)=(a x+x a) a \text { for } a, x \in R .
$$
Now,
$$
\begin{gathered}
a(a x+x a)=a(a x)+a(x a) \\
=a^2 x+a x a \\
=0+a x a=a x a .
\end{gathered}
$$
Again,
$$
\begin{gathered}
(a x+x a) a=(a x) a+(x a) a \\
=a x a+x a^2 \\
=a x a+0=a x a .
\end{gathered}
$$
It follows that,
$$
a(a x+x a)=(a x+x a) a, \text { for } x, a \in R .
$$
This shows that $a x+x a$ commutes with $a$. This completes the proof.
\end{proof}

\paragraph{Exercise 4.2.9} Let $p$ be an odd prime and let $1 + \frac{1}{2} + ... + \frac{1}{p - 1} = \frac{a}{b}$, where $a, b$ are integers. Show that $p \mid a$.
\begin{proof}
    First we prove for prime $p=3$ and then for all prime $p>3$.
Let us take $p=3$. Then the sum
$$
\frac{1}{1}+\frac{1}{2}+\ldots+\frac{1}{(p-1)}
$$
becomes
$$
1+\frac{1}{3-1}=1+\frac{1}{2}=\frac{3}{2} .
$$
Therefore in this case $\quad \frac{a}{b}=\frac{3}{2} \quad$ implies $3 \mid a$, i.e. $p \mid a$.
Now for odd prime $p>3$.
Let us consider $f(x)=(x-1)(x-2) \ldots(x-(p-1))$.
Now, by Fermat, we know that the coefficients of $f(x)$ other than the $x^{p-1}$ and $x^0$ are divisible by $p$.
So if,
$$
\begin{array}{r}
f(x)=x^{p-1}+\sum_{i=0}^{p-2} a_i x^i \\
\text { and } p>3 .
\end{array}
$$
Then $p \mid a_2$, and
$$
f(p) \equiv a_1 p+a_0 \quad\left(\bmod p^3\right)
$$
But we see that
$$
f(x)=(-1)^{p-1} f(p-x) \text { for any } x,
$$
so if $p$ is odd,
$$
f(p)=f(0)=a_0,
$$
So it follows that:
$$
0=f(p)-a_0 \equiv a_1 p \quad\left(\bmod p^3\right)
$$
Therefore,
$$
0 \equiv a_1 \quad\left(\bmod p^2\right) .
$$
Hence,
$$
0 \equiv a_1 \quad(\bmod p) .
$$
Now our sum is just $\frac{a_1}{(p-1) !}=\frac{a}{b}$.
It follows that $p$ divides $a$. This completes the proof.
\end{proof}



\paragraph{Exercise 4.3.1} If $R$ is a commutative ring and $a \in R$, let $L(a) = \{x \in R \mid xa = 0\}$. Prove that $L(a)$ is an ideal of $R$.
\begin{proof}
    First, note that if $x \in L(a)$ and $y \in L(a)$ then $x a=0$ and $y a=0$, so that
$$
\begin{aligned}
x a-y a & =0 \\
(x-y) a & =0,
\end{aligned}
$$
i.e. $L(a)$ is an additive subgroup of $R$. (We have used the criterion that $H$ is a subgroup of $G$ if for any $h_1, h_2 \in H$ we have that $h_1 h_2^{-1} \in H$. 

Now we prove the conclusion. Let $r \in R$ and $b \in L(a)$, then $b a=0$, and so $x b a=0$ which by associativity of multiplication in $R$ is equivalent to
$$
(x b) a=0,
$$
so that $x b \in L(a)$. Since $R$ is commutative, (1) implies that $(bx)a=0$, so that $b x \in L(a)$, which concludes the proof that $L(a)$ is an ideal.
\end{proof}



\paragraph{Exercise 4.3.25} Let $R$ be the ring of $2 \times 2$ matrices over the real numbers; suppose that $I$ is an ideal of $R$. Show that $I = (0)$ or $I = R$.
\begin{proof}
    Suppose that $I$ is a nontrivial ideal of $R$, and let
$$
A=\left(\begin{array}{ll}
a & b \\
c & d
\end{array}\right)
$$
where not all of $a, b, c d$ are zero. Suppose, without loss of generality -- our steps would be completely analogous, modulo some different placement of 1 s in our matrices, if we assumed some other element to be nonzero -- that $a \neq 0$. Then we have that
$$
\left(\begin{array}{ll}
1 & 0 \\
0 & 0
\end{array}\right)\left(\begin{array}{ll}
a & b \\
c & d
\end{array}\right)=\left(\begin{array}{ll}
a & b \\
0 & 0
\end{array}\right) \in I
$$
and so
$$
\left(\begin{array}{ll}
a & b \\
0 & 0
\end{array}\right)\left(\begin{array}{ll}
1 & 0 \\
0 & 0
\end{array}\right)=\left(\begin{array}{ll}
a & 0 \\
0 & 0
\end{array}\right) \in I
$$
so that
$$
\left(\begin{array}{ll}
x & 0 \\
0 & 0
\end{array}\right) \in I
$$
for any real $x$. Now, also for any real $x$,
$$
\left(\begin{array}{ll}
x & 0 \\
0 & 0
\end{array}\right)\left(\begin{array}{ll}
0 & 1 \\
0 & 0
\end{array}\right)=\left(\begin{array}{ll}
0 & x \\
0 & 0
\end{array}\right) \in I .
$$
Likewise
$$
\left(\begin{array}{ll}
0 & 0 \\
1 & 0
\end{array}\right)\left(\begin{array}{ll}
0 & x \\
0 & 0
\end{array}\right)=\left(\begin{array}{ll}
0 & 0 \\
0 & x
\end{array}\right) \in I
$$
and
$$
\left(\begin{array}{ll}
0 & 0 \\
0 & x
\end{array}\right)\left(\begin{array}{ll}
0 & 0 \\
1 & 0
\end{array}\right)=\left(\begin{array}{ll}
0 & 0 \\
x & 0
\end{array}\right)
$$
Thus, as
$$
\left(\begin{array}{ll}
a & b \\
c & d
\end{array}\right)=\left(\begin{array}{ll}
a & 0 \\
0 & 0
\end{array}\right)+\left(\begin{array}{ll}
0 & b \\
0 & 0
\end{array}\right)+\left(\begin{array}{ll}
0 & 0 \\
c & 0
\end{array}\right)+\left(\begin{array}{ll}
0 & 0 \\
0 & d
\end{array}\right)
$$
and since all the terms on the right side are in $I$ and $I$ is an additive group, it follows that
$$
\left(\begin{array}{ll}
a & b \\
c & d
\end{array}\right)
$$
for arbitrary $a, b, c, d$ is in $I$, i.e. $I=R$
Note that the intuition for picking these matrices is that, if we denote by $E_{i j}$ the matrix with 1 at position $(i, j)$ and 0 elsewhere, then
$$
E_{i j}\left(\begin{array}{ll}
a_{1,1} & a_{1,2} \\
a_{2,1} & a_{2,2}
\end{array}\right) E_{n m}=a_{j, n} E_{i m}
$$
\end{proof}



\paragraph{Exercise 4.4.9} Show that $(p - 1)/2$ of the numbers $1, 2, \ldots, p - 1$ are quadratic residues and $(p - 1)/2$ are quadratic nonresidues $\mod p$.
\begin{proof}
    To find all the quadratic residues $\bmod p$ among the integers $1,2, \ldots, p-1$, we compute the least positive residues modulo $p$ of the squares of the integers $1,2, \ldots, p-1\}$.

Since there are $p-1$ squares to consider, and since each congruence $x^2 \equiv a (\bmod p)$ has either zero or two solutions, there must be exactly $\frac{(p-1)}{2}$ quadratic residues mod $p$ among the integers $1,2, \ldots, p-1$.
The remaining
$$
(p-1)-\frac{(p-1)}{2}=\frac{(p-1)}{2}
$$
positive integers less than $p-1$ are quadratic non-residues of $\bmod p$.
\end{proof}



\paragraph{Exercise 4.5.16} Let $F = \mathbb{Z}_p$ be the field of integers $\mod p$, where $p$ is a prime, and let $q(x) \in F[x]$ be irreducible of degree $n$. Show that $F[x]/(q(x))$ is a field having at exactly $p^n$ elements.
\begin{proof}
    In the previous problem we have shown that any for any $p(x) \in F[x]$, we have that
$$
p(x)+(q(x))=a_{n-1} x^{n-1}+\cdots+a_1 x+a_0+(q(x))
$$
for some $a_{n-1}, \ldots, a_0 \in F$, and that there are $p^n$ choices for these numbers, so that $F[x] /(q(x)) \leq p^n$. In order to show that equality holds, we have to show that each of these choices induces a different element of $F[x] /(q(x))$; in other words, that each different polynomial of degree $n-1$ or lower belongs to a different coset of $(q(x))$ in $F[x]$.

Suppose now, then, that
$$
a_{n-1} x^{n-1}+\cdots+a_1 x+a_0+(q(x))=b_{n-1} x^{n-1}+\cdots+b_1 x+b_0+(q(x))
$$
which is equivalent with $\left(a_{n-1}-b_{n-1}\right)^{n-1}+\cdots\left(a_1-b_1\right) x+\left(a_0-b_0\right) \in(q(x))$, which is in turn equivalent with there being a $w(x) \in F[x]$ such that
$$
q(x) w(x)=\left(a_{n-1}-b_{n-1}\right)^{n-1}+\cdots\left(a_1-b_1\right) x+\left(a_0-b_0\right) .
$$
Degree of the right hand side is strictly smaller than $n$, while the degree of the left hand side is greater or equal to $n$ except if $w(x)=0$, so that if equality is hold we must have that $w(x)=0$, but then since polynomials are equal iff all of their coefficient are equal we get that $a_{n-1}-b_{n-1}=$ $0, \ldots, a_1-b_1=0, a_0-b_0=0$, i.e.
$$
a_{n-1}=b_{n-1}, \ldots, a_1=b_1, a_0=b_0
$$
which is what we needed to prove.
\end{proof}



\paragraph{Exercise 4.5.23} Let $F = \mathbb{Z}_7$ and let $p(x) = x^3 - 2$ and $q(x) = x^3 + 2$ be in $F[x]$. Show that $p(x)$ and $q(x)$ are irreducible in $F[x]$ and that the fields $F[x]/(p(x))$ and $F[x]/(q(x))$ are isomorphic.
\begin{proof}
    We have that $p(x)$ and $q(x)$ are irreducible if they have no roots in $\mathbb{Z}_7$, which can easily be checked. E.g. for $p(x)$ we have that $p(0)=5, p(1)=6, p(2)=6, p(3)=4, p(4)=6$, $p(5)=4, p(6)=4$, and similarly for $q(x)$.

We have that every element of $F[x] /(p(x))$ is equal to $a x^2+b x+c+(p(x))$, and likewise for $F[x] /(q(x))$. We consider a map $\tau$ : $F[x] /(p(x)) \rightarrow F[x] /(q(x))$ given by
$$
\tau\left(a x^2+b x+c+(p(x))\right)=a x^2-b x+c+(q(x)) .
$$
This map is obviously onto, and since $|F[x] /(p(x))|=|F[x] /(q(x))|=7^3$ by Problem 16, it is also one-to-one. We claim that it is a homomorphism. Additivity of $\tau$ is immediate by the linearity of addition of polynomial coefficient, so we just have to check the multiplicativity; if $n=a x^2+b x+$ $c+(p(x))$ and $m=d x^2+e x+f+(p(x))$ then
$$
\begin{aligned}
\tau(n m) & =\tau\left(a d x^4+(a e+b d) x^3+(a f+b e+c d) x^2+(b f+c e) x+c f+(p(x))\right) \\
& =\tau\left(2 a d x+2(a e+b d)+(a f+b e+c d) x^2+(b f+c e) x+c f+(p(x))\right) \\
& =\tau\left((a f+b e+c d) x^2+(b f+c e+2 a d) x+(c f+2 a e+2 b d)+(p(x))\right) \\
& =(a f+b e+c d) x^2-(b f+c e+2 a d) x+c f+2 a e+2 b d+(q(x)) \\
& =a d x^4-(a e+b d) x^3+(a f+b e+c d) x^2-(b f+c e) x+c f+(q(x)) \\
& =\left(a x^2-b x+c+(q(x))\right)\left(d x^2-e x+f+(q(x))\right) \\
& =\tau(n) \tau(m) .
\end{aligned}
$$
where in the second equality we used that $x^3+p(x)=2+p(x)$ and in the fifth we used that $x^3+$ $q(x)=-2+q(x)$
\end{proof}

\paragraph{Exercise 4.5.25} If $p$ is a prime, show that $q(x) = 1 + x + x^2 + \cdots x^{p - 1}$ is irreducible in $Q[x]$.
\begin{proof}
    Lemma: Let $F$ be a field and $f(x) \in F[x]$. If $c \in F$ and $f(x+c)$ is irreducible in $F[x]$, then $f(x)$ is irreducible in $F[x]$.
Proof of the Lemma: Suppose that $f(x)$ is reducible, i.e., there exist non-constant $g(x), h(x) \in F[x]$ so that
$$
f(x)=g(x) h(x) .
$$
In particular, then we have
$$
f(x+c)=g(x+c) h(x+c) .
$$
Note that $g(x+c)$ and $h(x+c)$ have the same degree at $g(x)$ and $h(x)$ respectively; in particular, they are non-constant polynomials. So our assumption is wrong.
Hence, $f(x)$ is irreducible in $F[x]$. This proves our Lemma.

Now recall the identity
$$
\frac{x^p-1}{x-1}=x^{p-1}+x^{p-2}+\ldots \ldots+x^2+x+1 .
$$
We prove that $f(x+1)$ is $\$$ |textbffirreducible in $\mathbb{Q}[x]$ and then apply the Lemma to conclude that $f(x)$ is irreducible in $\mathbb{Q}[x] .3 \$$ Note that
$$
\begin{aligned}
& f(x+1)=\frac{(x+1)^p-1}{x} \\
& =\frac{x^p+p x^{p-1}+\ldots+p x}{x} \\
& =x^{p-1}+p x^{p-2}+\ldots .+p .
\end{aligned}
$$
Using that the binomial coefficients occurring above are all divisible by $p$, we have that $f(x+1)$ is irreducible $\mathbb{Q}[x]$ by Eisenstein's criterion applied with prime $p$. 

Then by the lemma $f(x)$ is irreducible $\mathbb{Q}[x]$. This completes the proof.
\end{proof}



\paragraph{Exercise 4.6.2} Prove that $f(x) = x^3 + 3x + 2$ is irreducible in $Q[x]$.
\begin{proof}    
Let us assume that $f(x)$ is reducible over $\mathbb{Q}[x]$.
Then there exists a rational root of $f(x)$.
Let $p / q$ be a rational root of $f(x)$, where $\operatorname{gcd}(p, q)=1$.
Then $f(p / q)=0$.
Now,
$$
\begin{aligned}
& f(p / q)=(p / q)^3+3(p / q)+2 \\
\Longrightarrow & (p / q)^3+3(p / q)+2=0 \\
\Longrightarrow & p^3+3 p q^2=-2 q^3 \\
\Longrightarrow & p\left(p^2+3 q^2\right)=-q^3
\end{aligned}
$$
It follows that, $p$ divides $q$ which is a contradiction to the fact that $\operatorname{gcd}(p, q)=1$.
This implies that $f(x)$ has no rational root.
Now we know that, a polynomial of degree two or three over a field $F$ is reducible if and only if it has a root in $F$.
Now $f(x)$ is a 3 degree polynomial having no root in $\mathbb{Q}$.
So, $f(x)$ is irreducible in $\mathbb{Q}[x]$.
This completes the proof.
\end{proof}



\paragraph{Exercise 4.6.3} Show that there is an infinite number of integers a such that $f(x) = x^7 + 15x^2 - 30x + a$ is irreducible in $Q[x]$.
\begin{proof}
    Via Eisenstein's criterion and observation that 5 divides 15 and $-30$, it is sufficient to find infinitely many $a$ such that 5 divides $a$, but $5^2=25$ doesn't divide $a$. For example $5 \cdot 2^k$ for $k=0,1, \ldots$ is one such infinite sequence.
\end{proof}

\paragraph{Exercise 5.1.8} If $F$ is a field of characteristic $p \neq 0$, show that $(a + b)^m = a^m + b^m$, where $m = p^n$, for all $a, b \in F$ and any positive integer $n$.
\begin{proof}
    Since $F$ is of characteristic $p$ and we have considered arbitrary two elements $a, b$ in $F$ we have
$$
\begin{aligned}
& p a=p b=0 \\
& \Longrightarrow p^n a=p^n b=0 \\
& \Longrightarrow m a=m b=0 \text {. } \\
&
\end{aligned}
$$
Now we know from Binomial Theorem that
$$
(a+b)^m=\sum_{i=0}^m\left(\begin{array}{c}
m \\
i
\end{array}\right) a^i b^{m-i}
$$
Here
$$
\left(\begin{array}{c}
m \\
i
\end{array}\right)=\frac{m !}{i !(m-i) !} .
$$
Now we know that for any integer $n$ and any integer $k$ satisfying $1 \leq k<n, n$ always divides $\left(\begin{array}{l}n \\ k\end{array}\right)$. So in our case for $i$ in the range $1 \leq i<m, m$ divides $\left(\begin{array}{c}m \\ i\end{array}\right)$. It follows that $p$ divides $\left(\begin{array}{c}m \\ i\end{array}\right)$, for $i$ satisfying $1 \leq i<m$, since $m=p^n$ for any integer $n$. Therefore other than the terms $a^m$ and $b^m$ in the expansion $\sum_{i=0}^m\left(\begin{array}{c}m \\ i\end{array}\right) a^i b^{m-i}$ will vanish due to char $p$ nature of $F$.
Hence we have
$$
\sum_{i=0}^m\left(\begin{array}{c}
m \\
i
\end{array}\right) a^i b^{m-i}=a^m+b^m
$$
This follows that, for all $a, b \in F$
$$
(a+b)^m=a^m+b^m .
$$
This completes the proof.
\end{proof}



\paragraph{Exercise 5.2.20} Let $V$ be a vector space over an infinite field $F$. Show that $V$ cannot be the set-theoretic union of a finite number of proper subspaces of $V$.
\begin{proof}
    Assume that $V$ can be written as the set-theoretic union of $n$ proper subspaces $U_1, U_2, \ldots, U_n$. Without loss of generality, we may assume that no $U_i$ is contained in the union of other subspaces.

Let $u \in U_i$ but $u \notin \bigcup_{j \neq i} U_j$ and $v \notin U_i$. Then, we have $(v + Fu) \cap U_i = \varnothing$, and $(v + Fu) \cap U_j$ for $j \neq i$ contains at most one vector, since otherwise $U_j$ would contain $u$.

Therefore, we have $|v + Fu| \leq |F| \leq n-1$. However, since $n$ is a finite natural number, this contradicts the fact that the field $F$ is finite.

Thus, our assumption that $V$ can be written as the set-theoretic union of proper subspaces is wrong, and the claim is proven.
\end{proof}

\paragraph{Exercise 5.3.7} If $a \in K$ is such that $a^2$ is algebraic over the subfield $F$ of $K$, show that a is algebraic over $F$.
\begin{proof}
    Since $a^2$ is algebraic over $F$, there exist a non-zero polynomial $f(x)$ in $F[x]$ such that $f\left(a^2\right)=0$. Consider a new polynomial $g(x)$ defined as $g(x)=f\left(x^2\right)$. Clearly $g(x) \in F[x]$ and $g(a)=f\left(a^2\right)= 0$.
\end{proof}



\paragraph{Exercise 5.3.10} Prove that $\cos 1^{\circ}$  is algebraic over $\mathbb{Q}$.
\begin{proof}
    Since $\left(\cos \left(1^{\circ}\right)+i \sin \left(1^{\circ}\right)\right)^{360}=1$, the number $\cos \left(1^{\circ}\right)+i \sin \left(1^{\circ}\right)$ is algebraic. And the real part and the imaginary part of an algebraic number are always algebraic numbers.
\end{proof}



\paragraph{Exercise 5.4.3} If $a \in C$ is such that $p(a) = 0$, where $p(x) = x^5 + \sqrt{2}x^3 + \sqrt{5}x^2 + \sqrt{7}x + \sqrt{11}$, show that $a$ is algebraic over $\mathbb{Q}$ of degree at most 80.
\begin{proof}
    Given $a \in \mathbb{C}$ such that $p(a)=0$, where
$$
p(x)=x^5+\sqrt{2} x^3+\sqrt{5} x^2+\sqrt{7} x+\sqrt{11}
$$
Here, we note that $p(x) \in \mathbb{Q}(\sqrt{2}, \sqrt{5}, \sqrt{7}, \sqrt{11})$ and
$$
\begin{aligned}
    {[Q(\sqrt{2}, \sqrt{5}, \sqrt{7}, \sqrt{11}): \mathbb{Q}] } & =[Q(\sqrt{2}, \sqrt{5}, \sqrt{7}, \sqrt{11}): Q(\sqrt{2}, \sqrt{5}, \sqrt{7})] \cdot[\mathbb{Q}(\sqrt{2}, \sqrt{5}, \sqrt{7}): \mathbb{Q}(\sqrt{2}, \sqrt{5})] \\
& \cdot[\mathbb{Q}(\sqrt{2}, \sqrt{5}): \mathbb{Q}(\sqrt{2})] \cdot[\mathbb{Q}(\sqrt{2}): \mathbb{Q}] \\
& =2 \cdot 2 \cdot 2 \cdot 2 \\
& =16
\end{aligned}
$$
Here, we note that $p(x)$ is of degree 5 over $\mathbb{Q}(\sqrt{2}, \sqrt{5}, \sqrt{7}, \sqrt{11})$. If $a$ is root of $p(x)$, then
$$
[Q(\sqrt{2}, \sqrt{5}, \sqrt{7}, \sqrt{11}, a): \mathbb{Q}]=[Q(\sqrt{2}, \sqrt{5}, \sqrt{7}, \sqrt{11}): Q(\sqrt{2}, \sqrt{5}, \sqrt{7}, \sqrt{11})] \cdot 15
$$
and $[Q(\sqrt{2}, \sqrt{5}, \sqrt{7}, \sqrt{11}): Q(\sqrt{2}, \sqrt{5}, \sqrt{7}, \sqrt{11})] \leq 5$. We get equality if $p(x)$ is irreducible over $Q(\sqrt{2}, \sqrt{5}, \sqrt{7}, \sqrt{11})$. This gives
$$
[Q(\sqrt{2}, \sqrt{5}, \sqrt{7}, \sqrt{11}, a): \mathbb{Q}] \leq 16 \cdot 5=80
$$
\end{proof}



\paragraph{Exercise 5.5.2} Prove that $x^3 - 3x - 1$ is irreducible over $\mathbb{Q}$.
\begin{proof}
Let $p(x)=x^3-3 x-1$. Then
$$
p(x+1)=(x+1)^3-3(x+1)-1=x^3+3 x^2-3
$$
We have $3|3,3| 0$ but $3 \nmid 1$ and $3^2 \nmid 3$. Thus the polynomial is irreducible over $\mathbb{Q}$ by 3 -Eisenstein criterion.
\end{proof}



\paragraph{Exercise 5.6.14} If $F$ is of characteristic $p \neq 0$, show that all the roots of $x^m - x$, where $m = p^n$, are distinct.
\begin{proof}
    Let us consider $f(x)=x^m-x$. Then $f \in F[x]$.
Claim: $f(x)$ has a multiple root in some extension of $F$ if and only if $f(x)$ is not relatively prime to its formal derivative, $f^{\prime}(x)$. 

Proof of the Claim: Let us assume that $f(x)$ has a multiple root in some extension of $F$. Let $y$ be a multiple root of $f(x)$. Then over a splitting field, we have
$$
f(x)=(x-y)^n g(x), \text { for some integer } n \geq 2 .
$$
Here $g(x)$ is a polynomial such that $g(y) \neq 0$. Now taking derivative of $f$ we get
$$
f^{\prime}(x)=n \cdot(x-y)^{n-1} g(x)+(x-y)^n g^{\prime}(x)
$$
here $g^{\prime}(x)$ implies derivative of $g$ with respect to $x$. Since we have $n \geq 2$, this implies $(n-1) \geq 1$. Hence, (1) shows that $f^{\prime}(x)$ has $y$ as a root. Therefore, $f(x)$ is not relatively prime to $f^{\prime}(x)$. We now prove the other direction.
Conversely, let us assume that $f(x)$ is not relatively prime to $f^{\prime}(x)$. Let $y$ is a root of both $f(x)$ and $f^{\prime}(x)$. Since $y$ is a root of $f(x)$, we can write
$$
f(x)=(x-y) \cdot g(x)
$$
for some polynomial $g(x)$. then taking derivative of $f(x)$ we have
$$
f^{\prime}(x)=g(x)+(x-y) \cdot g^{\prime}(x)
$$
where $g^{\prime}(x)$ is the derivative of $g(x)$ with respect to $x$. Since $y$ is a root of $f^{\prime}(x)$ also we have
$$
f^{\prime}(y)=0
$$
Then we have
$$
\begin{aligned}
& f^{\prime}(y)=g(y)+(y-y) \cdot g^{\prime}(y) \\
\Longrightarrow & f^{\prime}(y)=g(y) \\
\Longrightarrow & g(y)=0 .
\end{aligned}
$$
This implies $y$ is a root of $g(x)$ also. Therefore we have
$$
g(x)=(x-y) \cdot h(x)
$$
for some polynomial $h(x)$. Now form (2) we have
$$
f(x)=(x-y)^2 \cdot h(x) .
$$
This follows that $y$ is a multiple root of $f(x)$. Therefore, $f(x)$ has a multiple root in some extension of the field $F$. This completes the proof of the Claim.

In our case, $f(x)=x^m-x$, where $m=p^n$. Now we calculate the derivative of $f$. That is
$$
f^{\prime}(x)=m x^{m-1}-1=-1(\bmod p) .
$$
By the above condition it follows that, $f^{\prime}$ has no root same as $f$, that is, $f(x)$ and $f^{\prime}(x)$ are relatively prime. Hence, $f(x)$ has no multiple root in $F$. Since $f(x)=x^m-x$ is a polynomial of degree $m$, it follows that $f(x)$ has $m$ distinct roots in $F$, where $m=p^n$. This completes the proof.
\end{proof}



\end{document}
