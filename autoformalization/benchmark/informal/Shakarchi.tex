\documentclass{article}

\title{\textbf{
Exercises from \\
\textit{Complex analysis} \\
by Elias M. Stein and Rami Shakarchi
}}

\date{}

\usepackage{amsmath}
\usepackage{amssymb}

\begin{document}
\maketitle

\paragraph{Exercise 1.13a} Suppose that $f$ is holomorphic in an open set $\Omega$. Prove that if $\text{Re}(f)$ is constant, then $f$ is constant.

\paragraph{Exercise 1.13b} Suppose that $f$ is holomorphic in an open set $\Omega$. Prove that if $\text{Im}(f)$ is constant, then $f$ is constant.

\paragraph{Exercise 1.13c} Suppose that $f$ is holomorphic in an open set $\Omega$. Prove that if $|f|$ is constant, then $f$ is constant.

\paragraph{Exercise 1.18} Let $f$ be a power series centered at the origin. Prove that $f$ has a power series expansion around any point in its disc of convergence.

\paragraph{Exercise 1.19a} Prove that the power series $\sum nz^n$ does not converge on any point of the unit circle.

\paragraph{Exercise 1.19b} Prove that the power series $\sum zn/n^2$ converges at every point of the unit circle.

\paragraph{Exercise 1.19c} Prove that the power series $\sum zn/n$ converges at every point of the unit circle except $z = 1$.

\paragraph{Exercise 1.22} Let $\mathbb{N} = {1, 2, 3, \ldots}$ denote the set of positive integers. A subset $S \subset \mathbb{N}$ is said to be in arithmetic progression if $S = {a, a + d, a + 2d, a + 3d, \ldots}$ where $a, d \in \mathbb{N}$. Here $d$ is called the step of $S$.  Show that $\mathbb{N}$ cannot be partitioned into a finite number of subsets that are in arithmetic progression with distinct steps (except for the trivial case $a = d = 1$).

\paragraph{Exercise 1.26} Suppose $f$ is continuous in a region $\Omega$. Prove that any two primitives of $f$ (if they exist) differ by a constant.

\paragraph{Exercise 2.2} Show that $\int_{0}^{\infty} \frac{\sin x}{x} d x=\frac{\pi}{2}$.

\paragraph{Exercise 2.5} Suppose $f$ is continuously complex differentiable on $\Omega$, and $T \subset \Omega$ is a triangle whose interior is also contained in $\Omega$. Apply Green’s theorem to show that $\int_T f(z) dz = 0$.

\paragraph{Exercise 2.6} Let $\Omega$ be an open subset of $\mathbb{C}$ and let $T \subset \Omega$ be a triangle whose interior is also contained in $\Omega$. Suppose that $f$ is a function holomorphic in $\Omega$ except possibly at a point w inside $T$. Prove that if $f$ is bounded near $w$, then $\int_T f(z) dz = 0$.

\paragraph{Exercise 2.9} Let $\Omega$ be a bounded open subset of $\mathbb{C}$, and $\varphi: \Omega \rightarrow \Omega$ a holomorphic function. Prove that if there exists a point $z_{0} \in \Omega$ such that $\varphi\left(z_{0}\right)=z_{0} \quad \text { and } \quad \varphi^{\prime}\left(z_{0}\right)=1$ then $\varphi$ is linear.

\paragraph{Exercise 2.13} Suppose $f$ is an analytic function defined everywhere in $\mathbb{C}$ and such that for each $z_0 \in \mathbb{C}$ at least one coefficient in the expansion $f(z) = \sum_{n=0}^\infty c_n(z - z_0)^n$ is equal to 0. Prove that $f$ is a polynomial.

\paragraph{Exercise 3.2} Evaluate the integral $\int_{-\infty}^{\infty} \frac{dx}{1 + x^4}$.

\paragraph{Exercise 3.3} Show that $ \int_{-\infty}^{\infty} \frac{\cos x}{x^2 + a^2} dx = \pi \frac{e^{-a}}{a}$ for $a > 0$.

\paragraph{Exercise 3.4} Show that $ \int_{-\infty}^{\infty} \frac{x \sin x}{x^2 + a^2} dx = \pi e^{-a}$ for $a > 0$.

\paragraph{Exercise 3.9} Show that $\int_0^1 \log(\sin \pi x) dx = - \log 2$.

\paragraph{Exercise 3.14} Prove that all entire functions that are also injective take the form $f(z) = az + b$, $a, b \in \mathbb{C}$ and $a \neq 0$.

\paragraph{Exercise 3.22} Show that there is no holomorphic function $f$ in the unit disc $D$ that extends continuously to $\partial D$ such that $f(z) = 1/z$ for $z \in \partial D$.

\paragraph{Exercise 4.4a} Suppose $Q$ is a polynomial of degree $\geq 2$ with distinct roots, none lying on the real axis. Calculate $\int_{-\infty}^\infty \frac{e^{-2 \pi ix \xi}}{Q(x)} dx$, $\xi \in \mathbb{R}$, in terms of the roots of $Q$.

% these below are from "problems" not "exercises"
\paragraph{Exercise 5.1} Prove that if $f$ is holomorphic in the unit disc, bounded and not identically zero, and $z_{1}, z_{2}, \ldots, z_{n}, \ldots$ are its zeros $\left(\left|z_{k}\right|<1\right)$, then $\sum_{n}\left(1-\left|z_{n}\right|\right)<\infty$.

\paragraph{Exercise 5.3} Show that $\sum \frac{z^{n}}{(n !)^{\alpha}}$ is an entire function of order $1 / \alpha$.

\end{document}
