\documentclass{article}

\title{\textbf{
Exercises from \\
\textit{Everything} \\
by All Authors
}}

\date{}

\usepackage{amsmath}
\usepackage{amssymb}
\usepackage{fullpage}

\begin{document}
\maketitle

\paragraph{Artin.exercise.2.3.2} Let $a, b$ be elements of a group $G$. Prove that there exists an element $g\in G$ such that $bab^{-1}=gag^{-1}$.

\paragraph{Artin.exercise.2.8.6} The center of the direct product of two groups is isomorphic to the direct product of the centers.

\paragraph{Artin.exercise.3.2.7} If $F$ is a field and $G$ is a field extension of $F$, then the inclusion map $F\to G$ is injective.

\paragraph{Artin.exercise.3.7.2} Let $V$ be a vector space over a field $K$, and let $\{W_i\}_{i\in I}$ be a family of subspaces of $V$. Prove that $\bigcap_{i\in I} W_i$ is nontrivial.

\paragraph{Artin.exercise.6.4.2} Prove that if $G$ is a finite group of order $pq$ where $p$ and $q$ are distinct primes, then $G$ is not simple.

\paragraph{Artin.exercise.6.4.12} Prove that there is no simple group of order $224$.

\paragraph{Artin.exercise.10.1.13} If $x$ is nilpotent, then $1+x$ is a unit.

\paragraph{Artin.exercise.10.4.7a} Let $R$ be a commutative ring with identity. Prove that if $I$ and $J$ are ideals of $R$ such that $I+J=R$, then $IJ=I\cap J$.

\paragraph{Artin.exercise.10.6.7} If $I$ is a nonzero ideal of $\mathbb{Z}[i]$, then there exists $z\in I$ such that $z$ is real.

\paragraph{Artin.exercise.11.2.13} Prove that if $a$ divides $b$ in $\mathbb{Z}[i]$, then $a$ divides $b$ in $\mathbb{Z}$.

\paragraph{Artin.exercise.11.4.6a} Prove that $X^2+1$ is irreducible over $\mathbb{F}_7$.

\paragraph{Artin.exercise.11.4.6c} Prove that $X^3-9$ is irreducible over $\mathbb{Z}_{31}$.

\paragraph{Artin.exercise.11.13.3} Prove that there exists a prime $p$ such that $p\geq N$ and $p+1\equiv 0\pmod{4}$.

\paragraph{Artin.exercise.13.6.10} Prove that the only element of the multiplicative group of a field is $-1$.

\paragraph{Axler.exercise.1.2} Prove that $(-1/2 + i\sqrt{3}/2)^3 = -1$.

\paragraph{Axler.exercise.1.4} If $v$ is a vector in a vector space $V$ over a field $F$, and $a$ is an element of $F$, then $av=0$ if and only if $a=0$ or $v=0$.

\paragraph{Axler.exercise.1.7} Prove that there exists a nonempty subset $U$ of $\mathbb{R}^2$ such that $cU=U$ for all $c\in\mathbb{R}$ and $U$ is not a subspace of $\mathbb{R}^2$.

\paragraph{Axler.exercise.1.9} Let $U$ and $W$ be subspaces of a vector space $V$. Prove that $U\cap W$ is a subspace of $V$ if and only if $U\subseteq W$ or $W\subseteq U$.

\paragraph{Axler.exercise.3.8} Let $L:V\to W$ be a linear map. Prove that there exists a subspace $U$ of $V$ such that $U\cap \ker L = \{0\}$ and $\operatorname{im} L = \operatorname{im} (L|_U)$.

\paragraph{Axler.exercise.5.1} Let $V$ be a vector space over a field $F$, and let $L:V\to V$ be a linear transformation. Let $U_1, \dots, U_n$ be subspaces of $V$ such that $L(U_i)=U_i$ for each $i$. Prove that $L(\sum_{i=1}^n U_i)=\sum_{i=1}^n U_i$.

\paragraph{Axler.exercise.5.11} If $S$ and $T$ are endomorphisms of a vector space $V$, then the eigenvalues of $ST$ are the same as the eigenvalues of $TS$.

\paragraph{Axler.exercise.5.13} Let $T$ be a linear transformation of a finite-dimensional vector space $V$ over a field $F$. Prove that if $T$ fixes every subspace of $V$ of codimension $1$, then $T$ is a scalar multiple of the identity.

\paragraph{Axler.exercise.5.24} Let $V$ be a finite-dimensional vector space over $\mathbb{R}$, and let $T:V\to V$ be a linear transformation such that $T(x)=cx$ for all $x\in V$ and some $c\in\mathbb{R}$. Prove that the rank of any subspace $U$ of $V$ is even.

\paragraph{Axler.exercise.6.3} If $a_1, \dots, a_n$ and $b_1, \dots, b_n$ are real numbers, then $(a_1b_1 + \dots + a_nb_n)^2 \leq (a_1^2 + \dots + a_n^2)(b_1^2 + \dots + b_n^2)$.

\paragraph{Axler.exercise.6.13} Let $V$ be a complex inner product space, and let $e_1, \dots, e_n$ be an orthonormal basis for $V$. Prove that $v\in V$ is in the span of $e_1, \dots, e_n$ if and only if $\|v\|^2 = \sum_{i=1}^n |\langle v, e_i\rangle|^2$.

\paragraph{Axler.exercise.7.5} If $V$ is a finite-dimensional complex inner product space of dimension at least $2$, then the set of all operators $T$ on $V$ such that $T^*T=TT^*$ is not a subspace of the space of all operators on $V$.

\paragraph{Axler.exercise.7.9} Let $T$ be a linear operator on a finite-dimensional inner product space $V$. Prove that $T$ is self-adjoint if and only if all of its eigenvalues are real.

\paragraph{Axler.exercise.7.11} Let $T$ be a linear operator on a finite-dimensional inner product space $V$ such that $T^*T=TT^*$. Prove that there exists a linear operator $S$ on $V$ such that $S^2=T$.

\paragraph{Dummit-Foote.exercise.1.1.2a} Find two integers $a$ and $b$ such that $a-b\neq b-a$.

\paragraph{Dummit-Foote.exercise.1.1.4} Prove that $(a\cdot b)\cdot c \equiv a\cdot (b\cdot c)$ in $\mathbb{Z}/n\mathbb{Z}$.

\paragraph{Dummit-Foote.exercise.1.1.15} If $a_1, \dots, a_n$ are elements of a group $G$, then $(a_1a_2\dots a_n)^{-1} = a_n^{-1}a_{n-1}^{-1}\dots a_1^{-1}$.

\paragraph{Dummit-Foote.exercise.1.1.17} If $x$ has order $n$, then $x^{-1}=x^{n-1}$.

\paragraph{Dummit-Foote.exercise.1.1.20} Prove that the order of an element $x$ of a group $G$ is equal to the order of $x^{-1}$.

\paragraph{Dummit-Foote.exercise.1.1.22b} Prove that the order of $ab$ is equal to the order of $ba$.

\paragraph{Dummit-Foote.exercise.1.1.29} Prove that a group $G$ is abelian if and only if $G\times G$ is abelian.

\paragraph{Dummit-Foote.exercise.1.3.8} Prove that the set of all permutations of $\mathbb{N}$ is infinite.

\paragraph{Dummit-Foote.exercise.1.6.11} Prove that the group $A\times B$ is isomorphic to the group $B\times A$.

\paragraph{Dummit-Foote.exercise.1.6.23} Let $G$ be a group and let $\sigma$ be an automorphism of $G$ such that $\sigma(g)=1$ implies $g=1$ and $\sigma(\sigma(g))=g$ for all $g\in G$. Prove that $G$ is abelian.

\paragraph{Dummit-Foote.exercise.2.1.13} Let $H$ be a subgroup of $\mathbb{Q}$. Prove that $H$ is either $\{0\}$ or $\mathbb{Q}$.

\paragraph{Dummit-Foote.exercise.2.4.16a} Let $H$ be a proper subgroup of $G$. Then there exists a maximal subgroup $M$ of $G$ containing $H$.

\paragraph{Dummit-Foote.exercise.2.4.16c} Let $H$ be a subgroup of $(\mathbb{Z}/n\mathbb{Z}, +)$. Prove that $H$ is a maximal subgroup if and only if $H$ is nontrivial and $H$ is the subgroup generated by a prime $p$.

\paragraph{Dummit-Foote.exercise.3.1.22a} If $H$ and $K$ are normal subgroups of $G$, then $H\cap K$ is a normal subgroup of $G$.

\paragraph{Dummit-Foote.exercise.3.2.8} If $H$ and $K$ are finite subgroups of $G$ with coprime orders, then $H\cap K = \{1\}$.

\paragraph{Dummit-Foote.exercise.3.2.16} If $p$ is prime and $a$ is coprime to $p$, then $a^p\equiv a\pmod{p}$.

\paragraph{Dummit-Foote.exercise.3.3.3} Let $H$ be a normal subgroup of $G$ of index $p$. Prove that if $K$ is a subgroup of $G$, then either $K\leq H$, or $H\cap K$ has index $p$ in $K$, or $H\cup K=G$.

\paragraph{Dummit-Foote.exercise.3.4.4} If $G$ is a finite commutative group, then for any divisor $n$ of $|G|$, there is a subgroup $H$ of $G$ such that $|H|=n$.

\paragraph{Dummit-Foote.exercise.3.4.5b} If $G$ is solvable and $H$ is a normal subgroup of $G$, then $G/H$ is solvable.

\paragraph{Dummit-Foote.exercise.4.2.8} If $H$ is a subgroup of $G$ of index $n$, then there is a normal subgroup $K$ of $H$ of index at most $n!$.

\paragraph{Dummit-Foote.exercise.4.2.9a} Let $G$ be a finite group of order $p^n$, where $p$ is prime. Prove that every subgroup of index $p$ is normal.

\paragraph{Dummit-Foote.exercise.4.4.2} Prove that a group of order $pq$ is cyclic, where $p$ and $q$ are distinct primes.

\paragraph{Dummit-Foote.exercise.4.4.6b} Prove that there exists a characteristic subgroup of a group that is not normal.

\paragraph{Dummit-Foote.exercise.4.4.8a} If $H$ is a normal subgroup of $K$ and $K$ is a normal subgroup of $G$, then $H$ is a normal subgroup of $G$.

\paragraph{Dummit-Foote.exercise.4.5.13} If $G$ is a group of order $56$, then $G$ has a normal Sylow $p$-subgroup for some prime $p$.

\paragraph{Dummit-Foote.exercise.4.5.15} If $G$ is a group of order $351$, then $G$ has a normal Sylow $p$-subgroup for some prime $p$.

\paragraph{Dummit-Foote.exercise.4.5.17} If $G$ is a group of order $105$, then $G$ has a Sylow $5$-subgroup and a Sylow $7$-subgroup.

\paragraph{Dummit-Foote.exercise.4.5.19} Prove that a group of order $6545$ is not simple.

\paragraph{Dummit-Foote.exercise.4.5.21} Prove that there is no simple group of order $2907$.

\paragraph{Dummit-Foote.exercise.4.5.23} Prove that there is no simple group of order $462$.

\paragraph{Dummit-Foote.exercise.4.5.33} Let $G$ be a finite group, $P$ a Sylow $p$-subgroup of $G$, and $H$ a subgroup of $G$. Prove that the Sylow $p$-subgroups of $H$ are precisely the subgroups of $H$ which are conjugate to $H\cap P$.

\paragraph{Dummit-Foote.exercise.7.1.2} Prove that if $u$ is a unit in a ring $R$, then $-u$ is a unit in $R$.

\paragraph{Dummit-Foote.exercise.7.1.12} If $K$ is a subring of a field $F$ and $1\in K$, then $K$ is a domain.

\paragraph{Dummit-Foote.exercise.7.2.2} A polynomial $p$ is divisible by $0$ if and only if there is a nonzero scalar $b$ such that $bp=0$.

\paragraph{Dummit-Foote.exercise.7.3.16} If $R$ is a ring and $\phi:R\to S$ is a surjective ring homomorphism, then $\phi(Z(R))\subset Z(S)$.

\paragraph{Dummit-Foote.exercise.7.4.27} Let $R$ be a commutative ring with $1\neq 0$. If $a$ is nilpotent and $b$ is an element of $R$, then $1-ab$ is a unit.

\paragraph{Dummit-Foote.exercise.8.2.4} Prove that a ring $R$ is a principal ideal ring if and only if for any two nonzero elements $a, b\in R$ there exist $r, s\in R$ such that $gcd(a, b) = ra + sb$.

\paragraph{Dummit-Foote.exercise.8.3.5a} Prove that if $n$ is a squarefree integer greater than $3$, then $2, 1+\sqrt{-n}, \sqrt{-n}$ are irreducible in $\mathbb{Z}[\sqrt{-n}]$.

\paragraph{Dummit-Foote.exercise.8.3.6b} Let $q$ be a prime congruent to $3$ modulo $4$. Prove that the quotient ring $\mathbb{Z}[i]/\langle q\rangle$ is a field of order $q^2$.

\paragraph{Dummit-Foote.exercise.9.1.10} Let $f_i(x)=x_ix_{i+1}$ for $i=1, 2, \dots$. Prove that the set of minimal primes of the ideal generated by $f_1, f_2, \dots$ is infinite.

\paragraph{Dummit-Foote.exercise.9.4.2a} Prove that $X^4 - 4X^3 + 6$ is irreducible over $\mathbb{Z}$.

\paragraph{Dummit-Foote.exercise.9.4.2c} Prove that $X^4 + 4X^3 + 6X^2 + 2X + 1$ is irreducible over $\mathbb{Z}$.

\paragraph{Dummit-Foote.exercise.9.4.9} Prove that $X^2-C\sqrt{2}$ is irreducible over $\mathbb{Q}(\sqrt{2})$.

\paragraph{Dummit-Foote.exercise.11.1.13} Prove that $\mathbb{R}^n$ is isomorphic to $\mathbb{R}$ as a $\mathbb{Q}$-vector space.

\paragraph{Herstein.exercise.2.1.18} If $G$ is a finite group of even order, then $G$ has an element of order $2$.

\paragraph{Herstein.exercise.2.1.26} If $G$ is a finite group, then every element of $G$ has finite order.

\paragraph{Herstein.exercise.2.2.3} Let $G$ be a group. Suppose that for some $n$, the following three statements hold:
\begin{enumerate}
\item $a^n b^n = b^n a^n$ for all $a, b \in G$.
\item $a^{n+1} b^{n+1} = b^{n+1} a^{n+1}$ for all $a, b \in G$.
\item $a^{n+2} b^{n+2} = b^{n+2} a^{n+2}$ for all $a, b \in G$.
\end{enumerate}
Prove that $G$ is abelian.

\paragraph{Herstein.exercise.2.2.6c} Let $G$ be a group and let $n$ be an integer greater than $1$. Suppose that for all $a, b\in G$, $(ab)^n=a^nb^n$. Prove that for all $a, b\in G$, $(aba^{-1}b^{-1})^{n(n-1)}=1$.

\paragraph{Herstein.exercise.2.3.16} Prove that a group $G$ is cyclic if and only if every subgroup of $G$ is either trivial or the whole group.

\paragraph{Herstein.exercise.2.5.23} Let $G$ be a group in which every subgroup is normal. Prove that for any $a, b \in G$, there exists $j \in \mathbb{Z}$ such that $b*a = a^j*b$.

\paragraph{Herstein.exercise.2.5.31} If $G$ is a finite group of order $p^nm$ where $p$ is prime and $p$ does not divide $m$, then the subgroup of order $p^n$ is characteristic.

\paragraph{Herstein.exercise.2.5.43} Prove that a group of order $9$ is abelian.

\paragraph{Herstein.exercise.2.5.52} Let $G$ be a finite group and let $\phi$ be an automorphism of $G$. If $|I|>3/4|G|$ and $\phi(x)=x^{-1}$ for all $x\in I$, then $\phi(x)=x^{-1}$ for all $x\in G$ and $G$ is abelian.

\paragraph{Herstein.exercise.2.7.7} If $N$ is a normal subgroup of $G$ and $\phi:G\to G'$ is a homomorphism, then $\phi(N)$ is a normal subgroup of $G'$.

\paragraph{Herstein.exercise.2.8.15} Let $G$ and $H$ be finite groups of order $pq$, where $p$ and $q$ are distinct primes with $q$ dividing $p-1$. Prove that $G$ and $H$ are isomorphic.

\paragraph{Herstein.exercise.2.10.1} Let $G$ be a group, $A$ a normal subgroup of $G$, and $b$ an element of $G$ of prime order. Prove that $A\cap \langle b\rangle = \{1\}$.

\paragraph{Herstein.exercise.2.11.7} If $P$ is a normal Sylow $p$-subgroup of $G$, then $P$ is characteristic.

\paragraph{Herstein.exercise.3.2.21} Let $\sigma$ and $\tau$ be permutations of a finite set $X$. Prove that if $\sigma(x)=x$ if and only if $\tau(x)\neq x$ for all $x\in X$, and $\tau\circ\sigma=1$, then $\sigma=1$ and $\tau=1$.

\paragraph{Herstein.exercise.4.1.34} Prove that the group of permutations of $\{1, 2, 3\}$ is isomorphic to the group of $2\times 2$ matrices over $\mathbb{Z}_2$.

\paragraph{Herstein.exercise.4.2.6} Prove that if $a^2=0$ then $a(ax+xa)=x+xa^2$.

\paragraph{Herstein.exercise.4.3.1} Let $R$ be a commutative ring and $a$ an element of $R$. Prove that the set of all elements $x$ of $R$ such that $xa=0$ is an ideal of $R$.

\paragraph{Herstein.exercise.4.4.9} If $p$ is an odd prime, then there exists a subset $S$ of $\mathbb{Z}_p$ of size $(p-1)/2$ such that $S$ is the set of squares modulo $p$.

\paragraph{Herstein.exercise.4.5.23} Prove that the polynomials $x^3-2$ and $x^3+2$ are irreducible over $\mathbb{Z}_7$ and that the quotient rings $\mathbb{Z}_7[x]/(x^3-2)$ and $\mathbb{Z}_7[x]/(x^3+2)$ are isomorphic.

\paragraph{Herstein.exercise.4.6.2} Prove that $X^3 + 3X + 2$ is irreducible over $\mathbb{Q}$.

\paragraph{Herstein.exercise.5.1.8} If $F$ is a field of characteristic $p$, then $(a+b)^p = a^p + b^p$.

\paragraph{Herstein.exercise.5.3.7} If $a$ is algebraic over $F$, then $a^2$ is algebraic over $F$.

\paragraph{Herstein.exercise.5.4.3} Let $p(x)=x^5+\sqrt{2}x^3+\sqrt{5}x^2+\sqrt{7}x+11$. Prove that if $p(a)=0$, then $a$ is a root of a polynomial of degree less than $80$ with integer coefficients.

\paragraph{Herstein.exercise.5.6.14} If $F$ is a field of characteristic $p$, then the number of roots of $x^p-x$ is $p$.

\paragraph{Ireland-Rosen.exercise.1.30} Prove that $\sum_{i=1}^n 1/(n+2)$ is not an integer.

\paragraph{Ireland-Rosen.exercise.2.4} Let $a$ be a nonzero integer. Prove that the function $f_a(x) = ax$ is a bijection from $\mathbb{Z}$ to $\mathbb{Z}$.

\paragraph{Ireland-Rosen.exercise.2.27a} Prove that the series $\sum_{p\in\mathbb{Z}, p\text{ squarefree}} 1/p$ diverges.

\paragraph{Ireland-Rosen.exercise.3.4} Prove that there are no integers $x$ and $y$ such that $3x^2+2=y^2$.

\paragraph{Ireland-Rosen.exercise.3.10} Prove that if $n$ is not prime and $n\neq 4$, then $n$ divides $(n-1)!$.

\paragraph{Ireland-Rosen.exercise.4.4} If $p=4t+1$ is prime, then $a$ is a primitive root modulo $p$ if and only if $-a$ is a primitive root modulo $p$.

\paragraph{Ireland-Rosen.exercise.4.6} Prove that $3$ is a primitive root modulo $p$ if $p=2^n+1$.

\paragraph{Ireland-Rosen.exercise.4.11} Let $p$ be a prime and let $k$ be a positive integer. Prove that $p^k$ divides $p^{k+s}-1$ if and only if $p$ divides $p^s-1$.

\paragraph{Ireland-Rosen.exercise.5.28} If $p$ is a prime congruent to $1$ modulo $4$, then there is an $x$ such that $x^4\equiv 2\pmod{p}$ if and only if there are integers $A$ and $B$ such that $p=A^2+64B^2$.

\paragraph{Ireland-Rosen.exercise.12.12} Prove that $\sin(\pi/12)$ is algebraic over $\mathbb{Q}$.

\paragraph{Munkres.exercise.13.3b} Prove that the following statement is false: If $X$ is any set, and if $S$ is any collection of subsets of $X$ such that each member of $S$ is either finite or empty or equal to $X$, then the union of the members of $S$ is either finite or empty or equal to $X$.

\paragraph{Munkres.exercise.13.4a2} There exists a family of topologies $\{T_i\}_{i\in I}$ on a set $X$ such that $\bigcap_{i\in I} T_i$ is not a topology on $X$.

\paragraph{Munkres.exercise.13.4b2} Let $X$ be a set and let $\{T_i\}_{i\in I}$ be a family of topologies on $X$. Prove that there is a unique topology $T$ on $X$ such that $T_i\subset T$ for all $i\in I$ and $T$ is the smallest topology on $X$ with this property.

\paragraph{Munkres.exercise.13.5b} Let $X$ be a set and let $A$ be a collection of subsets of $X$. Prove that the topology generated by $A$ is the intersection of all topologies on $X$ that contain $A$.

\paragraph{Munkres.exercise.13.8a} Prove that the collection of all open intervals with rational endpoints is a topological basis for the real line.

\paragraph{Munkres.exercise.16.1} Let $X$ be a topological space, $Y$ a subset of $X$, and $A$ a subset of $Y$. Prove that a subset $U$ of $A$ is open in $A$ if and only if $U$ is open in $Y$.

\paragraph{Munkres.exercise.16.6} Prove that the collection of all open rectangles in the plane is a basis for the Euclidean topology.

\paragraph{Munkres.exercise.18.8a} Let $X$ and $Y$ be topological spaces, and let $f$ and $g$ be continuous functions from $X$ into $Y$. Prove that the set $\{x\in X: f(x)\leq g(x)\}$ is closed in $X$.

\paragraph{Munkres.exercise.18.13} Let $X$ be a topological space, $Y$ a $T_2$ space, and $A$ a subset of $X$. Let $f$ be a continuous mapping of $A$ into $Y$. Prove that if $g$ is a continuous mapping of $\overline{A}$ into $Y$ such that $g(x)=f(x)$ for all $x \in A$, then $g$ is the only continuous mapping of $\overline{A}$ into $Y$ with this property.

\paragraph{Munkres.exercise.20.2} Prove that the product topology on $\mathbb{R}^2$ is metrizable.

\paragraph{Munkres.exercise.21.6b} Prove that the sequence of functions $f_n(x)=x^n$ does not converge uniformly on any interval $I$.

\paragraph{Munkres.exercise.22.2a} Let $p:X\to Y$ be a continuous map. Then $p$ is a quotient map if and only if there is a continuous map $f:Y\to X$ such that $p\circ f = id_Y$.

\paragraph{Munkres.exercise.22.5} Let $X$ and $Y$ be topological spaces, and let $p:X\to Y$ be an open mapping. Let $A$ be an open subset of $X$. Prove that the restriction of $p$ to $A$ is an open mapping.

\paragraph{Munkres.exercise.23.3} Let $X$ be a topological space, and let $\{A_n\}_{n=1}^\infty$ be a sequence of connected subsets of $X$. Suppose that $A_0$ is a connected subset of $X$ such that $A_0\cap A_n\neq \emptyset$ for all $n$. Prove that $A_0\cup (\cup_{n=1}^\infty A_n)$ is connected.

\paragraph{Munkres.exercise.23.6} Let $X$ be a topological space, and let $A$ and $C$ be subsets of $X$. Suppose that $C$ is connected, that $C\cap A$ is nonempty, and that $C\cap A^c$ is nonempty. Prove that $C\cap\partial A$ is nonempty.

\paragraph{Munkres.exercise.23.11} Let $X$ and $Y$ be topological spaces, and let $p:X\to Y$ be a quotient map. If $Y$ is connected and each fiber $p^{-1}(y)$ is connected, then $X$ is connected.

\paragraph{Munkres.exercise.24.3a} If $f:I\to I$ is continuous, then $f$ has a fixed point.

\paragraph{Munkres.exercise.25.9} If $G$ is a topological group and $C$ is the connected component of the identity, then $C$ is a normal subgroup of $G$.

\paragraph{Munkres.exercise.26.12} Suppose $X$ and $Y$ are topological spaces, $p:X\to Y$ is a continuous surjection, and $p^{-1}(y)$ is compact for each $y\in Y$. If $Y$ is compact, then $X$ is compact.

\paragraph{Munkres.exercise.28.4} A topological space $X$ is countably compact if and only if it is limit point compact.

\paragraph{Munkres.exercise.28.6} If $X$ is a compact metric space and $f:X\to X$ is an isometry, then $f$ is a bijection.

\paragraph{Munkres.exercise.29.4} Prove that the space $\mathbb{N}^I$ is not locally compact.

\paragraph{Munkres.exercise.30.10} Let $X_1, X_2, \dots$ be topological spaces. Suppose that for each $i$, there is a countable dense subset $S_i$ of $X_i$. Prove that there is a countable dense subset of the product space $X_1\times X_2\times \dots$.

\paragraph{Munkres.exercise.31.1} Let $X$ be a regular space. Prove that for any two points $x, y \in X$, there exist open sets $U$ and $V$ such that $x \in U$, $y \in V$, and $U \cap V = \emptyset$.

\paragraph{Munkres.exercise.31.3} Prove that the order topology on a partially ordered set is regular.

\paragraph{Munkres.exercise.32.2a} If $X_i$ is a topological space for each $i\in I$, and if $\prod_{i\in I}X_i$ is a $T_2$ space, then each $X_i$ is a $T_2$ space.

\paragraph{Munkres.exercise.32.2c} Prove that if $X_i$ is a normal space for each $i\in I$, then $\prod_{i\in I}X_i$ is normal.

\paragraph{Munkres.exercise.33.7} Let $X$ be a locally compact Hausdorff space. Prove that for each closed set $A$ and each point $x$ not in $A$, there is a continuous function $f:X\to [0,1]$ such that $f(x)=1$ and $f(A)=\{0\}$.

\paragraph{Munkres.exercise.34.9} If $X$ is a compact space, and $X_1$ and $X_2$ are closed subsets of $X$ such that $X_1\cup X_2=X$, and $X_1$ and $X_2$ are metrizable, then $X$ is metrizable.

\paragraph{Munkres.exercise.43.2} Let $X$ be a metric space, $Y$ a complete metric space, and $A$ a subset of $X$. Suppose that $f:A\to Y$ is uniformly continuous. Prove that there exists a unique continuous function $g:\overline{A}\to Y$ such that $g(x)=f(x)$ for all $x\in A$.

\paragraph{Pugh.exercise.2.26} A set $U$ is open if and only if for each $x\in U$, $x$ is not a cluster point of $U^c$.

\paragraph{Pugh.exercise.2.32a} Prove that the set of all natural numbers is clopen.

\paragraph{Pugh.exercise.2.46} Let $A$ and $B$ be compact sets in a metric space $M$ such that $A$ and $B$ are disjoint and nonempty. Prove that there exist points $a_0\in A$ and $b_0\in B$ such that $d(a_0, b_0)$ is less than or equal to $d(a, b)$ for all $a\in A$ and $b\in B$.

\paragraph{Pugh.exercise.2.92} Let $X$ be a topological space, and let $\{S_n\}_{n=1}^\infty$ be a sequence of nonempty compact subsets of $X$ such that $S_n\subset S_{n+1}$ for all $n$. Prove that $\bigcap_{n=1}^\infty S_n$ is nonempty.

\paragraph{Pugh.exercise.3.1} Let $f$ be a function from $\mathbb{R}$ to $\mathbb{R}$ such that $|f(x)-f(y)|\leq |x-y|^2$ for all $x, y\in\mathbb{R}$. Prove that $f$ is constant.

\paragraph{Pugh.exercise.3.63a} Prove that the function $f(x)=\frac{1}{x(\log x)^p}$ converges to $0$ as $x$ tends to infinity, for any $p>1$.

\paragraph{Pugh.exercise.4.15a} Let $F$ be a set of functions from $\mathbb{R}$ to $\mathbb{R}$. Prove that $F$ is equicontinuous at $x$ if and only if there exists a function $\mu$ such that $\mu(x)\geq 0$ for all $x$, $\mu(x)\to 0$ as $x\to 0$, and $|f(s)-f(t)|\leq \mu(|s-t|)$ for all $f\in F$.

\paragraph{Putnam.exercise.1998.b6} Prove that there exists a positive integer $n$ such that the equation $x^2=n^3+an^2+bn+c$ has no integer solution.

\paragraph{Putnam.exercise.1999.b4} Let $f$ be a function of class $C^3$ on $\mathbb{R}$ such that $f^{(n)}(x)>0$ for $n=0, 1, 2, 3$ and $f^{(3)}(x)\leq f(x)$ for all $x$. Prove that $f'(x)<2f(x)$ for all $x$.

\paragraph{Putnam.exercise.2001.a5} Prove that there is a unique pair of positive integers $(a, n)$ such that $a^{n+1} - (a+1)^n = 2001$.

\paragraph{Putnam.exercise.2014.a5} Let $P_n(x) = (n+1)x^n + (n+1)x^{n-1} + \dots + (n+1)x + (n+1)$. Prove that $P_j$ and $P_k$ are coprime for $j\neq k$.

\paragraph{Putnam.exercise.2018.a5} Let $f$ be a continuous function on $\mathbb{R}$ such that $f(0)=0$, $f(1)=1$, and $f(x)\geq 0$ for all $x$. Prove that there exists a positive integer $n$ and a point $x$ such that $f^{(n)}(x)=0$.

\paragraph{Putnam.exercise.2018.b4} Let $x_0, x_1, x_2, \dots$ be a sequence of real numbers defined by $x_0=a$, $x_1=a$, and $x_n=2x_{n-1}x_{n-2}-x_{n-3}$ for $n\geq 2$. Prove that if $x_n=0$ for some $n$, then $x_n=x_{n+k}$ for all $k\geq 0$.

\paragraph{Rudin.exercise.1.1b} If $x$ is irrational and $y$ is rational and nonzero, then $xy$ is irrational.

\paragraph{Rudin.exercise.1.4} Let $S$ be a nonempty set of real numbers which is bounded above. Prove that there is a number $b$ such that $b\leq s$ for every $s\in S$.

\paragraph{Rudin.exercise.1.8} Prove that there is no linear order on $\mathbb{C}$.

\paragraph{Rudin.exercise.1.12} If $z_1, \dots, z_n$ are complex, then $|z_1 + z_2 + \dots + z_n|\leq |z_1| + |z_2| + \dots + |z_n|$.

\paragraph{Rudin.exercise.1.14} If $z$ is a complex number of modulus $1$, prove that $(1+z)^2+(1-z)^2=4$.

\paragraph{Rudin.exercise.1.17} If $x$ and $y$ are vectors in $\mathbb{R}^n$, prove that $\|x+y\|^2 + \|x-y\|^2 = 2\|x\|^2 + 2\|y\|^2$.

\paragraph{Rudin.exercise.1.18b} Prove that the statement $\forall x\in\mathbb{R}, \exists y\in\mathbb{R}, y\neq 0 \land xy=0$ is false.

\paragraph{Rudin.exercise.2.19a} If $A$ and $B$ are disjoint closed sets, then $A$ and $B$ are separated.

\paragraph{Rudin.exercise.2.25} Prove that every compact metric space has a countable basis.

\paragraph{Rudin.exercise.2.27b} Let $E$ be a non-countable subset of $\mathbb{R}^k$ and let $P$ be the set of points $x$ such that for every neighborhood $U$ of $x$, the set $P\cap E$ is non-countable. Prove that $E\setminus P$ is countable.

\paragraph{Rudin.exercise.2.29} Let $U$ be an open set in $\mathbb{R}$. Prove that there exists a sequence of open intervals $\{(a_n, b_n)\}_{n=1}^\infty$ such that $U=\bigcup_{n=1}^\infty (a_n, b_n)$.

\paragraph{Rudin.exercise.3.2a} Prove that $\lim_{n\to\infty} \sqrt{n^2+n} - n = 1/2$.

\paragraph{Rudin.exercise.3.5} If $a_n$ and $b_n$ are bounded sequences, then $\limsup(a_n+b_n)\leq \limsup a_n + \limsup b_n$.

\paragraph{Rudin.exercise.3.7} Prove that if $\sum_{i=1}^\infty a_i$ converges, then $\sum_{i=1}^\infty \frac{\sqrt{a_i}}{n}$ converges.

\paragraph{Rudin.exercise.3.13} If $\sum_{i=1}^\infty |a_i|$ and $\sum_{i=1}^\infty |b_i|$ converge, then $\sum_{i=1}^\infty \sum_{j=1}^{i+1} a_jb_{i-j}$ converges.

\paragraph{Rudin.exercise.3.21} Let $X$ be a complete metric space. Suppose that $\{E_n\}_{n=1}^\infty$ is a sequence of nonempty closed subsets of $X$ such that $E_n\supset E_{n+1}$ for all $n$ and $\lim_{n\to\infty} diam(E_n)=0$. Prove that $\bigcap_{n=1}^\infty E_n$ is a singleton.

\paragraph{Rudin.exercise.4.1a} There exists a function $f:\mathbb{R}\to\mathbb{R}$ such that $f$ is not continuous but for each $x\in\mathbb{R}$, the function $y\mapsto f(x+y)-f(x-y)$ is continuous at $0$.

\paragraph{Rudin.exercise.4.3} Let $f$ be a continuous function from a metric space $X$ into $\mathbb{R}$. Prove that the set $f^{-1}(0)$ is closed.

\paragraph{Rudin.exercise.4.4b} Suppose $f$ and $g$ are continuous functions on a metric space $X$ and that $S$ is a dense subset of $X$. Prove that if $f(x)=g(x)$ for all $x \in S$, then $f(x)=g(x)$ for all $x \in X$.

\paragraph{Rudin.exercise.4.5b} There exists a set $E$ and a function $f:E\to\mathbb{R}$ such that $f$ is continuous on $E$ but there is no continuous function $g:\mathbb{R}\to\mathbb{R}$ such that $f(x)=g(x)$ for all $x\in E$.

\paragraph{Rudin.exercise.4.8a} If $f$ is uniformly continuous on a bounded set $E$, then $f(E)$ is bounded.

\paragraph{Rudin.exercise.4.11a} If $f:X\to Y$ is uniformly continuous and $x_n$ is a Cauchy sequence in $X$, then $f(x_n)$ is a Cauchy sequence in $Y$.

\paragraph{Rudin.exercise.4.15} If $f$ is continuous and open, then $f$ is monotone.

\paragraph{Rudin.exercise.4.21a} Suppose $K$ and $F$ are disjoint compact and closed subsets of a metric space $X$. Prove that there is a positive number $\delta$ such that $d(p,q)\geq \delta$ for all $p\in K$ and $q\in F$.

\paragraph{Rudin.exercise.5.1} Let $f$ be a function from $\mathbb{R}$ to $\mathbb{R}$ such that $|f(x)-f(y)|\leq (x-y)^2$ for all $x, y \in \mathbb{R}$. Prove that $f$ is constant.

\paragraph{Rudin.exercise.5.3} Suppose $g$ is continuous and $g'$ is bounded. Prove that for each $\epsilon>0$ there is a $\delta>0$ such that the function $x\mapsto x+\epsilon g(x)$ is one-to-one on the interval $(-\delta, \delta)$.

\paragraph{Rudin.exercise.5.5} Suppose that $f$ is differentiable on $\mathbb{R}$ and that $\lim_{x\to\infty} f'(x) = 0$. Prove that $\lim_{x\to\infty} f(x+1) - f(x) = 0$.

\paragraph{Rudin.exercise.5.7} Suppose that $f$ and $g$ are differentiable at $x$, that $g'(x)\neq 0$, and that $f(x)=g(x)=0$. Prove that $\lim_{t\to x}\frac{f(t)}{g(t)}=\frac{f'(x)}{g'(x)}$.

\paragraph{Rudin.exercise.5.17} Let $f$ be a function defined on $[-1, 1]$ and differentiable on $(-1, 1)$. Suppose that $f(-1)=0$, $f(0)=0$, $f(1)=1$, and $f'(0)=0$. Prove that there exists a point $x$ in $(-1, 1)$ such that $f'''(x)\geq 3$.

\paragraph{Shakarchi.exercise.1.13b} Let $f$ be a differentiable function on an open set $\Omega$ in $\mathbb{C}$. Suppose that the imaginary part of $f$ is constant on $\Omega$. Prove that $f$ is constant on $\Omega$.

\paragraph{Shakarchi.exercise.1.19a} Let $z$ be a complex number of modulus $1$. Let $s_n = 1 + 2z + 3z^2 + \dots + nz^{n-1}$. Prove that $s_n$ does not converge.

\paragraph{Shakarchi.exercise.1.19c} Let $z$ be a complex number of modulus $1$ and $z\neq 1$. Let $s_n = \sum_{i=1}^n i z^i/i$. Prove that $s_n$ converges.

\paragraph{Shakarchi.exercise.2.2} Prove that $\lim_{y\to\infty} \int_0^y \frac{\sin x}{x} dx = \frac{\pi}{2}$.

\paragraph{Shakarchi.exercise.2.13} Let $f$ be a function from $\mathbb{C}$ to $\mathbb{C}$. Suppose that for each $z_0\in\mathbb{C}$ there is an open set $S$ containing $z_0$ and a sequence of complex numbers $\{c_n\}$ such that for each $z\in S$ the series $\sum_{n=0}^\infty c_n(z-z_0)^n$ converges to $f(z)$ and $c_n=0$ for all but finitely many $n$. Prove that $f$ is a polynomial.

\paragraph{Shakarchi.exercise.3.4} Prove that $\lim_{y\to\infty} \int_{-y}^y \frac{x\sin x}{x^2+a^2} dx = \pi e^{-a}$.

\paragraph{Shakarchi.exercise.3.14} If $f$ is a differentiable function from $\mathbb{C}$ to $\mathbb{C}$ which is one-to-one, then $f$ is of the form $f(z)=az+b$ for some $a, b\in\mathbb{C}$ with $a\neq 0$.

\paragraph{Shakarchi.exercise.5.1} Let $f$ be a non-zero complex-valued function defined on the unit disk $D$ and differentiable on $D$. Suppose that $f$ has only finitely many zeros in $D$. Prove that the series $\sum_{n=1}^\infty (1-z_n)$ converges, where $z_n$ is the $n$th zero of $f$ in $D$.
\end{document}