\documentclass{article}

\title{ProofNet NL Statements}
\author{Zhangir Azerbayev}
\date{Summer 2022}

\usepackage{amsmath}
\usepackage{amssymb}
\usepackage{mathrsfs}
\usepackage{parskip}
\usepackage{fullpage}

\begin{document}
\maketitle
\section*{df}
\subsection*{chapter 1}
\subsubsection*{section 1}
1a: Prove that the the operation $\star$ on $\mathbb{Z}$ defined by $a \star b=a-b$ is not associative. 

2a: Prove the the operation $\star$ on $\mathbb{Z}$ defined by $a\star b=a-b$ is not commutative. 

3: Prove that the addition of residue classes $\mathbb{Z}/n\mathbb{Z}$ is associative. 

4: Prove that the multiplication of residue class $\mathbb{Z}/n\mathbb{Z}$ is associative. 

5: Prove that for all $n>1$ that $\mathbb{Z}/n\mathbb{Z}$ is not a group under multiplication of residue classes. 

15: Prove that $(a_1a_2\dots a_n)^{-1} = a_n^{-1}a_{n-1}^{-1}\dots a_1^{-1}$ for all $a_1, a_2, \dots, a_n\in G$. 

16: Let $x$ be an element of $G$. Prove that $x^2=1$ if and only if $|x|$ is either $1$ or $2$. 

17: Let $x$ be an element of $G$. Prove that if $|x|=n$ for some positive integer $n$ then $x^{-1}=x^{n-1}$. 

18: Let $x$ and $y$ be elements of $G$. Prove that $xy=yx$ if and only if $y^{-1}xy=x$ if and only if $x^{-1}y^{-1}xy=1$.

20: For $x$ an element in $G$ show that $x$ and $x^{-1}$ have the same order.

22a: If $x$ and $g$ are elements of the group $G$, prove that $|x|=\left|g^{-1} x g\right|$. 

22b: Deduce that $|a b|=|b a|$ for all $a, b \in G$.

25: Prove that if $x^{2}=1$ for all $x \in G$ then $G$ is abelian.

29: Prove that $A \times B$ is an abelian group if and only if both $A$ and $B$ are abelian.

34: If $x$ is an element of infinite order in $G$, prove that the elements $x^{n}, n \in \mathbb{Z}$ are all distinct.

\subsubsection*{section 3}
8: Prove that if $\Omega=\{1,2,3, \ldots\}$ then $S_{\Omega}$ is an infinite group

\subsubsection*{section 6}
4: Prove that the multiplicative groups $\mathbb{R}-\{0\}$ and $\mathbb{C}-\{0\}$ are not isomorphic.

11: Let $A$ and $B$ be groups. Prove that $A \times B \cong B \times A$.

17: Let $G$ be any group. Prove that the map from $G$ to itself defined by $g \mapsto g^{-1}$ is a homomorphism if and only if $G$ is abelian.

23: Let $G$ be a finite group which possesses an automorphism $\sigma$ such that $\sigma(g)=g$ if and only if $g=1$. If $\sigma^{2}$ is the identity map from $G$ to $G$, prove that $G$ is abelian. 

\subsubsection*{section 7}
5: Prove that the kernel of an action of the group $G$ on a set $A$ is the same as the kernel of the corresponding permutation representation $G\to S_A$. 

6: Prove that a group $G$ acts faithfully on a set $A$ if and only if the kernel of the action is the set consisting only of the identity. 

\subsection*{chapter 2}
\subsubsection*{section 1}
5: Prove that $G$ cannot have a subgroup $H$ with $|H|=n-1$, where $n=|G|>2$.

13: Let $H$ be a subgroup of the additive group of rational numbers with the property that $1 / x \in H$ for every nonzero element $x$ of $H$. Prove that $H=0$ or $\mathbb{Q}$.

\subsubsection*{section 2}
4: Prove that if $H$ is a subgroup of $G$ then $H$ is generated by the set $H-\{1\}$.

13: Prove that the multiplicative group of positive rational numbers is generated by the set $\left\{\frac{1}{p} \mid \text{$p$ is a prime} \right\}$. 

16a: A subgroup $M$ of a group $G$ is called a maximal subgroup if $M \neq G$ and the only subgroups of $G$ which contain $M$ are $M$ and $G$. Prove that if $H$ is a proper subgroup of the finite group $G$ then there is a maximal subgroup of $G$ containing $H$.

16c: Show that if $G=\langle x\rangle$ is a cyclic group of order $n \geq 1$ then a subgroup $H$ is maximal if and only $H=\left\langle x^{p}\right\rangle$ for some prime $p$ dividing $n$.

\subsection*{Chapter 3}
\subsubsection*{section 1}
3a: Let $A$ be an abelian group and let $B$ be a subgroup of $A$. Prove that $A / B$ is abelian.

22a: Prove that if $H$ and $K$ are normal subgroups of a group $G$ then their intersection $H \cap K$ is also a normal subgroup of $G$.

22b: Prove that the intersection of an arbitrary nonempty collection of normal subgroups of a group is a normal subgroup (do not assume the collection is countable).

\subsubsection*{section 2}
8: Prove that if $H$ and $K$ are finite subgroups of $G$ whose orders are relatively prime then $H \cap K=1$.

11: Let $H \leq K \leq G$. Prove that $|G: H|=|G: K| \cdot|K: H|$ (do not assume $G$ is finite).

16: Use Lagrange's Theorem in the multiplicative group $(\mathbb{Z} / p \mathbb{Z})^{\times}$to prove Fermat's Little Theorem: if $p$ is a prime then $a^{p} \equiv a(\bmod p)$ for all $a \in \mathbb{Z}$.

21a: Prove that $\mathbb{Q}$ has no proper subgroups of finite index.

\subsubsection*{section 3}
3: Prove that if $H$ is a normal subgroup of $G$ of prime index $p$ then for all $K \leq G$ either $K \leq H$ or $G=H K$ and $|K: K \cap H|=p$.

\subsubsection*{section 4}
1: Prove that if $G$ is an abelian simple group then $G \cong Z_{p}$ for some prime $p$ (do not assume $G$ is a finite group).

4: Use Cauchy's Theorem and induction to show that a finite abelian group has a subgroup of order $n$ for each positive divisor $n$ of its order. 

5a: Prove that subgroups of a solvable group are solvable.

5b: Prove that quotient groups of a solvable group are solvable.

11: Prove that if $H$ is a nontrivial normal subgroup of the solvable group $G$ then there is a nontrivial subgroup $A$ of $H$ with $A \unlhd G$ and $A$ abelian.

\subsection*{Chapter 4}
\subsubsection*{section 2}
8: Prove that if $H$ has finite index $n$ then there is a normal subgroup $K$ of $G$ with $K \leq H$ and $|G: K| \leq n$ !.

9a: Prove that if $p$ is a prime and $G$ is a group of order $p^{\alpha}$ for some $\alpha \in \mathbb{Z}^{+}$, then every subgroup of index $p$ is normal in $G$. 

14: Let $G$ be a finite group of composite order $n$ with the property that $G$ has a subgroup of order $k$ for each positive integer $k$ dividing $n$. Prove that $G$ is not simple.

\subsubsection*{section 3}
5: If the center of $G$ is of index $n$, prove that every conjugacy class has at most $n$ elements.

26: Let $G$ be a transitive permutation group on the finite set $A$ with $|A|>1$. Show that there is some $\sigma \in G$ such that $\sigma(a) \neq a$ for all $a \in A$ (such an element $\sigma$ is called fixed point free ).

27: Let $g_{1}, g_{2}, \ldots, g_{r}$ be representatives of the conjugacy classes of the finite group $G$ and assume these elements pairwise commute. Prove that $G$ is abelian.


\subsubsection*{section 4}
2: Prove that if $G$ is a n abelian group of order $p q$, where $p$ and $q$ are distinct primes, then $G$ is cyclic.

6a: Prove that characteristic subgroups are normal.

7: If $H$ is the unique subgroup of a given order in a group $G$ prove $H$ is characteristic in $G$.

8a: Let $G$ be a group with subgroups $H$ and $K$ with $H \leq K$. Prove that if $H$ is characteristic in $K$ and $K$ is normal in $G$ then $H$ is normal in $G$.

\subsubsection*{section 5}
1: Prove that if $P \in \mathrm{Syl}_{p}(G)$ and $H$ is a subgroup of $G$ containing $P$ then $P \in\mathrm{Syl}_{p}(H)$.

13: Prove that a group of order 56 has a normal Sylow $p$-subgroup for some prime $p$ dividing its order.

14: Prove that a group of order 312 has a normal Sylow $p$-subgroup for some prime $p$ dividing its order.

15: Prove that a group of order 351 has a normal Sylow $p$-subgroup for some prime $p$ dividing its order.

16: Let $|G|=p q r$, where $p, q$ and $r$ are primes with $p<q<r$. Prove that $G$ has a normal Sylow subgroup for either $p, q$ or $r$.

17: Prove that if $|G|=105$ then $G$ has a normal Sylow 5 -subgroup and a normal Sylow 7-subgroup.

18: Prove that a group of order 200 has a normal Sylow 5-subgroup.

19: Prove that if $|G|=6545$ then $G$ is not simple.

20: Prove that if $|G|=1365$ then $G$ is not simple.

21: Prove that if $|G|=2907$ then $G$ is not simple.

22: Prove that if $|G|=132$ then $G$ is not simple.

23: Prove that if $|G|=462$ then $G$ is not simple.

28: Let $G$ be a group of order 105. Prove that if a Sylow 3-subgroup of $G$ is normal then $G$ is abelian.

33: Let $P$ be a normal Sylow $p$-subgroup of $G$ and let $H$ be any subgroup of $G$. Prove that $P \cap H$ is the unique Sylow $p$-subgroup of $H$.

\subsection*{Chapter 5}
\subsubsection*{section 4}
2: Prove that a subgroup $H$ of $G$ is normal if and only if $[G, H] \leq H$.

\subsection*{Chapter 7}
\subsubsection{section 1}
2: Prove that if $u$ is a unit in $R$ then so is $-u$.

11: Prove that if $R$ is an integral domain and $x^{2}=1$ for some $x \in R$ then $x=\pm 1$.

12: Prove that any subring of a field which contains the identity is an integral domain.

15: A ring $R$ is called a Boolean ring if $a^{2}=a$ for all $a \in R$. Prove that every Boolean ring is commutative.

\subsubsection*{section 2}
2: Let $p(x)=a_{n} x^{n}+a_{n-1} x^{n-1}+\cdots+a_{1} x+a_{0}$ be an element of the polynomial ring $R[x]$. Prove that $p(x)$ is a zero divisor in $R[x]$ if and only if there is a nonzero $b \in R$ such that $b p(x)=0$. 

4: Prove that if $R$ is an integral domain then the ring of formal power series $R[[x]]$ is also an integral domain.

12: Let $G=\left\{g_{1}, \ldots, g_{n}\right\}$ be a finite group. Prove that the element $N=g_{1}+g_{2}+\ldots+g_{n}$ is in the center of the group ring $R G$.

\subsubsection*{section 3}
16: Let $\varphi: R \rightarrow S$ be a surjective homomorphism of rings. Prove that the image of the center of $R$ is contained in the center of $S$. 

28: Prove that an integral domain has characteristic $p$, where $p$ is either a prime or 0

37: An ideal $N$ is called nilpotent if $N^{n}$ is the zero ideal for some $n \geq 1$. Prove that the ideal $p \mathbb{Z} / p^{m} \mathbb{Z}$ is a nilpotent ideal in the ring $\mathbb{Z} / p^{m} \mathbb{Z}$.

\subsubsection*{section 4}
27: Let $R$ be a commutative ring with $1 \neq 0$. Prove that if $a$ is a nilpotent element of $R$ then $1-a b$ is a unit for all $b \in R$.

\section*{Chapter 8}
\subsubsection*{section 1}
12: Let $N$ be a positive integer. Let $M$ be an integer relatively prime to $N$ and let $d$ be an integer relatively prime to $\varphi(N)$, where $\varphi$ denotes Euler's $\varphi$-function. Prove that if $M_{1} \equiv M^{d}(\bmod N)$ then $M \equiv M_{1}^{d^{\prime}}(\bmod N)$ where $d^{\prime}$ is the inverse of $d$ $\bmod \varphi(N): d d^{\prime} \equiv 1(\bmod \varphi(N))$

\subsubsection*{section 2}
4: Let $R$ be an integral domain. Prove that if the following two conditions hold then $R$ is a Principal Ideal Domain: (i) any two nonzero elements $a$ and $b$ in $R$ have a greatest common divisor which can be written in the form $r a+s b$ for some $r, s \in R$, and (ii) if $a_{1}, a_{2}, a_{3}, \ldots$ are nonzero elements of $R$ such that $a_{i+1} \mid a_{i}$ for all $i$, then there is a positive integer $N$ such that $a_{n}$ is a unit times $a_{N}$ for all $n \geq N$.

\subsubsection*{section 3}
4: Prove that if an integer is the sum of two rational squares, then it is the sum of two integer squares (for example, $13=(1 / 5)^{2}+(18 / 5)^{2}=2^{2}+3^{2}$ ).

5a: Let $R=\mathbb{Z}[\sqrt{-n}]$ where $n$ is a squarefree integer greater than 3. Prove that $2, \sqrt{-n}$ and $1+\sqrt{-n}$ are irreducibles in $R$.

6a: Prove that the quotient ring $\mathbb{Z}[i] /(1+i)$ is a field of order 2 .

6b: Let $q \in \mathbb{Z}$ be a prime with $q \equiv 3 \bmod 4$. Prove that the quotient ring $\mathbb{Z}[i] /(q)$ is a field with $q^{2}$ elements.

\subsection*{chapter 9}
\subsubsection*{section 1}
6: Prove that $(x, y)$ is not a principal ideal in $\mathbb{Q}[x, y]$.

10: Prove that the ring $\mathbb{Z}\left[x_{1}, x_{2}, x_{3}, \ldots\right] /\left(x_{1} x_{2}, x_{3} x_{4}, x_{5} x_{6}, \ldots\right)$ contains infinitely many minimal prime ideals (cf. Exercise 36 of Section 7.4).

\subsubsection*{section 3}
2: Prove that if $f(x)$ and $g(x)$ are polynomials with rational coefficients whose product $f(x) g(x)$ has integer coefficients, then the product of any coefficient of $g(x)$ with any coefficient of $f(x)$ is an integer.

\subsubsection*{section 4}
2a: Prove that $x^4-4x^3+6$ is irreducible in $\mathbb{Z}[x]$. 

2b: Prove that $x^6+30x^5-15x^+6x-120$ is irreducible in $\mathbb{Z}[x]$. 

2c: Prove that $x^4+4x^3+6x^2+2x+1$ is irreducible in $\mathbb{Z}[x]$. 

2d: Prove that $\frac{(x+2)^p-2^p}{x}$, where $p$ is an odd prime, is irreducible in $\mathbb{Z}[x]$. 

9: Prove that the polynomial $x^{2}-\sqrt{2}$ is irreducible over $\mathbb{Z}[\sqrt{2}]$. You may assume that $\mathbb{Z}[\sqrt{2}]$ is a U.F.D. 

11: Prove that $x^2+y^2-1$ is irreducible in $\mathbb{Q}[x,y]$. 
\section*{Rudin}
\subsection*{chapter 1}
1: If $r$ is rational $(r \neq 0)$ and $x$ is irrational, prove that $r+x$ and $r x$ are irrational.

2: Prove that there is no rational number whose square is $12$. 

4: Let $E$ be a nonempty subset of an ordered set; suppose $\alpha$ is a lower bound of $E$ and $\beta$ is an upper bound of $E$. Prove that $\alpha \leq \beta$.

5: Let $A$ be a nonempty set of real numbers which is bounded below. Let $-A$ be the set of all numbers $-x$, where $x \in A$. Prove that $\inf A=-\sup (-A)$

8: Prove that no order can be defined in the complex field that turns it into an ordered field. Hint: $-1$ is a square.

14: If $z$ is a complex number such that $|z|=1$, that is, such that $z \bar{z}=1$, compute $|1+z|^{2}+|1-z|^{2}$. 

17: Prove that
$|\mathbf{x}+\mathbf{y}|^{2}+|\mathbf{x}-\mathbf{y}|^{2}=2|\mathbf{x}|^{2}+2|\mathbf{y}|^{2}$ if $\mathbf{x} \in R^{k}$ and $\mathbf{y} \in R^{k}$.

18a: If $k \geq 2$ and $\mathbf{x} \in R^{k}$, prove that there exists $\mathbf{y} \in R^{k}$ such that $\mathbf{y} \neq 0$ but $\mathbf{x} \cdot \mathbf{y}=0$

25: Prove that every compact metric space $K$ has a countable base. 

27a: Suppose $E\subset\mathbb{R}^k$ is uncountable, and let $P$ be the set of condensation points of $E$. Prove that $P$ is perfect. 

27b: Suppose $E\subset\mathbb{R}^k$ is uncountable, and let $P$ be the set of condensation points of $E$. Prove that at most countably many point of $E$ are not in $P$. 

28: Prove that every closed set in a separable metric space is the union of a (possibly empty) perfect set and a set which is at most countable. 

29: Prove that every open set in $\mathbb{R}$ is the union of an at most countable collection of disjoint segments. 

\subsection*{chapter 2}
19a: If $A$ and $B$ are disjoint closed sets in some metric space $X$, prove that they are separated.

24: Let $X$ be a metric space in which every infinite subset has a limit point. Prove that $X$ is separable. Hint: Fix $\delta>0$, and pick $x_{1} \in X$. Having chosen $x_{1}, \ldots, x_{J} \in X$,

\subsection*{chapter 3}
1a: Prove that convergence of $\left\{s_{n}\right\}$ implies convergence of $\left\{\left|s_{n}\right|\right\}$. 

3: If $s_{1}=\sqrt{2}$, and $s_{n+1}=\sqrt{2+\sqrt{s_{n}}} \quad(n=1,2,3, \ldots),$ prove that $\left\{s_{n}\right\}$ converges, and that $s_{n}<2$ for $n=1,2,3, \ldots$.

5: For any two real sequences $\left\{a_{n}\right\},\left\{b_{n}\right\}$, prove that $\limsup _{n \rightarrow \infty}\left(a_{n}+b_{n}\right) \leq \limsup _{n \rightarrow \infty} a_{n}+\limsup _{n \rightarrow \infty} b_{n},$ provided the sum on the right is not of the form $\infty-\infty$.

7: Prove that the convergence of $\Sigma a_{n}$ implies the convergence of $\sum \frac{\sqrt{a_{n}}}{n}$ if $a_n\geq 0$.

8: If $\Sigma a_{n}$ converges, and if $\left\{b_{n}\right\}$ is monotonic and bounded, prove that $\Sigma a_{n} b_{n}$ converges.

13: Prove that the Cauchy product of two absolutely convergent series converges absolutely.

20: 20. Suppose $\left\{p_{n}\right\}$ is a Cauchy sequence in a metric space $X$, and some subsequence $\left\{p_{n l}\right\}$ converges to a point $p \in X$. Prove that the full sequence $\left\{p_{n}\right\}$ converges to $p$.

21: If $\left\{E_{n}\right\}$ is a sequence of closed nonempty and bounded sets in a complete metric space $X$, if $E_{n} \supset E_{n+1}$, and if $\lim _{n \rightarrow \infty} \operatorname{diam} E_{n}=0,$ then $\bigcap_{1}^{\infty} E_{n}$ consists of exactly one point.

22: Suppose $X$ is a nonempty complete metric space, and $\left\{G_{n}\right\}$ is a sequence of dense open subsets of $X$. Prove Baire's theorem, namely, that $\bigcap_{1}^{\infty} G_{n}$ is not empty. Hint: Find a shrinking sequence of neighborhoods $E_{n}$ such that $E_{n} \subset G_{n}$. 

\subsection*{chapter 4}
2a: If $f$ is a continuous mapping of a metric space $X$ into a metric space $Y$, prove that $f(\overline{E}) \subset \overline{f(E)}$ for every set $E \subset X$. ($\overline{E}$ denotes the closure of $E$). 

3: Let $f$ be a continuous real function on a metric space $X$. Let $Z(f)$ (the zero set of $f$ ) be the set of all $p \in X$ at which $f(p)=0$. Prove that $Z(f)$ is closed.

4a: Let $f$ and $g$ be continuous mappings of a metric space $X$ into a metric space $Y$, and let $E$ be a dense subset of $X$. Prove that $f(E)$ is dense in $f(X)$. 

5a: If $f$ is a real continuous function defined on a closed set $E \subset R^{1}$, prove that there exist continuous real functions $g$ on $R^{1}$ such that $g(x)=f(x)$ for all $x \in E$.

6: If $f$ is defined on $E$, the graph of $f$ is the set of points $(x, f(x))$, for $x \in E$. In particular, if $E$ is a set of real numbers, and $f$ is real-valued, the graph of $f$ is a subset of the plane. Suppose $E$ is compact, and prove that $f$ is continuous on $E$ if and only if its graph is compact.

8a: Let $f$ be a real uniformly continuous function on the bounded set $E$ in $R^{1}$. Prove that $f$ is bounded on $E$.

11a: Suppose $f$ is a uniformly continuous mapping of a metric space $X$ into a metric space $Y$ and prove that $\left\{f\left(x_{n}\right)\right\}$ is a Cauchy sequence in 
12: A uniformly continuous function of a uniformly continuous function is uniformly continuous.

14: Let $I=[0,1]$ be the closed unit interval. Suppose $f$ is a continuous mapping of $I$ into $I$. Prove that $f(x)=x$ for at least one $x \in I$.

15: Prove that every continuous open mapping of $R^{1}$ into $R^{1}$ is monotonic.

19: Suppose $f$ is a real function with domain $R^{1}$ which has the intermediate value property: If $f(a)<c<f(b)$, then $f(x)=c$ for some $x$ between $a$ and $b$. Suppose also, for every rational $r$, that the set of all $x$ with $f(x)=r$ is closed. Prove that $f$ is continuous.

21a: Suppose $K$ and $F$ are disjoint sets in a metric space $X, K$ is compact, $F$ is closed. Prove that there exists $\delta>0$ such that $d(p, q)>\delta$ if $p \in K, q \in F$. 

24: Assume that $f$ is a continuous real function defined in $(a, b)$ such that $f\left(\frac{x+y}{2}\right) \leq \frac{f(x)+f(y)}{2}$ for all $x, y \in(a, b)$. Prove that $f$ is convex.

26a: Suppose $X, Y, Z$ are metric spaces, and $Y$ is compact. Let $f$ map $X$ into $Y$, let $g$ be a continuous one-to-one mapping of $Y$ into $Z$, and put $h(x)=g(f(x))$ for $x \in X$. Prove that $f$ is uniformly continuous if $h$ is uniformly continuous.

\subsection*{chapter 5}
1: Let $f$ be defined for all real $x$, and suppose that $|f(x)-f(y)| \leq(x-y)^{2}$for all real $x$ and $y$. Prove that $f$ is constant.

2: Suppose $f^{\prime}(x)>0$ in $(a, b)$. Prove that $f$ is strictly increasing in $(a, b)$, and let $g$ be its inverse function. Prove that $g$ is differentiable, and that$g^{\prime}(f(x))=\frac{1}{f^{\prime}(x)} \quad(a<x<b)$

3: Suppose $g$ is a real function on $R^{1}$, with bounded derivative (say $\left|g^{\prime}\right| \leq M$ ). Fix $\varepsilon>0$, and define $f(x)=x+\varepsilon g(x)$. Prove that $f$ is one-to-one if $\varepsilon$ is small enough.

4: If $C_{0}+\frac{C_{1}}{2}+\cdots+\frac{C_{n-1}}{n}+\frac{C_{n}}{n+1}=0,$ where $C_{0}, \ldots, C_{n}$ are real constants, prove that the equation $C_{0}+C_{1} x+\cdots+C_{n-1} x^{n-1}+C_{n} x^{n}=0$ has at least one real root between 0 and 1 .

5: Suppose $f$ is defined and differentiable for every $x>0$, and $f^{\prime}(x) \rightarrow 0$ as $x \rightarrow+\infty$. Put $g(x)=f(x+1)-f(x)$. Prove that $g(x) \rightarrow 0$ as $x \rightarrow+\infty$.

6: Suppose (a) $f$ is continuous for $x \geq 0$, (b) $f^{\prime}(x)$ exists for $x>0$, (c) $f(0)=0$, (d) $f^{\prime}$ is monotonically increasing. Put $g(x)=\frac{f(x)}{x} \quad(x>0)$ and prove that $g$ is monotonically increasing.

7: Suppose $f^{\prime}(x), g^{\prime}(x)$ exist, $g^{\prime}(x) \neq 0$, and $f(x)=g(x)=0$. Prove that $\lim _{t \rightarrow x} \frac{f(t)}{g(t)}=\frac{f^{\prime}(x)}{g^{\prime}(x)}.$

15: Suppose $a \in R^{1}, f$ is a twice-differentiable real function on $(a, \infty)$, and $M_{0}, M_{1}, M_{2}$ are the least upper bounds of $|f(x)|,\left|f^{\prime}(x)\right|,\left|f^{\prime \prime}(x)\right|$, respectively, on $(a, \infty)$. Prove that $M_{1}^{2} \leq 4 M_{0} M_{2} .$

16: Suppose $f$ is a real, three times differentiable function on $[-1,1]$, such that
$f(-1)=0, \quad f(0)=0, \quad f(1)=1, \quad f^{\prime}(0)=0 .$ Prove that $f^{(3)}(x) \geq 3$ for some $x \in(-1,1)$.

\subsection*{chapter 5}
1: Suppose $\alpha$ increases on $[a, b], a \leq x_{0} \leq b, \alpha$ is continuous at $x_{0}, f\left(x_{0}\right)=1$, and $f(x)=0$ if $x \neq x_{0}$. Prove that $f \in \mathscr{R}(\alpha)$ and that $\int f d \alpha=0$.

2: Suppose $f \geq 0, f$ is continuous on $[a, b]$, and $\int_{a}^{b} f(x) d x=0$. Prove that $f(x)=0$ for all $x \in[a, b]$.

4: If $f(x)=0$ for all irrational $x, f(x)=1$ for all rational $x$, prove that $f \notin \mathscr{R}$ on $[a, b]$ for any $a<b$.

6: Let $P$ be the Cantor set. Let $f$ be a bounded real function on $[0,1]$ which is continuous at every point outside $P$. Prove that $f \in \mathscr{R}$ on $[0,1]$.

\section*{Munkres}
\subsection*{chapter 2}
\subsubsection*{section 13}
5a: Show that if $\mathcal{A}$ is a basis for a topology on $X$, then the topology generated by $\mathcal{A}$ equals the intersection of all topologies on $X$ that contain $\mathcal{A}$. 

\subsubsection*{section 16}
4: A map $f: X \rightarrow Y$ is said to be an open map if for every open set $U$ of $X$, the set $f(U)$ is open in $Y$. Show that $\pi_{1}: X \times Y \rightarrow X$ and $\pi_{2}: X \times Y \rightarrow Y$ are open maps.

\subsubsection*{section 17}
2: Show that if $A$ is closed in $Y$ and $Y$ is closed in $X$, then $A$ is closed in $X$.

3: Show that if $A$ is closed in $X$ and $B$ is closed in $Y$, then $A \times B$ is closed in $X \times Y$.

4: Show that if $U$ is open in $X$ and $A$ is closed in $X$, then $U-A$ is open in $X$, and $A-U$ is closed in $X$.

\subsubsection*{18}
8a: Let $Y$ be an ordered set in the order topology. Let $f, g: X \rightarrow Y$ be continuous. Show that the set $\{x \mid f(x) \leq g(x)\}$ is closed in $X$.

8b: Let $Y$ be an ordered set in the order topology. Let $f, g: X \rightarrow Y$ be continuous. Let $h: X \rightarrow Y$ be the function $h(x)=\min \{f(x), g(x)\}.$ Show that $h$ is continuous. [Hint: Use the pasting lemma.] 

13: Let $A \subset X$; let $f: A \rightarrow Y$ be continuous; let $Y$ be Hausdorff. Show that if $f$ may be extended to a continuous function $g: \bar{A} \rightarrow Y$, then $g$ is uniquely determined by $f$.

\subsubsection*{21}

6a: Define $f_{n}:[0,1] \rightarrow \mathbb{R}$ by the equation $f_{n}(x)=x^{n}$. Show that the sequence $\left(f_{n}(x)\right)$ converges for each $x \in[0,1]$. 

6b: Define $f_{n}:[0,1] \rightarrow \mathbb{R}$ by the equation $f_{n}(x)=x^{n}$. Show that the sequence $\left(f_{n}\right)$ does not converge uniformly.

8: Let $X$ be a topological space and let $Y$ be a metric space. Let $f_{n}: X \rightarrow Y$ be a sequence of continuous functions. Let $x_{n}$ be a sequence of points of $X$ converging to $x$. Show that if the sequence $\left(f_{n}\right)$ converges uniformly to $f$, then $\left(f_{n}\left(x_{n}\right)\right)$ converges to $f(x)$.

\subsubsection*{22}
1: Let $p: X \rightarrow Y$ be a continuous map. Show that if there is a continuous map $f: Y \rightarrow X$ such that $p \circ f$ equals the identity map of $Y$, then $p$ is a quotient map.

2a: Let $p: X \rightarrow Y$ be a continuous map. Show that if there is a continuous map $f: Y \rightarrow X$ such that $p \circ f$ equals the identity map of $Y$, then $p$ is a quotient map.

2b: If $A \subset X$, a retraction of $X$ onto $A$ is a continuous map $r: X \rightarrow A$ such that $r(a)=a$ for each $a \in A$. Show that a retraction is a quotient map.

3: Let $H$ be a subspace of $G$. Show that if $H$ is also a subgroup of $G$, then both $H$ and $\bar{H}$ are topological groups.
\end{document}
