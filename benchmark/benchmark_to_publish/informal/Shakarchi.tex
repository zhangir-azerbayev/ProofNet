\documentclass{article}

\title{\textbf{
Exercises from \\
\textit{Complex analysis} \\
by Elias M. Stein and Rami Shakarchi
}}

\date{}

\usepackage{amsmath}
\usepackage{amssymb}

\begin{document}
\maketitle


\paragraph{Exercise 1.13a} Suppose that $f$ is holomorphic in an open set $\Omega$. Prove that if $\text{Re}(f)$ is constant, then $f$ is constant.
\begin{proof}
Let $f(z)=f(x, y)=u(x, y)+i v(x, y)$, where $z=x+i y$.
Since $\operatorname{Re}(f)=$ constant,
$$
\frac{\partial u}{\partial x}=0, \frac{\partial u}{\partial y}=0 .
$$
By the Cauchy-Riemann equations,
$$
\frac{\partial v}{\partial x}=-\frac{\partial u}{\partial y}=0 .
$$
Thus, in $\Omega$,
$$
f^{\prime}(z)=\frac{\partial f}{\partial x}=\frac{\partial u}{\partial x}+i \frac{\partial v}{\partial x}=0+0=0 .
$$
3
Thus $f(z)$ is constant.
\end{proof}



\paragraph{Exercise 1.13b} Suppose that $f$ is holomorphic in an open set $\Omega$. Prove that if $\text{Im}(f)$ is constant, then $f$ is constant.
\begin{proof}
Let $f(z)=f(x, y)=u(x, y)+i v(x, y)$, where $z=x+i y$.
Since $\operatorname{Im}(f)=$ constant,
$$
\frac{\partial v}{\partial x}=0, \frac{\partial v}{\partial y}=0 .
$$
By the Cauchy-Riemann equations,
$$
\frac{\partial u}{\partial x}=\frac{\partial v}{\partial y}=0 .
$$
Thus in $\Omega$,
$$
f^{\prime}(z)=\frac{\partial f}{\partial x}=\frac{\partial u}{\partial x}+i \frac{\partial v}{\partial x}=0+0=0 .
$$
Thus $f$ is constant.
\end{proof}


\paragraph{Exercise 1.13c} Suppose that $f$ is holomorphic in an open set $\Omega$. Prove that if $|f|$ is constant, then $f$ is constant.
\begin{proof}
Let $f(z)=f(x, y)=u(x, y)+i v(x, y)$, where $z=x+i y$.
We first give a mostly correct argument; the reader should pay attention to find the difficulty. Since $|f|=\sqrt{u^2+v^2}$ is constant,
$$
\left\{\begin{array}{l}
0=\frac{\partial\left(u^2+v^2\right)}{\partial x}=2 u \frac{\partial u}{\partial x}+2 v \frac{\partial v}{\partial x} . \\
0=\frac{\partial\left(u^2+v^2\right)}{\partial y}=2 u \frac{\partial u}{\partial y}+2 v \frac{\partial v}{\partial y} .
\end{array}\right.
$$
Plug in the Cauchy-Riemann equations and we get
$$
\begin{gathered}
u \frac{\partial v}{\partial y}+v \frac{\partial v}{\partial x}=0 \\
-u \frac{\partial v}{\partial x}+v \frac{\partial v}{\partial y}=0 \\
(1.14) \Rightarrow \frac{\partial v}{\partial x}=\frac{v}{u} \frac{\partial v}{\partial y}
\end{gathered}
$$
Plug (1.15) into (1.13) and we get
$$
\frac{u^2+v^2}{u} \frac{\partial v}{\partial y}=0 .
$$
So $u^2+v^2=0$ or $\frac{\partial v}{\partial y}=0$.

If $u^2+v^2=0$, then, since $u, v$ are real, $u=v=0$, and thus $f=0$ which is constant.

Thus we may assume $u^2+v^2$ equals a non-zero constant, and we may divide by it. We multiply both sides by $u$ and find $\frac{\partial v}{\partial y}=0$, then by (1.15), $\frac{\partial v}{\partial x}=0$, and by Cauchy-Riemann, $\frac{\partial u}{\partial x}=0$.
$$
f^{\prime}=\frac{\partial f}{\partial x}=\frac{\partial u}{\partial x}+i \frac{\partial v}{\partial x}=0 .
$$
Thus $f$ is constant.
Why is the above only mostly a proof? The problem is we have a division by $u$, and need to make sure everything is well-defined. Specifically, we need to know that $u$ is never zero. We do have $f^{\prime}=0$ except at points where $u=0$, but we would need to investigate that a bit more.
Let's return to
$$
\left\{\begin{array}{l}
0=\frac{\partial\left(u^2+v^2\right)}{\partial x}=2 u \frac{\partial u}{\partial x}+2 v \frac{\partial v}{\partial x} . \\
0=\frac{\partial\left(u^2+v^2\right)}{\partial y}=2 u \frac{\partial u}{\partial y}+2 v \frac{\partial v}{\partial y} .
\end{array}\right.
$$
Plug in the Cauchy-Riemann equations and we get
$$
\begin{array}{r}
u \frac{\partial v}{\partial y}+v \frac{\partial v}{\partial x}=0 \\
-u \frac{\partial v}{\partial x}+v \frac{\partial v}{\partial y}=0 .
\end{array}
$$
We multiply the first equation $u$ and the second by $v$, and obtain
$$
\begin{aligned}
u^2 \frac{\partial v}{\partial y}+u v \frac{\partial v}{\partial x} & =0 \\
-u v \frac{\partial v}{\partial x}+v^2 \frac{\partial v}{\partial y} & =0 .
\end{aligned}
$$
Adding the two yields
$$
u^2 \frac{\partial v}{\partial y}+v^2 \frac{\partial v}{\partial y}=0,
$$
or equivalently
$$
\left(u^2+v^2\right) \frac{\partial v}{\partial y}=0 .
$$
We now argue in a similar manner as before, except now we don't have the annoying $u$ in the denominator. If $u^2+v^2=0$ then $u=v=0$, else we can divide by $u^2+v^2$ and find $\partial v / \partial y=0$. Arguing along these lines finishes the proof.
\end{proof}



\paragraph{Exercise 1.19a} Prove that the power series $\sum nz^n$ does not converge on any point of the unit circle.


\paragraph{Exercise 1.19b} Prove that the power series $\sum zn/n^2$ converges at every point of the unit circle.


\paragraph{Exercise 1.19c} Prove that the power series $\sum zn/n$ converges at every point of the unit circle except $z = 1$.


\paragraph{Exercise 1.22} Let $\mathbb{N} = {1, 2, 3, \ldots}$ denote the set of positive integers. A subset $S \subset \mathbb{N}$ is said to be in arithmetic progression if $S = {a, a + d, a + 2d, a + 3d, \ldots}$ where $a, d \in \mathbb{N}$. Here $d$ is called the step of $S$.  Show that $\mathbb{N}$ cannot be partitioned into a finite number of subsets that are in arithmetic progression with distinct steps (except for the trivial case $a = d = 1$).


\paragraph{Exercise 1.26} Suppose $f$ is continuous in a region $\Omega$. Prove that any two primitives of $f$ (if they exist) differ by a constant.


\paragraph{Exercise 2.2} Show that $\int_{0}^{\infty} \frac{\sin x}{x} d x=\frac{\pi}{2}$.


\paragraph{Exercise 2.9} Let $\Omega$ be a bounded open subset of $\mathbb{C}$, and $\varphi: \Omega \rightarrow \Omega$ a holomorphic function. Prove that if there exists a point $z_{0} \in \Omega$ such that $\varphi\left(z_{0}\right)=z_{0} \quad \text { and } \quad \varphi^{\prime}\left(z_{0}\right)=1$ then $\varphi$ is linear.


\paragraph{Exercise 2.13} Suppose $f$ is an analytic function defined everywhere in $\mathbb{C}$ and such that for each $z_0 \in \mathbb{C}$ at least one coefficient in the expansion $f(z) = \sum_{n=0}^\infty c_n(z - z_0)^n$ is equal to 0. Prove that $f$ is a polynomial.


\paragraph{Exercise 3.3} Show that $ \int_{-\infty}^{\infty} \frac{\cos x}{x^2 + a^2} dx = \pi \frac{e^{-a}}{a}$ for $a > 0$.


\paragraph{Exercise 3.4} Show that $ \int_{-\infty}^{\infty} \frac{x \sin x}{x^2 + a^2} dx = \pi e^{-a}$ for $a > 0$.


\paragraph{Exercise 3.9} Show that $\int_0^1 \log(\sin \pi x) dx = - \log 2$.


\paragraph{Exercise 3.14} Prove that all entire functions that are also injective take the form $f(z) = az + b$, $a, b \in \mathbb{C}$ and $a \neq 0$.


\paragraph{Exercise 3.22} Show that there is no holomorphic function $f$ in the unit disc $D$ that extends continuously to $\partial D$ such that $f(z) = 1/z$ for $z \in \partial D$.


\paragraph{Exercise 5.1} Prove that if $f$ is holomorphic in the unit disc, bounded and not identically zero, and $z_{1}, z_{2}, \ldots, z_{n}, \ldots$ are its zeros $\left(\left|z_{k}\right|<1\right)$, then $\sum_{n}\left(1-\left|z_{n}\right|\right)<\infty$.


\end{document}
