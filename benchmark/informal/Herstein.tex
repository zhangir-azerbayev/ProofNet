\documentclass{article}

\title{\textbf{
Exercises from \\
\textit{Abstract Algebra} \\
by I. N. Herstein
}}

\date{}

\usepackage{amsmath}
\usepackage{amssymb}

\begin{document}
\maketitle

\paragraph{Exercise 2.1.18} If $G$ is a finite group of even order, show that there must be an element $a \neq e$ such that $a=a^{-1}$.

\paragraph{Exercise 2.1.21} Show that a group of order 5 must be abelian.

\paragraph{Exercise 2.1.26} If $G$ is a finite group, prove that, given $a \in G$, there is a positive integer $n$, depending on $a$, such that $a^n = e$.

\paragraph{Exercise 2.1.27} If $G$ is a finite group, prove that there is an integer $m > 0$ such that $a^m = e$ for all $a \in G$.

\paragraph{Exercise 2.2.3} If $G$ is a group in which $(a b)^{i}=a^{i} b^{i}$ for three consecutive integers $i$, prove that $G$ is abelian.

\paragraph{Exercise 2.2.5} Let $G$ be a group in which $(a b)^{3}=a^{3} b^{3}$ and $(a b)^{5}=a^{5} b^{5}$ for all $a, b \in G$. Show that $G$ is abelian.

\paragraph{Exercise 2.2.6c} Let $G$ be a group in which $(a b)^{n}=a^{n} b^{n}$ for some fixed integer $n>1$ for all $a, b \in G$. For all $a, b \in G$, prove that $\left(a b a^{-1} b^{-1}\right)^{n(n-1)}=e$. 

\paragraph{Exercise 2.3.17} If $G$ is a group and $a, x \in G$, prove that $C\left(x^{-1} a x\right)=x^{-1} C(a) x$

\paragraph{Exercise 2.3.19} If $M$ is a subgroup of $G$ such that $x^{-1} M x \subset M$ for all $x \in G$, prove that actually $x^{-1} M x=M$. 

\paragraph{Exercise 2.3.16} If a group $G$ has no proper subgroups, prove that $G$ is cyclic of order $p$, where $p$ is a prime number.

\paragraph{Exercise 2.3.21} If $A, B$ are subgroups of $G$ such that $b^{-1} Ab \subset A$ for all $b \in B$, show that $AB$ is a subgroup of $G$.

\paragraph{Exercise 2.3.22} If $A$ and $B$ are finite subgroups, of orders $m$ and $n$, respectively, of the abelian group $G$, prove that $AB$ is a subgroup of order $mn$ if $m$ and $n$ are relatively prime.

\paragraph{Exercise 2.3.28} Let $M, N$ be subgroups of $G$ such that $x^{-1}Mx \subset M$ and $x^{-1} N x \subset N$ for all $x \in G$. Prove that $MN$ is a subgroup of $G$ and that $x^{-1} (MN)x \subset MN$ for all $x \in G$.

\paragraph{Exercise 2.3.29} If $M$ is a subgroup of $G$ such that $x^{-1}Mx \subset M$ for all $x \in G$, prove that actually $x^{-1}Mx = M$.

\paragraph{Exercise 2.4.8} If every right coset of $H$ in $G$ is a left coset of $H$ in $G$, prove that $aHa^{-1} = H$ for all $a \in G$.

\paragraph{Exercise 2.4.26} Let $G$ be a group, $H$ a subgroup of $G$, and let $S$ be the set of all distinct right cosets of $H$ in $G$, $T$ the set of all left cosets of $H$ in $G$. Prove that there is a 1-1 mapping of $S$ onto $T$.

\paragraph{Exercise 2.4.32} Let $G$ be a finite group, $H$ a subgroup of $G$. Let $f(a)$ be the least positive $m$ such that $a^m \in H$. Prove that $f(a) \mid o(a)$, where $o(a)$ is an order of $a$.

\paragraph{Exercise 2.4.36} If $a > 1$ is an integer, show that $n \mid \varphi(a^n - 1)$, where $\phi$ is the Euler $\varphi$-function.

\paragraph{Exercise 2.5.23} Let $G$ be a group such that all subgroups of $G$ are normal in $G$. If $a, b \in G$, prove that $ba = a^jb$ for some $j$.

\paragraph{Exercise 2.5.30} Suppose that $|G| = pm$, where $p \nmid m$ and $p$ is a prime. If $H$ is a normal subgroup of order $p$ in $G$, prove that $H$ is characteristic.

\paragraph{Exercise 2.5.31} Suppose that $G$ is an abelian group of order $p^nm$ where $p \nmid m$ is a prime.  If $H$ is a subgroup of $G$ of order $p^n$, prove that $H$ is a characteristic subgroup of $G$.

\paragraph{Exercise 2.5.37} If $G$ is a nonabelian group of order 6, prove that $G \simeq S_3$.

\paragraph{Exercise 2.5.43} Prove that a group of order 9 must be abelian.

\paragraph{Exercise 2.5.44} Prove that a group of order $p^2$, $p$ a prime, has a normal subgroup of order $p$.

\paragraph{Exercise 2.5.52} Let $G$ be a finite group and $\varphi$ an automorphism of $G$ such that $\varphi(x) = x^{-1}$ for more than three-fourths of the elements of $G$. Prove that $\varphi(y) = y^{-1}$ for all $y \in G$, and so $G$ is abelian.

\paragraph{Exercise 2.6.15} If $G$ is an abelian group and if $G$ has an element of order $m$ and one of order $n$, where $m$ and $n$ are relatively prime, prove that $G$ has an element of order $mn$.

\paragraph{Exercise 2.7.3} Let $G$ be the group of nonzero real numbers under multiplication and let $N = \{1, -1\}$. Prove that $G / N \simeq$ positive real numbers under multiplication.

\paragraph{Exercise 2.7.7} If $\varphi$ is a homomorphism of $G$ onto $G'$ and $N \triangleleft G$, show that $\varphi(N) \triangleleft G'$.

\paragraph{Exercise 2.8.7} If $G$ is a group with subgroups $A, B$ of orders $m, n$, respectively, where $m$ and $n$ are relatively prime, prove that the subset of $G$, $AB = \{ab \mid a \in A, b \in B\}$, has $mn$ distinct elements.

\paragraph{Exercise 2.8.12} Prove that any two nonabelian groups of order 21 are isomorphic.

\paragraph{Exercise 2.8.15} Prove that if $p > q$ are two primes such that $q \mid p - 1$, then any two nonabelian groups of order $pq$ are isomorphic.

\paragraph{Exercise 2.9.2} If $G_1$ and $G_2$ are cyclic groups of orders $m$ and $n$, respectively, prove that $G_1 \times G_2$ is cyclic if and only if $m$ and $n$ are relatively prime.

\paragraph{Exercise 2.10.1} Let $A$ be a normal subgroup of a group $G$, and suppose that $b \in G$ is an element of prime order $p$, and that $b \not\in A$. Show that $A \cap (b) = (e)$.

\paragraph{Exercise 2.11.6} If $P$ is a $p$-Sylow subgroup of $G$ and $P \triangleleft G$, prove that $P$ is the only $p$-Sylow subgroup of $G$.

\paragraph{Exercise 2.11.7} If $P \triangleleft G$, $P$ a $p$-Sylow subgroup of $G$, prove that $\varphi(P) = P$ for every automorphism $\varphi$ of $G$.

\paragraph{Exercise 2.11.22} Show that any subgroup of order $p^{n-1}$ in a group $G$ of order $p^n$ is normal in $G$.

\paragraph{Exercise 3.2.21} If $\sigma, \tau$ are two permutations that disturb no common element and $\sigma \tau = e$, prove that $\sigma = \tau = e$.

\paragraph{Exercise 3.2.23} Let $\sigma, \tau$ be two permutations such that they both have decompositions into disjoint cycles of cycles of lengths $m_1, m_2, \ldots, m_k$. Prove that for some permutation $\beta, \tau = \beta \sigma \beta^{-1}$.

\paragraph{Exercise 3.3.2} If $\sigma$ is a $k$-cycle, show that $\sigma$ is an odd permutation if $k$ is even, and is an even permutation if $k$ is odd.

\paragraph{Exercise 3.3.9} If $n \geq 5$ and $(e) \neq N \subset A_n$ is a normal subgroup of $A_n$, show that $N$ must contain a 3-cycle.

\paragraph{Exercise 4.1.19} Show that there is an infinite number of solutions to $x^2 = -1$ in the quaternions.

\paragraph{Exercise 4.1.34} Let $T$ be the group of $2\times 2$ matrices $A$ with entries in the field $\mathbb{Z}_2$ such that $\det A$ is not equal to 0. Prove that $T$ is isomorphic to $S_3$, the symmetric group of degree 3.

\paragraph{Exercise 4.2.5} Let $R$ be a ring in which $x^3 = x$ for every $x \in R$. Prove that $R$ is commutative.

\paragraph{Exercise 4.2.6} If $a^2 = 0$ in $R$, show that $ax + xa$ commutes with $a$.

\paragraph{Exercise 4.2.9} Let $p$ be an odd prime and let $1 + \frac{1}{2} + ... + \frac{1}{p - 1} = \frac{a}{b}$, where $a, b$ are integers. Show that $p \mid a$.

\paragraph{Exercise 4.3.1} If $R$ is a commutative ring and $a \in R$, let $L(a) = \{x \in R \mid xa = 0\}$. Prove that $L(a)$ is an ideal of $R$.

\paragraph{Exercise 4.3.25} Let $R$ be the ring of $2 \times 2$ matrices over the real numbers; suppose that $I$ is an ideal of $R$. Show that $I = (0)$ or $I = R$.

\paragraph{Exercise 4.4.9} Show that $(p - 1)/2$ of the numbers $1, 2, \ldots, p - 1$ are quadratic residues and $(p - 1)/2$ are quadratic nonresidues $\mod p$.

\paragraph{Exercise 4.5.12} If $F \subset K$ are two fields and $f(x), g(x) \in F[x]$ are relatively prime in $F[x]$, show that they are relatively prime in $K[x]$.

\paragraph{Exercise 4.5.16} Let $F = \mathbb{Z}_p$ be the field of integers $\mod p$, where $p$ is a prime, and let $q(x) \in F[x]$ be irreducible of degree $n$. Show that $F[x]/(q(x))$ is a field having at exactly $p^n$ elements.

\paragraph{Exercise 4.5.23} Let $F = \mathbb{Z}_7$ and let $p(x) = x^3 - 2$ and $q(x) = x^3 + 2$ be in $F[x]$. Show that $p(x)$ and $q(x)$ are irreducible in $F[x]$ and that the fields $F[x]/(p(x))$ and $F[x]/(q(x))$ are isomorphic.

\paragraph{Exercise 4.5.25} If $p$ is a prime, show that $q(x) = 1 + x + x^2 + \cdots x^{p - 1}$ is irreducible in $Q[x]$.

\paragraph{Exercise 4.6.2} Prove that $f(x) = x^3 + 3x + 2$ is irreducible in $Q[x]$.

\paragraph{Exercise 4.6.3} Show that there is an infinite number of integers a such that $f(x) = x^7 + 15x^2 - 30x + a$ is irreducible in $Q[x]$.

\paragraph{Exercise 5.1.8} If $F$ is a field of characteristic $p \neq 0$, show that $(a + b)^m = a^m + b^m$, where $m = p^n$, for all $a, b \in F$ and any positive integer $n$.

\paragraph{Exercise 5.2.20} Let $V$ be a vector space over an infinite field $F$. Show that $V$ cannot be the set-theoretic union of a finite number of proper subspaces of $V$.

\paragraph{Exercise 5.3.7} If $a \in K$ is such that $a^2$ is algebraic over the subfield $F$ of $K$, show that a is algebraic over $F$.

\paragraph{Exercise 5.3.10} Prove that $\cos 1^{\circ}$  is algebraic over $\mathbb{Q}$.

\paragraph{Exercise 5.4.3} If $a \in C$ is such that $p(a) = 0$, where $p(x) = x^5 + \sqrt{2}x^3 + \sqrt{5}x^2 + \sqrt{7}x + \sqrt{11}$, show that $a$ is algebraic over $\mathbb{Q}$ of degree at most 80.

\paragraph{Exercise 5.5.2} Prove that $x^3 - 3x - 1$ is irreducible over $\mathbb{Q}$.

\paragraph{Exercise 5.6.3} Let $\mathbb{Q}$ be the rational field and let $p(x) = x^4 + x^3 + x^2 + x + 1$.  Show that there is an extension $K$ of $Q$ with $[K:Q] = 4$ over which $p(x)$ splits into linear factors.

\paragraph{Exercise 5.6.14} If $F$ is of characteristic $p \neq 0$, show that all the roots of $x^m - x$, where $m = p^n$, are distinct.

\end{document}
