\documentclass{article}

\title{\textbf{
Exercises from \\
\textit{Abstract Algebra} \\
by I. N. Herstein
}}

\date{}

\usepackage{amsmath}
\usepackage{amssymb}

\begin{document}
\maketitle

\paragraph{Exercise 2.1.18} If $G$ is a finite group of even order, show that there must be an element $a \neq e$ such that $a=a^{-1}$.

\paragraph{Exercise 2.2.3} If $G$ is a group in which $(a b)^{i}=a^{i} b^{i}$ for three consecutive integers $i$, prove that $G$ is abelian.

\paragraph{Exercise 2.2.5} Let $G$ be a group in which $(a b)^{3}=a^{3} b^{3}$ and $(a b)^{5}=a^{5} b^{5}$ for all $a, b \in G$. Show that $G$ is abelian.

\paragraph{Exercise 2.2.6c} Let $G$ be a group in which $(a b)^{n}=a^{n} b^{n}$ for some fixed integer $n>1$ for all $a, b \in G$. For all $a, b \in G$, prove that $\left(a b a^{-1} b^{-1}\right)^{n(n-1)}=e$. 

\paragraph{Exercise 2.3.17} If $G$ is a group and $a, x \in G$, prove that $C\left(x^{-1} a x\right)=x^{-1} C(a) x$

\paragraph{Exercise 2.3.19} If $M$ is a subgroup of $G$ such that $x^{-1} M x \subset M$ for all $x \in G$, prove that actually $x^{-1} M x=M$. 

\paragraph{
\end{document}
